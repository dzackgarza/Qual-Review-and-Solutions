%% ================================================================================
%% This LaTeX file was created by AbiWord.                                         
%% AbiWord is a free, Open Source word processor.                                  
%% More information about AbiWord is available at http://www.abisource.com/        
%% ================================================================================

\documentclass[a4paper,portrait,12pt]{article}
\usepackage[latin1]{inputenc}
\usepackage{calc}
\usepackage{setspace}
\usepackage{fixltx2e}
\usepackage{graphicx}
\usepackage{multicol}
\usepackage[normalem]{ulem}
%% Please revise the following command, if your babel
%% package does not support en-US
\usepackage[en]{babel}
\usepackage{color}
\usepackage{hyperref}
 
\begin{document}


\begin{flushleft}
Study Guide for Algebra Qualifying Exam
\end{flushleft}


\begin{flushleft}
Group Theory
\end{flushleft}


\begin{flushleft}
subgroups and quotient groups
\end{flushleft}


\begin{flushleft}
Lagrange's Theorem
\end{flushleft}


\begin{flushleft}
fundamental homomorphism theorems
\end{flushleft}


\begin{flushleft}
group actions with applications to the structure of groups such as
\end{flushleft}


\begin{flushleft}
the Sylow Theorems
\end{flushleft}


\begin{flushleft}
group constructions such as:
\end{flushleft}


\begin{flushleft}
direct and semi-direct products
\end{flushleft}


\begin{flushleft}
structures of special types of groups such as:
\end{flushleft}


\begin{flushleft}
p-groups
\end{flushleft}


\begin{flushleft}
dihedral, symmetric and alternating groups, cycle decompositions
\end{flushleft}


\begin{flushleft}
the simplicity of An, for n $\geq$ 5
\end{flushleft}


\begin{flushleft}
free groups, generators and relations
\end{flushleft}


\begin{flushleft}
solvable groups
\end{flushleft}


\begin{flushleft}
References: [1,3,4]
\end{flushleft}


\begin{flushleft}
Linear Algebra
\end{flushleft}


\begin{flushleft}
determinants
\end{flushleft}


\begin{flushleft}
eigenvalues and eigenvectors
\end{flushleft}


\begin{flushleft}
Cayley-Hamilton Theorem
\end{flushleft}


\begin{flushleft}
canonical forms for matrices
\end{flushleft}


\begin{flushleft}
linear groups (GLn , SLn, On, Un)
\end{flushleft}


\begin{flushleft}
dual spaces, dual bases, induced dual map, double duals
\end{flushleft}


\begin{flushleft}
finite-dimensional spectral theorem
\end{flushleft}


\begin{flushleft}
References: [1,2,4]
\end{flushleft}


\begin{flushleft}
Foundations
\end{flushleft}


\begin{flushleft}
Zorn's Lemma and its uses in various existence theorems such as that of a basis
\end{flushleft}


\begin{flushleft}
for a vector space or existence of maximal ideals.
\end{flushleft}


\begin{flushleft}
References: [1,3,4]
\end{flushleft}


\begin{flushleft}
Theory of Rings and Modules
\end{flushleft}


\begin{flushleft}
basic properties of ideals and quotient rings
\end{flushleft}


\begin{flushleft}
fundamental homomorphism theorems for rings and modules
\end{flushleft}


\begin{flushleft}
characterizations and properties of special domains such as:
\end{flushleft}


\begin{flushleft}
Euclidean implies PID implies UFD
\end{flushleft}


\begin{flushleft}
classification of finitely generated modules over PIDs with emphasis on Euclidean
\end{flushleft}


\begin{flushleft}
domains
\end{flushleft}


\begin{flushleft}
applications to the structure of:
\end{flushleft}


\begin{flushleft}
finitely generated abelian groups
\end{flushleft}


\begin{flushleft}
canonical forms of matrices
\end{flushleft}


\begin{flushleft}
References: [1,3,4]
\end{flushleft}





\begin{flushleft}
\newpage
Field Theory
\end{flushleft}


\begin{flushleft}
algebraic extensions of fields
\end{flushleft}


\begin{flushleft}
fundamental theorem of Galois theory
\end{flushleft}


\begin{flushleft}
properties of finite fields
\end{flushleft}


\begin{flushleft}
separable extensions
\end{flushleft}


\begin{flushleft}
computations of Galois groups of polynomials of small degree and cyclotomic
\end{flushleft}


\begin{flushleft}
polynomials
\end{flushleft}


\begin{flushleft}
solvability of polynomials by radicals
\end{flushleft}


\begin{flushleft}
References: [1,3,4]
\end{flushleft}


\begin{flushleft}
As a general rule, students are responsible for knowing both the theory (proofs) and practical
\end{flushleft}


\begin{flushleft}
applications (e.g. how to find the Jordan or rational canonical form of a given matrix, or the
\end{flushleft}


\begin{flushleft}
Galois group of a given polynomial) of the topics mentioned. A supplement to this study guide
\end{flushleft}


\begin{flushleft}
is available at:
\end{flushleft}


\begin{flushleft}
http://www.math.uga.edu/sites/default/files/PDFs/Graduate/QualsStudyGuides/AlgebraPhDqualr
\end{flushleft}


\begin{flushleft}
emarks.pdf
\end{flushleft}


\begin{flushleft}
References
\end{flushleft}


\begin{flushleft}
[1] David Dummit and Richard Foote, Abstract Algebra, Wiley, 2003.
\end{flushleft}


\begin{flushleft}
[2] Kenneth Hoffman and Ray Kunze, Linear Algebra, Prentice-Hall, 1971.
\end{flushleft}


\begin{flushleft}
[3] Thomas W. Hungerford, Algebra, Springer, 1974.
\end{flushleft}


\begin{flushleft}
[4] Roy Smith, Algebra Course Notes (843-1 through 845-3), http://www.math.uga.edu/\~{}roy/,
\end{flushleft}


1996.


\begin{flushleft}
[Revised December 2016]
\end{flushleft}





\newpage
\newpage



\end{document}
