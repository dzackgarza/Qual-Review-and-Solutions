\subsection{Exercises}

\begin{problem}[Hungerford 4.4.1]
Show the following:
\begin{enumerate}
    \item For any abelian group $A$ and any positive integer $m$,
    $$
    \mathrm{Hom}(\mathbb{Z}_m, A) \cong A[m] \coloneqq \{ a\in A \mid ma = 0\}.
    $$
    \item $\mathrm{Hom}(\mathbb{Z}_m, \mathbb{Z}_n) \cong \mathbb{Z}_{\mathrm{gcd}(m,n)}$.
    \item As a $\mathbb{Z}-$module, $\mathbb{Z}_m^* = 0$.
    \item For each $k\geq 1$, $\mathbb{Z}_m$ is a $\mathbb{Z}_{mk}-$module, and as a $\mathbb{Z}_{mk}$ module, $\mathbb{Z}_m^* \cong \mathbb{Z}_m$.
\end{enumerate}
\end{problem}

\begin{problem}[Hungerford 4.4.3]
Let $\pi: \mathbb{Z} \to \mathbb{Z}_2$ be the canonical epimorphism. Show that the induced map $\overline{\pi}: \mathrm{Hom}(\mathbb Z_2, \mathbb Z) \to \mathrm{Hom}(\mathbb Z_2, \mathbb Z_2)$ is the zero map. Conclude that $\overline{\pi}$ is not an epimorphism.
\end{problem}

\begin{problem}[Hungerford 4.4.5]
Let $R$ be a unital ring, show that there is a ring homomorphism $\mathrm{Hom}_R(R, R) \to R^{op}$ where $\mathrm{Hom}_R$ denotes left $R-$module homomorphisms. Conclude that if $R$ is commutative, then there is a ring isomorphism $\mathrm{Hom}_R(R, R) \cong R$.
\end{problem}

\begin{problem}[Hungerford 4.4.9]
Show that for any homomorphism $f: A \to B$ of left $R-$modules the following diagram is commutative:

\begin{center}
\begin{tikzcd}
A \arrow[rr, "\theta_A"] \arrow[dd, "f"] &  & A^{**} \arrow[dd, "f^*"] \\
                                         &  &                          \\
B \arrow[rr, "\theta_B"]                 &  & B^{**}                  
\end{tikzcd}
\end{center}

where $\theta_A, \theta_B$ are as in Theorem 4.12 and $f^*$ is the map induced on $A^{**} \coloneqq \mathrm{Hom}_R(\mathrm{Hom}(A, R), R)$ by the map $$\overline f: \mathrm{Hom}(B, R) \to \mathrm{Hom}_R(A, R).$$
\end{problem}


\begin{problem}[Hungerford 4.6.2]
Show that every free module over a unital integral domain is torsion-free. Show that the converse is false.
\end{problem}

\begin{problem}[Hungerford 4.6.3]
Let $A$ be a cyclic $R-$module of order $r \in R$.
\begin{enumerate}
    \item Show that if $s$ is relatively prime to $r$, then $sA = A$ and $A[s] = 0$.
    \item If $s$ divides $r$, so $sk = r$, then $sA \cong R/(k)$ and $A[s] \cong R/(s)$.
\end{enumerate}
\end{problem}

\begin{problem}[Hungerford 4.6.6]
Let $A, B$ be cyclic modules over $R$ of nonzero orders $r,s$ respectively, where $r$ is \textit{not} relatively prime to $s$. Show that the invariant factors of $A \oplus B$ are $\mathrm{gcd}(r, s)$ and $\mathrm{lcm}(r, s)$.
\end{problem}

\newpage
\subsection{Qual Problems}

\begin{problem}
Let $R$ be a PID. Let $n > 0$ and $A \in M_n(R)$ be a square $n\times n$ matrix with coefficients in $R$.

Consider the $R$-module $M \coloneqq R^n / \mathrm{im}(A)$.

\begin{enumerate}
    \item Give a necessary and sufficient condition for $M$ to be a torsion module (i.e. every nonzero element is torsion).
    Justify your answer.
    
    \item Let $F$ be a field and now let $R \coloneqq F[x]$. Give an example of an integer $n>0$ and an $n \times n$ square matrix $A \in M_n(R)$ such that $M \coloneqq R^n/\mathrm{im}(A)$ is isomorphic as an $R-$module to $R \times F$.
\end{enumerate}
\end{problem}

\begin{problem}
\begin{enumerate}
    \item State the structure theorem for finitely generated modules over a PID.
    \item Find the decomposition of the $\mathbb{Z}-$module $M$ generated by $w,x,y,z$ satisfying the relations
    \begin{align*}
        3w + 12y + 3x + 6z &=0 \\
        6y &= 0 \\
        -3w -3x + 6y &= 0.
    \end{align*}
\end{enumerate}
\end{problem}

\begin{problem}
Let $R$ be a commutative ring and $M$ an $R-$module.
\begin{enumerate}
    \item Define what a torsion element of $M$ is .
    \item Given an example of a ring $R$ and a cyclic $R-$module $M$ such that $M$ is infinite and $M$ contains a nontrivial torsion element $m$.
    Justify why $m$ is torsion.
    \item Show that if $R$ is a domain, then the subset of elements of $M$ that are torsion is an $R-$submodule of $M$. Clearly show where the hypothesis that $R$ is a domain is used.
\end{enumerate}
\end{problem}
