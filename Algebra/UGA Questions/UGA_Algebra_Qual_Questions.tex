\input{"preamble.tex"}

\addbibresource{UGA\_Algebra\_Qual\_Questions.bib}

\let\Begin\begin
\let\End\end
\newcommand\wrapenv[1]{#1}

\makeatletter
\def\ScaleWidthIfNeeded{%
 \ifdim\Gin@nat@width>\linewidth
    \linewidth
  \else
    \Gin@nat@width
  \fi
}
\def\ScaleHeightIfNeeded{%
  \ifdim\Gin@nat@height>0.9\textheight
    0.9\textheight
  \else
    \Gin@nat@width
  \fi
}
\makeatother

\setkeys{Gin}{width=\ScaleWidthIfNeeded,height=\ScaleHeightIfNeeded,keepaspectratio}%

\title{
\textbf{
    UGA Algebra Qualifying Exam Questions (Spring 2011 -- Spring 2021)
  }
  }







\begin{document}

\date{}
\author{D. Zack Garza}
\maketitle


\newpage

% Note: addsec only in KomaScript
\addsec{Table of Contents}
\tableofcontents
\newpage

\hypertarget{group-theory-general}{%
\section{Group Theory: General}\label{group-theory-general}}

\hypertarget{spring-2020-2-work}{%
\subsection{\texorpdfstring{Spring 2020 \#2
\(\work\)}{Spring 2020 \#2 \textbackslash work}}\label{spring-2020-2-work}}

Let \(H\) be a normal subgroup of a finite group \(G\) where the order
of \(H\) and the index of \(H\) in \(G\) are relatively prime. Prove
that no other subgroup of \(G\) has the same order as \(H\).

\todo[inline]{Work this problem.}

\hypertarget{spring-2019-4-done}{%
\subsection{\texorpdfstring{Spring 2019 \#4
\(\done\)}{Spring 2019 \#4 \textbackslash done}}\label{spring-2019-4-done}}

For a finite group \(G\), let \(c(G)\) denote the number of conjugacy
classes of \(G\).

\begin{enumerate}
\def\labelenumi{\alph{enumi}.}
\item
  Prove that if two elements of \(G\) are chosen uniformly at
  random,then the probability they commute is precisely
  \begin{align*}
  \frac{c(G)}{{\left\lvert {G} \right\rvert}}
  .\end{align*}
\item
  State the class equation for a finite group.
\item
  Using the class equation (or otherwise) show that the probability in
  part (a) is at most
  \begin{align*}
  \frac 1 2 + \frac 1 {2[G : Z(G)]}
  .\end{align*}
\end{enumerate}

\begin{quote}
Here, as usual, \(Z(G)\) denotes the center of \(G\).
\end{quote}

Section omitted.

Section omitted.

Section omitted.

\hypertarget{spring-2012-2-work}{%
\subsection{\texorpdfstring{Spring 2012 \#2
\(\work\)}{Spring 2012 \#2 \textbackslash work}}\label{spring-2012-2-work}}

Let \(G\) be a finite group and \(p\) a prime number such that there is
a normal subgroup \(H{~\trianglelefteq~}G\) with
\({\left\lvert {H} \right\rvert} = p^i > 1\).

\begin{enumerate}
\def\labelenumi{\alph{enumi}.}
\item
  Show that \(H\) is a subgroup of any Sylow \(p{\hbox{-}}\)subgroup of
  \(G\).
\item
  Show that \(G\) contains a nonzero abelian normal subgroup of order
  divisible by \(p\).
\end{enumerate}

\hypertarget{spring-2017-1-work}{%
\subsection{\texorpdfstring{Spring 2017 \#1
\(\work\)}{Spring 2017 \#1 \textbackslash work}}\label{spring-2017-1-work}}

Let \(G\) be a finite group and \(\pi: G\to \operatorname{Sym}(G)\) the
Cayley representation.

\begin{quote}
(Recall that this means that for an element \(x\in G\), \(\pi(x)\) acts
by left translation on \(G\).)
\end{quote}

Prove that \(\pi(x)\) is an odd permutation \(\iff\) the order
\({\left\lvert {\pi(x)} \right\rvert}\) of \(\pi(x)\) is even and
\({\left\lvert {G} \right\rvert} / {\left\lvert {\pi(x)} \right\rvert}\)
is odd.

\hypertarget{fall-2016-1-work}{%
\subsection{\texorpdfstring{Fall 2016 \#1
\(\work\)}{Fall 2016 \#1 \textbackslash work}}\label{fall-2016-1-work}}

Let \(G\) be a finite group and \(s, t\in G\) be two distinct elements
of order 2. Show that subgroup of \(G\) generated by \(s\) and \(t\) is
a dihedral group.

\begin{quote}
Recall that the dihedral groups of order \(2m\) for \(m\geq 2\) are of
the form
\begin{align*}
D_{2m} = \left\langle{\sigma, \tau {~\mathrel{\Big|}~}\sigma^m = 1 = \tau^2, \tau \sigma = \sigma^{-1}\tau}\right\rangle
.\end{align*}
\end{quote}

\hypertarget{fall-2015-1-work}{%
\subsection{\texorpdfstring{Fall 2015 \#1
\(\work\)}{Fall 2015 \#1 \textbackslash work}}\label{fall-2015-1-work}}

Let \(G\) be a group containing a subgroup \(H\) not equal to \(G\) of
finite index. Prove that \(G\) has a normal subgroup which is contained
in every conjugate of \(H\) which is of finite index.

\hypertarget{spring-2015-1-work}{%
\subsection{\texorpdfstring{Spring 2015 \#1
\(\work\)}{Spring 2015 \#1 \textbackslash work}}\label{spring-2015-1-work}}

For a prime \(p\), let \(G\) be a finite \(p{\hbox{-}}\)group and let
\(N\) be a normal subgroup of \(G\) of order \(p\). Prove that \(N\) is
contained in the center of \(G\).

\hypertarget{fall-2014-6-work}{%
\subsection{\texorpdfstring{Fall 2014 \#6
\(\work\)}{Fall 2014 \#6 \textbackslash work}}\label{fall-2014-6-work}}

Let \(G\) be a group and \(H, K < G\) be subgroups of finite index. Show
that
\begin{align*}
[G: H\cap K] \leq [G: H] ~ [G:K]
.\end{align*}

\hypertarget{spring-2013-3-work}{%
\subsection{\texorpdfstring{Spring 2013 \#3
\(\work\)}{Spring 2013 \#3 \textbackslash work}}\label{spring-2013-3-work}}

Let \(P\) be a finite \(p{\hbox{-}}\)group. Prove that every nontrivial
normal subgroup of \(P\) intersects the center of \(P\) nontrivially.

\hypertarget{fall-2019-midterm-1-work}{%
\subsection{\texorpdfstring{Fall 2019 Midterm \#1
\(\work\)}{Fall 2019 Midterm \#1 \textbackslash work}}\label{fall-2019-midterm-1-work}}

Let \(G\) be a group of order \(p^2q\) for \(p, q\) prime. Show that
\(G\) has a nontrivial normal subgroup.

\hypertarget{fall-2019-midterm-4-work}{%
\subsection{\texorpdfstring{Fall 2019 Midterm \#4
\(\work\)}{Fall 2019 Midterm \#4 \textbackslash work}}\label{fall-2019-midterm-4-work}}

Let \(p\) be a prime. Show that
\(S_p = \left\langle{\tau, \sigma}\right\rangle\) where \(\tau\) is a
transposition and \(\sigma\) is a \(p{\hbox{-}}\)cycle.

\hypertarget{fall-2019-midterm-5-work}{%
\subsection{\texorpdfstring{Fall 2019 Midterm \#5
\(\work\)}{Fall 2019 Midterm \#5 \textbackslash work}}\label{fall-2019-midterm-5-work}}

Let \(G\) be a nonabelian group of order \(p^3\) for \(p\) prime. Show
that \(Z(G) = [G, G]\).

\hypertarget{spring-2021-2-work}{%
\subsection{\texorpdfstring{Spring 2021 \#2
\(\work\)}{Spring 2021 \#2 \textbackslash work}}\label{spring-2021-2-work}}

Let \(H {~\trianglelefteq~}G\) be a normal subgroup of a finite group
\(G\), where the order of \(H\) is the smallest prime \(p\) dividing
\({\left\lvert {G} \right\rvert}\). Prove that \(H\) is contained in the
center of \(G\).

\hypertarget{groups-sylow-theory}{%
\section{Groups: Sylow Theory}\label{groups-sylow-theory}}

\hypertarget{fall-2019-1-done}{%
\subsection{\texorpdfstring{Fall 2019 \#1
\(\done\)}{Fall 2019 \#1 \textbackslash done}}\label{fall-2019-1-done}}

Let \(G\) be a finite group with \(n\) distinct conjugacy classes. Let
\(g_1 \cdots g_n\) be representatives of the conjugacy classes of \(G\).
Prove that if \(g_i g_j = g_j g_i\) for all \(i, j\) then \(G\) is
abelian.

Section omitted.

Section omitted.

\hypertarget{fall-2019-midterm-2-work}{%
\subsection{\texorpdfstring{Fall 2019 Midterm \#2
\(\work\)}{Fall 2019 Midterm \#2 \textbackslash work}}\label{fall-2019-midterm-2-work}}

Let \(G\) be a finite group and let \(P\) be a sylow
\(p{\hbox{-}}\)subgroup for \(p\) prime. Show that \(N(N(P)) = N(P)\)
where \(N\) is the normalizer in \(G\).

\hypertarget{fall-2013-2-work}{%
\subsection{\texorpdfstring{Fall 2013 \#2
\(\work\)}{Fall 2013 \#2 \textbackslash work}}\label{fall-2013-2-work}}

Let \(G\) be a group of order 30.

\begin{enumerate}
\def\labelenumi{\alph{enumi}.}
\item
  Show that \(G\) has a subgroup of order 15.
\item
  Show that every group of order 15 is cyclic.
\item
  Show that \(G\) is isomorphic to some semidirect product
  \({\mathbb{Z}}_{15} \rtimes{\mathbb{Z}}_2\).
\item
  Exhibit three nonisomorphic groups of order 30 and prove that they are
  not isomorphic. You are not required to use your answer to (c).
\end{enumerate}

\hypertarget{spring-2014-2-work}{%
\subsection{\texorpdfstring{Spring 2014 \#2
\(\work\)}{Spring 2014 \#2 \textbackslash work}}\label{spring-2014-2-work}}

Let \(G\subset S_9\) be a Sylow-3 subgroup of the symmetric group on 9
letters.

\begin{enumerate}
\def\labelenumi{\alph{enumi}.}
\item
  Show that \(G\) contains a subgroup \(H\) isomorphic to
  \({\mathbb{Z}}_3 \times{\mathbb{Z}}_3 \times{\mathbb{Z}}_3\) by
  exhibiting an appropriate set of cycles.
\item
  Show that \(H\) is normal in \(G\).
\item
  Give generators and relations for \(G\) as an abstract group, such
  that all generators have order 3. Also exhibit elements of \(S_9\) in
  cycle notation corresponding to these generators.
\item
  Without appealing to the previous parts of the problem, show that
  \(G\) contains an element of order 9.
\end{enumerate}

\hypertarget{fall-2014-2-work}{%
\subsection{\texorpdfstring{Fall 2014 \#2
\(\work\)}{Fall 2014 \#2 \textbackslash work}}\label{fall-2014-2-work}}

Let \(G\) be a group of order 96.

\begin{enumerate}
\def\labelenumi{\alph{enumi}.}
\item
  Show that \(G\) has either one or three 2-Sylow subgroups.
\item
  Show that either \(G\) has a normal subgroup of order 32, or a normal
  subgroup of order 16.
\end{enumerate}

\hypertarget{spring-2016-3-work}{%
\subsection{\texorpdfstring{Spring 2016 \#3
\(\work\)}{Spring 2016 \#3 \textbackslash work}}\label{spring-2016-3-work}}

\begin{enumerate}
\def\labelenumi{\alph{enumi}.}
\item
  State the three Sylow theorems.
\item
  Prove that any group of order 1225 is abelian.
\item
  Write down exactly one representative in each isomorphism class of
  abelian groups of order 1225.
\end{enumerate}

\hypertarget{spring-2017-2-work}{%
\subsection{\texorpdfstring{Spring 2017 \#2
\(\work\)}{Spring 2017 \#2 \textbackslash work}}\label{spring-2017-2-work}}

\begin{enumerate}
\def\labelenumi{\alph{enumi}.}
\item
  How many isomorphism classes of abelian groups of order 56 are there?
  Give a representative for one of each class.
\item
  Prove that if \(G\) is a group of order 56, then either the Sylow-2
  subgroup or the Sylow-7 subgroup is normal.
\item
  Give two non-isomorphic groups of order 56 where the Sylow-7 subgroup
  is normal and the Sylow-2 subgroup is \emph{not} normal. Justify that
  these two groups are not isomorphic.
\end{enumerate}

\hypertarget{fall-2017-2-work}{%
\subsection{\texorpdfstring{Fall 2017 \#2
\(\work\)}{Fall 2017 \#2 \textbackslash work}}\label{fall-2017-2-work}}

\begin{enumerate}
\def\labelenumi{\alph{enumi}.}
\item
  Classify the abelian groups of order 36.

  \begin{quote}
  For the rest of the problem, assume that \(G\) is a non-abelian group
  of order 36. You may assume that the only subgroup of order 12 in
  \(S_4\) is \(A_4\) and that \(A_4\) has no subgroup of order 6.
  \end{quote}
\item
  Prove that if the 2-Sylow subgroup of \(G\) is normal, \(G\) has a
  normal subgroup \(N\) such that \(G/N\) is isomorphic to \(A_4\).
\item
  Show that if \(G\) has a normal subgroup \(N\) such that \(G/N\) is
  isomorphic to \(A_4\) and a subgroup \(H\) isomorphic to \(A_4\) it
  must be the direct product of \(N\) and \(H\).
\item
  Show that the dihedral group of order 36 is a non-abelian group of
  order 36 whose Sylow-2 subgroup is not normal.
\end{enumerate}

\hypertarget{fall-2012-2-work}{%
\subsection{\texorpdfstring{Fall 2012 \#2
\(\work\)}{Fall 2012 \#2 \textbackslash work}}\label{fall-2012-2-work}}

Let \(G\) be a group of order 30.

\begin{enumerate}
\def\labelenumi{\alph{enumi}.}
\item
  Show that \(G\) contains normal subgroups of orders 3, 5, and 15.
\item
  Give all possible presentations and relations for \(G\).
\item
  Determine how many groups of order 30 there are up to isomorphism.
\end{enumerate}

\hypertarget{fall-2018-1-done}{%
\subsection{\texorpdfstring{Fall 2018 \#1
\(\done\)}{Fall 2018 \#1 \textbackslash done}}\label{fall-2018-1-done}}

Let \(G\) be a finite group whose order is divisible by a prime number
\(p\). Let \(P\) be a normal \(p{\hbox{-}}\)subgroup of \(G\) (so
\({\left\lvert {P} \right\rvert} = p^c\) for some \(c\)).

\begin{enumerate}
\def\labelenumi{\alph{enumi}.}
\item
  Show that \(P\) is contained in every Sylow \(p{\hbox{-}}\)subgroup of
  \(G\).
\item
  Let \(M\) be a maximal proper subgroup of \(G\). Show that either
  \(P \subseteq M\) or \(|G/M | = p^b\) for some \(b \leq c\).
\end{enumerate}

Section omitted.

Section omitted.

\hypertarget{fall-2019-2-done}{%
\subsection{\texorpdfstring{Fall 2019 \#2
\(\done\)}{Fall 2019 \#2 \textbackslash done}}\label{fall-2019-2-done}}

Let \(G\) be a group of order 105 and let \(P, Q, R\) be Sylow 3, 5, 7
subgroups respectively.

\begin{enumerate}
\def\labelenumi{\alph{enumi}.}
\item
  Prove that at least one of \(Q\) and \(R\) is normal in \(G\).
\item
  Prove that \(G\) has a cyclic subgroup of order 35.
\item
  Prove that both \(Q\) and \(R\) are normal in \(G\).
\item
  Prove that if \(P\) is normal in \(G\) then \(G\) is cyclic.
\end{enumerate}

Section omitted.

Section omitted.

\hypertarget{spring-2021-3-work}{%
\subsection{\texorpdfstring{Spring 2021 \#3
\(\work\)}{Spring 2021 \#3 \textbackslash work}}\label{spring-2021-3-work}}

\begin{enumerate}
\def\labelenumi{\alph{enumi}.}
\item
  Show that every group of order \(p^2\) with \(p\) prime is abelian.
\item
  State the 3 Sylow theorems.
\item
  Show that any group of order \(4225 = 5^2 13^2\) is abelian.
\item
  Write down one representative from each isomorphism class of abelian
  groups of order 4225.
\end{enumerate}

\hypertarget{fall-2020-1-work}{%
\subsection{\texorpdfstring{Fall 2020 \#1
\(\work\)}{Fall 2020 \#1 \textbackslash work}}\label{fall-2020-1-work}}

\begin{enumerate}
\def\labelenumi{\alph{enumi}.}
\item
  Using Sylow theory, show that every group of order \(2p\) where \(p\)
  is prime is not simple.
\item
  Classify all groups of order \(2p\) and justify your answer. For the
  nonabelian group(s), give a presentation by generators and relations.
\end{enumerate}

\hypertarget{fall-2020-2-work}{%
\subsection{\texorpdfstring{Fall 2020 \#2
\(\work\)}{Fall 2020 \#2 \textbackslash work}}\label{fall-2020-2-work}}

Let \(G\) be a group of order 60 whose Sylow 3-subgroup is normal.

\begin{enumerate}
\def\labelenumi{\alph{enumi}.}
\item
  Prove that \(G\) is solvable.
\item
  Prove that the Sylow 5-subgroup is also normal.
\end{enumerate}

\hypertarget{groups-group-actions}{%
\section{Groups: Group Actions}\label{groups-group-actions}}

\hypertarget{fall-2012-1-work}{%
\subsection{\texorpdfstring{Fall 2012 \#1
\(\work\)}{Fall 2012 \#1 \textbackslash work}}\label{fall-2012-1-work}}

Let \(G\) be a finite group and \(X\) a set on which \(G\) acts.

\begin{enumerate}
\def\labelenumi{\alph{enumi}.}
\item
  Let \(x\in X\) and
  \(G_x \coloneqq\left\{{g\in G {~\mathrel{\Big|}~}g\cdot x = x}\right\}\).
  Show that \(G_x\) is a subgroup of \(G\).
\item
  Let \(x\in X\) and
  \(G\cdot x \coloneqq\left\{{g\cdot x {~\mathrel{\Big|}~}g\in G}\right\}\).
  Prove that there is a bijection between elements in \(G\cdot x\) and
  the left cosets of \(G_x\) in \(G\).
\end{enumerate}

\hypertarget{fall-2015-2-work}{%
\subsection{\texorpdfstring{Fall 2015 \#2
\(\work\)}{Fall 2015 \#2 \textbackslash work}}\label{fall-2015-2-work}}

Let \(G\) be a finite group, \(H\) a \(p{\hbox{-}}\)subgroup, and \(P\)
a sylow \(p{\hbox{-}}\)subgroup for \(p\) a prime. Let \(H\) act on the
left cosets of \(P\) in \(G\) by left translation.

Prove that this is an orbit under this action of length 1.

Prove that \(xP\) is an orbit of length 1 \(\iff H\) is contained in
\(xPx^{-1}\).

\hypertarget{spring-2016-5-work}{%
\subsection{\texorpdfstring{Spring 2016 \#5
\(\work\)}{Spring 2016 \#5 \textbackslash work}}\label{spring-2016-5-work}}

Let \(G\) be a finite group acting on a set \(X\). For \(x\in X\), let
\(G_x\) be the stabilizer of \(x\) and \(G\cdot x\) be the orbit of
\(x\).

\begin{enumerate}
\def\labelenumi{\alph{enumi}.}
\item
  Prove that there is a bijection between the left cosets \(G/G_x\) and
  \(G\cdot x\).
\item
  Prove that the center of every finite \(p{\hbox{-}}\)group \(G\) is
  nontrivial by considering that action of \(G\) on \(X=G\) by
  conjugation.
\end{enumerate}

\hypertarget{fall-2017-1-work}{%
\subsection{\texorpdfstring{Fall 2017 \#1
\(\work\)}{Fall 2017 \#1 \textbackslash work}}\label{fall-2017-1-work}}

Suppose the group \(G\) acts on the set \(A\). Assume this action is
faithful (recall that this means that the kernel of the homomorphism
from \(G\) to \(\operatorname{Sym}(A)\) which gives the action is
trivial) and transitive (for all \(a, b\) in \(A\), there exists \(g\)
in \(G\) such that \(g \cdot a = b\).)

\begin{enumerate}
\def\labelenumi{\alph{enumi}.}
\item
  For \(a \in A\), let \(G_a\) denote the stabilizer of \(a\) in \(G\).
  Prove that for any \(a \in A\),
  \begin{align*}
  \bigcap_{\sigma\in G} \sigma G_a \sigma^{-1}= \left\{{1}\right\}
  .\end{align*}
\item
  Suppose that \(G\) is abelian. Prove that \(|G| = |A|\). Deduce that
  every abelian transitive subgroup of \(S_n\) has order \(n\).
\end{enumerate}

\hypertarget{fall-2018-2-done}{%
\subsection{\texorpdfstring{Fall 2018 \#2
\(\done\)}{Fall 2018 \#2 \textbackslash done}}\label{fall-2018-2-done}}

\begin{enumerate}
\def\labelenumi{\alph{enumi}.}
\item
  Suppose the group \(G\) acts on the set \(X\) . Show that the
  stabilizers of elements in the same orbit are conjugate.
\item
  Let \(G\) be a finite group and let \(H\) be a proper subgroup. Show
  that the union of the conjugates of \(H\) is strictly smaller than
  \(G\), i.e.
  \begin{align*}
  \bigcup_{g\in G} gHg^{-1}\subsetneq G
  \end{align*}
\item
  Suppose \(G\) is a finite group acting transitively on a set \(S\)
  with at least 2 elements. Show that there is an element of \(G\) with
  no fixed points in \(S\).
\end{enumerate}

Section omitted.

Section omitted.

\hypertarget{groups-classification}{%
\section{Groups: Classification}\label{groups-classification}}

\hypertarget{spring-2020-1-work}{%
\subsection{\texorpdfstring{Spring 2020 \#1
\(\work\)}{Spring 2020 \#1 \textbackslash work}}\label{spring-2020-1-work}}

\begin{enumerate}
\def\labelenumi{\alph{enumi}.}
\item
  Show that any group of order 2020 is solvable.
\item
  Give (without proof) a classification of all abelian groups of order
  2020.
\item
  Describe one nonabelian group of order 2020.
\end{enumerate}

\todo[inline]{Work this problem.}

\hypertarget{spring-2019-3-done}{%
\subsection{\texorpdfstring{Spring 2019 \#3
\(\done\)}{Spring 2019 \#3 \textbackslash done}}\label{spring-2019-3-done}}

How many isomorphism classes are there of groups of order 45?

Describe a representative from each class.

Section omitted.

Section omitted.

\todo[inline]{Revisit, seems short.}

\hypertarget{spring-2012-3-work}{%
\subsection{\texorpdfstring{Spring 2012 \#3
\(\work\)}{Spring 2012 \#3 \textbackslash work}}\label{spring-2012-3-work}}

Let \(G\) be a group of order 70.

\begin{enumerate}
\def\labelenumi{\alph{enumi}.}
\item
  Show that \(G\) is not simple.
\item
  Exhibit 3 nonisomorphic groups of order 70 and prove that they are not
  isomorphic.
\end{enumerate}

\hypertarget{fall-2016-3-work}{%
\subsection{\texorpdfstring{Fall 2016 \#3
\(\work\)}{Fall 2016 \#3 \textbackslash work}}\label{fall-2016-3-work}}

How many groups are there up to isomorphism of order \(pq\) where
\(p<q\) are prime integers?

\hypertarget{spring-2018-1-done}{%
\subsection{\texorpdfstring{Spring 2018 \#1
\(\done\)}{Spring 2018 \#1 \textbackslash done}}\label{spring-2018-1-done}}

\begin{enumerate}
\def\labelenumi{\alph{enumi}.}
\item
  Use the Class Equation (equivalently, the conjugation action of a
  group on itself) to prove that any \(p{\hbox{-}}\)group (a group whose
  order is a positive power of a prime integer \(p\)) has a nontrivial
  center.
\item
  Prove that any group of order \(p^2\) (where \(p\) is prime) is
  abelian.
\item
  Prove that any group of order \(5^2 \cdot 7^2\) is abelian.
\item
  Write down exactly one representative in each isomorphism class of
  groups of order \(5^2 \cdot 7^2\).
\end{enumerate}

Section omitted.

Section omitted.

\hypertarget{groups-simple-and-solvable}{%
\section{Groups: Simple and Solvable}\label{groups-simple-and-solvable}}

\hypertarget{star-fall-2016-7-work}{%
\subsection{\texorpdfstring{\(\star\) Fall 2016 \#7
\(\work\)}{\textbackslash star Fall 2016 \#7 \textbackslash work}}\label{star-fall-2016-7-work}}

\begin{enumerate}
\def\labelenumi{\alph{enumi}.}
\item
  Define what it means for a group \(G\) to be \emph{solvable}.
\item
  Show that every group \(G\) of order 36 is solvable.
\end{enumerate}

\begin{quote}
Hint: you can use that \(S_4\) is solvable.
\end{quote}

\hypertarget{spring-2015-4-work}{%
\subsection{\texorpdfstring{Spring 2015 \#4
\(\work\)}{Spring 2015 \#4 \textbackslash work}}\label{spring-2015-4-work}}

Let \(N\) be a positive integer, and let \(G\) be a finite group of
order \(N\).

\begin{enumerate}
\def\labelenumi{\alph{enumi}.}
\item
  Let \(\operatorname{Sym}G\) be the set of all bijections from
  \(G\to G\) viewed as a group under composition. Note that
  \(\operatorname{Sym}G \cong S_N\). Prove that the Cayley map
  \begin{align*}
  C: G&\to \operatorname{Sym}G\\
  g &\mapsto (x\mapsto gx)
  \end{align*}
  is an injective homomorphism.
\item
  Let \(\Phi: \operatorname{Sym}G\to S_N\) be an isomorphism. For
  \(a\in G\) define \(\varepsilon(a) \in \left\{{\pm 1}\right\}\) to be
  the sign of the permutation \(\Phi(C(a))\). Suppose that \(a\) has
  order \(d\). Prove that \(\varepsilon(a) = -1 \iff d\) is even and
  \(N/d\) is odd.
\item
  Suppose \(N> 2\) and \(n\equiv 2 \pmod 4\). Prove that \(G\) is not
  simple.
\end{enumerate}

\begin{quote}
Hint: use part (b).
\end{quote}

\hypertarget{spring-2014-1-work}{%
\subsection{\texorpdfstring{Spring 2014 \#1
\(\work\)}{Spring 2014 \#1 \textbackslash work}}\label{spring-2014-1-work}}

Let \(p, n\) be integers such that \(p\) is prime and \(p\) does not
divide \(n\). Find a real number \(k = k (p, n)\) such that for every
integer \(m\geq k\), every group of order \(p^m n\) is not simple.

\hypertarget{fall-2013-1-work}{%
\subsection{\texorpdfstring{Fall 2013 \#1
\(\work\)}{Fall 2013 \#1 \textbackslash work}}\label{fall-2013-1-work}}

Let \(p, q\) be distinct primes.

\begin{enumerate}
\def\labelenumi{\alph{enumi}.}
\item
  Let
  \(\mkern 1.5mu\overline{\mkern-1.5muq\mkern-1.5mu}\mkern 1.5mu \in {\mathbb{Z}}_p\)
  be the class of \(q\pmod p\) and let \(k\) denote the order of
  \(\mkern 1.5mu\overline{\mkern-1.5muq\mkern-1.5mu}\mkern 1.5mu\) as an
  element of \({\mathbb{Z}}_p^{\times}\). Prove that no group of order
  \(pq^k\) is simple.
\item
  Let \(G\) be a group of order \(pq\), and prove that \(G\) is not
  simple.
\end{enumerate}

\hypertarget{spring-2013-4-work}{%
\subsection{\texorpdfstring{Spring 2013 \#4
\(\work\)}{Spring 2013 \#4 \textbackslash work}}\label{spring-2013-4-work}}

Define a \emph{simple group}. Prove that a group of order 56 can not be
simple.

\hypertarget{fall-2019-midterm-3-work}{%
\subsection{\texorpdfstring{Fall 2019 Midterm \#3
\(\work\)}{Fall 2019 Midterm \#3 \textbackslash work}}\label{fall-2019-midterm-3-work}}

Show that there exist no simple groups of order 148.

\hypertarget{commutative-algebra}{%
\section{Commutative Algebra}\label{commutative-algebra}}

\hypertarget{spring-2020-5-done}{%
\subsection{\texorpdfstring{Spring 2020 \#5
\(\done\)}{Spring 2020 \#5 \textbackslash done}}\label{spring-2020-5-done}}

Let \(R\) be a ring and \(f: M\to N\) and \(g: N\to M\) be
\(R{\hbox{-}}\)module homomorphisms such that
\(g\circ f = \operatorname{id}_M\). Show that
\(N \cong \operatorname{im}f \oplus \ker g\).

\hypertarget{fall-2019-3-work}{%
\subsection{\texorpdfstring{Fall 2019 \#3
\(\work\)}{Fall 2019 \#3 \textbackslash work}}\label{fall-2019-3-work}}

Let \(R\) be a ring with the property that for every
\(a \in R, a^2 = a\).

\begin{enumerate}
\def\labelenumi{\alph{enumi}.}
\item
  Prove that \(R\) has characteristic 2.
\item
  Prove that \(R\) is commutative.
\end{enumerate}

Section omitted.

Section omitted.

Section omitted.

\hypertarget{fall-2019-6-done}{%
\subsection{\texorpdfstring{Fall 2019 \#6
\(\done\)}{Fall 2019 \#6 \textbackslash done}}\label{fall-2019-6-done}}

Let \(R\) be a commutative ring with multiplicative identity. Assume
Zorn's Lemma.

\begin{enumerate}
\def\labelenumi{\alph{enumi}.}
\item
  Show that
  \begin{align*}
  N = \{r \in R \mathrel{\Big|}r^n = 0 \text{ for some } n > 0\}
  \end{align*}
  is an ideal which is contained in any prime ideal.
\item
  Let \(r\) be an element of \(R\) not in \(N\). Let \(S\) be the
  collection of all proper ideals of \(R\) not containing any positive
  power of \(r\). Use Zorn's Lemma to prove that there is a prime ideal
  in \(S\).
\item
  Suppose that \(R\) has exactly one prime ideal \(P\) . Prove that
  every element \(r\) of \(R\) is either nilpotent or a unit.
\end{enumerate}

Section omitted.

Section omitted.

\hypertarget{spring-2019-6-done}{%
\subsection{\texorpdfstring{Spring 2019 \#6
\(\done\)}{Spring 2019 \#6 \textbackslash done}}\label{spring-2019-6-done}}

Let \(R\) be a commutative ring with 1.

\begin{quote}
Recall that \(x \in R\) is nilpotent iff \(xn = 0\) for some positive
integer \(n\).
\end{quote}

\begin{enumerate}
\def\labelenumi{\alph{enumi}.}
\item
  Show that every proper ideal of \(R\) is contained within a maximal
  ideal.
\item
  Let \(J(R)\) denote the intersection of all maximal ideals of \(R\).
  Show that \(x \in J(R) \iff 1 + rx\) is a unit for all \(r \in R\).
\item
  Suppose now that \(R\) is finite. Show that in this case \(J(R)\)
  consists precisely of the nilpotent elements in R.
\end{enumerate}

Section omitted.

Section omitted.

\hypertarget{fall-2018-7-done}{%
\subsection{\texorpdfstring{Fall 2018 \#7
\(\done\)}{Fall 2018 \#7 \textbackslash done}}\label{fall-2018-7-done}}

Let \(R\) be a commutative ring.

\begin{enumerate}
\def\labelenumi{\alph{enumi}.}
\item
  Let \(r \in R\). Show that the map
  \begin{align*}
  r\bullet : R &\to R \\
  x &\mapsto r x
  .\end{align*}
  is an \(R{\hbox{-}}\)module endomorphism of \(R\).
\item
  We say that \(r\) is a \textbf{zero-divisor} if \(r\bullet\) is not
  injective. Show that if \(r\) is a zero-divisor and \(r \neq 0\), then
  the kernel and image of \(R\) each consist of zero-divisors.
\item
  Let \(n \geq 2\) be an integer. Show: if \(R\) has exactly \(n\)
  zero-divisors, then \(\#R \leq n^2\) .
\item
  Show that up to isomorphism there are exactly two commutative rings
  \(R\) with precisely 2 zero-divisors.
\end{enumerate}

\begin{quote}
You may use without proof the following fact: every ring of order 4 is
isomorphic to exactly one of the following:
\begin{align*}
\frac{ {\mathbb{Z}}}{ 4{\mathbb{Z}}}, \quad
\frac{ \frac{  {\mathbb{Z}}}{ 2{\mathbb{Z}}} [t]}{(t^2 + t + 1)}, \quad
\frac{ \frac{ {\mathbb{Z}}}{ 2{\mathbb{Z}}} [t]}{ (t^2 - t)}, \quad
\frac{ \frac{ {\mathbb{Z}}}{2{\mathbb{Z}}}[t]}{(t^2 )}
.\end{align*}
\end{quote}

Section omitted.

Section omitted.

\hypertarget{spring-2018-5-work}{%
\subsection{\texorpdfstring{Spring 2018 \#5
\(\work\)}{Spring 2018 \#5 \textbackslash work}}\label{spring-2018-5-work}}

Let
\begin{align*}
M=\left(\begin{array}{ll}{a} & {b} \\ {c} & {d}\end{array}\right)
\quad \text{and} \quad 
N=\left(\begin{array}{cc}{x} & {u} \\ {-y} & {-v}\end{array}\right)
\end{align*}

over a commutative ring \(R\), where \(b\) and \(x\) are units of \(R\).
Prove that
\begin{align*}
M N=\left(\begin{array}{ll}{0} & {0} \\ {0} & {*}\end{array}\right)
\implies MN = 0
.\end{align*}

\hypertarget{spring-2018-8-work}{%
\subsection{\texorpdfstring{Spring 2018 \#8
\(\work\)}{Spring 2018 \#8 \textbackslash work}}\label{spring-2018-8-work}}

Let \(R = C[0, 1]\) be the ring of continuous real-valued functions on
the interval \([0, 1]\). Let I be an ideal of \(R\).

\begin{enumerate}
\def\labelenumi{\alph{enumi}.}
\item
  Show that if \(f \in I, a \in [0, 1]\) are such that \(f (a) \neq 0\),
  then there exists \(g \in I\) such that \(g(x) \geq 0\) for all
  \(x \in [0, 1]\), and \(g(x) > 0\) for all \(x\) in some open
  neighborhood of \(a\).
\item
  If \(I \neq R\), show that the set
  \(Z(I) = \{x \in [0, 1] {~\mathrel{\Big|}~}f(x) = 0 \text{ for all } f \in I\}\)
  is nonempty.
\item
  Show that if \(I\) is maximal, then there exists \(x_0 \in [0, 1]\)
  such that \(I = \{ f \in R {~\mathrel{\Big|}~}f (x_0 ) = 0\}\).
\end{enumerate}

\hypertarget{fall-2017-5-work}{%
\subsection{\texorpdfstring{Fall 2017 \#5
\(\work\)}{Fall 2017 \#5 \textbackslash work}}\label{fall-2017-5-work}}

A ring \(R\) is called \emph{simple} if its only two-sided ideals are
\(0\) and \(R\).

\begin{enumerate}
\def\labelenumi{\alph{enumi}.}
\item
  Suppose \(R\) is a commutative ring with 1. Prove \(R\) is simple if
  and only if \(R\) is a field.
\item
  Let \(k\) be a field. Show the ring \(M_n (k)\), \(n \times n\)
  matrices with entries in \(k\), is a simple ring.
\end{enumerate}

\hypertarget{fall-2017-6-work}{%
\subsection{\texorpdfstring{Fall 2017 \#6
\(\work\)}{Fall 2017 \#6 \textbackslash work}}\label{fall-2017-6-work}}

For a ring \(R\), let \(U(R)\) denote the multiplicative group of units
in \(R\). Recall that in an integral domain \(R\), \(r \in R\) is called
\emph{irreducible} if \(r\) is not a unit in R, and the only divisors of
\(r\) have the form \(ru\) with \(u\) a unit in \(R\).

We call a non-zero, non-unit \(r \in R\) \emph{prime} in \(R\) if
\(r \bigm|ab \implies r \bigm|a\) or \(r \bigm|b\). Consider the ring
\(R = \{a + b \sqrt{-5}{~\mathrel{\Big|}~}a, b \in Z\}\).

\begin{enumerate}
\def\labelenumi{\alph{enumi}.}
\item
  Prove \(R\) is an integral domain.
\item
  Show \(U(R) = \{\pm1\}\).
\item
  Show \(3, 2 + \sqrt{-5}\), and \(2 - \sqrt{-5}\) are irreducible in
  \(R\).
\item
  Show 3 is not prime in \(R\).
\item
  Conclude \(R\) is not a PID.
\end{enumerate}

\hypertarget{spring-2017-3-work}{%
\subsection{\texorpdfstring{Spring 2017 \#3
\(\work\)}{Spring 2017 \#3 \textbackslash work}}\label{spring-2017-3-work}}

Let \(R\) be a commutative ring with 1. Suppose that \(M\) is a free
\(R{\hbox{-}}\)module with a finite basis \(X\).

\begin{enumerate}
\def\labelenumi{\alph{enumi}.}
\item
  Let \(I {~\trianglelefteq~}R\) be a proper ideal. Prove that \(M/IM\)
  is a free \(R/I{\hbox{-}}\)module with basis \(X'\), where \(X'\) is
  the image of \(X\) under the canonical map \(M\to M/IM\).
\item
  Prove that any two bases of \(M\) have the same number of elements.
  You may assume that the result is true when \(R\) is a field.
\end{enumerate}

\hypertarget{spring-2017-4-work}{%
\subsection{\texorpdfstring{Spring 2017 \#4
\(\work\)}{Spring 2017 \#4 \textbackslash work}}\label{spring-2017-4-work}}

\begin{enumerate}
\def\labelenumi{\alph{enumi}.}
\item
  Let \(R\) be an integral domain with quotient field \(F\). Suppose
  that \(p(x), a(x), b(x)\) are monic polynomials in \(F[x]\) with
  \(p(x) = a(x) b(x)\) and with \(p(x) \in R[x]\), \(a(x)\) not in
  \(R[x]\), and both \(a(x), b(x)\) not constant.

  Prove that \(R\) is not a UFD.

  \begin{quote}
  (You may assume Gauss' lemma)
  \end{quote}
\item
  Prove that \({\mathbb{Z}}[2\sqrt{2}]\) is not a UFD.

  \begin{quote}
  Hint: let \(p(x) = x^2-2\).
  \end{quote}
\end{enumerate}

\hypertarget{spring-2016-8-work}{%
\subsection{\texorpdfstring{Spring 2016 \#8
\(\work\)}{Spring 2016 \#8 \textbackslash work}}\label{spring-2016-8-work}}

Let \(R\) be a simple rng (a nonzero ring which is not assume to have a
1, whose only two-sided ideals are \((0)\) and \(R\)) satisfying the
following two conditions:

\begin{enumerate}
\def\labelenumi{\roman{enumi}.}
\tightlist
\item
  \(R\) has no zero divisors, and
\item
  If \(x\in R\) with \(x\neq 0\) then \(2x\neq 0\), where
  \(2x\coloneqq x+x\).
\end{enumerate}

Prove the following:

\begin{enumerate}
\def\labelenumi{\alph{enumi}.}
\item
  For each \(x\in R\) there is one and only one element \(y\in R\) such
  that \(x = 2y\).
\item
  Suppose \(x,y\in R\) such that \(x\neq 0\) and \(2(xy) = x\), then
  \(yz = zy\) for all \(z\in R\).
\end{enumerate}

\begin{quote}
You can get partial credit for (b) by showing it in the case \(R\) has a
1.
\end{quote}

\hypertarget{fall-2015-3-work}{%
\subsection{\texorpdfstring{Fall 2015 \#3
\(\work\)}{Fall 2015 \#3 \textbackslash work}}\label{fall-2015-3-work}}

Let \(R\) be a rng (a ring without 1) which contains an element \(u\)
such that for all \(y\in R\), there exists an \(x\in R\) such that
\(xu=y\).

Prove that \(R\) contains a maximal left ideal.

\begin{quote}
Hint: imitate the proof (using Zorn's lemma) in the case where \(R\)
does have a 1.
\end{quote}

\hypertarget{fall-2015-4-work}{%
\subsection{\texorpdfstring{Fall 2015 \#4
\(\work\)}{Fall 2015 \#4 \textbackslash work}}\label{fall-2015-4-work}}

Let \(R\) be a PID and \((a_1) < (a_2) < \cdots\) be an ascending chain
of ideals in \(R\). Prove that for some \(n\), we have \((a_j) = (a_n)\)
for all \(j\geq n\).

\hypertarget{spring-2015-7-work}{%
\subsection{\texorpdfstring{Spring 2015 \#7
\(\work\)}{Spring 2015 \#7 \textbackslash work}}\label{spring-2015-7-work}}

Let \(R\) be a commutative ring, and \(S\subset R\) be a nonempty subset
that does not contain 0 such that for all \(x, y\in S\) we have
\(xy\in S\). Let \({\mathcal{I}}\) be the set of all ideals
\(I{~\trianglelefteq~}R\) such that \(I\cap S = \emptyset\).

Show that for every ideal \(I\in {\mathcal{I}}\), there is an ideal
\(J\in {\mathcal{I}}\) such that \(I\subset J\) and \(J\) is not
properly contained in any other ideal in \({\mathcal{I}}\).

Prove that every such ideal \(J\) is prime.

\hypertarget{fall-2014-7-work}{%
\subsection{\texorpdfstring{Fall 2014 \#7
\(\work\)}{Fall 2014 \#7 \textbackslash work}}\label{fall-2014-7-work}}

Give a careful proof that \({\mathbb{C}}[x, y]\) is not a PID.

\hypertarget{fall-2014-8-work}{%
\subsection{\texorpdfstring{Fall 2014 \#8
\(\work\)}{Fall 2014 \#8 \textbackslash work}}\label{fall-2014-8-work}}

Let \(R\) be a nonzero commutative ring without unit such that \(R\)
does not contain a proper maximal ideal. Prove that for all \(x\in R\),
the ideal \(xR\) is proper.

\begin{quote}
You may assume the axiom of choice.
\end{quote}

\hypertarget{spring-2014-5-work}{%
\subsection{\texorpdfstring{Spring 2014 \#5
\(\work\)}{Spring 2014 \#5 \textbackslash work}}\label{spring-2014-5-work}}

Let \(R\) be a commutative ring and \(a\in R\). Prove that \(a\) is not
nilpotent \(\iff\) there exists a commutative ring \(S\) and a ring
homomorphism \(\phi: R\to S\) such that \(\phi(a)\) is a unit.

\begin{quote}
Note: by definition, \(a\) is nilpotent \(\iff\) there is a natural
number \(n\) such that \(a^n = 0\).
\end{quote}

\hypertarget{spring-2014-6-work}{%
\subsection{\texorpdfstring{Spring 2014 \#6
\(\work\)}{Spring 2014 \#6 \textbackslash work}}\label{spring-2014-6-work}}

\(R\) be a commutative ring with identity and let \(n\) be a positive
integer.

\begin{enumerate}
\def\labelenumi{\alph{enumi}.}
\item
  Prove that every surjective \(R{\hbox{-}}\)linear endomorphism
  \(T: R^n \to R^n\) is injective.
\item
  Show that an injective \(R{\hbox{-}}\)linear endomorphism of \(R^n\)
  need not be surjective.
\end{enumerate}

\hypertarget{fall-2013-3-work}{%
\subsection{\texorpdfstring{Fall 2013 \#3
\(\work\)}{Fall 2013 \#3 \textbackslash work}}\label{fall-2013-3-work}}

\begin{enumerate}
\def\labelenumi{\alph{enumi}.}
\item
  Define \emph{prime ideal}, give an example of a nontrivial ideal in
  the ring \({\mathbb{Z}}\) that is not prime, and prove that it is not
  prime.
\item
  Define \emph{maximal ideal}, give an example of a nontrivial maximal
  ideal in \({\mathbb{Z}}\) and prove that it is maximal.
\end{enumerate}

\hypertarget{fall-2013-4-work}{%
\subsection{\texorpdfstring{Fall 2013 \#4
\(\work\)}{Fall 2013 \#4 \textbackslash work}}\label{fall-2013-4-work}}

Let \(R\) be a commutative ring with \(1\neq 0\). Recall that \(x\in R\)
is \emph{nilpotent} iff \(x^n = 0\) for some positive integer \(n\).

\begin{enumerate}
\def\labelenumi{\alph{enumi}.}
\item
  Show that the collection of nilpotent elements in \(R\) forms an
  ideal.
\item
  Show that if \(x\) is nilpotent, then \(x\) is contained in every
  prime ideal of \(R\).
\item
  Suppose \(x\in R\) is not nilpotent and let
  \(S = \left\{{x^n {~\mathrel{\Big|}~}n\in {\mathbb{N}}}\right\}\).
  There is at least on ideal of \(R\) disjoint from \(S\), namely
  \((0)\).
\end{enumerate}

By Zorn's lemma the set of ideals disjoint from \(S\) has a maximal
element with respect to inclusion, say \(I\). In other words, \(I\) is
disjoint from \(S\) and if \(J\) is any ideal disjoint from \(S\) with
\(I\subseteq J \subseteq R\) then \(J=I\) or \(J=R\).

Show that \(I\) is a prime ideal.

\begin{enumerate}
\def\labelenumi{\alph{enumi}.}
\setcounter{enumi}{3}
\tightlist
\item
  Deduce from (a) and (b) that the set of nilpotent elements of \(R\) is
  the intersection of all prime ideals of \(R\).
\end{enumerate}

\hypertarget{spring-2013-1-work}{%
\subsection{\texorpdfstring{Spring 2013 \#1
\(\work\)}{Spring 2013 \#1 \textbackslash work}}\label{spring-2013-1-work}}

Let \(R\) be a commutative ring.

\begin{enumerate}
\def\labelenumi{\alph{enumi}.}
\item
  Define a \emph{maximal ideal} and prove that \(R\) has a maximal
  ideal.
\item
  Show than an element \(r\in R\) is not invertible \(\iff r\) is
  contained in a maximal ideal.
\item
  Let \(M\) be an \(R{\hbox{-}}\)module, and recall that for
  \(0\neq \mu \in M\), the \emph{annihilator} of \(\mu\) is the set
  \begin{align*}
  \operatorname{Ann}(\mu) = \left\{{r\in R {~\mathrel{\Big|}~}r\mu = 0}\right\}
  .\end{align*}
  Suppose that \(I\) is an ideal in \(R\) which is maximal with respect
  to the property that there exists an element \(\mu \in M\) such that
  \(I = \operatorname{Ann}(\mu)\) for some \(\mu \in M\). In other
  words, \(I = \operatorname{Ann}(\mu)\) but there does not exist
  \(\nu\in M\) with \(J = \operatorname{Ann}(\nu) \subsetneq R\) such
  that \(I\subsetneq J\).
\end{enumerate}

Prove that \(I\) is a prime ideal.

\hypertarget{spring-2013-2-work}{%
\subsection{\texorpdfstring{Spring 2013 \#2
\(\work\)}{Spring 2013 \#2 \textbackslash work}}\label{spring-2013-2-work}}

\begin{enumerate}
\def\labelenumi{\alph{enumi}.}
\item
  Define a \emph{Euclidean domain}.
\item
  Define a \emph{unique factorization domain}.
\item
  Is a Euclidean domain an UFD? Give either a proof or a counterexample
  with justification.
\item
  Is a UFD a Euclidean domain? Give either a proof or a counterexample
  with justification.
\end{enumerate}

\hypertarget{spring-2021-5-work}{%
\subsection{\texorpdfstring{Spring 2021 \#5
\(\work\)}{Spring 2021 \#5 \textbackslash work}}\label{spring-2021-5-work}}

Suppose that \(f(x) \in ({\mathbb{Z}}/n{\mathbb{Z}})[x]\) is a zero
divisor. Show that there is a nonzero
\(a\in {\mathbb{Z}}/n{\mathbb{Z}}\) with \(af(x) = 0\).

\hypertarget{spring-2021-6}{%
\subsection{Spring 2021 \#6}\label{spring-2021-6}}

\begin{enumerate}
\def\labelenumi{\alph{enumi}.}
\item
  Carefully state the definition of \textbf{Noetherian} for a
  commutative ring \(R\).
\item
  Let \(R\) be a subset of \({\mathbb{Z}}[x]\) consisting of all
  polynomials
  \begin{align*}
  f(x) = a_ 0 + a_1 x + a_2 x^2 + \cdots + a_nx^n
  \end{align*}
  such that \(a_k\) is even for \(1\leq k \leq n\). Show that \(R\) is a
  subring of \({\mathbb{Z}}[x]\).
\item
  Show that \(R\) is not Noetherian.
\end{enumerate}

\begin{quote}
\emph{Hint: consider the ideal generated by
\(\left\{{ 2x^k {~\mathrel{\Big|}~}1\leq k \in {\mathbb{Z}}}\right\}\).}
\end{quote}

\hypertarget{fields-and-galois-theory}{%
\section{Fields and Galois Theory}\label{fields-and-galois-theory}}

\hypertarget{star-fall-2016-5-work}{%
\subsection{\texorpdfstring{\(\star\) Fall 2016 \#5
\(\work\)}{\textbackslash star Fall 2016 \#5 \textbackslash work}}\label{star-fall-2016-5-work}}

How many monic irreducible polynomials over \({\mathbb{F}}_p\) of prime
degree \(\ell\) are there? Justify your answer.

\hypertarget{star-fall-2013-7-work}{%
\subsection{\texorpdfstring{\(\star\) Fall 2013 \#7
\(\work\)}{\textbackslash star Fall 2013 \#7 \textbackslash work}}\label{star-fall-2013-7-work}}

Let \(F = {\mathbb{F}}_2\) and let
\(\mkern 1.5mu\overline{\mkern-1.5muF\mkern-1.5mu}\mkern 1.5mu\) denote
its algebraic closure.

\begin{enumerate}
\def\labelenumi{\alph{enumi}.}
\item
  Show that
  \(\mkern 1.5mu\overline{\mkern-1.5muF\mkern-1.5mu}\mkern 1.5mu\) is
  not a finite extension of \(F\).
\item
  Suppose that
  \(\alpha \in \mkern 1.5mu\overline{\mkern-1.5muF\mkern-1.5mu}\mkern 1.5mu\)
  satisfies \(\alpha^{17} = 1\) and \(\alpha\neq 1\). Show that
  \(F(\alpha)/F\) has degree 8.
\end{enumerate}

\hypertarget{fall-2019-4-done}{%
\subsection{\texorpdfstring{Fall 2019 \#4
\(\done\)}{Fall 2019 \#4 \textbackslash done}}\label{fall-2019-4-done}}

Let \(F\) be a finite field with \(q\) elements. Let \(n\) be a positive
integer relatively prime to \(q\) and let \(\omega\) be a primitive
\(n\)th root of unity in an extension field of \(F\). Let
\(E = F [\omega]\) and let \(k = [E : F]\).

\begin{enumerate}
\def\labelenumi{\alph{enumi}.}
\item
  Prove that \(n\) divides \(q^{k}-1\).
\item
  Let \(m\) be the order of \(q\) in
  \({\mathbb{Z}}/n{\mathbb{Z}}^{\times}\). Prove that \(m\) divides
  \(k\).
\item
  Prove that \(m = k\).
\end{enumerate}

\todo[inline]{Revisit, tricky!}

Section omitted.

Section omitted.

\hypertarget{fall-2019-7-done}{%
\subsection{\texorpdfstring{Fall 2019 \#7
\(\done\)}{Fall 2019 \#7 \textbackslash done}}\label{fall-2019-7-done}}

Let \(\zeta_n\) denote a primitive \(n\)th root of 1
\(\in {\mathbb{Q}}\). You may assume the roots of the minimal polynomial
\(p_n(x)\) of \(\zeta_n\) are exactly the primitive \(n\)th roots of 1.

Show that the field extension \({\mathbb{Q}}(\zeta_n )\) over
\({\mathbb{Q}}\) is Galois and prove its Galois group is
\(({\mathbb{Z}}/n{\mathbb{Z}})^{\times}\).

How many subfields are there of \({\mathbb{Q}}(\zeta_{20} )\)?

Section omitted.

Section omitted.

\hypertarget{spring-2019-2-done}{%
\subsection{\texorpdfstring{Spring 2019 \#2
\(\done\)}{Spring 2019 \#2 \textbackslash done}}\label{spring-2019-2-done}}

Let \(F = {\mathbb{F}}_p\) , where \(p\) is a prime number.

\begin{enumerate}
\def\labelenumi{\alph{enumi}.}
\item
  Show that if \(\pi(x) \in F[x]\) is irreducible of degree \(d\), then
  \(\pi(x)\) divides \(x^{p^d} - x\).
\item
  Show that if \(\pi(x) \in F[x]\) is an irreducible polynomial that
  divides \(x^{p^n} - x\), then \(\deg \pi(x)\) divides \(n\).
\end{enumerate}

Section omitted.

Section omitted.

\hypertarget{spring-2019-8-done}{%
\subsection{\texorpdfstring{Spring 2019 \#8
\(\done\)}{Spring 2019 \#8 \textbackslash done}}\label{spring-2019-8-done}}

Let \(\zeta = e^{2\pi i/8}\).

\begin{enumerate}
\def\labelenumi{\alph{enumi}.}
\item
  What is the degree of \({\mathbb{Q}}(\zeta)/{\mathbb{Q}}\)?
\item
  How many quadratic subfields of \({\mathbb{Q}}(\zeta)\) are there?
\item
  What is the degree of \({\mathbb{Q}}(\zeta, \sqrt[4] 2)\) over
  \({\mathbb{Q}}\)?
\end{enumerate}

Section omitted.

Section omitted.

\hypertarget{fall-2018-3-done}{%
\subsection{\texorpdfstring{Fall 2018 \#3
\(\done\)}{Fall 2018 \#3 \textbackslash done}}\label{fall-2018-3-done}}

Let \(F \subset K \subset L\) be finite degree field extensions. For
each of the following assertions, give a proof or a counterexample.

\begin{enumerate}
\def\labelenumi{\alph{enumi}.}
\item
  If \(L/F\) is Galois, then so is \(K/F\).
\item
  If \(L/F\) is Galois, then so is \(L/K\).
\item
  If \(K/F\) and \(L/K\) are both Galois, then so is \(L/F\).
\end{enumerate}

Section omitted.

Section omitted.

\hypertarget{spring-2018-2-done}{%
\subsection{\texorpdfstring{Spring 2018 \#2
\(\done\)}{Spring 2018 \#2 \textbackslash done}}\label{spring-2018-2-done}}

Let \(f(x) = x^4 - 4x^2 + 2 \in {\mathbb{Q}}[x]\).

\begin{enumerate}
\def\labelenumi{\alph{enumi}.}
\item
  Find the splitting field \(K\) of \(f\), and compute
  \([K: {\mathbb{Q}}]\).
\item
  Find the Galois group \(G\) of \(f\), both as an explicit group of
  automorphisms, and as a familiar abstract group to which it is
  isomorphic.
\item
  Exhibit explicitly the correspondence between subgroups of \(G\) and
  intermediate fields between \({\mathbb{Q}}\) and \(k\).
\end{enumerate}

\todo[inline]{Not the nicest proof! Would be better to replace the ad-hoc computations at the end.}

Section omitted.

Section omitted.

\hypertarget{spring-2018-3-done}{%
\subsection{\texorpdfstring{Spring 2018 \#3
\(\done\)}{Spring 2018 \#3 \textbackslash done}}\label{spring-2018-3-done}}

Let \(K\) be a Galois extension of \({\mathbb{Q}}\) with Galois group
\(G\), and let \(E_1 , E_2\) be intermediate fields of \(K\) which are
the splitting fields of irreducible \(f_i (x) \in {\mathbb{Q}}[x]\).

Let \(E = E_1 E_2 \subset K\).

Let \(H_i = \operatorname{Gal}(K/E_i)\) and
\(H = \operatorname{Gal}(K/E)\).

\begin{enumerate}
\def\labelenumi{\alph{enumi}.}
\item
  Show that \(H = H_1 \cap H_2\).
\item
  Show that \(H_1 H_2\) is a subgroup of \(G\).
\item
  Show that
  \begin{align*}
  \operatorname{Gal}(K/(E_1 \cap E_2 )) = H_1 H_2
  .\end{align*}
\end{enumerate}

Section omitted.

Section omitted.

\hypertarget{spring-2020-4-work}{%
\subsection{\texorpdfstring{Spring 2020 \#4
\(\work\)}{Spring 2020 \#4 \textbackslash work}}\label{spring-2020-4-work}}

Let \(f(x) = x^4-2 \in {\mathbb{Q}}[x]\).

\begin{enumerate}
\def\labelenumi{\alph{enumi}.}
\item
  Define what it means for a finite extension field \(E\) of a field
  \(F\) to be a Galois extension.
\item
  Determine the Galois group \(\operatorname{Gal}(E/{\mathbb{Q}})\) for
  the polynomial \(f(x)\), and justify your answer carefully.
\item
  Exhibit a subfield \(K\) in \((b)\) such that
  \({\mathbb{Q}}\leq K \leq E\) with \(K\) not a Galois extension over
  \({\mathbb{Q}}\). Explain.
\end{enumerate}

\hypertarget{spring-2020-3-work}{%
\subsection{\texorpdfstring{Spring 2020 \#3
\(\work\)}{Spring 2020 \#3 \textbackslash work}}\label{spring-2020-3-work}}

Let \(E\) be an extension field of \(F\) and \(\alpha\in E\) be
algebraic of odd degree over \(F\).

\begin{enumerate}
\def\labelenumi{\alph{enumi}.}
\item
  Show that \(F(\alpha) = F(\alpha^2)\).
\item
  Prove that \(\alpha^{2020}\) is algebraic of odd degree over \(F\).
\end{enumerate}

\hypertarget{fall-2017-4-work}{%
\subsection{\texorpdfstring{Fall 2017 \#4
\(\work\)}{Fall 2017 \#4 \textbackslash work}}\label{fall-2017-4-work}}

\begin{enumerate}
\def\labelenumi{\alph{enumi}.}
\item
  Let \(f (x)\) be an irreducible polynomial of degree 4 in
  \({\mathbb{Q}}[x]\) whose splitting field \(K\) over \({\mathbb{Q}}\)
  has Galois group \(G = S_4\).

  Let \(\theta\) be a root of \(f(x)\). Prove that
  \({\mathbb{Q}}[\theta]\) is an extension of \({\mathbb{Q}}\) of degree
  4 and that there are no intermediate fields between \({\mathbb{Q}}\)
  and \({\mathbb{Q}}[\theta]\).
\item
  Prove that if \(K\) is a Galois extension of \({\mathbb{Q}}\) of
  degree 4, then there is an intermediate subfield between \(K\) and
  \({\mathbb{Q}}\).
\end{enumerate}

\hypertarget{fall-2017-3-work}{%
\subsection{\texorpdfstring{Fall 2017 \#3
\(\work\)}{Fall 2017 \#3 \textbackslash work}}\label{fall-2017-3-work}}

Let \(F\) be a field. Let \(f(x)\) be an irreducible polynomial in
\(F[x]\) of degree \(n\) and let \(g(x)\) be any polynomial in \(F[x]\).
Let \(p(x)\) be an irreducible factor (of degree \(m\)) of the
polynomial \(f(g(x))\).

Prove that \(n\) divides \(m\). Use this to prove that if \(r\) is an
integer which is not a perfect square, and \(n\) is a positive integer
then every irreducible factor of \(x^{2n} - r\) over \({\mathbb{Q}}[x]\)
has even degree.

\hypertarget{spring-2017-7-work}{%
\subsection{\texorpdfstring{Spring 2017 \#7
\(\work\)}{Spring 2017 \#7 \textbackslash work}}\label{spring-2017-7-work}}

Let \(F\) be a field and let \(f(x) \in F[x]\).

\begin{enumerate}
\def\labelenumi{\alph{enumi}.}
\item
  Define what a splitting field of \(f(x)\) over \(F\) is.
\item
  Let \(F\) now be a finite field with \(q\) elements. Let \(E/F\) be a
  finite extension of degree \(n>0\). Exhibit an explicit polynomial
  \(g(x) \in F[x]\) such that \(E/F\) is a splitting field of \(g(x)\)
  over \(F\). Fully justify your answer.
\item
  Show that the extension \(E/F\) in (b) is a Galois extension.
\end{enumerate}

\hypertarget{spring-2017-8-work}{%
\subsection{\texorpdfstring{Spring 2017 \#8
\(\work\)}{Spring 2017 \#8 \textbackslash work}}\label{spring-2017-8-work}}

\begin{enumerate}
\def\labelenumi{\alph{enumi}.}
\item
  Let \(K\) denote the splitting field of \(x^5 - 2\) over
  \({\mathbb{Q}}\). Show that the Galois group of \(K/{\mathbb{Q}}\) is
  isomorphic to the group of invertible matrices
  \begin{align*}
  \left(\begin{array}{ll}
  a & b \\
  0 & 1
  \end{array}\right) 
  {\quad \operatorname{where} \quad} a\in {\mathbb{F}}_5^{\times}\text{ and } b\in {\mathbb{F}}_5
  .\end{align*}
\item
  Determine all intermediate fields between \(K\) and \({\mathbb{Q}}\)
  which are Galois over \({\mathbb{Q}}\).
\end{enumerate}

\hypertarget{fall-2016-4-work}{%
\subsection{\texorpdfstring{Fall 2016 \#4
\(\work\)}{Fall 2016 \#4 \textbackslash work}}\label{fall-2016-4-work}}

Set \(f(x) = x^3 - 5 \in {\mathbb{Q}}[x]\).

\begin{enumerate}
\def\labelenumi{\alph{enumi}.}
\item
  Find the splitting field \(K\) of \(f(x)\) over \({\mathbb{Q}}\).
\item
  Find the Galois group \(G\) of \(K\) over \({\mathbb{Q}}\).
\item
  Exhibit explicitly the correspondence between subgroups of \(G\) and
  intermediate fields between \({\mathbb{Q}}\) and \(K\).
\end{enumerate}

\hypertarget{spring-2016-2-work}{%
\subsection{\texorpdfstring{Spring 2016 \#2
\(\work\)}{Spring 2016 \#2 \textbackslash work}}\label{spring-2016-2-work}}

Let \(K = {\mathbb{Q}}[\sqrt 2 + \sqrt 5]\).

\begin{enumerate}
\def\labelenumi{\alph{enumi}.}
\item
  Find \([K: {\mathbb{Q}}]\).
\item
  Show that \(K/{\mathbb{Q}}\) is Galois, and find the Galois group
  \(G\) of \(K/{\mathbb{Q}}\).
\item
  Exhibit explicitly the correspondence between subgroups of \(G\) and
  intermediate fields between \({\mathbb{Q}}\) and \(K\).
\end{enumerate}

\hypertarget{spring-2016-6-work}{%
\subsection{\texorpdfstring{Spring 2016 \#6
\(\work\)}{Spring 2016 \#6 \textbackslash work}}\label{spring-2016-6-work}}

Let \(K\) be a Galois extension of a field \(F\) with \([K: F] = 2015\).
Prove that \(K\) is an extension by radicals of the field \(F\).

\hypertarget{fall-2015-5-work}{%
\subsection{\texorpdfstring{Fall 2015 \#5
\(\work\)}{Fall 2015 \#5 \textbackslash work}}\label{fall-2015-5-work}}

Let \(u = \sqrt{2 + \sqrt{2}}\), \(v = \sqrt{2 - \sqrt{2}}\), and
\(E = {\mathbb{Q}}(u)\).

\begin{enumerate}
\def\labelenumi{\alph{enumi}.}
\item
  Find (with justification) the minimal polynomial \(f(x)\) of \(u\)
  over \({\mathbb{Q}}\).
\item
  Show \(v\in E\), and show that \(E\) is a splitting field of \(f(x)\)
  over \({\mathbb{Q}}\).
\item
  Determine the Galois group of \(E\) over \({\mathbb{Q}}\) and
  determine all of the intermediate fields \(F\) such that
  \({\mathbb{Q}}\subset F \subset E\).
\end{enumerate}

\hypertarget{fall-2015-6-work}{%
\subsection{\texorpdfstring{Fall 2015 \#6
\(\work\)}{Fall 2015 \#6 \textbackslash work}}\label{fall-2015-6-work}}

\begin{enumerate}
\def\labelenumi{\alph{enumi}.}
\item
  Let \(G\) be a finite group. Show that there exists a field extension
  \(K/F\) with \(\operatorname{Gal}(K/F) = G\).

  \begin{quote}
  You may assume that for any natural number \(n\) there is a field
  extension with Galois group \(S_n\).
  \end{quote}
\item
  Let \(K\) be a Galois extension of \(F\) with
  \({\left\lvert {\operatorname{Gal}(K/F)} \right\rvert} = 12\). Prove
  that there exists an intermediate field \(E\) of \(K/F\) with
  \([E: F] = 3\).
\item
  With \(K/F\) as in (b), does an intermediate field \(L\) necessarily
  exist satisfying \([L: F] = 2\)? Give a proof or counterexample.
\end{enumerate}

\hypertarget{spring-2015-2-work}{%
\subsection{\texorpdfstring{Spring 2015 \#2
\(\work\)}{Spring 2015 \#2 \textbackslash work}}\label{spring-2015-2-work}}

Let \({\mathbb{F}}\) be a finite field.

\begin{enumerate}
\def\labelenumi{\alph{enumi}.}
\item
  Give (with proof) the decomposition of the additive group
  \(({\mathbb{F}}, +)\) into a direct sum of cyclic groups.
\item
  The \emph{exponent} of a finite group is the least common multiple of
  the orders of its elements. Prove that a finite abelian group has an
  element of order equal to its exponent.
\item
  Prove that the multiplicative group \(({\mathbb{F}}^{\times}, \cdot)\)
  is cyclic.
\end{enumerate}

\hypertarget{spring-2015-5-work}{%
\subsection{\texorpdfstring{Spring 2015 \#5
\(\work\)}{Spring 2015 \#5 \textbackslash work}}\label{spring-2015-5-work}}

Let \(f(x) = x^4 - 5 \in {\mathbb{Q}}[x]\).

\begin{enumerate}
\def\labelenumi{\alph{enumi}.}
\item
  Compute the Galois group of \(f\) over \({\mathbb{Q}}\).
\item
  Compute the Galois group of \(f\) over \({\mathbb{Q}}(\sqrt{5})\).
\end{enumerate}

\hypertarget{fall-2014-1-work}{%
\subsection{\texorpdfstring{Fall 2014 \#1
\(\work\)}{Fall 2014 \#1 \textbackslash work}}\label{fall-2014-1-work}}

Let \(f\in {\mathbb{Q}}[x]\) be an irreducible polynomial and \(L\) a
finite Galois extension of \({\mathbb{Q}}\). Let
\(f(x) = g_1(x)g_2(x)\cdots g_r(x)\) be a factorization of \(f\) into
irreducibles in \(L[x]\).

\begin{enumerate}
\def\labelenumi{\alph{enumi}.}
\item
  Prove that each of the factors \(g_i(x)\) has the same degree.
\item
  Give an example showing that if \(L\) is not Galois over
  \({\mathbb{Q}}\), the conclusion of part (a) need not hold.
\end{enumerate}

\hypertarget{fall-2014-3-work}{%
\subsection{\texorpdfstring{Fall 2014 \#3
\(\work\)}{Fall 2014 \#3 \textbackslash work}}\label{fall-2014-3-work}}

Consider the polynomial \(f(x) = x^4 - 7 \in {\mathbb{Q}}[x]\) and let
\(E/{\mathbb{Q}}\) be the splitting field of \(f\).

\begin{enumerate}
\def\labelenumi{\alph{enumi}.}
\item
  What is the structure of the Galois group of \(E/{\mathbb{Q}}\)?
\item
  Give an explicit description of all of the intermediate subfields
  \({\mathbb{Q}}\subset K \subset E\) in the form
  \(K = {\mathbb{Q}}(\alpha), {\mathbb{Q}}(\alpha, \beta), \cdots\)
  where \(\alpha, \beta\), etc are complex numbers. Describe the
  corresponding subgroups of the Galois group.
\end{enumerate}

\hypertarget{spring-2014-3-work}{%
\subsection{\texorpdfstring{Spring 2014 \#3
\(\work\)}{Spring 2014 \#3 \textbackslash work}}\label{spring-2014-3-work}}

Let \(F\subset C\) be a field extension with \(C\) algebraically closed.

\begin{enumerate}
\def\labelenumi{\alph{enumi}.}
\item
  Prove that the intermediate field \(C_{\text{alg}} \subset C\)
  consisting of elements algebraic over \(F\) is algebraically closed.
\item
  Prove that if \(F\to E\) is an algebraic extension, there exists a
  homomorphism \(E\to C\) that is the identity on \(F\).
\end{enumerate}

\hypertarget{spring-2014-4-work}{%
\subsection{\texorpdfstring{Spring 2014 \#4
\(\work\)}{Spring 2014 \#4 \textbackslash work}}\label{spring-2014-4-work}}

Let \(E\subset {\mathbb{C}}\) denote the splitting field over
\({\mathbb{Q}}\) of the polynomial \(x^3 - 11\).

\begin{enumerate}
\def\labelenumi{\alph{enumi}.}
\item
  Prove that if \(n\) is a squarefree positive integer, then
  \(\sqrt{n}\not\in E\).

  \begin{quote}
  Hint: you can describe all quadratic extensions of \({\mathbb{Q}}\)
  contained in \(E\).
  \end{quote}
\item
  Find the Galois group of \((x^3 - 11)(x^2 - 2)\) over
  \({\mathbb{Q}}\).
\item
  Prove that the minimal polynomial of \(11^{1/3} + 2^{1/2}\) over
  \({\mathbb{Q}}\) has degree 6.
\end{enumerate}

\hypertarget{fall-2013-5-work}{%
\subsection{\texorpdfstring{Fall 2013 \#5
\(\work\)}{Fall 2013 \#5 \textbackslash work}}\label{fall-2013-5-work}}

Let \(L/K\) be a finite extension of fields.

\begin{enumerate}
\def\labelenumi{\alph{enumi}.}
\item
  Define what it means for \(L/K\) to be \emph{separable}.
\item
  Show that if \(K\) is a finite field, then \(L/K\) is always
  separable.
\item
  Give an example of a finite extension \(L/K\) that is not separable.
\end{enumerate}

\hypertarget{fall-2013-6-work}{%
\subsection{\texorpdfstring{Fall 2013 \#6
\(\work\)}{Fall 2013 \#6 \textbackslash work}}\label{fall-2013-6-work}}

Let \(K\) be the splitting field of \(x^4-2\) over \({\mathbb{Q}}\) and
set \(G = \operatorname{Gal}(K/{\mathbb{Q}})\).

\begin{enumerate}
\def\labelenumi{\alph{enumi}.}
\item
  Show that \(K/{\mathbb{Q}}\) contains both \({\mathbb{Q}}(i)\) and
  \({\mathbb{Q}}(\sqrt[4]{2})\) and has degree 8 over \({\mathbb{Q}}\)/
\item
  Let \(N = \operatorname{Gal}(K/{\mathbb{Q}}(i))\) and
  \(H = \operatorname{Gal}(K/{\mathbb{Q}}(\sqrt[4]{2}))\). Show that
  \(N\) is normal in \(G\) and \(NH = G\).

  \begin{quote}
  Hint: what field is fixed by \(NH\)?
  \end{quote}
\item
  Show that \(\operatorname{Gal}(K/{\mathbb{Q}})\) is generated by
  elements \(\sigma, \tau\), of orders 4 and 2 respectively, with
  \(\tau \sigma\tau^{-1}= \sigma^{-1}\).

  \begin{quote}
  Equivalently, show it is the dihedral group of order 8.
  \end{quote}
\item
  How many distinct quartic subfields of \(K\) are there? Justify your
  answer.
\end{enumerate}

\hypertarget{spring-2013-7-work}{%
\subsection{\texorpdfstring{Spring 2013 \#7
\(\work\)}{Spring 2013 \#7 \textbackslash work}}\label{spring-2013-7-work}}

Let \(f(x) = g(x) h(x) \in {\mathbb{Q}}[x]\) and \(E,B,C/{\mathbb{Q}}\)
be the splitting fields of \(f,g,h\) respectively.

\begin{enumerate}
\def\labelenumi{\alph{enumi}.}
\item
  Prove that \(\operatorname{Gal}(E/B)\) and \(\operatorname{Gal}(E/C)\)
  are normal subgroups of \(\operatorname{Gal}(E/{\mathbb{Q}})\).
\item
  Prove that
  \(\operatorname{Gal}(E/B) \cap\operatorname{Gal}(E/C) = \left\{{1}\right\}\).
\item
  If \(B\cap C = {\mathbb{Q}}\), show that
  \(\operatorname{Gal}(E/B) \operatorname{Gal}(E/C) = \operatorname{Gal}(E/{\mathbb{Q}})\).
\item
  Under the hypothesis of (c), show that
  \(\operatorname{Gal}(E/{\mathbb{Q}}) \cong \operatorname{Gal}(E/B) \times \operatorname{Gal}(E/C)\).
\item
  Use (d) to describe
  \(\operatorname{Gal}({\mathbb{Q}}[\alpha]/{\mathbb{Q}})\) where
  \(\alpha = \sqrt 2 + \sqrt 3\).
\end{enumerate}

\hypertarget{spring-2013-8-work}{%
\subsection{\texorpdfstring{Spring 2013 \#8
\(\work\)}{Spring 2013 \#8 \textbackslash work}}\label{spring-2013-8-work}}

Let \(F\) be the field with 2 elements and \(K\) a splitting field of
\(f(x) = x^6 + x^3 + 1\) over \(F\). You may assume that \(f\) is
irreducible over \(F\).

\begin{enumerate}
\def\labelenumi{\alph{enumi}.}
\item
  Show that if \(r\) is a root of \(f\) in \(K\), then \(r^9 = 1\) but
  \(r^3\neq 1\).
\item
  Find \(\operatorname{Gal}(K/F)\) and express each intermediate field
  between \(F\) and \(K\) as \(F(\beta)\) for an appropriate
  \(\beta \in K\).
\end{enumerate}

\hypertarget{fall-2012-3-work}{%
\subsection{\texorpdfstring{Fall 2012 \#3
\(\work\)}{Fall 2012 \#3 \textbackslash work}}\label{fall-2012-3-work}}

Let \(f(x) \in {\mathbb{Q}}[x]\) be an irreducible polynomial of degree
5. Assume that \(f\) has all but two roots in \({\mathbb{R}}\). Compute
the Galois group of \(f(x)\) over \({\mathbb{Q}}\) and justify your
answer.

\hypertarget{fall-2012-4-work}{%
\subsection{\texorpdfstring{Fall 2012 \#4
\(\work\)}{Fall 2012 \#4 \textbackslash work}}\label{fall-2012-4-work}}

Let \(f(x) \in {\mathbb{Q}}[x]\) be a polynomial and \(K\) be a
splitting field of \(f\) over \({\mathbb{Q}}\). Assume that
\([K:{\mathbb{Q}}] = 1225\) and show that \(f(x)\) is solvable by
radicals.

\hypertarget{spring-2012-1-work}{%
\subsection{\texorpdfstring{Spring 2012 \#1
\(\work\)}{Spring 2012 \#1 \textbackslash work}}\label{spring-2012-1-work}}

Suppose that \(F\subset E\) are fields such that \(E/F\) is Galois and
\({\left\lvert {\operatorname{Gal}(E/F)} \right\rvert} = 14\).

\begin{enumerate}
\def\labelenumi{\alph{enumi}.}
\item
  Show that there exists a unique intermediate field \(K\) with
  \(F\subset K \subset E\) such that \([K: F] = 2\).
\item
  Assume that there are at least two distinct intermediate subfields
  \(F \subset L_1, L_2 \subset E\) with \([L_i: F]= 7\). Prove that
  \(\operatorname{Gal}(E/F)\) is nonabelian.
\end{enumerate}

\hypertarget{spring-2012-4-work}{%
\subsection{\texorpdfstring{Spring 2012 \#4
\(\work\)}{Spring 2012 \#4 \textbackslash work}}\label{spring-2012-4-work}}

Let \(f(x) = x^7 - 3\in {\mathbb{Q}}[x]\) and \(E/{\mathbb{Q}}\) be a
splitting field of \(f\) with \(\alpha \in E\) a root of \(f\).

\begin{enumerate}
\def\labelenumi{\alph{enumi}.}
\item
  Show that \(E\) contains a primitive 7th root of unity.
\item
  Show that \(E\neq {\mathbb{Q}}(\alpha)\).
\end{enumerate}

\hypertarget{fall-2019-midterm-6-work}{%
\subsection{\texorpdfstring{Fall 2019 Midterm \#6
\(\work\)}{Fall 2019 Midterm \#6 \textbackslash work}}\label{fall-2019-midterm-6-work}}

Compute the Galois group of
\(f(x) = x^3-3x -3\in {\mathbb{Q}}[x]/{\mathbb{Q}}\).

\hypertarget{fall-2019-midterm-7-work}{%
\subsection{\texorpdfstring{Fall 2019 Midterm \#7
\(\work\)}{Fall 2019 Midterm \#7 \textbackslash work}}\label{fall-2019-midterm-7-work}}

Show that a field \(k\) of characteristic \(p\neq 0\) is perfect
\(\iff\) for every \(x\in k\) there exists a \(y\in k\) such that
\(y^p=x\).

\hypertarget{fall-2019-midterm-8-work}{%
\subsection{\texorpdfstring{Fall 2019 Midterm \#8
\(\work\)}{Fall 2019 Midterm \#8 \textbackslash work}}\label{fall-2019-midterm-8-work}}

Let \(k\) be a field of characteristic \(p\neq 0\) and \(f\in k[x]\)
irreducible. Show that \(f(x) = g(x^{p^d})\) where \(g(x) \in k[x]\) is
irreducible and separable.

Conclude that every root of \(f\) has the same multiplicity \(p^d\) in
the splitting field of \(f\) over \(k\).

\hypertarget{fall-2019-midterm-9-work}{%
\subsection{\texorpdfstring{Fall 2019 Midterm \#9
\(\work\)}{Fall 2019 Midterm \#9 \textbackslash work}}\label{fall-2019-midterm-9-work}}

Let \(n\geq 3\) and \(\zeta_n\) be a primitive \(n\)th root of unity.
Show that
\([{\mathbb{Q}}(\zeta_n + \zeta_n^{-1}): {\mathbb{Q}}] = \phi(n)/2\) for
\(\phi\) the totient function. 10.

Let \(L/K\) be a finite normal extension.

\begin{enumerate}
\def\labelenumi{\alph{enumi}.}
\item
  Show that if \(L/K\) is cyclic and \(E/K\) is normal with \(L/E/K\)
  then \(L/E\) and \(E/K\) are cyclic.
\item
  Show that if \(L/K\) is cyclic then there exists exactly one extension
  \(E/K\) of degree \(n\) with \(L/E/K\) for each divisor \(n\) of
  \([L:K]\).
\end{enumerate}

\hypertarget{spring-2021-4-work}{%
\subsection{\texorpdfstring{Spring 2021 \#4
\(\work\)}{Spring 2021 \#4 \textbackslash work}}\label{spring-2021-4-work}}

Define
\begin{align*}
f(x) \coloneqq x^4 + 4x^2 + 64 \in {\mathbb{Q}}[x]
.\end{align*}

\begin{enumerate}
\def\labelenumi{\alph{enumi}.}
\item
  Find the splitting field \(K\) of \(f\) over \({\mathbb{Q}}\).
\item
  Find the Galois group \(G\) of \(f\).
\item
  Exhibit explicitly the correspondence between subgroups of \(G\) and
  intermediate fields between \({\mathbb{Q}}\) and \(K\).
\end{enumerate}

\hypertarget{spring-2021-7-done}{%
\subsection{\texorpdfstring{Spring 2021 \#7
\(\done\)}{Spring 2021 \#7 \textbackslash done}}\label{spring-2021-7-done}}

Let \(p\) be a prime number and let \(F\) be a field of characteristic
\(p\). Show that if \(a\in F\) is not a \(p\)th power in \(F\), then
\(x^p-a \in F[x]\) is irreducible.

Section omitted.

Section omitted.

Section omitted.

\hypertarget{fall-2020-3-work}{%
\subsection{\texorpdfstring{Fall 2020 \#3
\(\work\)}{Fall 2020 \#3 \textbackslash work}}\label{fall-2020-3-work}}

\begin{enumerate}
\def\labelenumi{\alph{enumi}.}
\item
  Define what it means for a finite extension of fields \(E\) over \(F\)
  to be a \emph{Galois} extension.
\item
  Determine the Galois group of \(f(x) = x^3 - 7\) over
  \({\mathbb{Q}}\), and justify your answer carefully.
\item
  Find all subfields of the splitting field of \(f(x)\) over
  \({\mathbb{Q}}\).
\end{enumerate}

\hypertarget{fall-2020-4-work}{%
\subsection{\texorpdfstring{Fall 2020 \#4
\(\work\)}{Fall 2020 \#4 \textbackslash work}}\label{fall-2020-4-work}}

Let \(K\) be a Galois extension of \(F\), and let
\(F \subset E \subset K\) be inclusions of fields. Let
\(G \coloneqq\operatorname{Gal}(K/F)\) and
\(H \coloneqq\operatorname{Gal}(K/E)\), and suppose \(H\) contains
\(N_G(P)\), where \(P\) is a Sylow \(p\)-subgroup of \(G\) for \(p\) a
prime. Prove that \([E: F] \equiv 1 \pmod p\).

\hypertarget{modules}{%
\section{Modules}\label{modules}}

\hypertarget{general-questions}{%
\subsection{General Questions}\label{general-questions}}

\hypertarget{fall-2018-6-done}{%
\subsubsection{\texorpdfstring{Fall 2018 \#6
\(\done\)}{Fall 2018 \#6 \textbackslash done}}\label{fall-2018-6-done}}

Let \(R\) be a commutative ring, and let \(M\) be an
\(R{\hbox{-}}\)module. An \(R{\hbox{-}}\)submodule \(N\) of \(M\) is
maximal if there is no \(R{\hbox{-}}\)module \(P\) with
\(N \subsetneq P \subsetneq M\).

\begin{enumerate}
\def\labelenumi{\alph{enumi}.}
\item
  Show that an \(R{\hbox{-}}\)submodule \(N\) of \(M\) is maximal
  \(\iff M /N\) is a simple \(R{\hbox{-}}\)module: i.e., \(M /N\) is
  nonzero and has no proper, nonzero \(R{\hbox{-}}\)submodules.
\item
  Let \(M\) be a \({\mathbb{Z}}{\hbox{-}}\)module. Show that a
  \({\mathbb{Z}}{\hbox{-}}\)submodule \(N\) of \(M\) is maximal
  \(\iff \#M /N\) is a prime number.
\item
  Let \(M\) be the \({\mathbb{Z}}{\hbox{-}}\)module of all roots of
  unity in \({\mathbb{C}}\) under multiplication. Show that there is no
  maximal \({\mathbb{Z}}{\hbox{-}}\)submodule of \(M\).
\end{enumerate}

Section omitted.

Section omitted.

\hypertarget{fall-2019-final-2-work}{%
\subsubsection{\texorpdfstring{Fall 2019 Final \#2
\(\work\)}{Fall 2019 Final \#2 \textbackslash work}}\label{fall-2019-final-2-work}}

Consider the \({\mathbb{Z}}{\hbox{-}}\)submodule \(N\) of
\({\mathbb{Z}}^3\) spanned by
\begin{align*}
f_1 &= [-1, 0, 1], \\
f_2 &= [2,-3,1], \\
f_3 &= [0, 3, 1], \\
f_4 &= [3,1,5]
.\end{align*}
Find a basis for \(N\) and describe \({\mathbb{Z}}^3/N\).

\hypertarget{spring-2018-6-work}{%
\subsubsection{\texorpdfstring{Spring 2018 \#6
\(\work\)}{Spring 2018 \#6 \textbackslash work}}\label{spring-2018-6-work}}

Let
\begin{align*}
M &= \{(w, x, y, z) \in {\mathbb{Z}}^4 {~\mathrel{\Big|}~}w + x + y + z \in 2{\mathbb{Z}}\} \\
N &= \left\{{
(w, x, y, z) \in {\mathbb{Z}}^4 {~\mathrel{\Big|}~}4\bigm|(w - x),~ 4\bigm|(x - y),~ 4\bigm|( y - z)
}\right\}
.\end{align*}

\begin{enumerate}
\def\labelenumi{\alph{enumi}.}
\item
  Show that \(N\) is a \({\mathbb{Z}}{\hbox{-}}\)submodule of \(M\) .
\item
  Find vectors \(u_1 , u_2 , u_3 , u_4 \in {\mathbb{Z}}^4\) and integers
  \(d_1 , d_2 , d_3 , d_4\) such that
  \begin{align*}
  \{
  u_1 , u_2 , u_3 , u_4 
  \} 
  && \text{is a free basis for }M
  \\
  \{
  d_1 u_1,~ d_2 u_2,~ d_3 u_3,~ d_4 u_4 
  \}
  && \text{is a free basis for }N
  \end{align*}
\item
  Use the previous part to describe \(M/N\) as a direct sum of cyclic
  \({\mathbb{Z}}{\hbox{-}}\)modules.
\end{enumerate}

\hypertarget{spring-2018-7-work}{%
\subsubsection{\texorpdfstring{Spring 2018 \#7
\(\work\)}{Spring 2018 \#7 \textbackslash work}}\label{spring-2018-7-work}}

Let \(R\) be a PID and \(M\) be an \(R{\hbox{-}}\)module. Let \(p\) be a
prime element of \(R\). The module \(M\) is called
\emph{\(\left\langle{p}\right\rangle{\hbox{-}}\)primary} if for every
\(m \in M\) there exists \(k > 0\) such that \(p^k m = 0\).

\begin{enumerate}
\def\labelenumi{\alph{enumi}.}
\item
  Suppose M is \(\left\langle{p}\right\rangle{\hbox{-}}\)primary. Show
  that if \(m \in M\) and
  \(t \in R, ~t \not\in \left\langle{p}\right\rangle\), then there
  exists \(a \in R\) such that \(atm = m\).
\item
  A submodule \(S\) of \(M\) is said to be \emph{pure} if
  \(S \cap r M = rS\) for all \(r \in R\). Show that if \(M\) is
  \(\left\langle{p}\right\rangle{\hbox{-}}\)primary, then \(S\) is pure
  if and only if \(S \cap p^k M = p^k S\) for all \(k \geq 0\).
\end{enumerate}

\hypertarget{fall-2016-6-work}{%
\subsubsection{\texorpdfstring{Fall 2016 \#6
\(\work\)}{Fall 2016 \#6 \textbackslash work}}\label{fall-2016-6-work}}

Let \(R\) be a ring and \(f: M\to N\) and \(g: N\to M\) be
\(R{\hbox{-}}\)module homomorphisms such that
\(g\circ f = \operatorname{id}_M\). Show that
\(N\cong \operatorname{im}f \oplus \ker g\).

\hypertarget{spring-2016-4-work}{%
\subsubsection{\texorpdfstring{Spring 2016 \#4
\(\work\)}{Spring 2016 \#4 \textbackslash work}}\label{spring-2016-4-work}}

Let \(R\) be a ring with the following commutative diagram of
\(R{\hbox{-}}\)modules, where each row represents a short exact sequence
of \(R{\hbox{-}}\)modules:

\begin{center}
\begin{tikzcd}
0 \ar[r] & A \ar[d, "\alpha"] \ar[r, "f"] & B \ar[d, "\beta"] \ar[r, "g"] & C \ar[r] \ar[d, "\gamma"] & 0 \\
0 \ar[r] & A' \ar[r, "f'"] & B'\ar[r, "g'"] & C' \ar[r] & 0 
\end{tikzcd}
\end{center}

Prove that if \(\alpha\) and \(\gamma\) are isomorphisms then \(\beta\)
is an isomorphism.

\hypertarget{spring-2015-8-work}{%
\subsubsection{\texorpdfstring{Spring 2015 \#8
\(\work\)}{Spring 2015 \#8 \textbackslash work}}\label{spring-2015-8-work}}

Let \(R\) be a PID and \(M\) a finitely generated \(R{\hbox{-}}\)module.

\begin{enumerate}
\def\labelenumi{\alph{enumi}.}
\item
  Prove that there are \(R{\hbox{-}}\)submodules
  \begin{align*}
  0 = M_0 \subset M_1 \subset \cdots \subset M_n = M
  \end{align*}
  such that for all \(0\leq i \leq n-1\), the module \(M_{i+1}/M_i\) is
  cyclic.
\item
  Is the integer \(n\) in part (a) uniquely determined by \(M\)? Prove
  your answer.
\end{enumerate}

\hypertarget{fall-2012-6-work}{%
\subsubsection{\texorpdfstring{Fall 2012 \#6
\(\work\)}{Fall 2012 \#6 \textbackslash work}}\label{fall-2012-6-work}}

Let \(R\) be a ring and \(M\) an \(R{\hbox{-}}\)module. Recall that
\(M\) is \emph{Noetherian} iff any strictly increasing chain of
submodule \(M_1 \subsetneq M_2 \subsetneq \cdots\) is finite. Call a
proper submodule \(M' \subsetneq M\) \emph{intersection-decomposable} if
it can not be written as the intersection of two proper submodules
\(M' = M_1\cap M_2\) with \(M_i \subsetneq M\).

Prove that for every Noetherian module \(M\), any proper submodule
\(N\subsetneq M\) can be written as a finite intersection
\(N = N_1 \cap\cdots \cap N_k\) of intersection-indecomposable modules.

\hypertarget{fall-2019-final-1-work}{%
\subsubsection{\texorpdfstring{Fall 2019 Final \#1
\(\work\)}{Fall 2019 Final \#1 \textbackslash work}}\label{fall-2019-final-1-work}}

Let \(A\) be an abelian group, and show \(A\) is a
\({\mathbb{Z}}{\hbox{-}}\)module in a unique way.

\hypertarget{fall-2020-6-work}{%
\subsubsection{\texorpdfstring{Fall 2020 \#6
\(\work\)}{Fall 2020 \#6 \textbackslash work}}\label{fall-2020-6-work}}

Let \(R\) be a ring with \(1\) and let \(M\) be a left
\(R{\hbox{-}}\)module. If \(I\) is a left ideal of \(R\), define
\begin{align*}
IM \coloneqq\left\{{ \sum_{i=1}^{N < \infty} a_i m_i {~\mathrel{\Big|}~}a_i \in I, m_i \in M, n\in {\mathbb{N}}}\right\}
,\end{align*}
i.e.~the set of finite sums of of elements of the form \(am\) where
\(a\in I, m\in M\).

\begin{enumerate}
\def\labelenumi{\alph{enumi}.}
\item
  Prove that \(IM \leq M\) is a submodule.
\item
  Let \(M, N\) be left \(R{\hbox{-}}\)modules, \(I\) a nilpotent left
  ideal of \(R\), and \(f: M\to N\) an \(R{\hbox{-}}\)module morphism.
  Prove that if the induced morphism
  \(\mkern 1.5mu\overline{\mkern-1.5muf\mkern-1.5mu}\mkern 1.5mu: M/IM \to N/IN\)
  is surjective, then \(f\) is surjective.
\end{enumerate}

\hypertarget{torsion-and-the-structure-theorem}{%
\subsection{Torsion and the Structure
Theorem}\label{torsion-and-the-structure-theorem}}

\hypertarget{star-fall-2019-5-done}{%
\subsubsection{\texorpdfstring{\(\star\) Fall 2019 \#5
\(\done\)}{\textbackslash star Fall 2019 \#5 \textbackslash done}}\label{star-fall-2019-5-done}}

Let \(R\) be a ring and \(M\) an \(R{\hbox{-}}\)module.

\begin{quote}
Recall that the set of torsion elements in M is defined by
\begin{align*}
\operatorname{Tor}(M) = \{m \in M {~\mathrel{\Big|}~}\exists r \in R, ~r \neq 0, ~rm = 0\}
.\end{align*}
\end{quote}

\begin{enumerate}
\def\labelenumi{\alph{enumi}.}
\item
  Prove that if \(R\) is an integral domain, then
  \(\operatorname{Tor}(M )\) is a submodule of \(M\) .
\item
  Give an example where \(\operatorname{Tor}(M )\) is not a submodule of
  \(M\).
\item
  If \(R\) has zero-divisors, prove that every non-zero
  \(R{\hbox{-}}\)module has non-zero torsion elements.
\end{enumerate}

Section omitted.

Section omitted.

\hypertarget{star-spring-2019-5-done}{%
\subsubsection{\texorpdfstring{\(\star\) Spring 2019 \#5
\(\done\)}{\textbackslash star Spring 2019 \#5 \textbackslash done}}\label{star-spring-2019-5-done}}

Let \(R\) be an integral domain. Recall that if \(M\) is an
\(R{\hbox{-}}\)module, the \emph{rank} of \(M\) is defined to be the
maximum number of \(R{\hbox{-}}\)linearly independent elements of \(M\)
.

\begin{enumerate}
\def\labelenumi{\alph{enumi}.}
\item
  Prove that for any \(R{\hbox{-}}\)module \(M\), the rank of
  \(\operatorname{Tor}(M )\) is 0.
\item
  Prove that the rank of \(M\) is equal to the rank of of
  \(M/\operatorname{Tor}(M )\).
\item
  Suppose that M is a non-principal ideal of \(R\).
\end{enumerate}

Prove that \(M\) is torsion-free of rank 1 but not free.

Section omitted.

:::\{.solution\} \envlist

Section omitted.

Section omitted.

Section omitted.

Section omitted.

Section omitted.

\hypertarget{star-spring-2020-6-done}{%
\subsubsection{\texorpdfstring{\(\star\) Spring 2020 \#6
\(\done\)}{\textbackslash star Spring 2020 \#6 \textbackslash done}}\label{star-spring-2020-6-done}}

Let \(R\) be a ring with unity.

\begin{enumerate}
\def\labelenumi{\alph{enumi}.}
\item
  Give a definition for a free module over \(R\).
\item
  Define what it means for an \(R{\hbox{-}}\)module to be torsion free.
\item
  Prove that if \(F\) is a free module, then any short exact sequence of
  \(R{\hbox{-}}\)modules of the following form splits:
  \begin{align*}
  0 \to N \to M \to F \to 0
  .\end{align*}
\item
  Let \(R\) be a PID. Show that any finitely generated
  \(R{\hbox{-}}\)module \(M\) can be expressed as a direct sum of a
  torsion module and a free module.
\end{enumerate}

\begin{quote}
You may assume that a finitely generated torsionfree module over a PID
is free.
\end{quote}

Section omitted.

\hypertarget{spring-2012-5-work}{%
\subsubsection{\texorpdfstring{Spring 2012 \#5
\(\work\)}{Spring 2012 \#5 \textbackslash work}}\label{spring-2012-5-work}}

Let \(M\) be a finitely generated module over a PID \(R\).

\begin{enumerate}
\def\labelenumi{\alph{enumi}.}
\item
  \(M_t\) be the set of torsion elements of \(M\), and show that \(M_t\)
  is a submodule of \(M\).
\item
  Show that \(M/M_t\) is torsion free.
\item
  Prove that \(M \cong M_t \oplus F\) where \(F\) is a free module.
\end{enumerate}

\hypertarget{spring-2017-5-work}{%
\subsubsection{\texorpdfstring{Spring 2017 \#5
\(\work\)}{Spring 2017 \#5 \textbackslash work}}\label{spring-2017-5-work}}

Let \(R\) be an integral domain and let \(M\) be a nonzero torsion
\(R{\hbox{-}}\)module.

\begin{enumerate}
\def\labelenumi{\alph{enumi}.}
\item
  Prove that if \(M\) is finitely generated then the annihilator in
  \(R\) of \(M\) is nonzero.
\item
  Give an example of a non-finitely generated torsion
  \(R{\hbox{-}}\)module whose annihilator is \((0)\), and justify your
  answer.
\end{enumerate}

\hypertarget{fall-2019-final-3-work}{%
\subsubsection{\texorpdfstring{Fall 2019 Final \#3
\(\work\)}{Fall 2019 Final \#3 \textbackslash work}}\label{fall-2019-final-3-work}}

Let \(R = k[x]\) for \(k\) a field and let \(M\) be the
\(R{\hbox{-}}\)module given by
\begin{align*}
M=\frac{k[x]}{(x-1)^{3}} \oplus \frac{k[x]}{\left(x^{2}+1\right)^{2}} \oplus \frac{k[x]}{(x-1)\left(x^{2}+1\right)^{4}} \oplus \frac{k[x]}{(x+2)\left(x^{2}+1\right)^{2}}
.\end{align*}
Describe the elementary divisors and invariant factors of \(M\).

\hypertarget{fall-2019-final-4-work}{%
\subsubsection{\texorpdfstring{Fall 2019 Final \#4
\(\work\)}{Fall 2019 Final \#4 \textbackslash work}}\label{fall-2019-final-4-work}}

Let \(I = (2, x)\) be an ideal in \(R = {\mathbb{Z}}[x]\), and show that
\(I\) is not a direct sum of nontrivial cyclic \(R{\hbox{-}}\)modules.

\hypertarget{fall-2019-final-5-work}{%
\subsubsection{\texorpdfstring{Fall 2019 Final \#5
\(\work\)}{Fall 2019 Final \#5 \textbackslash work}}\label{fall-2019-final-5-work}}

Let \(R\) be a PID.

\begin{enumerate}
\def\labelenumi{\alph{enumi}.}
\item
  Classify irreducible \(R{\hbox{-}}\)modules up to isomorphism.
\item
  Classify indecomposable \(R{\hbox{-}}\)modules up to isomorphism.
\end{enumerate}

\hypertarget{fall-2019-final-6-work}{%
\subsubsection{\texorpdfstring{Fall 2019 Final \#6
\(\work\)}{Fall 2019 Final \#6 \textbackslash work}}\label{fall-2019-final-6-work}}

Let \(V\) be a finite-dimensional \(k{\hbox{-}}\)vector space and
\(T:V\to V\) a non-invertible \(k{\hbox{-}}\)linear map. Show that there
exists a \(k{\hbox{-}}\)linear map \(S:V\to V\) with \(T\circ S = 0\)
but \(S\circ T\neq 0\).

\hypertarget{fall-2019-final-7-work}{%
\subsubsection{\texorpdfstring{Fall 2019 Final \#7
\(\work\)}{Fall 2019 Final \#7 \textbackslash work}}\label{fall-2019-final-7-work}}

Let \(A\in M_n({\mathbb{C}})\) with \(A^2 = A\). Show that \(A\) is
similar to a diagonal matrix, and exhibit an explicit diagonal matrix
similar to \(A\).

\hypertarget{fall-2019-final-10-work}{%
\subsubsection{\texorpdfstring{Fall 2019 Final \#10
\(\work\)}{Fall 2019 Final \#10 \textbackslash work}}\label{fall-2019-final-10-work}}

Show that the eigenvalues of a Hermitian matrix \(A\) are real and that
\(A = PDP^{-1}\) where \(P\) is an invertible matrix with orthogonal
columns.

\hypertarget{fall-2020-7-work}{%
\subsubsection{\texorpdfstring{Fall 2020 \#7
\(\work\)}{Fall 2020 \#7 \textbackslash work}}\label{fall-2020-7-work}}

Let \(A \in \operatorname{Mat}(n\times n, {\mathbb{R}})\) be arbitrary.
Make \({\mathbb{R}}^n\) into an \({\mathbb{R}}[x]{\hbox{-}}\)module by
letting \(f(x).\mathbf{v} \coloneqq f(A)(\mathbf{v})\) for
\(f(\mathbf{v})\in {\mathbb{R}}[x]\) and
\(\mathbf{v} \in {\mathbb{R}}^n\). Suppose that this induces the
following direct sum decomposition:
\begin{align*}
{\mathbb{R}}^n \cong
{ {\mathbb{R}}[x] \over \left\langle{ (x-1)^3 }\right\rangle }
\oplus
{ {\mathbb{R}}[x] \over \left\langle{ (x^2+1)^2 }\right\rangle }
\oplus
{ {\mathbb{R}}[x] \over \left\langle{ (x-1)(x^2-1)(x^2+1)^4 }\right\rangle }
\oplus
{ {\mathbb{R}}[x] \over \left\langle{ (x+2)(x^2+1)^2 }\right\rangle }
.\end{align*}

\begin{enumerate}
\def\labelenumi{\alph{enumi}.}
\item
  Determine the elementary divisors and invariant factors of \(A\).
\item
  Determine the minimal polynomial of \(A\).
\item
  Determine the characteristic polynomial of \(A\).
\end{enumerate}

\hypertarget{linear-algebra-diagonalizability}{%
\section{Linear Algebra:
Diagonalizability}\label{linear-algebra-diagonalizability}}

\hypertarget{fall-2017-7-work}{%
\subsection{\texorpdfstring{Fall 2017 \#7
\(\work\)}{Fall 2017 \#7 \textbackslash work}}\label{fall-2017-7-work}}

Let \(F\) be a field and let \(V\) and \(W\) be vector spaces over \(F\)
.

Make \(V\) and \(W\) into \(F[x]{\hbox{-}}\)modules via linear operators
\(T\) on \(V\) and \(S\) on \(W\) by defining \(X \cdot v = T (v)\) for
all \(v \in V\) and \(X \cdot w = S(w)\) for all w \(\in\) W .

Denote the resulting \(F[x]{\hbox{-}}\)modules by \(V_T\) and \(W_S\)
respectively.

\begin{enumerate}
\def\labelenumi{\alph{enumi}.}
\item
  Show that an \(F[x]{\hbox{-}}\)module homomorphism from \(V_T\) to
  \(W_S\) consists of an \(F{\hbox{-}}\)linear transformation
  \(R : V \to W\) such that \(RT = SR\).
\item
  Show that \(VT \cong WS\) as \(F[x]{\hbox{-}}\)modules \(\iff\) there
  is an \(F{\hbox{-}}\)linear isomorphism \(P : V \to W\) such that
  \(T = P^{-1}SP\).
\item
  Recall that a module \(M\) is \emph{simple} if \(M \neq 0\) and any
  proper submodule of \(M\) must be zero. Suppose that \(V\) has
  dimension 2. Give an example of \(F\), \(T\) with \(V_T\) simple.
\item
  Assume \(F\) is algebraically closed. Prove that if \(V\) has
  dimension 2, then any \(V_T\) is not simple.
\end{enumerate}

\hypertarget{spring-2015-3-work}{%
\subsection{\texorpdfstring{Spring 2015 \#3
\(\work\)}{Spring 2015 \#3 \textbackslash work}}\label{spring-2015-3-work}}

Let \(F\) be a field and \(V\) a finite dimensional
\(F{\hbox{-}}\)vector space, and let \(A, B: V\to V\) be commuting
\(F{\hbox{-}}\)linear maps. Suppose there is a basis \({\mathcal{B}}_1\)
with respect to which \(A\) is diagonalizable and a basis
\({\mathcal{B}}_2\) with respect to which \(B\) is diagonalizable.

Prove that there is a basis \({\mathcal{B}}_3\) with respect to which
\(A\) and \(B\) are both diagonalizable.

\hypertarget{fall-2016-2-work}{%
\subsection{\texorpdfstring{Fall 2016 \#2
\(\work\)}{Fall 2016 \#2 \textbackslash work}}\label{fall-2016-2-work}}

Let \(A, B\) be two \(n\times n\) matrices with the property that
\(AB = BA\). Suppose that \(A\) and \(B\) are diagonalizable. Prove that
\(A\) and \(B\) are \emph{simultaneously} diagonalizable.

\hypertarget{spring-2019-1-done}{%
\subsection{\texorpdfstring{Spring 2019 \#1
\(\done\)}{Spring 2019 \#1 \textbackslash done}}\label{spring-2019-1-done}}

Let \(A\) be a square matrix over the complex numbers. Suppose that
\(A\) is nonsingular and that \(A^{2019}\) is diagonalizable over
\({\mathbb{C}}\).

Show that \(A\) is also diagonalizable over \({\mathbb{C}}\).

Section omitted.

Section omitted.

\hypertarget{linear-algebra-misc}{%
\section{Linear Algebra: Misc}\label{linear-algebra-misc}}

\hypertarget{star-spring-2012-6-work}{%
\subsection{\texorpdfstring{\(\star\) Spring 2012 \#6
\(\work\)}{\textbackslash star Spring 2012 \#6 \textbackslash work}}\label{star-spring-2012-6-work}}

Let \(k\) be a field and let the group
\(G = \operatorname{GL}(m, k) \times\operatorname{GL}(n, k)\) acts on
the set of \(m\times n\) matrices \(M_{m, n}(k)\) as follows:
\begin{align*}
(A, B) \cdot X = AXB^{-1}
\end{align*}
where \((A, B) \in G\) and \(X\in M_{m, n}(k)\).

\begin{enumerate}
\def\labelenumi{\alph{enumi}.}
\item
  State what it means for a group to act on a set. Prove that the above
  definition yields a group action.
\item
  Exhibit with justification a subset \(S\) of \(M_{m, n}(k)\) which
  contains precisely one element of each orbit under this action.
\end{enumerate}

\hypertarget{star-spring-2014-7-work}{%
\subsection{\texorpdfstring{\(\star\) Spring 2014 \#7
\(\work\)}{\textbackslash star Spring 2014 \#7 \textbackslash work}}\label{star-spring-2014-7-work}}

Let \(G = \operatorname{GL}(3, {\mathbb{Q}}[x])\) be the group of
invertible \(3\times 3\) matrices over \({\mathbb{Q}}[x]\). For each
\(f\in {\mathbb{Q}}[x]\), let \(S_f\) be the set of \(3\times 3\)
matrices \(A\) over \({\mathbb{Q}}[x]\) such that \(\det(A) = c f(x)\)
for some nonzero constant \(c\in {\mathbb{Q}}\).

\begin{enumerate}
\def\labelenumi{\alph{enumi}.}
\item
  Show that for \((P, Q) \in G\times G\) and \(A\in S_f\), the formula
  \begin{align*}
  (P, Q)\cdot A \coloneqq PAQ^{-1}
  \end{align*}
  gives a well defined map \(G\times G \times S_f \to S_f\) and show
  that this map gives a group action of \(G\times G\) on \(S_f\).
\item
  For \(f(x) = x^3(x^2+1)^2\), give one representative from each orbit
  of the group action in (a), and justify your assertion.
\end{enumerate}

\hypertarget{fall-2012-7-work}{%
\subsection{\texorpdfstring{Fall 2012 \#7
\(\work\)}{Fall 2012 \#7 \textbackslash work}}\label{fall-2012-7-work}}

Let \(k\) be a field of characteristic zero and \(A, B \in M_n(k)\) be
two square \(n\times n\) matrices over \(k\) such that \(AB - BA = A\).
Prove that \(\det A = 0\).

Moreover, when the characteristic of \(k\) is 2, find a counterexample
to this statement.

\hypertarget{fall-2012-8-work}{%
\subsection{\texorpdfstring{Fall 2012 \#8
\(\work\)}{Fall 2012 \#8 \textbackslash work}}\label{fall-2012-8-work}}

Prove that any nondegenerate matrix \(X\in M_n({\mathbb{R}})\) can be
written as \(X = UT\) where \(U\) is orthogonal and \(T\) is upper
triangular.

\hypertarget{fall-2012-5-work}{%
\subsection{\texorpdfstring{Fall 2012 \#5
\(\work\)}{Fall 2012 \#5 \textbackslash work}}\label{fall-2012-5-work}}

Let \(U\) be an infinite-dimensional vector space over a field \(k\),
\(f: U\to U\) a linear map, and
\(\left\{{u_1, \cdots, u_m}\right\} \subset U\) vectors such that \(U\)
is generated by
\(\left\{{u_1, \cdots, u_m, f^d(u_1), \cdots, f^d(u_m)}\right\}\) for
some \(d\in {\mathbb{N}}\).

Prove that \(U\) can be written as a direct sum \(U \cong V\oplus W\)
such that

\begin{enumerate}
\def\labelenumi{\arabic{enumi}.}
\tightlist
\item
  \(V\) has a basis consisting of some vector
  \(v_1, \cdots v_n, f^d(v_1), \cdots, f^d(v_n)\) for some
  \(d\in {\mathbb{N}}\), and
\item
  \(W\) is finite-dimensional.
\end{enumerate}

Moreover, prove that for any other decomposition
\(U \cong V' \oplus W'\), one has \(W' \cong W\).

\hypertarget{fall-2015-7-work}{%
\subsection{\texorpdfstring{Fall 2015 \#7
\(\work\)}{Fall 2015 \#7 \textbackslash work}}\label{fall-2015-7-work}}

\begin{enumerate}
\def\labelenumi{\alph{enumi}.}
\item
  Show that two \(3\times 3\) matrices over \({\mathbb{C}}\) are similar
  \(\iff\) their characteristic polynomials are equal and their minimal
  polynomials are equal.
\item
  Does the conclusion in (a) hold for \(4\times 4\) matrices? Justify
  your answer with a proof or counterexample.
\end{enumerate}

\hypertarget{fall-2014-4-work}{%
\subsection{\texorpdfstring{Fall 2014 \#4
\(\work\)}{Fall 2014 \#4 \textbackslash work}}\label{fall-2014-4-work}}

Let \(F\) be a field and \(T\) an \(n\times n\) matrix with entries in
\(F\). Let \(I\) be the ideal consisting of all polynomials
\(f\in F[x]\) such that \(f(T) =0\).

Show that the following statements are equivalent about a polynomial
\(g\in I\):

\begin{enumerate}
\def\labelenumi{\alph{enumi}.}
\item
  \(g\) is irreducible.
\item
  If \(k\in F[x]\) is nonzero and of degree strictly less than \(g\),
  then \(k[T]\) is an invertible matrix.
\end{enumerate}

\hypertarget{fall-2015-8-work}{%
\subsection{\texorpdfstring{Fall 2015 \#8
\(\work\)}{Fall 2015 \#8 \textbackslash work}}\label{fall-2015-8-work}}

Let \(V\) be a vector space over a field \(F\) and \(V {}^{ \check{} }\)
its dual. A \emph{symmetric bilinear form} \(({-}, {-})\) on \(V\) is a
map \(V\times V\to F\) satisfying
\begin{align*}
(av_1 + b v_2, w) = a(v_1, w) + b(v_2, w) {\quad \operatorname{and} \quad} (v_1, v_2) = (v_2, v_1)
\end{align*}
for all \(a, b\in F\) and \(v_1, v_2 \in V\). The form is
\emph{nondegenerate} if the only element \(w\in V\) satisfying
\((v, w) = 0\) for all \(v\in V\) is \(w=0\).

Suppose \(({-}, {-})\) is a nondegenerate symmetric bilinear form on
\(V\). If \(W\) is a subspace of \(V\), define
\begin{align*}
W^{\perp} \coloneqq\left\{{v\in V {~\mathrel{\Big|}~}(v, w) = 0 \text{ for all } w\in W}\right\}
.\end{align*}

\begin{enumerate}
\def\labelenumi{\alph{enumi}.}
\item
  Show that if \(X, Y\) are subspaces of \(V\) with \(Y\subset X\), then
  \(X^{\perp} \subseteq Y^{\perp}\).
\item
  Define an injective linear map
  \begin{align*}
  \psi: Y^{\perp}/X^{\perp} \hookrightarrow(X/Y) {}^{ \check{} }
  \end{align*}
  which is an isomorphism if \(V\) is finite dimensional.
\end{enumerate}

\hypertarget{fall-2018-4-done}{%
\subsection{\texorpdfstring{Fall 2018 \#4
\(\done\)}{Fall 2018 \#4 \textbackslash done}}\label{fall-2018-4-done}}

Let \(V\) be a finite dimensional vector space over a field (the field
is not necessarily algebraically closed).

Let \(\phi : V \to V\) be a linear transformation. Prove that there
exists a decomposition of \(V\) as \(V = U \oplus W\) , where \(U\) and
\(W\) are \(\phi{\hbox{-}}\)invariant subspaces of \(V\) ,
\({\left.{{\phi}} \right|_{{U}} }\) is nilpotent, and
\({\left.{{\phi}} \right|_{{W}} }\) is nonsingular.

\todo[inline]{Revisit.}

Section omitted.

\hypertarget{fall-2018-5-done}{%
\subsection{\texorpdfstring{Fall 2018 \#5
\(\done\)}{Fall 2018 \#5 \textbackslash done}}\label{fall-2018-5-done}}

Let \(A\) be an \(n \times n\) matrix.

\begin{enumerate}
\def\labelenumi{\alph{enumi}.}
\item
  Suppose that \(v\) is a column vector such that the set
  \(\{v, Av, . . . , A^{n-1} v\}\) is linearly independent. Show that
  any matrix \(B\) that commutes with \(A\) is a polynomial in \(A\).
\item
  Show that there exists a column vector \(v\) such that the set
  \(\{v, Av, . . . , A^{n-1} v\}\) is linearly independent \(\iff\) the
  characteristic polynomial of \(A\) equals the minimal polynomial of A.
\end{enumerate}

Section omitted.

Section omitted.

Section omitted.

\hypertarget{fall-2019-8-work}{%
\subsection{\texorpdfstring{Fall 2019 \#8
\(\work\)}{Fall 2019 \#8 \textbackslash work}}\label{fall-2019-8-work}}

Let \(\{e_1, \cdots, e_n \}\) be a basis of a real vector space \(V\)
and let
\begin{align*}
\Lambda \coloneqq\left\{{ \sum r_i e_i \mathrel{\Big|}r_i \in {\mathbb{Z}}}\right\}
\end{align*}

Let \(\cdot\) be a non-degenerate (\(v \cdot w = 0\) for all
\(w \in V \iff v = 0\)) symmetric bilinear form on \(V\) such that the
Gram matrix \(M = (e_i \cdot e_j )\) has integer entries.

Define the dual of \(\Lambda\) to be
\begin{align*}
\Lambda  {}^{ \check{} }\coloneqq\{v \in V {~\mathrel{\Big|}~}v \cdot x \in {\mathbb{Z}}\text{ for all } x \in \Lambda
\}
.\end{align*}

\begin{enumerate}
\def\labelenumi{\alph{enumi}.}
\item
  Show that \(\Lambda \subset \Lambda {}^{ \check{} }\).
\item
  Prove that \(\det M \neq 0\) and that the rows of \(M^{-1}\) span
  \(\Lambda {}^{ \check{} }\).
\item
  Prove that \(\det M = |\Lambda {}^{ \check{} }/\Lambda|\).
\end{enumerate}

\todo[inline]{Todo, missing part (c).}

Section omitted.

Section omitted.

\hypertarget{spring-2013-6-done}{%
\subsection{\texorpdfstring{Spring 2013 \#6
\(\done\)}{Spring 2013 \#6 \textbackslash done}}\label{spring-2013-6-done}}

Let \(V\) be a finite dimensional vector space over a field \(F\) and
let \(T: V\to V\) be a linear operator with characteristic polynomial
\(f(x) \in F[x]\).

\begin{enumerate}
\def\labelenumi{\alph{enumi}.}
\item
  Show that \(f(x)\) is irreducible in \(F[x] \iff\) there are no proper
  nonzero subspaces \(W< V\) with \(T(W) \subseteq W\).
\item
  If \(f(x)\) is irreducible in \(F[x]\) and the characteristic of \(F\)
  is 0, show that \(T\) is diagonalizable when we extend the field to
  its algebraic closure.
\end{enumerate}

\todo[inline]{Is there a proof without matrices? What if $V$ is infinite dimensional?}
\todo[inline]{How to extend basis?}

Section omitted.

Section omitted.

\hypertarget{fall-2020-8-work}{%
\subsection{\texorpdfstring{Fall 2020 \#8
\(\work\)}{Fall 2020 \#8 \textbackslash work}}\label{fall-2020-8-work}}

Let \(A\in \operatorname{Mat}(n\times n, {\mathbb{C}})\) such that the
group generated by \(A\) under multiplication is finite. Show that
\begin{align*}
\operatorname{Tr}(A^{-1}) ={\overline{{\operatorname{Tr}(A) }}}
,\end{align*}
where \({\overline{{({-})}}}\) denotes taking the complex conjugate and
\(\operatorname{Tr}({-})\) is the trace.

\hypertarget{linear-algebra-canonical-forms}{%
\section{Linear Algebra: Canonical
Forms}\label{linear-algebra-canonical-forms}}

\hypertarget{star-spring-2012-8-work}{%
\subsection{\texorpdfstring{\(\star\) Spring 2012 \#8
\(\work\)}{\textbackslash star Spring 2012 \#8 \textbackslash work}}\label{star-spring-2012-8-work}}

Let \(V\) be a finite-dimensional vector space over a field \(k\) and
\(T:V\to V\) a linear transformation.

\begin{enumerate}
\def\labelenumi{\alph{enumi}.}
\item
  Provide a definition for the \emph{minimal polynomial} in \(k[x]\) for
  \(T\).
\item
  Define the \emph{characteristic polynomial} for \(T\).
\item
  Prove the Cayley-Hamilton theorem: the linear transformation \(T\)
  satisfies its characteristic polynomial.
\end{enumerate}

\hypertarget{star-spring-2020-8-work}{%
\subsection{\texorpdfstring{\(\star\) Spring 2020 \#8
\(\work\)}{\textbackslash star Spring 2020 \#8 \textbackslash work}}\label{star-spring-2020-8-work}}

Let \(T:V\to V\) be a linear transformation where \(V\) is a
finite-dimensional vector space over \({\mathbb{C}}\). Prove the
Cayley-Hamilton theorem: if \(p(x)\) is the characteristic polynomial of
\(T\), then \(p(T) = 0\). You may use canonical forms.

\hypertarget{star-spring-2012-7-work}{%
\subsection{\texorpdfstring{\(\star\) Spring 2012 \#7
\(\work\)}{\textbackslash star Spring 2012 \#7 \textbackslash work}}\label{star-spring-2012-7-work}}

Consider the following matrix as a linear transformation from
\(V\coloneqq{\mathbb{C}}^5\) to itself:
\begin{align*}
A=\left(\begin{array}{ccccc}
-1 & 1 & 0 & 0 & 0 \\
-4 & 3 & 1 & 0 & 0 \\
0 & 0 & 1 & 0 & 1 \\
0 & 0 & 0 & 1 & 0 \\
0 & 0 & 0 & 0 & 2
\end{array}\right)
.\end{align*}

\begin{enumerate}
\def\labelenumi{\alph{enumi}.}
\item
  Find the invariant factors of \(A\).
\item
  Express \(V\) in terms of a direct sum of indecomposable
  \({\mathbb{C}}[x]{\hbox{-}}\)modules.
\item
  Find the Jordan canonical form of \(A\).
\end{enumerate}

\hypertarget{fall-2019-final-8-work}{%
\subsection{\texorpdfstring{Fall 2019 Final \#8
\(\work\)}{Fall 2019 Final \#8 \textbackslash work}}\label{fall-2019-final-8-work}}

Exhibit the rational canonical form for

\begin{itemize}
\tightlist
\item
  \(A\in M_6({\mathbb{Q}})\) with minimal polynomial
  \((x-1)(x^2 + 1)^2\).
\item
  \(A\in M_{10}({\mathbb{Q}})\) with minimal polynomial
  \((x^2+1)^2(x^3 + 1)\).
\end{itemize}

\hypertarget{fall-2019-final-9-work}{%
\subsection{\texorpdfstring{Fall 2019 Final \#9
\(\work\)}{Fall 2019 Final \#9 \textbackslash work}}\label{fall-2019-final-9-work}}

Exhibit the rational and Jordan canonical forms for the following matrix
\(A\in M_4({\mathbb{C}})\):
\begin{align*}
  A=\left(\begin{array}{cccc}
  2 & 0 & 0 & 0 \\
  1 & 1 & 0 & 0 \\
  -2 & -2 & 0 & 1 \\
  -2 & 0 & -1 & -2
  \end{array}\right)
  .\end{align*}

\hypertarget{spring-2016-7-work}{%
\subsection{\texorpdfstring{Spring 2016 \#7
\(\work\)}{Spring 2016 \#7 \textbackslash work}}\label{spring-2016-7-work}}

Let \(D = {\mathbb{Q}}[x]\) and let \(M\) be a
\({\mathbb{Q}}[x]{\hbox{-}}\)module such that
\begin{align*}
M \cong \frac{\mathbb{Q}[x]}{(x-1)^{3}} \oplus \frac{\mathbb{Q}[x]}{\left(x^{2}+1\right)^{3}} \oplus \frac{\mathbb{Q}[x]}{(x-1)\left(x^{2}+1\right)^{5}} \oplus \frac{\mathbb{Q}[x]}{(x+2)\left(x^{2}+1\right)^{2}}
.\end{align*}

Determine the elementary divisors and invariant factors of \(M\).

\hypertarget{spring-2020-7-work}{%
\subsection{\texorpdfstring{Spring 2020 \#7
\(\work\)}{Spring 2020 \#7 \textbackslash work}}\label{spring-2020-7-work}}

Let
\begin{align*}
A=\left[\begin{array}{ccc}
2 & 0 & 0 \\
4 & 6 & 1 \\
-16 & -16 & -2
\end{array}\right] \in M_{3}(\mathrm{C})
.\end{align*}

\begin{enumerate}
\def\labelenumi{\alph{enumi}.}
\item
  Find the Jordan canonical form \(J\) of \(A\).
\item
  Find an invertible matrix \(P\) such that \(P^{-1}A P = J\).

  \begin{quote}
  You should not need to compute \(P^{-1}\).
  \end{quote}
\item
  Write down the minimal polynomial of \(A\).
\end{enumerate}

\hypertarget{spring-2019-7-done}{%
\subsection{\texorpdfstring{Spring 2019 \#7
\(\done\)}{Spring 2019 \#7 \textbackslash done}}\label{spring-2019-7-done}}

Let \(p\) be a prime number. Let \(A\) be a \(p \times p\) matrix over a
field \(F\) with 1 in all entries except 0 on the main diagonal.

Determine the Jordan canonical form (JCF) of \(A\)

\begin{enumerate}
\def\labelenumi{\alph{enumi}.}
\item
  When \(F = {\mathbb{Q}}\),
\item
  When \(F = {\mathbb{F}}_p\).
\end{enumerate}

\begin{quote}
Hint: In both cases, all eigenvalues lie in the ground field. In each
case find a matrix \(P\) such that \(P^{-1}AP\) is in JCF.
\end{quote}

Section omitted.

Section omitted.

Section omitted.

\hypertarget{spring-2018-4-work}{%
\subsection{\texorpdfstring{Spring 2018 \#4
\(\work\)}{Spring 2018 \#4 \textbackslash work}}\label{spring-2018-4-work}}

Let
\begin{align*}
A=\left[\begin{array}{lll}{0} & {1} & {-2} \\ {1} & {1} & {-3} \\ {1} & {2} & {-4}\end{array}\right] \in M_{3}(\mathbb{C})
\end{align*}

\begin{enumerate}
\def\labelenumi{\alph{enumi}.}
\item
  Find the Jordan canonical form \(J\) of \(A\).
\item
  Find an invertible matrix \(P\) such that \(P^{-1}AP = J\).

  \begin{quote}
  You should not need to compute \(P^{-1}\).
  \end{quote}
\end{enumerate}

\hypertarget{spring-2017-6-work}{%
\subsection{\texorpdfstring{Spring 2017 \#6
\(\work\)}{Spring 2017 \#6 \textbackslash work}}\label{spring-2017-6-work}}

Let \(A\) be an \(n\times n\) matrix with all entries equal to \(0\)
except for the \(n-1\) entries just above the diagonal being equal to 2.

\begin{enumerate}
\def\labelenumi{\alph{enumi}.}
\item
  What is the Jordan canonical form of \(A\), viewed as a matrix in
  \(M_n({\mathbb{C}})\)?
\item
  Find a nonzero matrix \(P\in M_n({\mathbb{C}})\) such that
  \(P^{-1}A P\) is in Jordan canonical form.
\end{enumerate}

\hypertarget{spring-2016-1-work}{%
\subsection{\texorpdfstring{Spring 2016 \#1
\(\work\)}{Spring 2016 \#1 \textbackslash work}}\label{spring-2016-1-work}}

Let
\begin{align*}
A=\left(\begin{array}{ccc}
-3 & 3 & -2 \\
-7 & 6 & -3 \\
1 & -1 & 2
\end{array}\right) \in M_{3}(\mathrm{C})
.\end{align*}

\begin{enumerate}
\def\labelenumi{\alph{enumi}.}
\item
  Find the Jordan canonical form \(J\) of \(A\).
\item
  Find an invertible matrix \(P\) such that \(P^{-1}A P = J\). You do
  not need to compute \(P^{-1}\).
\end{enumerate}

\hypertarget{spring-2015-6-work}{%
\subsection{\texorpdfstring{Spring 2015 \#6
\(\work\)}{Spring 2015 \#6 \textbackslash work}}\label{spring-2015-6-work}}

Let \(F\) be a field and \(n\) a positive integer, and consider
\begin{align*}
A=\left[\begin{array}{ccc}
1 & \dots & 1 \\
& \ddots & \\
1 & \dots & 1
\end{array}\right] \in M_{n}(F)
.\end{align*}

Show that \(A\) has a Jordan normal form over \(F\) and find it.

\begin{quote}
Hint: treat the cases \(n\cdot 1 \neq 0\) in \(F\) and \(n\cdot 1 = 0\)
in \(F\) separately.
\end{quote}

\hypertarget{fall-2014-5-work}{%
\subsection{\texorpdfstring{Fall 2014 \#5
\(\work\)}{Fall 2014 \#5 \textbackslash work}}\label{fall-2014-5-work}}

Let \(T\) be a \(5\times 5\) complex matrix with characteristic
polynomial \(\chi(x) = (x-3)^5\) and minimal polynomial
\(m(x) = (x-3)^2\). Determine all possible Jordan forms of \(T\).

\hypertarget{spring-2013-5-work}{%
\subsection{\texorpdfstring{Spring 2013 \#5
\(\work\)}{Spring 2013 \#5 \textbackslash work}}\label{spring-2013-5-work}}

Let \(T: V\to V\) be a linear map from a 5-dimensional
\({\mathbb{C}}{\hbox{-}}\)vector space to itself and suppose
\(f(T) = 0\) where \(f(x) = x^2 + 2x + 1\).

\begin{enumerate}
\def\labelenumi{\alph{enumi}.}
\item
  Show that there does not exist any vector \(v\in V\) such that
  \(Tv = v\), but there \emph{does} exist a vector \(w\in V\) such that
  \(T^2 w= w\).
\item
  Give all of the possible Jordan canonical forms of \(T\).
\end{enumerate}

\hypertarget{spring-2021-1-work}{%
\subsection{\texorpdfstring{Spring 2021 \#1
\(\work\)}{Spring 2021 \#1 \textbackslash work}}\label{spring-2021-1-work}}

Let m
\begin{align*}
A \coloneqq
\begin{bmatrix}
r & 1 & -1 \\
-6 & -1 & 2 \\
2 & 1 & 1
\end{bmatrix}
\in \operatorname{Mat}(3\times 3, {\mathbb{C}})
.\end{align*}

\begin{enumerate}
\def\labelenumi{\alph{enumi}.}
\item
  Find the Jordan canonical form \(J\) of \(A\).
\item
  Find an invertible matrix \(P\) such that \(J = P ^{-1}A P\).
\end{enumerate}

\begin{quote}
You should not need to compute \(P^{-1}\)
\end{quote}

\begin{enumerate}
\def\labelenumi{\alph{enumi}.}
\setcounter{enumi}{2}
\tightlist
\item
  Write down the minimal polynomial of \(A\).
\end{enumerate}

\hypertarget{fall-2020-5-work}{%
\subsection{\texorpdfstring{Fall 2020 \#5
\(\work\)}{Fall 2020 \#5 \textbackslash work}}\label{fall-2020-5-work}}

Consider the following matrix:
\begin{align*}
B \coloneqq
\begin{bmatrix}
1 & 3 & 3
\\
2 & 2 & 3
\\
-1 & -2 & -2
\end{bmatrix}
.\end{align*}

\begin{enumerate}
\def\labelenumi{\alph{enumi}.}
\item
  Find the minimal polynomial of \(B\).
\item
  Find a \(3\times 3\) matrix \(J\) in Jordan canonical form such that
  \(B = JPJ^{-1}\) where \(P\) is an invertible matrix.
\end{enumerate}


\printbibliography[title=Bibliography]


\end{document}
