%% ================================================================================
%% This LaTeX file was created by AbiWord.                                         
%% AbiWord is a free, Open Source word processor.                                  
%% More information about AbiWord is available at http://www.abisource.com/        
%% ================================================================================

\documentclass[a4paper,portrait,12pt]{article}
\usepackage[latin1]{inputenc}
\usepackage{calc}
\usepackage{setspace}
\usepackage{fixltx2e}
\usepackage{graphicx}
\usepackage{multicol}
\usepackage[normalem]{ulem}
%% Please revise the following command, if your babel
%% package does not support en-US
\usepackage{babel}
\usepackage{color}
\usepackage{hyperref}
 
\begin{document}


\begin{flushleft}
Study Guide for Complex Analysis Exam
\end{flushleft}


\begin{flushleft}
I. Calculus and Undergraduate Analysis
\end{flushleft}


\begin{flushleft}
Continuity and differentiation in one and several real variables
\end{flushleft}


\begin{flushleft}
Inverse and implicit function theorems
\end{flushleft}


\begin{flushleft}
Compactness and connectedness in analysis
\end{flushleft}


\begin{flushleft}
Uniform convergence and uniform continuity
\end{flushleft}


\begin{flushleft}
Riemann integrals
\end{flushleft}


\begin{flushleft}
Contour integrals and Green's theorem
\end{flushleft}


\begin{flushleft}
Reference: [3].
\end{flushleft}


\begin{flushleft}
II. Preliminary Topics in Complex Analysis
\end{flushleft}


\begin{flushleft}
Complex arithmetic
\end{flushleft}


\begin{flushleft}
Analyticity, harmonic functions, and the Cauchy-Riemann equations
\end{flushleft}


\begin{flushleft}
Contour Integration in C
\end{flushleft}


\begin{flushleft}
References: [1] Chapters 1, 2; [2] Chapters 1, 2, 4; [4] Chapter 1.
\end{flushleft}


\begin{flushleft}
III. Cauchy's Theorem and its consequences
\end{flushleft}


\begin{flushleft}
Cauchy's theorem and integral formula, Morera's theorem, Schwarz reflection
\end{flushleft}


\begin{flushleft}
Uniform convergence of analytic functions
\end{flushleft}


\begin{flushleft}
Taylor and Laurent expansions
\end{flushleft}


\begin{flushleft}
Maximum modulus principle and Schwarz's lemma
\end{flushleft}


\begin{flushleft}
Liouville's theorem and the Fundamental theorem of algebra
\end{flushleft}


\begin{flushleft}
Residue theorem and applications
\end{flushleft}


\begin{flushleft}
Singularities and meromorphic functions, including the Casorati-Weierstrass theorem
\end{flushleft}


\begin{flushleft}
Rouche's theorem, the argument principle, and the open mapping theorem
\end{flushleft}


\begin{flushleft}
Estimates using Cauchy Integral Formula: Cauchy inequalities and, more generally,
\end{flushleft}


\begin{flushleft}
bounds on holomorphic functions and their derivatives on compact sets
\end{flushleft}


\begin{flushleft}
References: [1] Chapters 4, 5, 6; [2] Chapters 5, 7, 8, 9; [4] Chapters 2, 3, 5, 8 (§2,3).
\end{flushleft}


\begin{flushleft}
IV. Conformal Mapping
\end{flushleft}


\begin{flushleft}
General properties of conformal mappings
\end{flushleft}


\begin{flushleft}
Analytic and mapping properties of linear fractional transformations
\end{flushleft}


\begin{flushleft}
Automorphisms of the disk, plane, and Riemann sphere
\end{flushleft}


\begin{flushleft}
References: [1] Chapters 3, 8; [2] Chapters 3, 4; [4] Chapter 8 (§1,2).
\end{flushleft}


\begin{flushleft}
References
\end{flushleft}


\begin{flushleft}
[1] L. Ahlfors, Complex Analysis, Third Edition, McGraw-Hill.
\end{flushleft}


\begin{flushleft}
[2] E. Hille, Analytic Function Theory, Vol. 1, Ginn and Company.
\end{flushleft}


\begin{flushleft}
[3] W. Rudin, Principles of Mathematical Analysis, Third Edition, McGraw-Hill.
\end{flushleft}


\begin{flushleft}
[4] E. M. Stein and R. Shakarchi, Complex Analysis, Princeton University Press.
\end{flushleft}


\begin{flushleft}
[Revised June 2007]
\end{flushleft}





\newpage



\end{document}
