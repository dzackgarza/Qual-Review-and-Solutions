\documentclass[reqno]{article}
\usepackage{amsmath, amssymb}
\usepackage[margin=1in]{geometry}
\usepackage{amsfonts}
\usepackage{mathrsfs}
\usepackage[arrow,matrix,curve,cmtip,ps]{xy}
\usepackage{wasysym}

\usepackage{epsfig}
\usepackage{parskip}
\usepackage{amsthm}

\allowdisplaybreaks



\newtheorem{theorem}{Theorem}[section]
\newtheorem{lemma}[theorem]{Lemma}
\newtheorem{proposition}[theorem]{Proposition}
\newtheorem{corollary}[theorem]{Corollary}
\newtheorem{notation}[theorem]{Notation}
\newtheorem*{theorem*}{Theorem}
\theoremstyle{remark}
\newtheorem{remark}[theorem]{Remark}
\newtheorem{definition}[theorem]{Definition}
\newtheorem{example}[theorem]{Example}
\newtheorem{question}{Question}


\newcommand\mc[1]{\marginpar{\sloppy\protect\footnotesize #1}}

%\renewcommand{\baselinestretch}{2}




%this has equations numbered within sections 1.1,1.2, ... 2.1,...
\numberwithin{equation}{section}

%-------------------------------------------
%       Begin Local Macros
%-------------------------------------------

\newcommand{\Z}{\mathbb{Z}}
\newcommand{\N}{\mathbb{N}}
\newcommand{\R}{\mathbb{R}}
\newcommand{\C}{\mathbb{C}}
\newcommand{\T}{\mathbb{T}}
\newcommand{\Q}{\mathbb{Q}}
\newcommand{\F}{\mathbb{F}}
\newcommand{\im}{\operatorname{im }}
\newcommand{\coker}{\operatorname{coker }}
\newcommand{\Ad}{\operatorname{Ad}}
\newcommand{\ext}{\operatorname{Ext}}
\newcommand{\prim}{\operatorname{Prim}}
\newcommand{\ind}{\operatorname{ind}}
\newcommand{\aut}{\operatorname{Aut}}
\newcommand{\rank}{\operatorname{rank}}
\newcommand{\rpt}{\mathbb{R}\mathbb{P}^2}
\newcommand{\wt}{\widetilde}
\newcommand{\st}{\, | \, }
\newcommand{\pa}{\partial}
\newcommand{\Mod}[1]{\ (\mathrm{mod}\ #1)}
\newcommand{\ran}{\operatorname{ran}}
\newcommand{\bh}{\operatorname{ball}}
\newcommand{\SR}{\mathscr{R}}


\def\wcheck#1{\smash{
        \mathop{#1}\limits^{\scriptscriptstyle{\frown}}}}
        

\newcommand{\gae}{\lower 2pt \hbox{$\, \buildrel {\scriptstyle >}\over {\scriptstyle
\sim}\,$}}

\newcommand{\lae}{\lower 2pt \hbox{$\, \buildrel {\scriptstyle <}\over {\scriptstyle
\sim}\,$}}

\newcommand{\MU}[1]{
\setbox0\hbox{$#1$}
\setbox1\hbox{$W$}
\ifdim\wd0>\wd1 #1^{\sim} \else \widetilde{#1} \fi
}


\title{Analysis Qual Problem Solutions}
\author{Analysis Qual Workshop May 2020}
\date

\begin{document}

\maketitle

This document is a collection of solutions to various problems written up by the participants of the May 2020 Analysis Qualifying Exam Workshop. This is in no way meant to be an official source for solutions to old exam problems, but is simply meant to be a repository of solutions to problems worked on during the workshop. You may use this to help you study for the analysis qualifying exam, but before you read the solutions to any problems, you should attempt the problem on your own. 


\section{May 2016 Qual}

\begin{enumerate}

\item (May 2016, \#1) Consider the function $f(x) = \frac{x}{1-x^2}$, $x \in (0,1)$. 

\begin{enumerate} 

\item By using the $\epsilon-\delta$ definition of the limit only, prove that $f$ is continuous on $(0,1)$. (Note: You are not allowed to trivialize the problem by using properties of limits). 

\begin{proof}
    Fix $x\in (0,1)$ and let $\epsilon > 0$. Then we have 
    $$
        \left |f(x) - f(y) \right| 
        = \left|\frac{x}{1-x^2} - \frac{y}{1-y^2}\right| 
        = \left| \frac{x(1-y^2) - y(1-x^2)}{(1-x^2)(1-y^2)} \right|
        = \left| \frac{x-y}{(1-x)(1+x)(1-y)(1+y)} \right|.
    $$
    Now, choose $\delta > 0$ such that $\delta < \min\{\frac{1}{2}(1-x)^2\epsilon, \frac{1}{2}(1-x)\}$. When 
    $x - \delta < y < x + \delta$,
    \begin{eqnarray*}
        |f(x) - f(y) | & = & 
        \left| \frac{x-y}{(1-x)(1+x)(1-y)(1+y)} \right| \\
        & \leq & \left| \frac{x-y}{(1-x)(1-y)} \right| 
        \leq  \left| \frac{x-y}{(1-x)(1-(x+ \frac{1}{2}(1 - x)))} \right| \\
        & = & \left| \frac{x-y}{(1-x)(1-(x+ \frac{1}{2}(1 - x)))} \right|
        = \left| \frac{2}{(1-x)^2} \right||x-y| \\
        & < & \epsilon.
    \end{eqnarray*}
    
    As our choice of $x\in (0,1)$ was arbitrary, we conclude that 
    $f$ is continuous on all of $(0,1)$.
\end{proof} 

\item Is $f$ uniformly continuous on $(0,1)$? Justify your answer. 

\begin{proof}
We will show that the function $f$ is not uniformly continuous. Consider the sequence $(x_n)_{n=1}^\infty$ in $(0,1)$ defined by $x_n = \frac{n}{n+1}$. Observe that 
$$
    f(x_n) = \frac{\frac{n}{n+1}}{1-\left(\frac{n}{n+1}\right)^2} 
    = \frac{n(n+1)}{(n+1)^2 - n^2} = \frac{n(n+1)}{[(n+1)-n][(n+1)+n]} 
    = \frac{n(n+1)}{2n+1}
$$  
Written as $x_n = 1 - \frac{1}{n+1}$, one can more easily see that $(x_n)_{n=1}^\infty$ converges to $1$ in $\R$, hence is Cauchy in $(0,1)$. Now, let $\delta > 0$ and choose $N\in \N$ such that 
$|x_n - x_m| < \delta $ when $n,m \geq N$. For $\epsilon < \frac{1}{8}$ we have 

\begin{eqnarray*}
      \left| f(x_n) - f(x_{n+1}) \right|
    &=& \left|\frac{n(n+1)}{2n+1} - \frac{(n+1)(n+2)}{2n+3} \right| 
    = \left|\frac{n(n+1)(2n+3) - (n+1)(n+2)(2n+1)}{(2n+1)(2n+3)}  \right| \\ 
    &=&  \left|\frac{(2n^3+5n^2+3n) - (2n^3+7n^2+7n+2)}{(2n+1)(2n+3)}  \right|
    = \left|\frac{ 2n^2+4n+2 }{4n^2 + 8n + 3}  \right| \\
    &\geq& \left| \frac{2n^2}{ 16n^2 } \right| =  \frac{1}{8}.
\end{eqnarray*}
    So for any $\delta > 0$, we see that there exists two points $x_n, x_{n+1} \in (0,1)$ such that $|x_n - x_{n+1}| < \delta$ when $n$ is sufficiently large 
    but $f(x_n) - f(x_{n+1}) | \not < \epsilon$. Therefore $f(x)$ is not uniformly 
    continuous.

\end{proof}

\end{enumerate} 

\item Let $\{a_k\}_{k=1}^\infty$ be a bounded sequence of real numbers and $E$ given by: 
$$
E:= \bigg\{s \in \R \, \colon \, \text{ the set } \{k \in \N \, \colon \, a_k \geq s\} \text{ has at most finitely many elements}\bigg\}. 
$$
Prove that $\limsup_{k \to \infty} a_k = \inf E$. 

\begin{proof}
Let $e \in E$. As there are only finitely many $a_k \geq s$, there exists some $N \in \mathbb{N}$ such that $a_k < e$ for all $k \geq N$. Define $T_k := \{a_k : k \geq n\}$. It is clear that $e$ is thus an upper bound for $T_N$. So, $$e \geq \sup T_N \geq \limsup a_k.$$
Thus, $\limsup a_k$ is a lower bound for $E$, meaning $\inf E \geq \limsup a_n$.\\
Conversely, suppose $k \in \mathbb{N}$. $$T_k = \{a_n : n \geq k \}.$$
So, $\sup T_k \geq a_n$ for all $a_n \in T_k$. Then, $\{a_k : a_k \geq \sup T_k\}$ must be finite, so $\{k \in \mathbb{N} : a_k \geq \sup T_k\}$ is finite. So, $\sup T_k \in E$ for all $k \in \N$. Since $\inf E$ is a lower bound for $E$, $\inf E \leq \sup T_k$ for all $k \in \N$. Thus, $$\inf E \leq \lim (\sup T_k) = \limsup a_k.$$
We have both inequalities, therefore $\limsup a_k = \inf E$.
\end{proof} 

\item Assume $(X,d)$ is a compact metric space.

\begin{enumerate}

\item Prove that $X$ is both complete and separable.

\begin{proof}

\end{proof}

\item Suppose $\{x_k\}_{k=1}^\infty \subseteq X$ is a sequence such that the series $\sum_{k=1}^\infty d(x_k, x_{k+1})$ converges. Prove that the sequence $\{x_k\}_{k=1}^\infty$ converges in $X$. 

\begin{proof}

\end{proof} 

\end{enumerate} 

\item Suppose that $f \colon [0,2] \to \R$ is continuous on $[0,2]$ , differentiable on $(0,2)$, and such that $f(0) = f(2) = 0$, $f(c) = 1$ for some $c \in (0,2)$. Prove that there exists $x \in (0,2)$ such that $|f'(x)| >1.$

\begin{proof}
%there's probably a much cleaner/quicker way to do this (please email scott if you see it...I'd like to know it)
    We will consider three cases. First, suppose $c<1$. Then, by the mean value theorem, there exists $x\in (0,c)$ such that $f'(x)(c-0)=f(c)-f(0)$ so $f'(x)=\frac{f(c)}{c}=\frac{1}{c}>1$ since $c<1$. Similarly, if $c>1$ then by the mean value theorem there exits $y\in (c,2)$ such that \[|f'(y)|=\left\lvert\frac{f(2)-f(c)}{2-c}\right\rvert=\left\lvert \frac{-f(c)}{2-c}\right\rvert=\left\lvert\frac{-1}{2-c}\right\rvert>1\]
    since $1<c<2$.
    
    % This paragraph argues that the only way to avoid a greater than 1 slope is having an inverted absolute value graph with vertex at (1,1), but that's not differentiable
    Now, suppose $c=1$. If there exists $x\in (0,1)$ such that $x<f(x)$ then by the mean value theorem on the interval $(0,x)$ there exists $s\in (0,x)$ such that $f'(s)=\frac{f(x)}{x}>1$ since $f(x)>x$. Likewise, if there exists $x\in (0,1)$ such that $x>f(x)$ then the mean value theorem on $(x,1)$ gives a point $t\in (x,1)$ such that $\left\lvert f'(t)\right\rvert=\left\lvert \frac{f(1)-f(x)}{1-x}\right\rvert=\left\lvert\frac{1-f(x)}{1-x}\right\rvert>1$ since $x>f(x)$. So, on $(0,1)$, if the proposition does not hold then $f(x)=x$.
    Similarly, if there exists $x\in (1,2)$ such that $f(x)>2-x$ then the mean value theorem yields a point $u\in (x,2)$ such that $|f'(u)|=\left\lvert \frac{f(2)-f(x)}{2-x}\right\rvert=\left\lvert \frac{-f(x)}{2-x}\right\rvert>1$ since $f(x)>2-x$. If there exists $y\in (1,2)$ such that $f(y)<2-y$ then again by the mean value theorem there exists $v\in (1,y)$ such that $|f'(v)|=\left\lvert\frac{f(y)-f(1)}{y-1}\right\rvert=\left\lvert\frac{f(y)-1}{y-1}\right\rvert>1$ since $f(y)<2-y$ so $|f(y)-1|>|y-1|$. So, on $(1,2)$ if the proposition does not hold then $f(x)=2-x$. However, notice that since $f(x)$ is differentiable at $x=1$ we cannot have $f(x)=x$ on $(0,1)$ and $f(x)=2-x$ on $(1,2)$.
\end{proof} 

\item Let $f_n(x) = n^\beta x(1-x^2)^n$, $x \in [0,1]$, $n \in \N$. 

\begin{enumerate} 

\item Prove that $\{f_n\}_{n=1}^\infty$ converges pointwise on $[0,1]$ for every $\beta \in \R$. 

\begin{proof} 

\end{proof} 

\item Show that the convergence in part (a) is uniform for all $\beta < \frac{1}{2}$, but not uniform for any $\beta \geq \frac{1}{2}$. 

\begin{proof} 

\end{proof} 

\end{enumerate} 

\item \begin{enumerate} 

\item Suppose $f \colon [-1,1] \to \R$ is a bounded function that is continuous at $0$. Let $\alpha(x) = -1$ for $x \in [-1,0]$ and $\alpha(x)=1$ for $x \in (0,1]$. Prove that $f \in \mathcal{R}(\alpha)[-1,1]$, i.e., $f$ is Riemann integrable with respect to $\alpha$ on $[-1,1]$, and $\int_{-1}^1 f d\alpha = 2f(0)$. 

\begin{proof} 
    Let $\epsilon>0$. Choose $\delta >0$ so that if $|x|<\delta$, then $|f(x)-f(0)|<\epsilon$. Let $P$ be a partition of $[-1,1]$ with $0 \in P$ and $\operatorname{mesh}(P)<\delta$. Then $|U(f,P,\alpha)-L(f,P,\alpha)|=|\sum_{i=1}^n(M_i-m_i)\Delta \alpha_i|=(|\sup_{x \in [0,x_k]}f(x)-\inf_{x \in [0,x_k]}f(x)|)2<4\epsilon$. Thus $f$ is integrable with respect to $\alpha$. Additionally, we have $L(f,P,\alpha)\leq 2f(0)\leq U(f,P,\alpha)$ for all partitions $P$ of the form described above, and so $\int_{-1}^1 f d\alpha = 2f(0)$.
\end{proof} 

\item Let $g \colon [0,1] \to \R$ be a continuous function such that $\int_0^1 g(x)x^{3k+2} dx = 0$ for all $k = 0, 1, 2, \ldots$. Prove that $g(x) =0$ for all $x \in [0,1]$. 

\begin{proof} 
    Since $g(x)$ is continuous, so is $g(x^{1/3})$. Thus by the Weierstrauss Approximation Theorem, we can find a sequence of polynomials $(p_n(x))\to g(x^{1/3})$ uniformly. Since this holds for all values $x\in [0,1]$, we have that $(p_n(x^3))$ converges to $g(x)$ uniformly. Then we have $(x^2p_n(x^3))$ converges to $x^2g(x)$ uniformly. Note that by assumption, $\int_0^1 g(x)x^2p_n(x^3)dx=0$, and so $0 = \lim_{n \to \infty}\int_0^1 g(x)x^2p_n(x^3)dx=\int_0^1 \lim_{n \to \infty}g(x)x^2p_n(x^3)dx=\int_0^1x^2g^2(x)dx$. Since $x^2g^2(x)$ is non-negative, and its integral is zero, we conclude that $x^2g^2(x)=0$ for all $x$. Therefore, we have $g(x)=0$.
\end{proof} 

\end{enumerate} 

\end{enumerate} 

\section{Metric Spaces and Topology}

\begin{enumerate} 

\item (May 2019, \#1) Let $(M, d_M)$, $(N, d_N)$ be metric spaces. Define $d_{M \times N} \colon (M \times N) \times (M \times N) \to \R$ by $$d_{M \times N}((x_1, y_1), (x_2, y_2)) := d_M(x_1, x_2) + d_N(y_1, y_2).$$
\begin{enumerate}
	
\item Prove that $(M \times N, d_{M \times N})$ is a metric space. 

\begin{proof} 
To prove that $(M \times N, d_{M \times N})$ is a metric space we must prove that $d_{M\times N}$ is a metric on $M \times N$.

\begin{itemize}
    \item Positive Definite-
    
    Let $(x_1,y_1), (x_2,y_2) \in M \times N$. Then  since $d_M$ is a metric on $M$, then $d_M(x_1,x_2)\geq 0$ for all $x_i,x_j \in M$ and $d_N$ is a metric on $N$ and likewise $d_N(y_1,y_2)\geq 0$ for any $y_i,y_j \in N.$
    
    Then by definition $d_{M\times N}((x_1,y_1),(x_2,y_2))=d_M(x_1,x_2)+d_N(y_1,y_2)\geq 0 + 0 =0.$ Hence since $(x_1,y_1),(x_2,y_2)$ are arbitrary, $d_{M\times N}((x_1,y_1),(x_2,y_2))\geq 0$ for all $(x_i,y_i),(x_j,y_j)\in M \times N$.
    
    Suppose that $d_{M \times N}((x_1,y_1),(x_2,y_2))=0.$ By definition $d_{M \times N}((x_1,y_1),(x_2,y_2))=d_M(x_1,x_2)+d_N(y_1,y_2)$. Therefore $d_M(x_1,x_2)+d_N(y_1,y_2)=0$, since $d_M, d_N$ are metrics, then $d_M(x_1,x_2)\geq 0, d_N(y_1,y_2)\geq 0$, which implies that $d_M(x_1,x_2)=d_N(y_1,y_2)=0$ and also since they are metrics we have that $x_1=x_2, y_1=y_2.$ Hence, $(x_1,y_1)=(x_2,y_2).$
    
    Now suppose that $(x_1,y_1)=(x_2,y_2).$ Then $x_1=x_2, y_1=y_2$ and for the metrics $d_M, d_N$ we would have $d_M(x_1,x_2)=0, d_N(y_1,y_2)=0.$ Thus $d_{M \times N}((x_1,y_1),(x_2,y_2))=d_M(x_1,x_2)+d_N(y_1,y_2)=0+0=0$.
    
    Therefore $d_{M \times N}((x_1,y_1),(x_2,y_2))=0$ if and only if $(x_1,y_1)=(x_2,y_2).$
    \item Symmetric
    
    Let $(x_1,y_1), (x_2,y_2) \in M \times N$. Then  since $d_M$ is a metric on $M$, then $d_M(x_1,x_2)=d_M(x_2,x_1)$ for all $x_i,x_j \in M$ and $d_N$ is a metric on $N$ and likewise $d_N(y_1,y_2)=d_N(y_2,y_1)$ for any $y_i,y_j \in N.$ Therefore,
    \begin{align*}
        d_{M \times N}((x_1,y_1),(x_2,y_2))&=d_M(x_1,x_2)+d_N(y_1,y_2)\\
        &=d_M(x_2,x_1)+d_N(y_2,y_1)\\
        &=d_{M \times N}((x_2,y_2),(x_1,y_1)).
    \end{align*}
    \item Triangle Inequality
    
    Since $d_M, d_N$ are metrics then for all $x_1,x_2,x_3 \in M, y_1,y_2,y_3 \in N$ we have that $d_M(x_1,x_2)\leq d_M(x_1,x_3)+d_M(x_3,x_2)$ and that $d_N(y_1,y_2)\leq d_N(y_1,y_3)+d_N(y_3,y_2).$ Therefore,
    \begin{align*}
        d_{M \times N}((x_1,y_1),(x_2,y_2))&=d_M(x_1,x_2)+d_N(y_1,y_2)\\
      d_M(x_1,x_2)+d_N(y_1,y_2) &\leq d_M(x_1,x_3)+d_M(x_3,x_2)+d_N(y_1,y_3)+d_N(y_3,y_2)\\
       d_M(x_1,x_3)+d_M(x_3,x_2)+d_N(y_1,y_3)+d_N(y_3,y_2) &=d_M(x_1,x_3)+d_N(y_1,y_3)+d_M(x_3,x_2)+d_N(y_3,y_2)\\
      d_M(x_1,x_3)+d_N(y_1,y_3)+d_M(x_3,x_2)+d_N(y_3,y_2)&=d_M((x_1,y_1),(x_3,y_3))+d_M((x_3,y_3),(x_2,y_2)). 
    \end{align*}
    
    Hence $ d_{M \times N}((x_1,y_1),(x_2,y_2))\leq d_M((x_1,y_1),(x_3,y_3))+d_M((x_3,y_3),(x_2,y_2)).$
\end{itemize}
Therefore $d_{M \times N}$ is a metric on $M \times N$ and $(M \times N, d_{M\times N})$ is a metric space.
\end{proof} 
	
\item Let $S \subseteq M$ and $T \subseteq N$ be compact sets in $(M, d_M)$ and $(N, d_N)$, respectively. Prove that $S \times T$ is a compact set in $(M \times N, d_{M \times N})$.

\begin{proof} 
By part a we showed that $(M \times N, d_{M \times N})$ is a metric space. Let $\{s_n,t_n\}$ be a sequence in $S \times T.$ Since $\{s_n\}$ is a sequence on a compact set $S$ in a metric space $(M,d_M)$ then it has a convergent subsequence ${s_{n_k}}.$ Let $\lim_{k \to \infty}{s_{n_k}}=s_0.$

Since $\{t_{n_k}\}$ is a sequence on a compact set $T$ in a metric space. Thus $\{t_{n_k}\}$ has a convergent subsequence $\{t_{n_{k_j}}\}.$ Let $\lim_{j\to \infty} t_{n_{k_j}}=t_0.$ Thus $\{s_{n_{k_j}}\}$ is a subsequence of $\{s_{n_k}\}.$ And since $\{s_{n_k}\}$ converges to $s_0$, then any subsequence also converges to $s_0.$

Let $\epsilon >0$ be given. Then for $\epsilon/2$ there exists $N_1, N_2\in \N$ such that for all $n_{k_j}\geq N_1, d_M(s_{n_{k_j}},s_0)<\epsilon/2$, and for all $n_{k_j}\geq N_2, d_N(t_{n_{k_j}},t_0)<\epsilon/2$. Choose $N=\text{Max}(\{N_1,N_2\}).$

Then $d_{M \times N}((s_{n_{k_j}},t_{n_{k_j}}),(s_0,t_0))=d_M(s_{n_{k_j}},s_0)+d_N(t_{n_{k_j}},t_0)<\epsilon/2 + \epsilon/2 = \epsilon.$

Therefore $d_{M \times N}((s_{n_{k_j}},t_{n_{k_j}}),(s_0,t_0))< \epsilon.$

Hence $\{(s_{n_{k_j}},t_{n_{k_j}})$ converges to $(s_0,t_0).$ Therefore $S \times T$ is sequentially compact and $S \times T$ is therefore compact.


\end{proof} 
	
\end{enumerate} 
	
\item (June 2003, \#1b,c) (b) Show by example that the union of infinitely many compact subsets of a metric space need not be compact. (c) If $(X,d)$ is a metric space and $K\subset X$ is compact, define $d(x_0,K)=\inf_{y\in K} d(x_0,y)$. Prove that there exists a point $y_0\in K$ such that $d(x_0,K)=d(x_0,y_0)$.

\begin{proof} 

\end{proof} 
		
\item (January 2009, \#4a) Consider the metric space $(\mathbb{Q},d)$ where $\mathbb{Q}$ denotes the rational numbers and $d(x,y)=|x-y|$. Let $E=\{x\in\mathbb{Q}:x>0,\,2<x^2<3\}$. Is $E$ closed and bounded in $\mathbb{Q}$?  Is $E$ compact in $\mathbb{Q}$? 

\begin{proof} 

\end{proof} 
	
\item (January 2011 \#3a) Let $(X,d)$ be a metric space, $K\subset X$ be compact, and $F\subset X$ be closed. If $K\cap F=\emptyset$, prove that there exists an $\epsilon>0$ so that $d(k,f)\geq \epsilon$ for all $k\in K$ and $f\in F$.

\begin{proof} 
We prove this by contrapositive. Suppose for all $\epsilon >0$, there exists $k \in K$, $f \in F$ such that $d(k,f)< \epsilon$. Then for all $n \in \N$, we can choose $k_n \in K$, $f_n \in F$ such that $d(k_n, f_n) < \frac{1}{n}$.

Since $k_n$ is a sequence in $K$, which is compact (and therefore sequentially compact), there exists a subsequence $k_{n_j} \subseteq k_n$ with the property that $k_{n_j}$ converges to some $k_0 \in K$. Find $N \in \N$ such that for $n \geq N$, $d(k_{n_j}, k_0) < \frac{\epsilon}{2}$ and $\frac{1}{n} < \frac{\epsilon}{2}$. Then
$$ d(f_{n_j}, k_0) \leq d(f_{n_j}, k_{n_j}) + d(k_{n_j}, k_0) < \frac{\epsilon}{2} + \frac{\epsilon}{2} = \epsilon$$

Thus, $f_{n_j}$ also converges to $k_0$, and since $F$ is closed, $k_0 \in F$. So $K \cap F \neq \emptyset$.
\end{proof} 
	
\item Let $(X,d)$ be an unbounded and connected metric space. Prove that for each $x_0 \in X$, the set $\{x \in X \, \colon \,  d(x,x_0) = r\}$ is nonempty. 

\begin{proof} 

\end{proof} 




\end{enumerate} 
\section{Sequences and Series}
    \begin{enumerate}
	
	\item (June 2013 \#1a) Let $a_n =\sqrt{n}\left(\sqrt{n+1}-\sqrt{n}\right)$.  Prove that $\lim_{n\to\infty}a_n=1/2$.
	
	\begin{proof}
	
	\end{proof}
		 	 
	\item (January 2014 \#2) (a) Produce sequences $\{a_n\},\,\{b_n\}$ of positive real numbers such that $$\liminf_{n\to\infty}(a_nb_n)>\left(\liminf_{n\to\infty} a_n\right) \left(\liminf_{n\to\infty} b_n\right).$$ (b) If $\{a_n\},\,\{b_n\}$ are sequences of positive real numbers and $\{a_n\}$ converges, prove that $$\liminf_{n\to\infty}(a_nb_n)=\left(\lim_{n\to\infty}a_n\right)\left(\liminf_{n\to\infty}b_n\right).$$
	
	\begin{proof}
	
	\end{proof}
	
	\item (May 2011 \#4a) Determine the values of $x\in\R$ for which $\displaystyle\sum_{n=1}^\infty \frac{x^n}{1+n|x|^n}$ converges, justifying your answer carefully.
	
	\begin{proof} 
	
	\end{proof} 
	
	\item (June 2005 \#3b) If the series $\sum_{n=0}^\infty a_n$ converges conditionally, show that the radius of convergence of the power series $\sum_{n=0}^\infty a_nx^n$ is 1.
	
	\begin{proof}
	
	\end{proof}
	
	\item (January 2011 \#5) Suppose $\{a_n\}$ is a sequence of positive real numbers such that $\lim_{n\to\infty}a_n=0$ and $\sum a_n$ diverges. Prove that for all $x>0$ there exist integers $n(1)<n(2)<\ldots$ such that $\sum_{k=1}^\infty a_{n(k)}=x$.\\
	(Note: Many variations on this problem are possible including more general rearrangements.  You may also wish to show that if $\sum a_n$ converges conditionally then given any $x\in\R$ there is a rearrangement of $\{b_n\}$ of $\{a_n\}$ such that $\sum b_n=r$.  See Rudin Thm. 3.54 for a further generalization.)
	
	\begin{proof}
	
	\end{proof}
	 
	\item (June 2008 \# 4b) Assume $\beta >0$, $a_n>0$, $n=1,2,\ldots$, and the series $\sum a_n$ is divergent.  Show that $\displaystyle \sum \frac{a_n}{\beta + a_n}$ is also divergent.
	
	\begin{proof}
	
	\end{proof} 
	
	
\end{enumerate}
\section{Continuity of Functions}

\begin{enumerate}
    \item (January 2009 #2a, Obfuscated) Is $f(x)=\text{ln}(x)$ uniformly continuous on $(0,1]?$ Prove your answer. 
    
    \begin{proof}
    
    \end{proof}
    
    (You may also wish to try the same question with $f(x)=sin(1/x)$ on $(0,\pi]$.)
    \begin{proof}
    
    \end{proof}
    
    	
	\item (January 2006 \#2b) Assume that $f$ is differentiable at $a$. Evaluate $$\lim_{x\to a}\frac{a^nf(x)-x^nf(a)}{x-a},\quad n\in\N.$$ 
	
	 \begin{proof}
    
    \end{proof}
	
	\item (June 2007 \#3a) Suppose that $f,g:\R\to\R$ are differentiable, that $f(x)\leq g(x)$ for all $x\in\R$, and that $f(x_0)=g(x_0)$ for some $x_0$. Prove that $f'(x_0)=g'(x_0)$.
	
	 \begin{proof}
    
    \end{proof}
	
	\item (June 2008 \#3a) Prove that if $f'$ exists and is bounded on $(a,b]$, then $\lim_{x\to a^+}f(x)$ exists.
	
	 \begin{proof}
    
    \end{proof}
	
	\item (January 2012 \#4b, extended) Let $f:\R\to\R$ be a differentiable function with $f'\in C(\R)$. Assume that there are $a,b\in\R$ with $\lim_{x\to\infty}f(x)=a$ and $\lim_{x\to\infty}f'(x)=b$. Prove that $b=0$. 
	Then, find a counterexample to show that the assumption  $\lim_{x\to\infty}f'(x)$ exists is necessary.
	
	 \begin{proof}
    
    \end{proof}
	
	\item (June 2012 \#1a) Suppose that $f:\R\to\R$ satisfies $f(0)=0$. Prove that $f$ is differentiable at $x=0$ if and only if there is a function $g:\R\to\R$ which is continuous at $x=0$ and satisfies $f(x)=xg(x)$ for all $x\in\R$.
	
	 \begin{proof}
    
    \end{proof}
    
\end{enumerate}

\section{Differential Calculus} 

\begin{enumerate}

	\item (June 2005 \#1a) Use the definition of the derivative to prove that if $f$ and $g$ are differentiable at $x$, then $fg$ is differentiable at $x$.
	
	\begin{proof}
    
    \end{proof}
	
	\item (January 2006 \#2b) Assume that $f$ is differentiable at $a$. Evaluate $$\lim_{x\to a}\frac{a^nf(x)-x^nf(a)}{x-a},\quad n\in\N.$$ 
	
	\begin{proof}
    
    \end{proof}
	
	\item (June 2007 \#3a) Suppose that $f,g:\R\to\R$ are differentiable, that $f(x)\leq g(x)$ for all $x\in\R$, and that $f(x_0)=g(x_0)$ for some $x_0$. Prove that $f'(x_0)=g'(x_0)$.
	
	\begin{proof}
    
    \end{proof}
	
	\item (June 2008 \#3a) Prove that if $f'$ exists and is bounded on $(a,b]$, then $\lim_{x\to a^+}f(x)$ exists.
	
	\begin{proof}
    
    \end{proof}
	
	\item (January 2012 \#4b, extended) Let $f:\R\to\R$ be a differentiable function with $f'\in C(\R)$. Assume that there are $a,b\in\R$ with $\lim_{x\to\infty}f(x)=a$ and $\lim_{x\to\infty}f'(x)=b$. Prove that $b=0$. 
	Then, find a counterexample to show that the assumption  $\lim_{x\to\infty}f'(x)$ exists is necessary.
	
	\begin{proof}
    
    \end{proof}
	
	\item (June 2012 \#1a) Suppose that $f:\R\to\R$ satisfies $f(0)=0$. Prove that $f$ is differentiable at $x=0$ if and only if there is a function $g:\R\to\R$ which is continuous at $x=0$ and satisfies $f(x)=xg(x)$ for all $x\in\R$.
	
	\begin{proof}
    
    \end{proof}
    
\end{enumerate}

\section{Integral Calculus}

\begin{enumerate} 

	\item (January 2006 \#4b) Suppose that $f$ is continuous and $f(x)\geq 0$ on $[0,1]$. If $f(0)>0$, prove that $\int_0^1 f(x)dx>0$.
	
	\begin{proof}
    
    \end{proof}
	
	\item (June 2005 \#1b) Use the definition of the Riemann integral to prove that if $f$ is bounded on $[a,b]$ and is continuous everywhere except for finitely many points in $(a,b)$, then $f\in\SR$ on $[a,b]$.
	
	\begin{proof}
    
    \end{proof}
	
	\item (January 2010 \#5) Suppose that $f:[a,b]\to\R$ is continuous, $f\geq 0$ on $[a,b]$, and put $M=\sup\{f(x):x\in[a,b]\}$. Prove that $$\lim_{p\to\infty}\left(\int_a^b f(x)^p\,dx\right)^{1/p}=M.$$
	
	\begin{proof}
    
    \end{proof}
		
	\item (January 2009 \#4b) Let $f$ be a continuous real-valued function on $[0,1]$. Prove that there exists at least one point $\xi\in[0,1]$ such that $\int_0^1 x^4 f(x)\,dx=\frac{1}{5}f(\xi)$.
	
	\begin{proof}
    Assume that $f$ is a continuous real-valued function on $[0,1]$. Then, by the Intermediate Value Theorem we have that $f$ attains its maximum and minimum on $[0,1]$. That is, for some $a,b\in[0,1]$, 
    
    $$f(a)=\min\limits_{[0,1]}f(x) \qquad \text{and} \qquad  f(b)=\max\limits_{[0,1]}f(x).$$
    
    We now have $f(a)\leq f(x)\leq f(b)$ for all $x\in[0,1]$. This gives 
    $$f(a)\int_0^1 x^4dx\leq \int_0^1 x^4f(x)dx\leq f(b)\int_0^1 x^4dx.$$
    
    By the Fundamental Theorem of Calculus we know that 
    
    $$\int_0^1x^4dx=\frac{1}{5}.$$
    
    Thus, it follows that 

$$\frac{1}{5}f(a)\leq\int_0^1 x^4f(x)dx\leq \frac{1}{5}f(b)$$
giving 

$$f(a)\leq 5\int_0^1 x^4f(x)dx\leq f(b).$$

By the Intermediate Value Theorem, there exists $\xi\in[0,1]$ such that 

$$f(\xi)=5\int_0^1 x^4f(x)dx.$$

Therefore, we have that there exists $\xi\in[0,1]$ such that $\int_0^1 x^4 f(x)dx=\frac{1}{5}f(\xi)$. 
   
    
    
    \end{proof}
	
	\item (June 2009 \#5b) Let $\phi$ be a real-valued function defined on $[0,1]$ such that $\phi$, $\phi'$, and $\phi''$ are continuous on $[0,1]$. Prove that $$\int_0^1 \cos x \frac{x\phi'(x)-\phi(x)+\phi(0)}{x^2}\,dx<\frac{3}{2}||\phi''||_\infty,$$
	where $||\phi''||_\infty = \sup_{[0,1]}|\phi''(x)|.$ Note that $3/2$ may not be the smallest possible constant.
	
	\begin{proof}
    
    \end{proof}
	
	\item (Essentialy June 2013 \#7) Prove Theorem~\ref{thm:d5-alpha}.
	
	\begin{proof}
    
    \end{proof}

\end{enumerate} 


\section{Sequences and Series of Functions} 

\begin{enumerate}
	\item (June 2010 \#6a) Let $f:[0,1]\to\R$ be continuous with $f(0)\neq f(1)$ and define $f_n(x)=f(x^n)$. Prove that $f_n$ does not converge uniformly on $[0,1]$. 
	
	
	\begin{proof}
    
    \end{proof}
	
	\item (January 2008 5a) Let $f_n(x) = \frac{x}{1+nx^2}$ for $n \in \N$. Let $\mathcal{F} := \{f_n \, \colon \, n = 1, 2, 3, \ldots\}$ and $[a,b]$ be any compact subset of $\R$. Is $\mathcal{F}$ equicontinuous? Justify your answer. 
	
	\begin{proof}
    
    \end{proof}
	
	\item (January 2005 \#4, June 2010 \#6b) If $f:[0,1]\to\R$ is continuous, prove that $$\displaystyle\lim_{n\to\infty}\int_0^1 f(x^n)\,dx=f(0).$$
	
	\begin{proof}
    
    \end{proof}
	
	\item (January 2020 4a) Let $M<\infty$ and $\mathcal{F} \subseteq C[a,b]$. Assume that each $f \in \mathcal{F}$ is differentiable on $(a,b)$ and satisfies $|f(a)| \leq M$ and $|f'(x)| \leq M$ for all $x \in (a,b)$. Prove that $\mathcal{F}$ is equicontinuous on $[a,b]$. 
	
	\begin{proof}
    
    \end{proof}
	
	  


		
	\item (June 2005 \#5) Suppose that $f\in C([0,1])$ and that $\displaystyle \int_0^1 f(x)x^n\,dx=0$ for all $n=99,100,101,\ldots$. Show that $f\equiv 0$.\\
	Note: Many variations on this problem exist. See June 2012 \#6b and others. 
	
	\begin{proof}
    
    \end{proof}
		
	\item (January 2005 \#3b) Suppose $f_n:[0,1]\to\R$ are continuous functions converging uniformly to $f:[0,1]\to\R$. Either prove that $\displaystyle\lim_{n\to\infty}\int_{1/n}^1 f_n(x)\,dx=\int_0^1 f(x)\,dx$ or give a counterexample.
	
	
	\begin{proof}
    
    \end{proof}

\end{enumerate}

\section{Miscellaneous Topics}
\subsection*{Bounded Variation}
\begin{enumerate}

    \item (January 2018) Let $f \colon [a,b] \to \R$. Suppose $f \in \text{BV}[a,b]$. Prove $f$ is the difference of two increasing functions.
    
    	\begin{proof}
    
    \end{proof}

    
	\item (January 2007, 6a) Let $f$ be a function of bounded variation on $[a,b]$. Furthermore, assume that for some $c>0$, $|f(x)| \geq c$ on $[a,b]$. Show that $g(x) = 1/f(x)$ is of bounded variation on $[a,b]$.
	
		\begin{proof}
    
    \end{proof}
	
	\item (January 2017, 2a) Define $f \colon [0,1] \to [-1,1]$ by 
	$$
	f(x):= \begin{cases} x\sin\big({\frac{1}{x}}\big) & 0 < x \leq 1 \\ 0 & x = 0 \end{cases}
	$$
	Determine, with justification, whether $f$ is if bounded variation on the interval $[0,1]$. 
	
		\begin{proof}
    
    \end{proof}
	
	\item (January 2020, 6a) Let $\{a_n\}_{n=1}^\infty \subseteq \R$ and a strictly increasing sequence $\{x_n\}_{n=1}^\infty \subseteq (0,1)$ be given. Assume that $\sum_{n=1}^\infty a_n$ is absolutely convergent, and define $\alpha \colon [0,1] \to \R$ by 
	$$
	\alpha(x):= \begin{cases} a_n &  x=x_n \\ 0 & \text{otherwise} \end{cases}.
	$$
    Prove or disprove: $\alpha$ has bounded variation on $[0,1]$. 
    
    	\begin{proof}
    
    \end{proof}
\end{enumerate}

\subsection*{Metric Spaces and Topology} 

\begin{enumerate} 

\item Find an example of a metric space $X$ and a subset $E \subseteq X$ such that $E$ is closed and bounded but not compact. 

	\begin{proof}
    
    \end{proof}

\item (May 2017 6) Let $(X,d)$ be a metric space. A function $f \colon X \to \R$ is said to be lower semi-continuous (l.s.c) if $f^{-1}(a,\infty)  = \{x \in X \, \colon \,  f(x)> a\}$ is open in $X$ for every $a \in \R$. Analogously, $f$ is upper semi-continuous (u.s.c) if $f^{-1}(-\infty, b) = \{x \in X \, \colon \,  f(x)<b\}$ is open in $X$ for every $b \in \R$. 

	\begin{proof}
    
    \end{proof}

\begin{enumerate} 

\item[(a)] Prove that a function $f \colon X \to \R$ is continuous if and only if $f$ is both l.s.c. and u.s.c. 

	\begin{proof}
    
    \end{proof}

\item[(b)] Prove that $f$ is lower semi-continuous if and only if $\liminf_{n \to \infty} f(x_n) \geq f(x)$  whenever $\{x_n\}_{n=1}^\infty \subseteq X$ such that $x_n \to x$ in $X$. 
	\begin{proof}
    
    \end{proof}

\end{enumerate} 

\item (January 2017 3) Let $(X,d)$ be a compact metric space. Suppose that $f_n \colon X \to [0,\infty)$ is a sequence of continuous functions with $f_n(x) \geq f_{n+1}(x)$ for all $n \in \N$ and $x \in X$, and such that $f_n \to 0$ pointwise on $X$. Prove that $\{f_n\}_{n=1}^\infty$ converges uniformly on $X$. 

	\begin{proof}
    
    \end{proof}

\end{enumerate} 

\subsection*{Integral Calculus} 

\begin{enumerate}

\item (June 2014 1)Define $\alpha \colon [-1,1] \to \R$ by 
$$
\alpha(x) := \begin{cases} -1 & x \in [-1,0] \\ 1 & x \in (0,1]. \end{cases}
$$
Let $f \colon [-1,1] \to \R$ be a function that is uniformly bounded on $[-1,1]$ and continuous at $x=0$, but not necessarily continuous for $x \neq 0$. Prove that $f$ is Riemann-Stieltjes integrable with respect to $\alpha$ over $[-1,1]$ and that $$\int_{-1}^1 f(x)d\alpha(x) = 2f(0).$$

	\begin{proof}
    
    \end{proof}


\item (June 2017 2) Prove : $f \in \mathcal{R}(\alpha)$ on $[a,b]$ if and only if for any $a <c<b$, $f \in \mathcal{R}(\alpha)$ on $[a,c]$ and on $[c,b]$. In addition, if either condition holds, then we have that 
$$
\int_a^c fd\alpha + \int_c^b fd\alpha = \int_a^b fd\alpha. 
$$

	\begin{proof}
    
    \end{proof}

\item (Spring 2017 7) Prove that if $f \in \mathcal{R}$ on $[a,b]$ and $\alpha \in C^1[a,b]$, then the Riemann integral $\int_a^b f(x)\alpha'(x)dx$ exists and $$\int_a^b f(x) d\alpha(x)= \int_a^b f(x)\alpha'(x)dx.$$

	\begin{proof}
    
    \end{proof}

\end{enumerate} 

\subsection*{Sequences and Series (and of Functions)} 

\begin{enumerate} 

\item (January 2006 1)  Let the power series series $\sum_{n=0}^\infty a_nx^n$ and $\sum_{n=0}^\infty b_nx^n$ have radii of convergence $R_1$ and $R_2$, respectively. 

	\begin{proof}
    
    \end{proof}

\begin{enumerate} 

\item[(a)] If $R_1 \neq R_2$, prove that the radius of convergence, $R$, of the power series $\sum_{n=0}^\infty (a_n+b_n)x^n$ is $\min\{R_1, R_2\}$. What can be said about $R$ when $R_1 = R_2$? 
	\begin{proof}
    
    \end{proof}

\item[(b)] Prove that the radius of convergence, $R$, of $\sum_{n=0}^\infty a_nb_nx^n$ satisfies $R \geq R_1R_2$. Show by means of example that this inequality can be strict.
	\begin{proof}
    
    \end{proof}

\end{enumerate} 

\item Show that the infinite series $\sum_{n=0}^\infty x^n2^{-nx}$ converges uniformly on $[0,B]$ for any $B > 0$. Does this series converge uniformly on $[0,\infty)$? 
	\begin{proof}
    
    \end{proof}

\item (January 2006 4a) Let $$f_n(x) = \begin{cases} \frac{1}{n}  & x \in (\frac{1}{2^{n+1}}, \frac{1}{2^n}] \\ 0 & \text{ otherwise}.\end{cases}$$
	\begin{proof}
    
    \end{proof}

Show that $\sum_{n=1}^\infty f_n$ does not satisfy the Weierstrass M-test but that it nevertheless converges uniformly on $\R$. 
	\begin{proof}
    
    \end{proof}

\item Let $f_n \colon [0,1) \to \R$ be the function defined by $$f_n(x):= \sum_{k=1}^n \frac{x^k}{1+x^k}.$$
	\begin{proof}
    
    \end{proof}

\begin{enumerate}

\item[(a)] Prove that $f_n$ converges to a function $f \colon [0,1) \to \R$. 
	\begin{proof}
    
    \end{proof}

\item[(b)] Prove that for every $0 < a < 1$ the convergence is uniform on $[0,a]$. 
	\begin{proof}
    
    \end{proof}

\item[(c)] Prove that $f$ is differentiable on $(0,1)$. 
	\begin{proof}
    
    \end{proof}

\end{enumerate} 

\end{enumerate} 

\subsection*{January 2019 Qualifying Exam}


\begin{enumerate}

\item[4] Suppose that $f: [0,1] \to \R$ is differentiable and $f(0) = 0$. Assume that there is a
$k > 0$ such that $$|f'(x)| \leq k|f(x)|$$ for all 
$x\in [0,1]$. Prove that $f(x) = 0$ for all $x\in [0,1]$.

\begin{proof}

    Let $0<\delta_1<1$, and fix $x_1 \in (0, \delta_1]$. Since $f(x)$ is differentiable on all of $[0,1]$, $f(x)$ is differentiable on all of $(0, \delta_1)$. So by the Mean 
    Value Theorem, there exists $x_2 \in (0, x_1)$ such that $$ f'(x_2) = \frac{f(x_1) - f(0)}{x_1-0} = \frac{f(x_1)}{x_1} .$$ Solving for $f(x_1)$, we get 
    $f(x_1) = f'(x_2)x_1$. So by hypothesis, $f(x_1) = f'(x_2) x_1 \leq k|f(x_2)|x_1$. Assume for $x_1, x_2, \ldots, x_{n-1} \in (0,1)$ the following conditions are satisfied for $j\in\{1,2,\ldots, n-1\}$. 
    \begin{eqnarray*}
    x_j &\in& (0,x_{j-1}) \\
    f(x_{j-1}) &=& f'(x_j)x_{j-1} \\
    f(x_1) &\leq& k^{j-1}|f(x_j)|(x_{j-1} \cdots x_2x_1)
    \end{eqnarray*}
    I now claim that this inductive process is true for $j=n$, given that it holds for all $j \leq n$. 
    We apply the Mean Value Theorem to find some $x_n \in (0, x_{n-1})$ such that $f'(x_n) = \frac{f(x_{n-1})}{x_{n-1}}$, then write $f(x_{n-1}) = f'(x_n)x_{n-1}.$ By our inductive hypothesis, we have 
    \begin{eqnarray*}
    |f(x_1)| &\leq& k^{n-2}|f(x_{n-1})|(x_{n-2}\cdots x_2x_1) \\
    &=& k^{n-2}|f'(x_n)x_{n-1}|(x_{n-2}\cdots x_2x_1)  \\
    &\leq& k^{n-2}(k|f(x_n)|)(x_{n-1}x_{n-2}\cdots x_2x_1) \\
    &=& k^{n-1}|f(x_n)|(x_{n-1}x_{n-2}\cdots x_2x_1).
    \end{eqnarray*}
    
    Thus our claim holds by induction. Now, since $f$ is a continuous function on the closed interval, we can apply the Extreme Value Theorem to find some $M>0$ for which $f(x) \leq M$ for all $x\in [0,1]$. Thus we get 
    $$|f(x_1)| \leq k^n M (x_n \cdots x_1)
    =(kx_n)(kx_{n-1})\cdots(kx_1)M
    $$ for all $n \in \N$. If $k < \frac{1}{x_1}$, then for any $\epsilon > 0$ we can find $N\in \N$ sufficiently large so that 
    $|f(x_1)| < \epsilon$. Otherwise, we set $\delta_1< \frac{1}{k}$ so that $kx_1< 1$.

\end{proof}

\end{enumerate}

\end{document} 
