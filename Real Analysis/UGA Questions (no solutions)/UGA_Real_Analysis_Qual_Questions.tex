\input{"preamble.tex"}

\addbibresource{UGA\_Real\_Analysis\_Qual\_Questions.bib}

\let\Begin\begin
\let\End\end
\newcommand\wrapenv[1]{#1}

\makeatletter
\def\ScaleWidthIfNeeded{%
 \ifdim\Gin@nat@width>\linewidth
    \linewidth
  \else
    \Gin@nat@width
  \fi
}
\def\ScaleHeightIfNeeded{%
  \ifdim\Gin@nat@height>0.9\textheight
    0.9\textheight
  \else
    \Gin@nat@width
  \fi
}
\makeatother

\setkeys{Gin}{width=\ScaleWidthIfNeeded,height=\ScaleHeightIfNeeded,keepaspectratio}%

\title{
\textbf{
    UGA Real Analysis Questions (Fall 2014 -- Spring 2021)
  }
  }







\begin{document}

\date{}
\author{D. Zack Garza}
\maketitle


\newpage

% Note: addsec only in KomaScript
\addsec{Table of Contents}
\tableofcontents
\newpage

\hypertarget{preface}{%
\section{Preface}\label{preface}}

I'd like to extend my gratitude to Peter Woolfitt for supplying many
solutions and checking many proofs of the rest in problem sessions. Many
other solutions contain input and ideas from other graduate students and
faculty members at UGA, along with questions and answers posted on Math
Stack Exchange or Math Overflow.

\hypertarget{undergraduate-analysis-uniform-convergence}{%
\section{Undergraduate Analysis: Uniform
Convergence}\label{undergraduate-analysis-uniform-convergence}}

\hypertarget{fall-2018-1-done}{%
\subsection{\texorpdfstring{Fall 2018 \# 1
\(\done\)}{Fall 2018 \# 1 \textbackslash done}}\label{fall-2018-1-done}}

Let \(f(x) = \frac 1 x\). Show that \(f\) is uniformly continuous on
\((1, \infty)\) but not on \((0,\infty)\).

Relevant concepts omitted.

Solution omitted.

\hypertarget{fall-2017-1-done}{%
\subsection{\texorpdfstring{Fall 2017 \# 1
\(\done\)}{Fall 2017 \# 1 \textbackslash done}}\label{fall-2017-1-done}}

Let
\begin{align*}
f(x) = \sum _{n=0}^{\infty} \frac{x^{n}}{n !}.
\end{align*}

Describe the intervals on which \(f\) does and does not converge
uniformly.

Relevant concepts omitted.

Solution omitted.

\hypertarget{fall-2014-1-done}{%
\subsection{\texorpdfstring{Fall 2014 \# 1
\(\done\)}{Fall 2014 \# 1 \textbackslash done}}\label{fall-2014-1-done}}

Let \(\left\{{f_n}\right\}\) be a sequence of continuous functions such
that \(\sum f_n\) converges uniformly.

Prove that \(\sum f_n\) is also continuous.

\envlist

Relevant concepts omitted.

Solution omitted.

\hypertarget{spring-2017-4-done}{%
\subsection{\texorpdfstring{Spring 2017 \# 4
\(\done\)}{Spring 2017 \# 4 \textbackslash done}}\label{spring-2017-4-done}}

Let \(f(x, y)\) on \([-1, 1]^2\) be defined by
\begin{align*}
f(x, y) = \begin{cases}
\frac{x y}{\left(x^{2}+y^{2}\right)^{2}} & (x, y) \neq (0, 0) \\
0 & (x, y) = (0, 0)
\end{cases}
\end{align*}
Determine if \(f\) is integrable.

Relevant concepts omitted.

Solution omitted.

\hypertarget{spring-2015-1-done}{%
\subsection{\texorpdfstring{Spring 2015 \# 1
\(\done\)}{Spring 2015 \# 1 \textbackslash done}}\label{spring-2015-1-done}}

Let \((X, d)\) and \((Y, \rho)\) be metric spaces, \(f: X\to Y\), and
\(x_0 \in X\).

Prove that the following statements are equivalent:

\begin{enumerate}
\def\labelenumi{\arabic{enumi}.}
\tightlist
\item
  For every \(\varepsilon > 0 \quad \exists \delta > 0\) such that
  \(\rho( f(x), f(x_0) ) < \varepsilon\) whenever
  \(d(x, x_0) < \delta\).
\item
  The sequence \(\left\{{f(x_n)}\right\}_{n=1}^\infty \to f(x_0)\) for
  every sequence \(\left\{{x_n}\right\} \to x_0\) in \(X\).
\end{enumerate}

Relevant concepts omitted.

Solution omitted.

\hypertarget{fall-2014-2-work}{%
\subsection{\texorpdfstring{Fall 2014 \# 2
\(\work\)}{Fall 2014 \# 2 \textbackslash work}}\label{fall-2014-2-work}}

Let \(I\) be an index set and \(\alpha: I \to (0, \infty)\).

\begin{enumerate}
\def\labelenumi{\alph{enumi}.}
\item
  Show that
  \begin{align*}
  \sum_{i \in I} a(i):=\sup _{\substack{ J \subset I \\ J \text { finite }}} \sum_{i \in J} a(i)<\infty \implies I \text{ is countable.}
  \end{align*}
\item
  Suppose \(I = {\mathbb{Q}}\) and
  \(\sum_{q \in \mathbb{Q}} a(q)<\infty\). Define
  \begin{align*}
  f(x):=\sum_{\substack{q \in \mathbb{Q}\\ q \leq x}} a(q).
  \end{align*}
  Show that \(f\) is continuous at \(x \iff x\not\in {\mathbb{Q}}\).
\end{enumerate}

\todo[inline]{Stuck on part b}

Solution omitted.

\hypertarget{spring-2014-2-done}{%
\subsection{\texorpdfstring{Spring 2014 \# 2
\(\done\)}{Spring 2014 \# 2 \textbackslash done}}\label{spring-2014-2-done}}

Let \(\left\{{a_n}\right\}\) be a sequence of real numbers such that
\begin{align*}
\left\{{b_n}\right\} \in \ell^2({\mathbb{N}}) \implies \sum a_n b_n < \infty.
\end{align*}
Show that \(\sum a_n^2 < \infty\).

\begin{quote}
Note: Assume \(a_n, b_n\) are all non-negative.
\end{quote}

\todo[inline]{Have someone check!}

Solution omitted.

\hypertarget{general-analysis}{%
\section{General Analysis}\label{general-analysis}}

\hypertarget{spring-2020-1-done}{%
\subsection{\texorpdfstring{Spring 2020 \# 1
\(\done\)}{Spring 2020 \# 1 \textbackslash done}}\label{spring-2020-1-done}}

Prove that if \(f: [0, 1] \to {\mathbb{R}}\) is continuous then
\begin{align*}
\lim_{k\to\infty} \int_0^1 kx^{k-1} f(x) \,dx = f(1)
.\end{align*}

Relevant concepts omitted.

Solution omitted.

\hypertarget{fall-2019-1-done}{%
\subsection{\texorpdfstring{Fall 2019 \# 1
\(\done\)}{Fall 2019 \# 1 \textbackslash done}}\label{fall-2019-1-done}}

Let \(\{a_n\}_{n=1}^\infty\) be a sequence of real numbers.

\begin{enumerate}
\def\labelenumi{\alph{enumi}.}
\item
  Prove that if \(\displaystyle\lim_{n\to \infty } a_n = 0\), then
  \begin{align*}
  \lim _{n \rightarrow \infty} \frac{a_{1}+\cdots+a_{n}}{n}=0
  \end{align*}
\item
  Prove that if \(\displaystyle\sum_{n=1}^{\infty} \frac{a_{n}}{n}\)
  converges, then
  \begin{align*}
  \lim _{n \rightarrow \infty} \frac{a_{1}+\cdots+a_{n}}{n}=0
  \end{align*}
\end{enumerate}

Solution omitted.

\hypertarget{fall-2018-4-done}{%
\subsection{\texorpdfstring{Fall 2018 \# 4
\(\done\)}{Fall 2018 \# 4 \textbackslash done}}\label{fall-2018-4-done}}

Let \(f\in L^1([0, 1])\). Prove that
\begin{align*}
\lim_{n \to \infty} \int_{0}^{1} f(x) {\left\lvert {\sin n x} \right\rvert} ~d x= \frac{2}{\pi} \int_{0}^{1} f(x) ~d x
\end{align*}

\begin{quote}
Hint: Begin with the case that \(f\) is the characteristic function of
an interval.
\end{quote}

\todo[inline]{Ask someone to check the last approximation part.}

Solution omitted.

\hypertarget{fall-2017-4-done}{%
\subsection{\texorpdfstring{Fall 2017 \# 4
\(\done\)}{Fall 2017 \# 4 \textbackslash done}}\label{fall-2017-4-done}}

Let
\begin{align*}
f_{n}(x) = n x(1-x)^{n}, \quad n \in {\mathbb{N}}.
\end{align*}

\begin{enumerate}
\def\labelenumi{\alph{enumi}.}
\item
  Show that \(f_n \to 0\) pointwise but not uniformly on \([0, 1]\).
\item
  Show that
  \begin{align*}
  \lim _{n \to \infty} \int _{0}^{1} n(1-x)^{n} \sin x \, dx = 0
  \end{align*}
\end{enumerate}

\begin{quote}
Hint for (a): Consider the maximum of \(f_n\).
\end{quote}

Solution omitted.

\hypertarget{spring-2017-3-work}{%
\subsection{\texorpdfstring{Spring 2017 \# 3
\(\work\)}{Spring 2017 \# 3 \textbackslash work}}\label{spring-2017-3-work}}

Let
\begin{align*}
f_{n}(x) = a e^{-n a x} - b e^{-n b x} \quad \text{ where } 0 < a < b.
\end{align*}

Show that

\begin{enumerate}
\def\labelenumi{\alph{enumi}.}
\tightlist
\item
  \(\sum_{n=1}^{\infty} \left|f_{n}\right|\) is not in
  \(L^{1}([0, \infty), m)\)
\end{enumerate}

\begin{quote}
Hint: \(f_n(x)\) has a root \(x_n\).
\end{quote}

\begin{enumerate}
\def\labelenumi{\alph{enumi}.}
\setcounter{enumi}{1}
\tightlist
\item

  \begin{align*}
  \sum_{n=1}^{\infty} f_{n} \text { is in } L^{1}([0, \infty), m) 
  {\quad \operatorname{and} \quad}
  \int _{0}^{\infty} \sum _{n=1}^{\infty} f_{n}(x) \,dm = \ln \frac{b}{a}
  \end{align*}
  \todo[inline]{Not complete.} \todo[inline]{Add concepts.}
  \todo[inline]{Walk through.}
\end{enumerate}

Solution omitted.

\hypertarget{fall-2016-1-done}{%
\subsection{\texorpdfstring{Fall 2016 \# 1
\(\done\)}{Fall 2016 \# 1 \textbackslash done}}\label{fall-2016-1-done}}

Define
\begin{align*}
f(x) = \sum_{n=1}^{\infty} \frac{1}{n^{x}}.
\end{align*}
Show that \(f\) converges to a differentiable function on
\((1, \infty)\) and that
\begin{align*}
f'(x)  =\sum_{n=1}^{\infty}\left(\frac{1}{n^{x}}\right)^{\prime}.
\end{align*}

\begin{quote}
Hint:
\begin{align*}
\left(\frac{1}{n^{x}}\right)' = -\frac{1}{n^{x}} \ln n
\end{align*}
\end{quote}

\todo[inline]{Add concepts.}

Solution omitted.

\hypertarget{fall-2016-5-done}{%
\subsection{\texorpdfstring{Fall 2016 \# 5
\(\done\)}{Fall 2016 \# 5 \textbackslash done}}\label{fall-2016-5-done}}

Let \(\phi\in L^\infty({\mathbb{R}})\). Show that the following limit
exists and satisfies the equality
\begin{align*}
\lim _{n \to \infty} \left(\int _{\mathbb{R}} \frac{|\phi(x)|^{n}}{1+x^{2}} \, dx \right) ^ {\frac{1}{n}} 
= {\left\lVert {\phi} \right\rVert}_\infty.
\end{align*}
\todo[inline]{Add concepts.}

Solution omitted.

\hypertarget{fall-2016-6-done}{%
\subsection{\texorpdfstring{Fall 2016 \# 6
\(\done\)}{Fall 2016 \# 6 \textbackslash done}}\label{fall-2016-6-done}}

Let \(f, g \in L^2({\mathbb{R}})\). Show that
\begin{align*}
\lim _{n \to \infty} \int _{{\mathbb{R}}} f(x) g(x+n) \,dx = 0
\end{align*}

\todo[inline]{Rewrite solution.}

Relevant concepts omitted.

Solution omitted.

\hypertarget{spring-2016-1-work}{%
\subsection{\texorpdfstring{Spring 2016 \# 1
\(\work\)}{Spring 2016 \# 1 \textbackslash work}}\label{spring-2016-1-work}}

For \(n\in {\mathbb{N}}\), define
\begin{align*}
e_{n} = \left (1+ {1\over n} \right)^{n} 
{\quad \operatorname{and} \quad}
E_{n} = \left( 1+ {1\over n} \right)^{n+1}
\end{align*}

Show that \(e_n < E_n\), and prove Bernoulli's inequality:
\begin{align*}
(1+x)^n \geq 1+nx && -1 < x < \infty  ,\,\, n\in {\mathbb{N}}
.\end{align*}

Use this to show the following:

\begin{enumerate}
\def\labelenumi{\arabic{enumi}.}
\tightlist
\item
  The sequence \(e_n\) is increasing.
\item
  The sequence \(E_n\) is decreasing.
\item
  \(2 < e_n < E_n < 4\).
\item
  \(\lim _{n \to \infty} e_{n} = \lim _{n \to \infty} E_{n}\).
\end{enumerate}

\hypertarget{fall-2015-1-work}{%
\subsection{\texorpdfstring{Fall 2015 \# 1
\(\work\)}{Fall 2015 \# 1 \textbackslash work}}\label{fall-2015-1-work}}

Define
\begin{align*}
f(x)=c_{0}+c_{1} x^{1}+c_{2} x^{2}+\ldots+c_{n} x^{n} \text { with } n \text { even and } c_{n}>0.
\end{align*}

Show that there is a number \(x_m\) such that \(f(x_m) \leq f(x)\) for
all \(x\in {\mathbb{R}}\).

\hypertarget{fall-2020-1}{%
\subsection{Fall 2020 \# 1}\label{fall-2020-1}}

Show that if \(x_n\) is a decreasing sequence of positive real numbers
such that \(\sum_{n=1}^\infty x_n\) converges, then
\begin{align*}
\lim_{n\to\infty} n x_n = 0.
\end{align*}

\hypertarget{fall-2020-3}{%
\subsection{Fall 2020 \# 3}\label{fall-2020-3}}

Let \(f\) be a non-negative Lebesgue measurable function on
\([1, \infty)\).

\begin{enumerate}
\def\labelenumi{\alph{enumi}.}
\item
  Prove that
  \begin{align*}  
  1 \leq \qty{
  {1 \over b-a} \int_a^b f(x) \,dx
  }\qty{
  {1\over b-a} \int_a^b {1 \over f(x)}\, dx
  }
  \end{align*}
  for any \(1\leq a < b <\infty\).
\item
  Prove that if \(f\) satisfies
  \begin{align*}  
  \int_1^t f(x) \, dx \leq t^2 \log(t)
  \end{align*}
  for all \(t\in [1, \infty)\), then
  \begin{align*}  
  \int_1^\infty {1\over f(x) \,dx} = \infty
  .\end{align*}
\end{enumerate}

\begin{quote}
Hint: write
\begin{align*}  
\int_1^\infty {1\over f(x) \, dx} = \sum_{k=0}^\infty \int_{2^k}^{2^{k+1}} {1 \over f(x)}\,dx
.\end{align*}
\end{quote}

\hypertarget{measure-theory-sets}{%
\section{Measure Theory: Sets}\label{measure-theory-sets}}

\hypertarget{spring-2020-2-done}{%
\subsection{\texorpdfstring{Spring 2020 \# 2
\(\done\)}{Spring 2020 \# 2 \textbackslash done}}\label{spring-2020-2-done}}

Let \(m_*\) denote the Lebesgue outer measure on \({\mathbb{R}}\).

\hypertarget{a.}{%
\subsubsection{a.}\label{a.}}

Prove that for every \(E\subseteq {\mathbb{R}}\) there exists a Borel
set \(B\) containing \(E\) such that
\begin{align*}
m_*(B) = m_*(E)
.\end{align*}

\hypertarget{b.}{%
\subsubsection{b.}\label{b.}}

Prove that if \(E\subseteq {\mathbb{R}}\) has the property that
\begin{align*}
m_*(A) = m_*(A\cap E) + m_*(A\cap E^c)
\end{align*}
for every set \(A\subseteq {\mathbb{R}}\), then there exists a Borel set
\(B\subseteq {\mathbb{R}}\) such that \(E = B\setminus N\) with
\(m_*(N) = 0\).

Be sure to address the case when \(m_*(E) = \infty\).

Solution omitted.

\hypertarget{fall-2019-3.-done}{%
\subsection{\texorpdfstring{Fall 2019 \# 3.
\(\done\)}{Fall 2019 \# 3. \textbackslash done}}\label{fall-2019-3.-done}}

Let \((X, \mathcal B, \mu)\) be a measure space with \(\mu(X) = 1\) and
\(\{B_n\}_{n=1}^\infty\) be a sequence of \(\mathcal B\)-measurable
subsets of \(X\), and
\begin{align*}
B \coloneqq\left\{{x\in X {~\mathrel{\Big|}~}x\in B_n \text{ for infinitely many } n}\right\}.
\end{align*}

\begin{enumerate}
\def\labelenumi{\alph{enumi}.}
\item
  Argue that \(B\) is also a \(\mathcal{B} {\hbox{-}}\)measurable subset
  of \(X\).
\item
  Prove that if \(\sum_{n=1}^\infty \mu(B_n) < \infty\) then
  \(\mu(B)= 0\).
\item
  Prove that if \(\sum_{n=1}^\infty \mu(B_n) = \infty\) \textbf{and} the
  sequence of set complements \(\left\{{B_n^c}\right\}_{n=1}^\infty\)
  satisfies
  \begin{align*}
  \mu\left(\bigcap_{n=k}^{K} B_{n}^{c}\right)=\prod_{n=k}^{K}\left(1-\mu\left(B_{n}\right)\right)
  \end{align*}
  for all positive integers \(k\) and \(K\) with \(k < K\), then
  \(\mu(B) = 1\).
\end{enumerate}

\begin{quote}
Hint: Use the fact that \(1 - x ≤ e^{-x}\) for all \(x\).
\end{quote}

Solution omitted.

\hypertarget{spring-2019-2-done}{%
\subsection{\texorpdfstring{Spring 2019 \# 2
\(\done\)}{Spring 2019 \# 2 \textbackslash done}}\label{spring-2019-2-done}}

Let \(\mathcal B\) denote the set of all Borel subsets of
\({\mathbb{R}}\) and \(\mu : \mathcal B \to [0, \infty)\) denote a
finite Borel measure on \({\mathbb{R}}\).

\hypertarget{a-2}{%
\subsubsection{a}\label{a-2}}

Prove that if \(\{F_k\}\) is a sequence of Borel sets for which
\(F_k \supseteq F_{k+1}\) for all \(k\), then
\begin{align*}
\lim _{k \rightarrow \infty} \mu\left(F_{k}\right)=\mu\left(\bigcap_{k=1}^{\infty} F_{k}\right)
\end{align*}

\hypertarget{b-2}{%
\subsubsection{b}\label{b-2}}

Suppose \(\mu\) has the property that \(\mu (E) = 0\) for every
\(E \in \mathcal B\) with Lebesgue measure \(m(E) = 0\).

Prove that for every \(\epsilon > 0\) there exists \(\delta > 0\) so
that if \(E \in \mathcal B\) with \(m(E) < δ\), then \(\mu(E) < ε\).

\todo[inline]{Add concepts.}

Solution omitted.

\todo[inline]{All messed up!}

\hypertarget{fall-2018-2-done}{%
\subsection{\texorpdfstring{Fall 2018 \# 2
\(\done\)}{Fall 2018 \# 2 \textbackslash done}}\label{fall-2018-2-done}}

Let \(E\subset {\mathbb{R}}\) be a Lebesgue measurable set. Show that
there is a Borel set \(B \subset E\) such that \(m(E\setminus B) = 0\).

\todo[inline]{Move this to review notes to clean things up.}

Solution omitted.

\hypertarget{spring-2018-1-done}{%
\subsection{\texorpdfstring{Spring 2018 \# 1
\(\done\)}{Spring 2018 \# 1 \textbackslash done}}\label{spring-2018-1-done}}

Define
\begin{align*}
E:=\left\{x \in \mathbb{R}:\left|x-\frac{p}{q}\right|<q^{-3} \text { for infinitely many } p, q \in \mathbb{N}\right\}.
\end{align*}

Prove that \(m(E) = 0\).

Solution omitted.

\hypertarget{fall-2017-2-done}{%
\subsection{\texorpdfstring{Fall 2017 \# 2
\(\done\)}{Fall 2017 \# 2 \textbackslash done}}\label{fall-2017-2-done}}

Let \(f(x) = x^2\) and
\(E \subset [0, \infty) \coloneqq{\mathbb{R}}^+\).

\begin{enumerate}
\def\labelenumi{\arabic{enumi}.}
\item
  Show that
  \begin{align*}
  m^*(E) = 0 \iff m^*(f(E)) = 0.
  \end{align*}
\item
  Deduce that the map
\end{enumerate}

\begin{align*}
\phi: \mathcal{L}({\mathbb{R}}^+) &\to \mathcal{L}({\mathbb{R}}^+) \\
E &\mapsto f(E)
\end{align*}
is a bijection from the class of Lebesgue measurable sets of
\([0, \infty)\) to itself.

\todo[inline]{Walk through.}

Solution omitted.

\hypertarget{spring-2017-2-done}{%
\subsection{\texorpdfstring{Spring 2017 \# 2
\(\done\)}{Spring 2017 \# 2 \textbackslash done}}\label{spring-2017-2-done}}

\hypertarget{a-5}{%
\subsubsection{a}\label{a-5}}

Let \(\mu\) be a measure on a measurable space \((X, \mathcal M)\) and
\(f\) a positive measurable function.

Define a measure \(\lambda\) by
\begin{align*}
\lambda(E):=\int_{E} f ~d \mu, \quad E \in \mathcal{M}
\end{align*}

Show that for \(g\) any positive measurable function,
\begin{align*}
\int_{X} g ~d \lambda=\int_{X} f g ~d \mu
\end{align*}

\hypertarget{b-5}{%
\subsubsection{b}\label{b-5}}

Let \(E \subset {\mathbb{R}}\) be a measurable set such that
\begin{align*}
\int_{E} x^{2} ~d m=0.
\end{align*}
Show that \(m(E) = 0\).

Solution omitted.

\hypertarget{fall-2016-4-done}{%
\subsection{\texorpdfstring{Fall 2016 \# 4
\(\done\)}{Fall 2016 \# 4 \textbackslash done}}\label{fall-2016-4-done}}

Let \((X, \mathcal M, \mu)\) be a measure space and suppose
\(\left\{{E_n}\right\} \subset \mathcal M\) satisfies
\begin{align*}
\lim _{n \rightarrow \infty} \mu\left(X \backslash E_{n}\right)=0.
\end{align*}

Define
\begin{align*}
G \coloneqq\left\{{x\in X {~\mathrel{\Big|}~}x\in E_n \text{ for only finitely many  } n}\right\}.
\end{align*}

Show that \(G \in \mathcal M\) and \(\mu(G) = 0\).

\todo[inline]{Add concepts.}

Solution omitted.

\hypertarget{spring-2016-3-work}{%
\subsection{\texorpdfstring{Spring 2016 \# 3
\(\work\)}{Spring 2016 \# 3 \textbackslash work}}\label{spring-2016-3-work}}

Let \(f\) be Lebesgue measurable on \({\mathbb{R}}\) and
\(E \subset {\mathbb{R}}\) be measurable such that
\begin{align*}
0<A=\int_{E} f(x) d x<\infty.
\end{align*}

Show that for every \(0 < t < 1\), there exists a measurable set
\(E_t \subset E\) such that
\begin{align*}
\int_{E_{t}} f(x) d x=t A.
\end{align*}

\hypertarget{spring-2016-5-work}{%
\subsection{\texorpdfstring{Spring 2016 \# 5
\(\work\)}{Spring 2016 \# 5 \textbackslash work}}\label{spring-2016-5-work}}

Let \((X, \mathcal M, \mu)\) be a measure space. For \(f\in L^1(\mu)\)
and \(\lambda > 0\), define
\begin{align*}
\phi(\lambda)=\mu(\{x \in X | f(x)>\lambda\}) 
\quad \text { and } \quad 
\psi(\lambda)=\mu(\{x \in X | f(x)<-\lambda\})
\end{align*}

Show that \(\phi, \psi\) are Borel measurable and
\begin{align*}
\int_{X}|f| ~d \mu=\int_{0}^{\infty}[\phi(\lambda)+\psi(\lambda)] ~d \lambda
\end{align*}

\hypertarget{fall-2015-2-work}{%
\subsection{\texorpdfstring{Fall 2015 \# 2
\(\work\)}{Fall 2015 \# 2 \textbackslash work}}\label{fall-2015-2-work}}

Let \(f: {\mathbb{R}}\to {\mathbb{R}}\) be Lebesgue measurable.

\begin{enumerate}
\def\labelenumi{\arabic{enumi}.}
\tightlist
\item
  Show that there is a sequence of simple functions \(s_n(x)\) such that
  \(s_n(x) \to f(x)\) for all \(x\in {\mathbb{R}}\).
\item
  Show that there is a Borel measurable function \(g\) such that
  \(g = f\) almost everywhere.
\end{enumerate}

\hypertarget{spring-2015-3-work}{%
\subsection{\texorpdfstring{Spring 2015 \# 3
\(\work\)}{Spring 2015 \# 3 \textbackslash work}}\label{spring-2015-3-work}}

Let \(\mu\) be a finite Borel measure on \({\mathbb{R}}\) and
\(E \subset {\mathbb{R}}\) Borel. Prove that the following statements
are equivalent:

\begin{enumerate}
\def\labelenumi{\arabic{enumi}.}
\tightlist
\item
  \(\forall \varepsilon > 0\) there exists \(G\) open and \(F\) closed
  such that
  \begin{align*}
  F \subseteq E \subseteq G \quad \text{and} \quad \mu(G\setminus F) < \varepsilon.
  \end{align*}
\item
  There exists a \(V \in G_\delta\) and \(H \in F_\sigma\) such that
  \begin{align*}
  H \subseteq E \subseteq V \quad \text{and}\quad \mu(V\setminus H) = 0
  \end{align*}
\end{enumerate}

\hypertarget{spring-2014-3-work}{%
\subsection{\texorpdfstring{Spring 2014 \# 3
\(\work\)}{Spring 2014 \# 3 \textbackslash work}}\label{spring-2014-3-work}}

Let \(f: {\mathbb{R}}\to {\mathbb{R}}\) and suppose
\begin{align*}
\forall x\in {\mathbb{R}},\quad f(x) \geq \limsup _{y \rightarrow x} f(y)
\end{align*}
Prove that \(f\) is Borel measurable.

\hypertarget{spring-2014-4-work}{%
\subsection{\texorpdfstring{Spring 2014 \# 4
\(\work\)}{Spring 2014 \# 4 \textbackslash work}}\label{spring-2014-4-work}}

Let \((X, \mathcal M, \mu)\) be a measure space and suppose \(f\) is a
measurable function on \(X\). Show that
\begin{align*}
\lim _{n \rightarrow \infty} \int_{X} f^{n} ~d \mu =
\begin{cases}
\infty & \text{or} \\
\mu(f^{-1}(1)),
\end{cases}
\end{align*}
and characterize the collection of functions of each type.

\hypertarget{spring-2017-1-done}{%
\subsection{\texorpdfstring{Spring 2017 \# 1
\(\done\)}{Spring 2017 \# 1 \textbackslash done}}\label{spring-2017-1-done}}

Let \(K\) be the set of numbers in \([0, 1]\) whose decimal expansions
do not use the digit \(4\).

\begin{quote}
We use the convention that when a decimal number ends with 4 but all
other digits are different from 4, we replace the digit \(4\) with
\(399\cdots\). For example, \(0.8754 = 0.8753999\cdots\).
\end{quote}

Show that \(K\) is a compact, nowhere dense set without isolated points,
and find the Lebesgue measure \(m(K)\).

Solution omitted.

\hypertarget{spring-2016-2-work}{%
\subsection{\texorpdfstring{Spring 2016 \# 2
\(\work\)}{Spring 2016 \# 2 \textbackslash work}}\label{spring-2016-2-work}}

Let \(0 < \lambda < 1\) and construct a Cantor set \(C_\lambda\) by
successively removing middle intervals of length \(\lambda\).

Prove that \(m(C_\lambda) = 0\).

\hypertarget{measure-theory-functions}{%
\section{Measure Theory: Functions}\label{measure-theory-functions}}

\hypertarget{fall-2016-2-done}{%
\subsection{\texorpdfstring{Fall 2016 \# 2
\(\done\)}{Fall 2016 \# 2 \textbackslash done}}\label{fall-2016-2-done}}

Let \(f, g: [a, b] \to {\mathbb{R}}\) be measurable with
\begin{align*}
\int_{a}^{b} f(x) ~d x=\int_{a}^{b} g(x) ~d x.
\end{align*}

Show that either

\begin{enumerate}
\def\labelenumi{\arabic{enumi}.}
\tightlist
\item
  \(f(x) = g(x)\) almost everywhere, or
\item
  There exists a measurable set \(E \subset [a, b]\) such that
  \begin{align*}]
  \int _{E} f(x) \, dx > \int _{E} g(x) \, dx
  \end{align*}
\end{enumerate}

\todo[inline]{Add concepts.}

Solution omitted.

\hypertarget{spring-2016-4-work}{%
\subsection{\texorpdfstring{Spring 2016 \# 4
\(\work\)}{Spring 2016 \# 4 \textbackslash work}}\label{spring-2016-4-work}}

Let \(E \subset {\mathbb{R}}\) be measurable with \(m(E) < \infty\).
Define
\begin{align*}
f(x)=m(E \cap(E+x)).
\end{align*}

Show that

\begin{enumerate}
\def\labelenumi{\arabic{enumi}.}
\tightlist
\item
  \(f\in L^1({\mathbb{R}})\).
\item
  \(f\) is uniformly continuous.
\item
  \(\lim _{|x| \to \infty} f(x) = 0\).
\end{enumerate}

\begin{quote}
Hint:
\begin{align*}
\chi_{E \cap(E+x)}(y)=\chi_{E}(y) \chi_{E}(y-x)
\end{align*}
\end{quote}

\hypertarget{spring-2021-1}{%
\subsection{Spring 2021 \# 1}\label{spring-2021-1}}

Let \((X, \mathcal{M},\mu)\) be a measure space and let
\(E_n \in \mathcal{M}\) be a measurable set for \(n\geq 1\). Let
\(f_n \coloneqq\chi_{E_n}\) be the indicator function of the set \(E\)
and show that

\begin{enumerate}
\def\labelenumi{\alph{enumi}.}
\item
  \(f_n \overset{n\to\infty}\to 1\) uniformly \(\iff\) there exists
  \(N\in |NN\) such that \(E_n = X\) for all \(n\geq N\).
\item
  \(f_n(x) \overset{n\to\infty}\to 1\) for almost every \(x\) \(\iff\)
  \begin{align*}
  \mu \qty{ \bigcap_{n \geq 0} \bigcup_{k \geq n} (X \setminus E_k) } = 0
  .\end{align*}
\end{enumerate}

\hypertarget{spring-2021-3}{%
\subsection{Spring 2021 \# 3}\label{spring-2021-3}}

Let \((X, \mathcal{M}, \mu)\) be a finite measure space and let
\(\left\{{ f_n}\right\}_{n=1}^{\infty } \subseteq L^1(X, \mu)\). Suppose
\(f\in L^1(X, \mu)\) such that \(f_n(x) \overset{n\to \infty }\to f(x)\)
for almost every \(x \in X\). Prove that for every \(\varepsilon> 0\)
there exists \(M>0\) and a set \(E\subseteq X\) such that
\(\mu(E) \leq \varepsilon\) and
\({\left\lvert {f_n(x)} \right\rvert}\leq M\) for all
\(x\in X\setminus E\) and all \(n\in {\mathbb{N}}\).

\hypertarget{fall-2020-2}{%
\subsection{Fall 2020 \# 2}\label{fall-2020-2}}

\begin{enumerate}
\def\labelenumi{\alph{enumi}.}
\item
  Let \(f: {\mathbb{R}}\to {\mathbb{R}}\). Prove that
  \begin{align*}
  f(x) \leq \liminf_{y\to x} f(y)~ \text{for each}~ x\in {{\mathbb{R}}} \iff \{ x\in {{\mathbb{R}}} \mathrel{\Big|}f(x) > a \}~\text{is open for all}~ a\in {{\mathbb{R}}}
  \end{align*}
\item
  Recall that a function \(f: {{\mathbb{R}}} \to {{\mathbb{R}}}\) is
  called \emph{lower semi-continuous} iff it satisfies either condition
  in part (a) above.
\end{enumerate}

Prove that if \(\mathcal{F}\) is an y family of lower semi-continuous
functions, then
\begin{align*}
g(x) = \sup\{ f(x) \mathrel{\Big|}f\in \mathcal{F}\}
\end{align*}
is Borel measurable.

\begin{quote}
Note that \(\mathcal{F}\) need not be a countable family.
\end{quote}

\hypertarget{integrals-convergence}{%
\section{Integrals: Convergence}\label{integrals-convergence}}

\hypertarget{fall-2019-2-done}{%
\subsection{\texorpdfstring{Fall 2019 \# 2
\(\done\)}{Fall 2019 \# 2 \textbackslash done}}\label{fall-2019-2-done}}

Prove that
\begin{align*}
\left| \frac{d^{n}}{d x^{n}} \frac{\sin x}{x}\right| \leq \frac{1}{n}
\end{align*}

for all \(x \neq 0\) and positive integers \(n\).

\begin{quote}
Hint: Consider \(\displaystyle\int_0^1 \cos(tx) dt\)
\end{quote}

Solution omitted.

\hypertarget{spring-2020-5-done}{%
\subsection{\texorpdfstring{Spring 2020 \# 5
\(\done\)}{Spring 2020 \# 5 \textbackslash done}}\label{spring-2020-5-done}}

Compute the following limit and justify your calculations:
\begin{align*}
\lim_{n\to\infty} \int_0^n \qty{1 + {x^2 \over n}}^{-(n+1)} \,dx
.\end{align*}

\todo[inline]{Not finished, flesh out.}
\todo[inline]{Walk through.}

Solution omitted.

\hypertarget{spring-2019-3-done}{%
\subsection{\texorpdfstring{Spring 2019 \# 3
\(\done\)}{Spring 2019 \# 3 \textbackslash done}}\label{spring-2019-3-done}}

Let \(\{f_k\}\) be any sequence of functions in \(L^2([0, 1])\)
satisfying \({\left\lVert {f_k} \right\rVert}_2 ≤ M\) for all
\(k ∈ {\mathbb{N}}\).

Prove that if \(f_k \to f\) almost everywhere, then \(f ∈ L^2([0, 1])\)
with \({\left\lVert {f} \right\rVert}_2 ≤ M\) and
\begin{align*}
\lim _{k \rightarrow \infty} \int_{0}^{1} f_{k}(x) dx = \int_{0}^{1} f(x) d x
\end{align*}

\begin{quote}
Hint: Try using Fatou's Lemma to show that
\({\left\lVert {f} \right\rVert}_2 ≤ M\) and then try applying Egorov's
Theorem.
\end{quote}

Solution omitted.

\hypertarget{fall-2018-6-done}{%
\subsection{\texorpdfstring{Fall 2018 \# 6
\(\done\)}{Fall 2018 \# 6 \textbackslash done}}\label{fall-2018-6-done}}

Compute the following limit and justify your calculations:
\begin{align*}
\lim_{n \rightarrow \infty} \int_{1}^{n} \frac{d x}{\left(1+\frac{x}{n}\right)^{n} \sqrt[n]{x}}
\end{align*}

\todo[inline]{Add concepts.}

Solution omitted.

\hypertarget{fall-2018-3-done}{%
\subsection{\texorpdfstring{Fall 2018 \# 3
\(\done\)}{Fall 2018 \# 3 \textbackslash done}}\label{fall-2018-3-done}}

Suppose \(f(x)\) and \(xf(x)\) are integrable on \({\mathbb{R}}\).
Define \(F\) by
\begin{align*}
F(t)\coloneqq\int _{-\infty}^{\infty} f(x) \cos (x t) dx
\end{align*}
Show that
\begin{align*}
F'(t)=-\int _{-\infty}^{\infty} x f(x) \sin (x t) dx
.\end{align*}

\todo[inline]{Walk through.}

Solution omitted.

\hypertarget{spring-2018-5-done}{%
\subsection{\texorpdfstring{Spring 2018 \# 5
\(\done\)}{Spring 2018 \# 5 \textbackslash done}}\label{spring-2018-5-done}}

Suppose that

\begin{itemize}
\tightlist
\item
  \(f_n, f \in L^1\),
\item
  \(f_n \to f\) almost everywhere, and
\item
  \(\int\left|f_{n}\right| \rightarrow \int|f|\).
\end{itemize}

Show that \(\int f_{n} \rightarrow \int f\).

Solution omitted.

\hypertarget{spring-2018-2-done}{%
\subsection{\texorpdfstring{Spring 2018 \# 2
\(\done\)}{Spring 2018 \# 2 \textbackslash done}}\label{spring-2018-2-done}}

Let
\begin{align*}
f_{n}(x):=\frac{x}{1+x^{n}}, \quad x \geq 0.
\end{align*}

\begin{enumerate}
\def\labelenumi{\alph{enumi}.}
\item
  Show that this sequence converges pointwise and find its limit. Is the
  convergence uniform on \([0, \infty)\)?
\item
  Compute
  \begin{align*}
  \lim _{n \rightarrow \infty} \int_{0}^{\infty} f_{n}(x) d x
  \end{align*}
\end{enumerate}

\todo[inline]{Add concepts.}

Solution omitted.

\hypertarget{fall-2016-3-done}{%
\subsection{\texorpdfstring{Fall 2016 \# 3
\(\done\)}{Fall 2016 \# 3 \textbackslash done}}\label{fall-2016-3-done}}

Let \(f\in L^1({\mathbb{R}})\). Show that
\begin{align*}
\lim _{x \to 0} \int _{{\mathbb{R}}} {\left\lvert {f(y-x)-f(y)} \right\rvert} \, dy = 0
\end{align*}
\todo[inline]{Missing some stuff.}

Solution omitted.

\hypertarget{fall-2015-3-work}{%
\subsection{\texorpdfstring{Fall 2015 \# 3
\(\work\)}{Fall 2015 \# 3 \textbackslash work}}\label{fall-2015-3-work}}

Compute the following limit:
\begin{align*}
\lim _{n \rightarrow \infty} \int_{1}^{n} \frac{n e^{-x}}{1+n x^{2}} \, \sin \left(\frac x n\right) \, dx
\end{align*}

\hypertarget{fall-2015-4-work}{%
\subsection{\texorpdfstring{Fall 2015 \# 4
\(\work\)}{Fall 2015 \# 4 \textbackslash work}}\label{fall-2015-4-work}}

Let \(f: [1, \infty) \to {\mathbb{R}}\) such that \(f(1) = 1\) and
\begin{align*}
f^{\prime}(x)= \frac{1} {x^{2}+f(x)^{2}}
\end{align*}

Show that the following limit exists and satisfies the equality
\begin{align*}
\lim _{x \rightarrow \infty} f(x) \leq 1 + \frac \pi 4
\end{align*}

\hypertarget{spring-2021-2}{%
\subsection{Spring 2021 \# 2}\label{spring-2021-2}}

Calculate the following limit, justifying each step of your calculation:

\begin{align*}
L \coloneqq\lim_{n\to \infty} \int_0^n { \cos\qty{x\over n} \over x^2 + \cos\qty{x\over n} }\,dx
.\end{align*}

\hypertarget{spring-2021-5}{%
\subsection{Spring 2021 \# 5}\label{spring-2021-5}}

Let \(f_n \in L^2([0, 1])\) for \(n\in {\mathbb{N}}\), and assume that

\begin{itemize}
\item
  \({\left\lVert {f_n} \right\rVert}_2 \leq n^{-51 \over 100}\) for all
  \(n\in {\mathbb{N}}\),
\item
  \(hat{f}_n\) is supported in the interval \([2^n, 2^{n+1}]\), so
  \begin{align*}
  \widehat{f}_n(\xi) \coloneqq\int_0^1 f_n(x) e^{2\pi i \xi \cdot x} \,dx= 0 && \text{for } \xi \not\in [2^n, 2^{n+1}]
  .\end{align*}
\end{itemize}

Prove that \(\sum_{n\in {\mathbb{N}}} f_n\) converges in the Hilbert
space \(L^2([0, 1])\).

\begin{quote}
Hint: Plancherel's identity may be helpful.
\end{quote}

\hypertarget{integrals-approximation}{%
\section{Integrals: Approximation}\label{integrals-approximation}}

\hypertarget{spring-2018-3-done}{%
\subsection{\texorpdfstring{Spring 2018 \# 3
\(\done\)}{Spring 2018 \# 3 \textbackslash done}}\label{spring-2018-3-done}}

Let \(f\) be a non-negative measurable function on \([0, 1]\).

Show that
\begin{align*}
\lim _{p \rightarrow \infty}\left(\int_{[0,1]} f(x)^{p} d x\right)^{\frac{1}{p}}=\|f\|_{\infty}.
\end{align*}

Solution omitted.

\hypertarget{spring-2018-4-done}{%
\subsection{\texorpdfstring{Spring 2018 \# 4
\(\done\)}{Spring 2018 \# 4 \textbackslash done}}\label{spring-2018-4-done}}

Let \(f\in L^2([0, 1])\) and suppose
\begin{align*}
\int _{[0,1]} f(x) x^{n} d x=0 \text { for all integers } n \geq 0.
\end{align*}
Show that \(f = 0\) almost everywhere.

Solution omitted.

\hypertarget{spring-2015-2-work}{%
\subsection{\texorpdfstring{Spring 2015 \# 2
\(\work\)}{Spring 2015 \# 2 \textbackslash work}}\label{spring-2015-2-work}}

Let \(f: {\mathbb{R}}\to {\mathbb{C}}\) be continuous with period 1.
Prove that
\begin{align*}
\lim _{N \rightarrow \infty} \frac{1}{N} \sum_{n=1}^{N} f(n \alpha)=\int_{0}^{1} f(t) d t \quad \forall \alpha \in {\mathbb{R}}\setminus{\mathbb{Q}}.
\end{align*}

\begin{quote}
Hint: show this first for the functions \(f(t) = e^{2\pi i k t}\) for
\(k\in {\mathbb{Z}}\).
\end{quote}

\hypertarget{fall-2014-4-work}{%
\subsection{\texorpdfstring{Fall 2014 \# 4
\(\work\)}{Fall 2014 \# 4 \textbackslash work}}\label{fall-2014-4-work}}

Let \(g\in L^\infty([0, 1])\) Prove that
\begin{align*}
\int _{[0,1]} f(x) g(x)\, dx = 0 
\quad\text{for all continuous } f:[0, 1] \to {\mathbb{R}}
\implies g(x) = 0 \text{ almost everywhere. }
\end{align*}

\hypertarget{l1}{%
\section{\texorpdfstring{\(L^1\)}{L\^{}1}}\label{l1}}

\hypertarget{spring-2020-3-done}{%
\subsection{\texorpdfstring{Spring 2020 \# 3
\(\done\)}{Spring 2020 \# 3 \textbackslash done}}\label{spring-2020-3-done}}

\begin{enumerate}
\def\labelenumi{\alph{enumi}.}
\item
  Prove that if \(g\in L^1({\mathbb{R}})\) then
  \begin{align*}
  \lim_{N\to \infty} \int _{{\left\lvert {x} \right\rvert} \geq N} {\left\lvert {f(x)} \right\rvert} \, dx = 0
  ,\end{align*}
  and demonstrate that it is not necessarily the case that
  \(f(x) \to 0\) as \({\left\lvert {x} \right\rvert}\to \infty\).
\item
  Prove that if \(f\in L^1([1, \infty])\) and is decreasing, then
  \(\lim_{x\to\infty}f(x) =0\) and in fact
  \(\lim_{x\to \infty} xf(x) = 0\).
\item
  If \(f: [1, \infty) \to [0, \infty)\) is decreasing with
  \(\lim_{x\to \infty} xf(x) = 0\), does this ensure that
  \(f\in L^1([1, \infty))\)?
\end{enumerate}

Solution omitted.

\hypertarget{fall-2019-5.-done}{%
\subsection{\texorpdfstring{Fall 2019 \# 5.
\(\done\)}{Fall 2019 \# 5. \textbackslash done}}\label{fall-2019-5.-done}}

\hypertarget{a-9}{%
\subsubsection{a}\label{a-9}}

Show that if \(f\) is continuous with compact support on
\({\mathbb{R}}\), then
\begin{align*}
\lim _{y \rightarrow 0} \int_{\mathbb{R}}|f(x-y)-f(x)| d x=0
\end{align*}

\hypertarget{b-9}{%
\subsubsection{b}\label{b-9}}

Let \(f\in L^1({\mathbb{R}})\) and for each \(h > 0\) let
\begin{align*}
\mathcal{A}_{h} f(x):=\frac{1}{2 h} \int_{|y| \leq h} f(x-y) d y
\end{align*}

\begin{enumerate}
\def\labelenumi{\roman{enumi}.}
\item
  Prove that \(\left\|\mathcal{A}_{h} f\right\|_{1} \leq\|f\|_{1}\) for
  all \(h > 0\).
\item
  Prove that \(\mathcal{A}_h f \to f\) in \(L^1({\mathbb{R}})\) as
  \(h \to 0^+\).
\end{enumerate}

\todo[inline]{Walk through.}

Solution omitted.

\hypertarget{fall-2017-3-done}{%
\subsection{\texorpdfstring{Fall 2017 \# 3
\(\done\)}{Fall 2017 \# 3 \textbackslash done}}\label{fall-2017-3-done}}

Let
\begin{align*}
S = {\operatorname{span}}_{\mathbb{C}}\left\{{\chi_{(a, b)} {~\mathrel{\Big|}~}a, b \in {\mathbb{R}}}\right\},
\end{align*}
the complex linear span of characteristic functions of intervals of the
form \((a, b)\).

Show that for every \(f\in L^1({\mathbb{R}})\), there exists a sequence
of functions \(\left\{{f_n}\right\} \subset S\) such that
\begin{align*}
\lim _{n \rightarrow \infty}\left\|f_{n}-f\right\|_{1}=0
\end{align*}

\todo[inline]{Walk through.}

Solution omitted.

\hypertarget{spring-2015-4-work}{%
\subsection{\texorpdfstring{Spring 2015 \# 4
\(\work\)}{Spring 2015 \# 4 \textbackslash work}}\label{spring-2015-4-work}}

Define
\begin{align*}
f(x, y):=\left\{\begin{array}{ll}{\frac{x^{1 / 3}}{(1+x y)^{3 / 2}}} & {\text { if } 0 \leq x \leq y} \\ {0} & {\text { otherwise }}\end{array}\right.
\end{align*}

Carefully show that \(f \in L^1({\mathbb{R}}^2)\).

\hypertarget{fall-2014-3-work}{%
\subsection{\texorpdfstring{Fall 2014 \# 3
\(\work\)}{Fall 2014 \# 3 \textbackslash work}}\label{fall-2014-3-work}}

Let \(f\in L^1({\mathbb{R}})\). Show that
\begin{align*}
\forall\varepsilon > 0 \exists \delta > 0 \text{ such that } \qquad 
m(E) < \delta 
\implies 
\int _{E} |f(x)| \, dx < \varepsilon
\end{align*}

\hypertarget{spring-2014-1-work}{%
\subsection{\texorpdfstring{Spring 2014 \# 1
\(\work\)}{Spring 2014 \# 1 \textbackslash work}}\label{spring-2014-1-work}}

\begin{enumerate}
\def\labelenumi{\arabic{enumi}.}
\item
  Give an example of a continuous \(f\in L^1({\mathbb{R}})\) such that
  \(f(x) \not\to 0\) as\({\left\lvert {x} \right\rvert} \to \infty\).
\item
  Show that if \(f\) is \emph{uniformly} continuous, then
  \begin{align*}
  \lim_{{\left\lvert {x} \right\rvert} \to \infty} f(x) = 0.
  \end{align*}
\end{enumerate}

\hypertarget{spring-2021-4}{%
\subsection{Spring 2021 \# 4}\label{spring-2021-4}}

Let \(f, g\) be Lebesgue integrable on \({\mathbb{R}}\) and let
\(g_n(x) \coloneqq g(x- n)\). Prove that
\begin{align*}
\lim_{n\to \infty } {\left\lVert {f + g_n} \right\rVert}_1 = {\left\lVert {f} \right\rVert}_1 + {\left\lVert {g} \right\rVert}_1
.\end{align*}

Relevant concepts omitted.

Solution omitted.

\hypertarget{fall-2020-4}{%
\subsection{Fall 2020 \# 4}\label{fall-2020-4}}

Prove that if \(xf(x) \in L^1({\mathbb{R}})\), then
\begin{align*}  
F(y) \coloneqq\int f(x) \cos(yx)\,  dx
\end{align*}
defines a \(C^1\) function.

\hypertarget{fubini-tonelli}{%
\section{Fubini-Tonelli}\label{fubini-tonelli}}

\hypertarget{spring-2020-4-done}{%
\subsection{\texorpdfstring{Spring 2020 \# 4
\(\done\)}{Spring 2020 \# 4 \textbackslash done}}\label{spring-2020-4-done}}

Let \(f, g\in L^1({\mathbb{R}})\). Argue that
\(H(x, y) \coloneqq f(y) g(x-y)\) defines a function in
\(L^1({\mathbb{R}}^2)\) and deduce from this fact that
\begin{align*}
(f\ast g)(x) \coloneqq\int_{\mathbb{R}}f(y) g(x-y) \,dy
\end{align*}
defines a function in \(L^1({\mathbb{R}})\) that satisfies
\begin{align*}
{\left\lVert {f\ast g} \right\rVert}_1 \leq {\left\lVert {f} \right\rVert}_1 {\left\lVert {g} \right\rVert}_1
.\end{align*}

Hint/strategy omitted.

Relevant concepts omitted.

Solution omitted.

\hypertarget{spring-2019-4-done}{%
\subsection{\texorpdfstring{Spring 2019 \# 4
\(\done\)}{Spring 2019 \# 4 \textbackslash done}}\label{spring-2019-4-done}}

Let \(f\) be a non-negative function on \({\mathbb{R}}^n\) and
\(\mathcal A = \{(x, t) ∈ {\mathbb{R}}^n \times {\mathbb{R}}: 0 ≤ t ≤ f (x)\}\).

Prove the validity of the following two statements:

\begin{enumerate}
\def\labelenumi{\alph{enumi}.}
\item
  \(f\) is a Lebesgue measurable function on
  \({\mathbb{R}}^n \iff \mathcal A\) is a Lebesgue measurable subset of
  \({\mathbb{R}}^{n+1}\)
\item
  If \(f\) is a Lebesgue measurable function on \({\mathbb{R}}^n\), then
  \begin{align*}
  m(\mathcal{A})=\int _{{\mathbb{R}}^{n}} f(x) d x=\int_{0}^{\infty} m\left(\left\{x \in {\mathbb{R}}^{n}: f(x) \geq t\right\}\right) dt
  \end{align*}
\end{enumerate}

\todo[inline]{Add concepts.}

Relevant concepts omitted.

Solution omitted.

\hypertarget{fall-2018-5-done}{%
\subsection{\texorpdfstring{Fall 2018 \# 5
\(\done\)}{Fall 2018 \# 5 \textbackslash done}}\label{fall-2018-5-done}}

Let \(f \geq 0\) be a measurable function on \({\mathbb{R}}\). Show that
\begin{align*}
\int _{{\mathbb{R}}} f = \int _{0}^{\infty} m(\{x: f(x)>t\}) dt
\end{align*}
:::\{.concept\} \envlist - Claim: If
\(E\subseteq {\mathbb{R}}^a \times{\mathbb{R}}^b\) is a measurable set,
then for almost every \(y\in {\mathbb{R}}^b\), the slice \(E^y\) is
measurable and
\begin{align*}
m(E) = \int_{{\mathbb{R}}^b} m(E^y) \,dy
.\end{align*}
- Set \(g = \chi_E\), which is non-negative and measurable, so apply
Tonelli. - Conclude that \(g^y = \chi_{E^y}\) is measurable, the
function \(y\mapsto \int g^y(x)\, dx\) is measurable, and
\(\int \int g^y(x)\,dx \,dy = \int g\). - But \(\int g = m(E)\) and
\(\int\int g^y(x) \,dx\,dy = \int m(E^y)\,dy\). :::

Solution omitted.

\hypertarget{fall-2015-5-work}{%
\subsection{\texorpdfstring{Fall 2015 \# 5
\(\work\)}{Fall 2015 \# 5 \textbackslash work}}\label{fall-2015-5-work}}

Let \(f, g \in L^1({\mathbb{R}})\) be Borel measurable.

\begin{enumerate}
\def\labelenumi{\arabic{enumi}.}
\tightlist
\item
  Show that
\end{enumerate}

\begin{itemize}
\tightlist
\item
  The function
  \begin{align*}F(x, y) \coloneqq f(x-y) g(y)\end{align*}
  is Borel measurable on \({\mathbb{R}}^2\), and
\item
  For almost every \(y\in {\mathbb{R}}\),
  \begin{align*}F_y(x) \coloneqq f(x-y)g(y)\end{align*}
  is integrable with respect to \(y\).
\end{itemize}

\begin{enumerate}
\def\labelenumi{\arabic{enumi}.}
\setcounter{enumi}{1}
\tightlist
\item
  Show that \(f\ast g \in L^1({\mathbb{R}})\) and
  \begin{align*}
  \|f * g\|_{1} \leq \|f\|_{1} \|g\|_{1}
  \end{align*}
\end{enumerate}

\hypertarget{spring-2014-5-work}{%
\subsection{\texorpdfstring{Spring 2014 \# 5
\(\work\)}{Spring 2014 \# 5 \textbackslash work}}\label{spring-2014-5-work}}

Let \(f, g \in L^1([0, 1])\) and for all \(x\in [0, 1]\) define
\begin{align*}
F(x) \coloneqq\int _{0}^{x} f(y) \, dy 
{\quad \operatorname{and} \quad}
G(x)\coloneqq\int _{0}^{x} g(y) \, dy.
\end{align*}

Prove that
\begin{align*}
\int _{0}^{1} F(x) g(x) \, dx = 
F(1) G(1) - \int _{0}^{1} f(x) G(x) \, dx
\end{align*}

\hypertarget{spring-2021-6}{%
\subsection{Spring 2021 \# 6}\label{spring-2021-6}}

\begin{warnings}

This problem may be much harder than expected. Recommended skip.

\end{warnings}

Let \(f: {\mathbb{R}}\times{\mathbb{R}}\to {\mathbb{R}}\) be a
measurable function and for \(x\in {\mathbb{R}}\) define the set
\begin{align*}
E_x \coloneqq\left\{{ y\in {\mathbb{R}}{~\mathrel{\Big|}~}\mu\qty{ z\in {\mathbb{R}}{~\mathrel{\Big|}~}f(x,z) = f(x, y) } > 0 }\right\} 
.\end{align*}
Show that the following set is a measurable subset of
\({\mathbb{R}}\times{\mathbb{R}}\):
\begin{align*}
E \coloneqq\bigcup_{x\in {\mathbb{R}}} \left\{{ x }\right\} \times E_x
.\end{align*}

\begin{quote}
Hint: consider the measurable function
\(h(x,y,z) \coloneqq f(x, y) - f(x, z)\).
\end{quote}

\hypertarget{l2-and-fourier-analysis}{%
\section{\texorpdfstring{\(L^2\) and Fourier
Analysis}{L\^{}2 and Fourier Analysis}}\label{l2-and-fourier-analysis}}

\hypertarget{spring-2020-6-done}{%
\subsection{\texorpdfstring{Spring 2020 \# 6
\(\done\)}{Spring 2020 \# 6 \textbackslash done}}\label{spring-2020-6-done}}

\hypertarget{a-11}{%
\subsubsection{a}\label{a-11}}

Show that
\begin{align*}
L^2([0, 1]) \subseteq L^1([0, 1]) {\quad \operatorname{and} \quad} \ell^1({\mathbb{Z}}) \subseteq \ell^2({\mathbb{Z}})
.\end{align*}

\hypertarget{b-11}{%
\subsubsection{b}\label{b-11}}

For \(f\in L^1([0, 1])\) define
\begin{align*}
\widehat{f}(n) \coloneqq\int _0^1 f(x) e^{-2\pi i n x} \, dx
.\end{align*}

Prove that if \(f\in L^1([0, 1])\) and
\(\left\{{\widehat{f}(n)}\right\} \in \ell^1({\mathbb{Z}})\) then
\begin{align*}
S_N f(x) \coloneqq\sum_{{\left\lvert {n} \right\rvert} \leq N} \widehat{f} (n) e^{2 \pi i n x}
.\end{align*}
converges uniformly on \([0, 1]\) to a continuous function \(g\) such
that \(g = f\) almost everywhere.

\begin{quote}
Hint: One approach is to argue that if \(f\in L^1([0, 1])\) with
\(\left\{{\widehat{f} (n)}\right\} \in \ell^1({\mathbb{Z}})\) then
\(f\in L^2([0, 1])\).
\end{quote}

Solution omitted.

\hypertarget{fall-2017-5-done}{%
\subsection{\texorpdfstring{Fall 2017 \# 5
\(\done\)}{Fall 2017 \# 5 \textbackslash done}}\label{fall-2017-5-done}}

Let \(\phi\) be a compactly supported smooth function that vanishes
outside of an interval \([-N, N]\) such that
\(\int _{{\mathbb{R}}} \phi(x) \, dx = 1\).

For \(f\in L^1({\mathbb{R}})\), define
\begin{align*}
K_{j}(x) \coloneqq j \phi(j x), 
\qquad 
f \ast K_{j}(x) \coloneqq\int_{{\mathbb{R}}} f(x-y) K_{j}(y) \, dy
\end{align*}
and prove the following:

\begin{enumerate}
\def\labelenumi{\arabic{enumi}.}
\item
  Each \(f\ast K_j\) is smooth and compactly supported.
\item

  \begin{align*}
  \lim _{j \to \infty} {\left\lVert {f * K_{j}-f} \right\rVert}_{1} = 0
  \end{align*}
\end{enumerate}

\begin{quote}
Hint:
\begin{align*}
\lim _{y \to 0} \int _{{\mathbb{R}}} |f(x-y)-f(x)| dy = 0
\end{align*}
\end{quote}

\todo[inline]{Add concepts.}

Solution omitted.

\hypertarget{spring-2017-5-work}{%
\subsection{\texorpdfstring{Spring 2017 \# 5
\(\work\)}{Spring 2017 \# 5 \textbackslash work}}\label{spring-2017-5-work}}

Let \(f, g \in L^2({\mathbb{R}})\). Prove that the formula
\begin{align*}
h(x) \coloneqq\int _{-\infty}^{\infty} f(t) g(x-t) \, dt
\end{align*}
defines a uniformly continuous function \(h\) on \({\mathbb{R}}\).

\hypertarget{spring-2015-6-work}{%
\subsection{\texorpdfstring{Spring 2015 \# 6
\(\work\)}{Spring 2015 \# 6 \textbackslash work}}\label{spring-2015-6-work}}

Let \(f \in L^1({\mathbb{R}})\) and \(g\) be a bounded measurable
function on \({\mathbb{R}}\).

\begin{enumerate}
\def\labelenumi{\arabic{enumi}.}
\tightlist
\item
  Show that the convolution \(f\ast g\) is well-defined, bounded, and
  uniformly continuous on \({\mathbb{R}}\).
\item
  Prove that one further assumes that \(g \in C^1({\mathbb{R}})\) with
  bounded derivative, then \(f\ast g \in C^1({\mathbb{R}})\) and
  \begin{align*}
  \frac{d}{d x}(f * g)=f *\left(\frac{d}{d x} g\right)
  \end{align*}
\end{enumerate}

\hypertarget{fall-2014-5-work}{%
\subsection{\texorpdfstring{Fall 2014 \# 5
\(\work\)}{Fall 2014 \# 5 \textbackslash work}}\label{fall-2014-5-work}}

\begin{enumerate}
\def\labelenumi{\arabic{enumi}.}
\item
  Let \(f \in C_c^0({\mathbb{R}}^n)\), and show
  \begin{align*}
  \lim _{t \to 0} \int_{{\mathbb{R}}^n} |f(x+t) - f(x)| \, dx = 0
  .\end{align*}
\item
  Extend the above result to \(f\in L^1({\mathbb{R}}^n)\) and show that
  \begin{align*}
  f\in L^1({\mathbb{R}}^n), \quad g\in L^\infty({\mathbb{R}}^n) \quad
  \implies f \ast g \text{ is bounded and uniformly continuous. }
  \end{align*}
\end{enumerate}

\hypertarget{fall-2020-5}{%
\subsection{Fall 2020 \# 5}\label{fall-2020-5}}

Suppose \(\phi\in L^1({\mathbb{R}})\) with
\begin{align*}  
\int \phi(x) \, dx = \alpha
.\end{align*}
For each \(\delta > 0\) and \(f\in L^1({\mathbb{R}})\), define
\begin{align*}  
A_\delta f(x) \coloneqq\int f(x-y) \delta^{-1} \phi\qty{\delta^{-1} y}\, dy
.\end{align*}

\begin{enumerate}
\def\labelenumi{\alph{enumi}.}
\item
  Prove that for all \(\delta > 0\),
  \begin{align*}  
  {\left\lVert {A_\delta f} \right\rVert}_1 \leq {\left\lVert {\phi} \right\rVert}_1 {\left\lVert {f} \right\rVert}_1
  .\end{align*}
\item
  Prove that
  \begin{align*}  
  A_\delta f \to \alpha f \text{ in } L^1({\mathbb{R}}) {\quad \operatorname{as} \quad} \delta\to 0^+
  .\end{align*}
\end{enumerate}

\begin{quote}
Hint: you may use without proof the fact that for all
\(f\in L^1({\mathbb{R}})\),
\begin{align*}  
\lim_{y\to 0} \int_{\mathbb{R}}{\left\lvert {f(x-y) - f(x)} \right\rvert}\, dx = 0
.\end{align*}
\end{quote}

\hypertarget{functional-analysis-general}{%
\section{Functional Analysis:
General}\label{functional-analysis-general}}

\hypertarget{fall-2019-4-done}{%
\subsection{\texorpdfstring{Fall 2019 \# 4
\(\done\)}{Fall 2019 \# 4 \textbackslash done}}\label{fall-2019-4-done}}

Let \(\{u_n\}_{n=1}^∞\) be an orthonormal sequence in a Hilbert space
\(\mathcal{H}\).

\hypertarget{a-14}{%
\subsubsection{a}\label{a-14}}

Prove that for every \(x ∈ \mathcal H\) one has
\begin{align*}
\displaystyle\sum_{n=1}^{\infty}\left|\left\langle x, u_{n}\right\rangle\right|^{2} \leq\|x\|^{2}
\end{align*}

\hypertarget{b-13}{%
\subsubsection{b}\label{b-13}}

Prove that for any sequence
\(\{a_n\}_{n=1}^\infty \in \ell^2({\mathbb{N}})\) there exists an
element \(x\in\mathcal H\) such that
\begin{align*}
a_n = {\left\langle {x},~{u_n} \right\rangle} \text{ for all } n\in {\mathbb{N}}
\end{align*}
and
\begin{align*}
{\left\lVert {x} \right\rVert}^2 = \sum_{n=1}^{\infty}\left|\left\langle x, u_{n}\right\rangle\right|^{2}
\end{align*}

Solution omitted.

\hypertarget{spring-2019-5-done}{%
\subsection{\texorpdfstring{Spring 2019 \# 5
\(\done\)}{Spring 2019 \# 5 \textbackslash done}}\label{spring-2019-5-done}}

\hypertarget{a-16}{%
\subsubsection{a}\label{a-16}}

Show that \(L^2([0, 1]) ⊆ L^1([0, 1])\) and argue that \(L^2([0, 1])\)
in fact forms a dense subset of \(L^1([0, 1])\).

\hypertarget{b-15}{%
\subsubsection{b}\label{b-15}}

Let \(Λ\) be a continuous linear functional on \(L^1([0, 1])\).

Prove the Riesz Representation Theorem for \(L^1([0, 1])\) by following
the steps below:

\begin{enumerate}
\def\labelenumi{\roman{enumi}.}
\tightlist
\item
  Establish the existence of a function \(g ∈ L^2([0, 1])\) which
  represents \(Λ\) in the sense that
  \begin{align*}
    Λ(f ) = f (x)g(x) dx \text{ for all } f ∈ L^2([0, 1]).
    \end{align*}
\end{enumerate}

\begin{quote}
Hint: You may use, without proof, the Riesz Representation Theorem for
\(L^2([0, 1])\).
\end{quote}

\begin{enumerate}
\def\labelenumi{\roman{enumi}.}
\setcounter{enumi}{1}
\tightlist
\item
  Argue that the \(g\) obtained above must in fact belong to
  \(L^∞([0, 1])\) and represent \(Λ\) in the sense that
  \begin{align*}
    \Lambda(f)=\int_{0}^{1} f(x) \overline{g(x)} d x \quad \text { for all } f \in L^{1}([0,1])
    \end{align*}
  with
  \begin{align*}
    \|g\|_{L^{\infty}([0,1])} = \|\Lambda\|_{L^{1}([0,1]) {}^{ \check{} }}
    \end{align*}
\end{enumerate}

Solution omitted.

\hypertarget{spring-2016-6-work}{%
\subsection{\texorpdfstring{Spring 2016 \# 6
\(\work\)}{Spring 2016 \# 6 \textbackslash work}}\label{spring-2016-6-work}}

Without using the Riesz Representation Theorem, compute
\begin{align*}
\sup \left\{\left|\int_{0}^{1} f(x) e^{x} d x\right| {~\mathrel{\Big|}~}f \in L^{2}([0,1], m),~~ \|f\|_{2} \leq 1\right\}
\end{align*}

\hypertarget{spring-2015-5-work}{%
\subsection{\texorpdfstring{Spring 2015 \# 5
\(\work\)}{Spring 2015 \# 5 \textbackslash work}}\label{spring-2015-5-work}}

Let \(\mathcal H\) be a Hilbert space.

\begin{enumerate}
\def\labelenumi{\arabic{enumi}.}
\tightlist
\item
  Let \(x\in \mathcal H\) and \(\left\{{u_n}\right\}_{n=1}^N\) be an
  orthonormal set. Prove that the best approximation to \(x\) in
  \(\mathcal H\) by an element in
  \({\operatorname{span}}_{\mathbb{C}}\left\{{u_n}\right\}\) is given by
  \begin{align*}
    \widehat{x} \coloneqq\sum_{n=1}^N {\left\langle {x},~{u_n} \right\rangle}u_n.
    \end{align*}
\item
  Conclude that finite dimensional subspaces of \(\mathcal H\) are
  always closed.
\end{enumerate}

\hypertarget{fall-2015-6-work}{%
\subsection{\texorpdfstring{Fall 2015 \# 6
\(\work\)}{Fall 2015 \# 6 \textbackslash work}}\label{fall-2015-6-work}}

Let \(f: [0, 1] \to {\mathbb{R}}\) be continuous. Show that
\begin{align*}
\sup \left\{\|f g\|_{1} {~\mathrel{\Big|}~}g \in L^{1}[0,1],~~ \|g\|_{1} \leq 1\right\}=\|f\|_{\infty}
\end{align*}

\hypertarget{fall-2014-6-work}{%
\subsection{\texorpdfstring{Fall 2014 \# 6
\(\work\)}{Fall 2014 \# 6 \textbackslash work}}\label{fall-2014-6-work}}

Let \(1 \leq p,q \leq \infty\) be conjugate exponents, and show that
\begin{align*}
f \in L^p({\mathbb{R}}^n) \implies \|f\|_{p} = \sup _{\|g\|_{q}=1}\left|\int f(x) g(x) d x\right|
\end{align*}

\hypertarget{functional-analysis-banach-spaces}{%
\section{Functional Analysis: Banach
Spaces}\label{functional-analysis-banach-spaces}}

\hypertarget{spring-2019-1-done}{%
\subsection{\texorpdfstring{Spring 2019 \# 1
\(\done\)}{Spring 2019 \# 1 \textbackslash done}}\label{spring-2019-1-done}}

Let \(C([0, 1])\) denote the space of all continuous real-valued
functions on \([0, 1]\).

\begin{enumerate}
\def\labelenumi{\alph{enumi}.}
\item
  Prove that \(C([0, 1])\) is complete under the uniform norm
  \({\left\lVert {f} \right\rVert}_u := \displaystyle\sup_{x\in [0,1]} |f (x)|\).
\item
  Prove that \(C([0, 1])\) is not complete under the
  \(L^1{\hbox{-}}\)norm
  \({\left\lVert {f} \right\rVert}_1 = \displaystyle\int_0^1 |f (x)| ~dx\).
\end{enumerate}

\todo[inline]{Add concepts.}

Solution omitted.

\hypertarget{spring-2017-6-done}{%
\subsection{\texorpdfstring{Spring 2017 \# 6
\(\done\)}{Spring 2017 \# 6 \textbackslash done}}\label{spring-2017-6-done}}

Show that the space \(C^1([a, b])\) is a Banach space when equipped with
the norm
\begin{align*}
\|f\|:=\sup _{x \in[a, b]}|f(x)|+\sup _{x \in[a, b]}\left|f^{\prime}(x)\right|.
\end{align*}

\todo[inline]{Add concepts.}

Solution omitted.

\hypertarget{fall-2017-6-done}{%
\subsection{\texorpdfstring{Fall 2017 \# 6
\(\done\)}{Fall 2017 \# 6 \textbackslash done}}\label{fall-2017-6-done}}

Let \(X\) be a complete metric space and define a norm
\begin{align*}
\|f\|:=\max \{|f(x)|: x \in X\}.
\end{align*}

Show that \((C^0({\mathbb{R}}), {\left\lVert {{-}} \right\rVert} )\)
(the space of continuous functions \(f: X\to {\mathbb{R}}\)) is
complete.

\todo[inline]{Add concepts.}
\todo[inline]{Shouldn't this be a supremum? The max may not exist?}
\todo[inline]{Review and clean up.}

Solution omitted.


\printbibliography[title=Bibliography]


\end{document}
