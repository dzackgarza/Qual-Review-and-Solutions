\documentclass[psamsfonts, 11pt, reqno]{amsart}
\setlength{\topmargin}{0pt}
\setlength{\footskip}{10pt}
\setlength{\oddsidemargin}{2.5cm}
\setlength{\evensidemargin}{2.5cm}
\setlength{\textwidth}{329pt}
\addtolength{\textheight}{50pt}
\usepackage{amsfonts,amsmath, amsthm, amssymb, latexsym, amscd, epsfig, pdfsync}
\usepackage[all]{xy}
\usepackage[mathscr]{eucal}
%\usepackage{layout}
%\addtolength{\topmargin}{-90pt}
%\addtolength{\textheight}{140pt}
\addtolength{\textwidth}{120pt}
\addtolength{\hoffset}{-60pt}
\everymath={\displaystyle}

\newcommand{\seriesk}{\sum\limits_{k=1}^{\infty}}
\newcommand{\seriesn}{\sum\limits_{n=1}^{\infty}}
\newcommand{\z}{\mathbb{Z}}
\newcommand{\n}{\mathbb{N}}
\newcommand{\R}{\mathbb{R}}
\newcommand{\q}{\mathbb{Q}}
\newcommand{\C}{\mathbb{C}}
\newcommand{\la}{\langle}
\newcommand{\ra}{\rangle}
\newcommand{\x}{\underline{\underline{x}}}
\renewcommand{\r}{\underline{\underline{r}}}
\renewcommand{\t}{\mathcal{T}}
\renewcommand{\d}{\mathcal{D}}
\newcommand{\D}{\mathcal{D}}
\newcommand{\pic}{\mathbb{P}}
\newcommand{\p}{\mathcal{P}}
\newcommand{\Q}{\mathcal{Q}}
\newcommand{\ds}{\displaystyle}
\renewcommand{\tilde}{\widetilde}
\newcommand{\inv}{^{-1}}
\renewcommand{\oplus}{\bigoplus}
\renewcommand{\Sigma}{\sum}
\renewcommand{\Pi}{\prod}
\renewcommand{\phi}{\varphi}

\newtheorem{definition}{Definition}
\newtheorem{thm}[definition]{Theorem}
\newtheorem{theorem}[definition]{Theorem}
%\theoremstyle{definition}
%\newtheorem{fact}[defn]{Fact}
%\newtheorem{note}[defn]{Note}
\newtheorem{lemma}[definition]{Lemma}
%\newtheorem{cor}[defn]{Corollary}
%\newtheorem{conjecture}[defn]{Conjecture}
%\newtheorem{problem}[defn]{Problem}
%\newtheorem{notation}[defn]{Notation}
%\newtheorem*{question}{Question}
\newtheorem{ex}[definition]{Example}
\newtheorem{example}[definition]{Example}
%\newtheorem*{results}{Known Results}

\title{QualProblems}
%\author(,\\
%}

\begin{document}
\begin{center}
\begin{large}
Topology Qual Workshop Day 3: Separation Axioms
\end{large}
\end{center}
\vspace{.25in}

\begin{enumerate}
\item (Jan '08 \# B9) Let $X = \mathbb{R}^2$ and define an equivalence relation on $X$ by $(x_1, x_2) \sim (y_1,y_2)$  iff
they are qual or $x_1 = y_1 = 0$.  Set $Y = X/ \sim$.  Show that $Y$ is Hausdorff but does not have a countable basis
for its topology.

\vfill

\item (June '06 \# A5) 
\begin{enumerate}
\item Recall that a topological space is called \emph{separable} if it contains a countable dense subset.  Prove that
a countable product of separable spaces (with the product topology is separable).

\item Prove that the result in the first part is false if the product is given the box topology.
\end{enumerate}
\vfill

\item (June '04 \# A3) Show that for a topological space $X$, if every $x \in X$ has a neighborhood whose closure
is a regular space, then $X$ is regular.
\vfill


\item (Jan '04 \# A2) Show that if the metric space $(X, d)$ is separable, then the metric topology on $X$ is second countable.
\vfill

\item (Jan '04 \# A4) Two subsets $A, B \subseteq X$ of the space $(X, \tau)$ are called \emph{separated} if
there are $U,V \in \tau$ with $A \subseteq U \subseteq X \setminus B$ and $B \subseteq V \subseteq X \setminus A$.  We
say that $X$ is \emph{completely normal} if $X$ is $T_1$ and if for every pair of separated subsets $A,B$ there are
$U,V \in \tau$ so that $A \subseteq U, B \subseteq V$ and $U \cap V = \emptyset$.  Show that a space $(X, \tau)$ is
completely normal $\Leftrightarrow$ every subset of $X$ is normal.
\vfill

\item (Jan '12 \# A2) Let $X$ and $Y$ be two topological spaces and let $X \times Y$ be endowed with the product topology.
Prove that if $X$ and $Y$ each have a countable dense subset, then so does $X \times Y$.

\vfill

\item (Jan '12 \# A4)
\begin{enumerate}
\item State the definition of a what it means for a topological space to be regular.
\item Prove that a subspace of a regular space is also regular.
\item Prove that a product of two regular spaces (equipped with the product topology) is also regular.

\end{enumerate}
\vfill 

\item (June '11 \# A3) Prove that every metrizable space is normal Hausdorff (aka $T_4$).

\vfill

\item (June '09 \# A4) Let $(X, \tau)$ and $(X, \tau ')$ be topological spaces with $\tau \subseteq \tau '$.
\begin{enumerate}
\item If $(X, \tau ')$ is normal, must $(X, \tau)$ also be normal?
\item If $(X, \tau ')$ is compact, must $(X, \tau)$ also be compact?
\end{enumerate}


\end{enumerate}

\end{document}