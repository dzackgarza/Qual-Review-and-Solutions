%\documentclass[exam]{amsart}
%\usepackage{amssymb}
%\voffset -1in \hoffset -1in \textwidth 6in \textheight 7in
\documentclass[12pt]{exam}
\newif\ifpdf                              % newvariable ifpdf
\ifx\pdfoutput\undefined
     \pdffalse                               % we are not running PDFLaTeX
\else
     \pdfoutput=1                         % we are running PDFLaTeX
     \pdftrue
\fi
\usepackage[pdftex]{graphicx}
\usepackage{verbatim}
\usepackage{amsthm}
\usepackage{amssymb}
\usepackage{latexsym}
\usepackage{amsfonts}
\usepackage{amstext}
\usepackage{amsbsy}
\usepackage{amsmath}
%\usepackage{euler}
\usepackage{txfonts}
%\usepackage{punk}
%\usepackage{theorem}
\usepackage{exscale}
%\renewcommand{\baselinestretch}{1.58}
\setlength{\textheight}{23cm} %\setlength{\textwidth}{18cm}
\setlength{\voffset}{-.5cm}% \setlength{\hoffset}{-2.5cm}

\theoremstyle{definition}
\newtheorem{ques}{\bf Question}

\newcommand{\A}{{\mathcal A}}
\newcommand{\B}{{\mathcal B}}
\newcommand{\RA}{\rm{I}\!\!\!Q}
\newcommand{\RE}{\rm{I}\!R}
\newcommand{\NA}{\rm{I}\!N}
\newcommand{\ZE}{\rm{Z}\!\!Z}
\newcommand{\IR}{\RE\!\! \setminus\! \RA}
\newcommand{\CO}{\rm{I}\!\!\!C}
\newcommand{\menos}{\! \setminus\!}
\newcommand{\contp}{\subset}
\newcommand{\cont}{\subseteq}
\newcommand{\vac}{\emptyset}
\newcommand{\cerr}{\overline}
\newcommand{\arco}{\widehat}
\newcommand{\invlim}{\hbox{lim}_{\leftarrow}}
\newcommand{\tiende}{\longrightarrow}
\newcommand{\mini}{\hbox{\rm m\ii n}}
\newcommand{\maxi}{\hbox{\rm m\'ax}}
\newcommand{\infi}{\hbox{\rm \ii nf}}
\newcommand{\ellim}{\hbox{\rm lim}}
%\newcommand{\limite}{\ellim_{m \tiende \infty}}
\newcommand{\limite}[1]{\begin{smallmatrix}\hbox{\rm lim}\\ x \rightarrow \infty\end{smallmatrix}#1}
\newcommand{\limiten}[1]{\begin{smallmatrix}\hbox{\rm lim}\\ x \rightarrow -\infty\end{smallmatrix}#1}
\newcommand{\limitex}[1]{\begin{smallmatrix} \vspace{.1cm} \\ \hbox{\rm lim}\\ x \rightarrow\, #1\end{smallmatrix}\;}

\newcommand{\pun}[1]{\frac{1}{#1}}
\newcommand{\inv}[3]{\begin{smallmatrix}{\hbox{\rm lim}}\vspace{-.8mm}\\ {\!\!\!\!{#1}\hspace{.2cm}\rightarrow{#2}}\!\!\!\!\!\!\!\!\!\!\!\!\!\!\!\!\!-- \vspace{-1.6mm}\end{smallmatrix}\hspace{.2cm} {#3}}
%\newcommand{\inv2}[3]{\begin{smallmatrix}{\hbox{\rm lim}}\vspace{-.8mm}\\ {\!\!\!\!{#1}\hspace{.2cm}\rightarrow{#2}^+}\!\!\!\!\!\!\!\!\!\!\!\!\!\!\!\!\!-- \vspace{-1.6mm}\end{smallmatrix}\hspace{.2cm} {#3}}

\newcommand{\f}[3]{#1:#2 \longrightarrow #3}
\newcommand{\Sum}[3]{\begin{smallmatrix}\hbox{$#1$}\\#2\end{smallmatrix}#3}
\newcommand{\Suma}[4]{\begin{smallmatrix}#3\\ \hbox{$#1$}\\#2\end{smallmatrix}#4}
\newcommand{\inte}[1]{#1^{\circ}}
\newcommand{\C}{\mathcal{C}}
\newcommand{\D}{\mathcal{D}}
\newcommand{\R}{\mathcal{R}}
\newcommand{\J}{\mathcal{S}}
\def\Z{{ \Bbb Z}}
\def\R{{ \Bbb R}}
\def\C{{ \Bbb C}}
\newcommand{\e}{\text{\Large{\emph e}}}
\newcommand{\x}{\text{\large{\emph x}}}
\newcommand{\fracc}[2]{\begin{array}{c}%
  #1 \\
\hline \vspace{-.4cm} \\
 #2 \\
\end{array}}%
%\setcounter{page}{0}

%\firstpageheader{Topology Ph.D. Qualifying Exam} This exam has been checked carefully for errors. If you find what
%you believe to be an error in a question, report this to the proctor. If the proctor's interpretation still seems
%unsatisfactory to you, you may modify the question so that in your view it is correctly stated, but not in such a
%way that it becomes trivial. {Page \thepage\ of \numpages}{Fall 2006}

\author{Gerard Thompson \and Mao-Pei Tsui }
\date{January 17, 2009}

\runningheader{Topology Ph.D. Qualifying Exam}{Page \thepage\ of
\numpages}{January 2009}

\firstpagefooter{Topology Ph.D. Qualifying Exam}{}{}
\runningfooter{}{}{}

\runningheadrule \firstpageheadrule

\begin{document}

\title{Topology Ph.D. Qualifying Exam}
\author{Gerard Thompson \and Mao-Pei Tsui }
\date{January 17, 2009}
\maketitle \noindent This examination  has been checked carefully
for errors. If you find what you believe to be an error in a
question, report this to the proctor. If the proctor's
interpretation still seems unsatisfactory to you,  you may modify
the question so that in your view it is correctly stated, but not in
such a way that it becomes trivial. If you feel that the examination
is on the long side do not panic. The grading will be adjusted
accordingly.

\setcounter{page}{1}

\section{Part One: Do six questions}

\begin{questions}

\question
%If $(X,d)$ is a metric space then $\{x\in X: d(x,x_0) < \epsilon \}$
%is said to be the open ball of radius $\epsilon$. Prove that an open
%ball is an open set.

Prove that in a metric space a compact subset is closed and bounded.
If you cannot do it for a general metric space do it for $\R^n$.

\question

Prove that in a complete metric space $(X,d)$ a subspace $Y$ of $X$
is complete if and only if it is a closed subspace of $X$.

\question

Define the term \emph{closure} $\overline{A}$ of a subset $A$ of a
topological space $X$. Prove that if $B$ is a closed subset of $X$
such that $A \subset B$ then $\overline{A} \subset B$ .

%\question

%Define what it means for $(X,d)$ to be a metric space. Then $d:X
%\times X:\rightarrow \R$: is $d$ continuous?
%Discuss. If you cannot answer in general do it for $X=\R$ with the usual  topology. %Choose a metric $d_1$ on $X$
%defined by $d_1((x,a),(y,b))=(d((x,y)^2+d((a,b)^2)^{\frac{1}{2}}$ and put $\delta=\frac{\epsilon}{2}$. Then if
%$d_1((x,a),(y,b)) < \frac{\epsilon}{2}$, $d((x,y) < \frac{\epsilon}{2}$ and $d((a,b) < \frac{\epsilon}{2}$. Now
%$|d((x,y)-d((a,b)| < d((x,y)+d((a,b) < \frac{\epsilon}{2}+\frac{\epsilon}{2}=\epsilon. $

\question

Let $(X,d)$ be a metric space and let $A \subset X$. If $x \in X$
define the distance of $x$ to $A$ to be inf $\{d(x,a): a \in A\}$.
Prove that the real-valued function on $X$ defined by $x \mapsto
d(x,A)$ is continuous.

\question

Prove that a closed subset of a compact topological space is
compact. Prove that in a Hausdorff topological space a compact
subset is closed.

%A set is said to have the finite complement topology if the closed sets are the finite sets together with the
%empty set. If $X$ is a topological space with an infinite number of points show that the diagonal $\Delta:=
%\{(x,x) | x \in X\}$ is n
%ot closed in the finite complement topology.\\
%Suppose that $x,y \in X$ with $x \neq y$ and that $ \in U$ and  $v \in V$ with $U, V$ open in $X$. Then $X-U$ and
%$X-V$ are both finite but since $X$ is infinite we can only have that $U \cap V \neq \emptyset$.

%A set is said to have the countable-closed topology if the closed
%sets are the countable sets together with the empty set. If $X$ is a
%topological space with an uncountably infinite number of points is
%the diagonal $\Delta:= \{(x,x) | x \in X\}$ closed in the finite
%complement topology? Justify your answer carefully.
%Suppose that $x,y \in X$ with $x \neq y$ and that $x \in U$ and  $v \in V$ with $U, V$ open in $X$. Then $X-U$ and
%$X-V$ are both finite but since $X$ is infinite we can only have that $U \cap V \neq \emptyset$.

\question

%Let $B$ be an open subset of a topological space $X$. Prove that a
%subset $A \subset B$ is relatively open in $B$ if and only if $A$ is
%open in $X$.

Define the term \emph{identification map} in the category of
topological spaces. Let $\pi:X \rightarrow Y$ be a surjective,
continuous map of topological spaces. Suppose that $\pi$ maps closed
sets to closed sets. Show that $\pi$ is an identification map. What
happens if we replace closed sets by open sets? Justify your
answers.
%\questiono

%A topological space $X$ is said to be \emph{locally connected} if any neighborhood of any point contains a
%connected neighborhood. Prove that the connected components of a locally connected space are both open and closed.

%\question

%Let $D$ be the open unit disk in the complex plane that is $D:=\{z |
%\,\, |z| < 1\}$. Let $\sim{}$ be an equivalence relation on $D$
%defined by $z_1 \sim{ z_2}$ if $|z_1 |=|z_2|$. Is the quotient space
%$D/\sim{}$ Hausdorff? Prove or disprove.

\question Prove that the product topological space $X \times Y $ is
Hausdorff if and only if $X$ and $Y$ are Hausdorff.

%Define compactness for a topological space. A collection of subsets is said to have the finite intersection
%property if every finite class of those subsets has non-empty intersection. Prove that a topological space is
%compact iff every collection of closed subsets that has the finite intersection property itself has  non-empty
%intersection.

%Define compactness for a topological space. Prove from your
%definition that the closed interval $[0,1]$ is compact.

\question

Prove that the open interval $(0,1)$ considered as a subset of $\R$
in the usual topology is not compact.

\question

Prove that $\R$ with the usual topology  is connected.


%2
%\item Prove that a function $f:W \rightarrow \Phi$ from a space
%$W$ into the topological product $\Phi$ is continuous iff, for each $\mu \in
%M$, the composition $p_{\mu} \circ f$ is continuous.\\

%\vspace{.45in}

%Let $\gamma$ be a given cover of a topological space $X$. Assume that for each member $A\in\gamma$, there is given
%a continuous map $f_A:A \rightarrow Y$ such that
%\[f_A \ | \ A\cap B=f_B \ | \ A \cap B\]
%\noindent for each pair of members $A$ and $B$ of $\gamma$. Then we may define a function $f:X \rightarrow Y$ by
%taking
%\[f(x)=f_A(x), \qquad \mbox{(if $x \in A \in \gamma$)}.\]
%Prove that if $\gamma$ is a finite closed cover of $X$, then the function $f$ is continuous.

%in \vspace{.25in}

%4
%\item Prove that every compact space is normal.

\question

Construct a topological space $X$ by starting with $\R$ with the
usual topology and defining $x$ to be equivalent to $y$ if $x-y$ is
rational. Show that the resulting quotient or identification space
$X$ has the indiscrete topology, that is, the only open sets are
$\emptyset $ and $X$.

%\question

%Let $X=\Pi_{\mu \in M}X_{\mu}$ and $Y=\Pi_{\mu \in M}Y_{\mu}$ be the
%Cartesian products of the topological spaces $(X_{\mu})_{\mu \in M}$
%and $(Y_{\mu})_{\mu \in M}$ and let $X$ and $Y$ have the product
%topologies, respectively. Prove that if for each $\mu \in M$ the
%maps $f_{\mu}:X_{\mu} \rightarrow Y_{\mu}$ are continuous then $f:X
%\rightarrow Y$ defined by $f(x)_{\mu}=f_{\mu}(x_{\mu})$ is
%continuous.

%\question

%Prove that the continuous image of a connected set is connected.
%Prove that a path-connected topological space is connected.

%Prove that if two connected sets $A$ and $B$ in a space $X$ have a common point $p$, then $A \cup B$ is connected.

%9
%\item Prove that a point $p$ of a space $X$ belongs to the closure $\overline{E}$ of a set $E$ in $X$ iff there
%exists a net $\Phi$ in $E$ which converges to $p$.

%\question

%A Hausdorff topological space is known to be \emph{locally compact}
%if every point has a compact neighborhood. Prove that every closed
%subspace of a locally compact space is locally compact.

%\vspace{.25in}
%\question

%It is a fact that every compact subset of a Hausdorff space is closed. Moreover a topological space is said to be
%\emph{normal} if every pair of disjoint closed sets can be separated by disjoint open sets. Prove that a compact
%Hausdorff space is normal.

%\question

%Let $X$ be a topological space. Let $A \subset$ X be connected.
%Prove that the closure $\overline{A}$ of $A$ is connected.

%\question

%Prove that if $f:X \mapsto Y$ is  continuous and surjective and $X$ is compact and $Y$ Hausdorff then $f$ is an
%identification map.

% Let $f:X \mapsto Y$ be a continuous map between compact Hausdorff spaces. Show that f is a homeomorphism if
%and only if it is injective and surjective.

\question A topological space $X$ is said to be \emph{locally
connected} if the connected components of each point form a base of
neighborhoods of $X$. Prove that in a locally connected space the
connected components of $X$ are open in $X$.

%Define the terms {\it topological space} and {\it neighborhood space}. Explain carefully how a topological space
%gives rise to a neighborhood space and conversely a neighborhood space gives rise to a topological space.

%Prove or disprove: in a compact topological space every infinite set has a limit point. If you cannot answer the
%question for a compact topological space answer it for a metric space.

% Let $X$ be a compact space and let $A_1 \supset A_ 2 \supset \cdots A_k \cdots$ be a descending chain of
%non-empty closed subsets of $X$. Show that the intersection $\cap_{k =1}^{\infty} A_k$ is not empty.

%Let $A$ be a subspace of a topological space $X$. Prove that $A$ is disconnected if and only if there exist two
%closed subsets $F$ and $G$ of $X$ such that $A \subset F \cup G$ and  $F \cap G \subset X-A$.

%Define the term ``connected component" for a topological space.
%Prove that a connected component is connected.

\question

True or false: in a compact topological space every infinite set has
a limit point. Justify your answer.


%Prove that a subset of $\R^n$ is compact if and only if it is closed and bounded.

%Is an arbitrary product of T$_1$ or Hausdorff spaces T$_1$ or Hausdorff?
%A closed subset of a locally compact  Hausdorff space is locally compact.

%\question

%A topological space $X$ is said to be \emph{regular} if disjoint
%singleton and closed sets can be separated by disjoint open sets.
%Prove that in regular space disjoint closed and compact sets can be
%separated by disjoint open sets.

\end{questions}

\section{Part Two: Do three questions}

\begin{questions}

\question

Define what is meant by a M\"obius band. Identify the space obtained
by identifying the boundary of a M\"obius band to a point. Give a
brief explanation.

\question

Prove that if $f,g:X \rightarrow S^{n-1}$ are continuous and both
not surjective then $f$ is homotopic to $g$.

\question

Let $X$ be a topological space and let $x_0 \in X$. Define the
product of homotopy classes of loops $[\alpha]_{x_0}$ based at $x_0$
and verify in detail that this product is associative.

\question Give the definitions of deformation retract and strong
deformation retract for topological spaces. Compute the fundamental
group of $\R^3-C$ where $C$ denotes the circle $x^2+y^2=1,z=0$.

%\question

%Let $$S^1 = \{(x,y)|x^2+y^2=1\}$$ and
%$$S^4=\{(x_1,x_2,x_3,x_4,x_5)|x_1^2+x_2^2+x_3^2+x_4^2+x_5^2=1\}.$$

%(i) Let $A$ be the antipodal map  on $S^1$ defined by $A(x,y)=(-x,-y)$. Show that $A$ is homotopic to the identity
%map on $S^1$.

%(ii) Let $B$ be the  map on $S^4$ defined by $B(x_1,x_2,x_3,x_4,x_5)=(-x_1,-x_2,x_3,-x_4,-x_5)$. Show that $B$ is
%homotopic to the identity map on $S^4$.

%\question
%the fundamental group of Recall that $S^n$ is defined to be
%$\{(x_1,x_2,...,x_n,x_{n+1}) \in \R^{n+1}
%|x_1^2+x_2^2+...+x_n^2+x_{n+1}^2=1\}$. We shall be interested in the
%values $n=1,2$. Then define $f,g: S^1 \rightarrow$ by
%$f(\cos(s),\sin(s))=(\cos(s),\sin(s),0)$ and
%$g(\cos(s),\sin(s))=(\cos(s),-\sin(s),0)$. Show that $f$ is
%homotopic to $g$.

%\question

%Recall that $S^k = \left\{(x_1, \cdots,  x_{k+1} )\Big| x_1^2+ \cdots x_{k+1}^2 = 1 \right\} \subset \R^{k+1}$.
%The antipodal map $A_k : S^k \mapsto S^k$ is the smooth map defined by $(x_1, \cdots, x_{k+1} ) \mapsto (-x_1,
%\cdots, -x_{k+1} )$.

%(i) Show that $A_1 : S^1 \mapsto S^1$ is homotopic to the identity
% map.

%(ii) Show that $A_k : S^k \mapsto S^k$ is homotopic to the identity map if $k$ is odd.

%We identify (x_1,x_2) with x_1+i x_2. One can construct the homotopy easily by f(t)= exp(i t\pi) (x_1+ i x_2) .
%Then f(0)=x_1+ i x_2 f(1)= -(x_1+ i x_2).

%\question

%(i) Let $X$ be a topological space. Let $f$, $g:I \mapsto X$ be two
%paths from $p$ to $q$. Show that $f\sim g$, that is, $f$ is
%homotopic to $g$ if and only if $f \cdot g^{-1} \sim c_p$ where
%$g^{-1}$ is the inverse path to $g$, $c_p$ denotes the constant path
%based at $p$ and the  $``\cdot"$ denotes the product of (compatible)
%paths.

%(ii) Show that $X$ is simply connected if and only if any two paths
%in $X$ with the same initial and terminal points are path homotopic.

%\question

%Let X and Y be  topological spaces.

%(i) Define what it means for X and Y to have the same homotopy type.

%(ii) A space is contractible if it is homotopy equivalent to the one-point space. Prove that X is contractible if
%and only if the identity map $id_X : X \mapsto X$ is homotopic to a map $r : X \mapsto X $ whose image is a single
%point.

%(iii) Suppose that $Y \subset X$. Define what it means for $Y$ to be a retract of $X$.

%(vi) Prove that a retract of a contractible space is contractible.

\question

(i) The polygonal symbol of a certain surface without boundary is
$xy^{-1}x^{-1}zwz^{-1}vyw^{-1}v^{-1}$. Identify the surface. What is
its Euler characteristic?

(ii) Explain how polygons with an even number of sides may be used
to classify surfaces without boundary. You do not need to give
detailed proofs.

\question

It is known that if $p: \tilde{X}: \rightarrow X$ is a covering
space and $x_0 \in X$ then the cardinality of $p^{-1}(x_0)$ is the
index of $p_{*}\pi_1( \tilde{X},y_0)$ in $\pi_1(X,x_0)$ where
$p(y_0)=x_0$. Use this fact to deduce that there is no $n$-sheeted
covering of the circle $S^1$ for any finite $n$.

\question

Compute the Euler characteristic of the $n$-sphere $S^n$ using the
standard triangulation of an $n$-simplex.


%\question

%Let $X=[0,1] \times [0,1]$ denote the rectangle in $\R^2$. Let
%$\sim$  be the equivalence relation generated by $(0,p) \sim (1,
%1-p)$ wher $0 \le p \le 1$. The quotient space $X/{\sim}$ is called
%the  M\"obius band. Show that $S^1$ is a retract of the M\"obius
%band.

%\question

%Compute the first three homology groups of the \emph{hollow} sphere
%$S^2$. You may use simplicial theory and a triangulation but do not
%simply say that $H_1(S^2)$ is the abelianization of $\pi_1(S^2)$.


%\question

%Let $D=\{ (x,y) \in \R^2 | x^2+ y^2 =1 \}$ be the closed unit disk. Prove that $D$ in cannot be retracted to the
%unit circle $S^1$.  Deduce that any continuous map $f : D \longrightarrow D$ has a fixed point. ({\it Hint:}\
%Consider the line joining $x$ to $f(x)$ where $x \in D$.)

%\question

%\noindent Compute the fundamental group of the (surface of) a sphere
%when three points on it are removed.

%\question

%Recall the definition of $S^n$ from question 1. By dividing $S^n$
%into two ``hemi-spheres" use the Sefeirt van Kampen Theorem to find
%$\pi_1(S^n)$ for $n \geq 1$.


%\question

%(i) Compute the fundamental group of the \emph{solid} torus in the figure below when the points $x$ and $y$ are
%identified.

%(ii) Compute the fundamental group of the \emph{hollow} torus in the figure below when the points $x$ and $y$ are
%identified.

%\begin{center}
%\begin{figure}[hhh]
  % Requires \usepackage{graphicx}
% \includegraphics[scale=.8]{topology1.jpg}
 %\includegraphics[scale=.8]{3_hole_torus.jpg}
%  \end{figure}
%\end{center}

\begin{center}
 \begin{figure}[hhh]
% \includegraphics[scale=.75]{3holetorus1.jpg}
 \end{figure}
\end{center}

%\question

%Let $P^2$ be the two-dimensional real projective space and $T^2$ be the two-dimensional torus.

%(i) What is $\pi_1(P^2)$? Explain your answer.

%(ii) The space $P^2$ can be obtained from the disk $D^2$ by identifying $x \sim -x$ if $\|x\| = 1$. Let $p\in
%\hbox {int}(D^2)$, the interior of $D^2$. Find the fundamental group of $P^2 -\{[p]\}$.

%(iii) Let $f:P^2 \mapsto T^2$ be a continuous map. Show that $f$ is null homotopic.

%\question

%It is well-known that the punctured plane $\R^2-(0,0)$ has the structure of a topological group with
%multiplication defined by $(x,y) \cdot (p,q) =(xp-yq,xq+yp)$ (induced by multiplication of complex numbers). Can
%one define a topological group structure on $\R^2 - \{(1,0),(-1,0)\}$? Explain.

%%Outline the main points in the construction of the fundamental group of a topological space.

\end{questions}

\end{document}

%\item

%The space $G$ is a topological group meaning that $G$ is a group and also a Hausdorff topological space such that
%the multiplication and map taking each element to its inverse are continuous operations. Given two loops based at
%the identity $e$ in $G$, say $\alpha (s)$ and $\beta (s)$, we have two ways to combine them: $\alpha \cdot \beta$
%(product of loops as in the definition of fundamental group) and secondly $\alpha \beta$ using the group
%multiplication.  Show, however, that these constructions give homotopic loops.

%\item Show that the fundamental group of a path-connected topological group is abelian. (Hint: Show that $\alpha
%\beta \sim \beta \alpha $.)

%\item Let $D^2=\{ (x,y) \in R^2 | x^2+ y^2 =1 \}$. Prove that the closed unit disk $D^2$ in ${\Bbb R}^2$ cannot be
%retracted to the unit circle $S^1$.  Deduce that any continuous map $f : D \longrightarrow D$ has a fixed point.
%({\it Hint:}\ Consider the line joining $x$ to $f(x)$ where $x \in D$.)


%\item For two continuous maps $f$, $g: X \mapsto S^n$ such that $f(x)\neq -g(x)$ for all $x\in S^n$, show that
%$f\simeq g$.

\end{document}

Some students have been asking me about qualifying problems. Here
are some comments about three of them. I hope they help.

GT


#3 (i)  (Part 2) Jan-2008

The answer here is a connected sum of three projective planes? Why?
First of all the symbol contains the edges z and z so it is
non-orientable. Now remember we can read the symbol beginning with
any letter here $zx^{-1}zy^{-1}xy$. Now use ru le 1 on P.93 of
Maunder so that symbol is changed to $wwxy^{-1}xy$. Using the same
rule again gives $wwuuxyy$ i.e. a connected sum of three projective
planes.

Alternatively use the following Theorem.

Given a surface symbol of a surface $M^2$ let $n$ be the number of
edges in the polygon (necessarily even) and let $m$ be the number of
distinct vertices in the polygon after the identifications of the
edges have been ma de. Then $\chi(M^2)=m-n/2+1$.

In the present case we have $n=6$ and $m=1$ (check!). Thus
$\chi(M^2)=1-6/2+1=-1$. We have $2-k=-1$ where $k$ is the number of
e handles, since $M^2$ is non-orientable. This gives $k=3$ and same
conclusion as before.



#6 (Part 2) April-2008

There are some problems with this question. Sayonita already asked
me about it. I could not remember at all where this question  came
from which is what happens when you get old.  Anyway I think that
the subcomplex |L| should be the simplex ACE together with its
faces. Think of the vertices of the hexagon as being the roots of
z^6=1: then f(z)=z/2. Can you do the problem now?


#4 (Part 2) April-2006

By a deformation we can change the problem to: two spheres one on
top of the other with the north pole of the lower sphere and south
pole of the upper sphere identified and piece of string joining the
north pole of the upper sphere to the south pole of the lower
sphere. The easiest argument to compute the fundamental group is to
"fill in" the two spheres; simplicially, the two- skeleton remains
the same but now we can contract the two balls to a point leaving a
circle so that the the fundamental group is $Z$. Alternatively, to
do it by Seifert-van Kampen take for U the union of the two spheres;
for V take the piece of string but now continue it through the
spheres to get a loop, a space homeomorphic to a circle. Now $U \cap
V$ is the part of the loop in the spheres so is contractible so has
trivial fundamental group. Also $U$ has trivial fundamental group by
an easy application of Seifert-van Kampen and the fundamental group
of $V$ is $Z$ as before.

Notes:

1. We cannot just retract the spheres to points: the sphere is not
contractible in any dimension.

2. When choosing U and V you must check that $U \cap V$ is connected

3. When choosing U and V they may be both open or both closed or
even more general than that. After all open intervals and closed
intervals on the real line have the same homotopy type: they are
both contractible.

math.elmo.net.cn

{\lambda}^{3}-d \left(
{d}^{2}{c}^{2}{a}^{4}{y}^{2}-{b}^{2}{a}^{4}{z}^
{2}{c}^{2}+{a}^{2}{c}^{4}{b}^{4}-{c}^{2}{a}^{4}{y}^{2}{b}^{2}+{d}^{2}{
a}^{2}{y}^{2}{c}^{4}-{b}^{2}{x}^{2}{a}^{2}{c}^{4}+{c}^{2}{b}^{4}{a}^{4
}+{b}^{2}{a}^{4}{c}^{4}-{a}^{2}{y}^{2}{c}^{4}{b}^{2}+{d}^{2}{b}^{2}{a}
^{4}{z}^{2}+{d}^{2}{b}^{2}{x}^{2}{c}^{4}-{a}^{2}{b}^{4}{z}^{2}{c}^{2}-
{c}^{2}{x}^{2}{b}^{4}{a}^{2}+{d}^{2}{a}^{2}{b}^{4}{z}^{2}+{d}^{2}{c}^{
2}{x}^{2}{b}^{4} \right) {\lambda}^{2}+{d}^{2} \left(
-{x}^{2}{b}^{4}{
a}^{2}{c}^{4}+{x}^{2}{b}^{4}{c}^{4}{d}^{2}+{b}^{4}{a}^{4}{c}^{4}-{b}^{
4}{a}^{4}{z}^{2}{c}^{2}+{b}^{4}{a}^{4}{d}^{2}{z}^{2}-{a}^{4}{y}^{2}{c}
^{4}{b}^{2}+{d}^{2}{a}^{4}{y}^{2}{c}^{4} \right)  \left(
{x}^{2}{b}^{2
}{c}^{2}{d}^{2}-{b}^{2}{x}^{2}{a}^{2}{c}^{2}+{a}^{2}{c}^{2}{b}^{4}+{b}
^{2}{a}^{4}{c}^{2}+{a}^{2}{b}^{2}{c}^{4}+{z}^{2}{a}^{2}{b}^{2}{d}^{2}-
{a}^{2}{b}^{2}{c}^{2}{z}^{2}+{y}^{2}{a}^{2}{c}^{2}{d}^{2}-{a}^{2}{b}^{
2}{c}^{2}{y}^{2} \right) \lambda-{d}^{3}{b}^{2}{c}^{2}{a}^{2} \left(
-
{x}^{2}{b}^{4}{a}^{2}{c}^{4}+{x}^{2}{b}^{4}{c}^{4}{d}^{2}+{b}^{4}{a}^{
4}{c}^{4}-{b}^{4}{a}^{4}{z}^{2}{c}^{2}+{b}^{4}{a}^{4}{d}^{2}{z}^{2}-{a
}^{4}{y}^{2}{c}^{4}{b}^{2}+{d}^{2}{a}^{4}{y}^{2}{c}^{4} \right) ^{2}
