%\documentclass[12pt]{amsart}
\documentclass[10pt]{article}
\parindent=0pt \parskip=8pt

\usepackage{verbatim}
\usepackage{fullpage}

\usepackage{amssymb,amsmath,amsthm}
\usepackage[mathscr]{eucal}

\usepackage{makeidx}
%====================================================================
%-------------------------------------------------------------------- 
%
%   isomath.tex  LaTeX macros for math conforming to ISO standards
%
%---------------------------------------- this is a LaTeX-2e document
%====================================================================     

%% Blackboard Bold

\newcommand*{\bbb}[1]{\mathbb{#1}}
\newcommand{\bC}{\bbb{C}}
\newcommand{\bN}{\bbb{N}}
\newcommand{\bQ}{\bbb{Q}}
\newcommand{\bR}{\bbb{R}}
\newcommand{\bZ}{\bbb{Z}}
\newcommand{\bF}{\bbb{F}}
\newcommand{\bFp}{\bbb{F}_p}
\newcommand{\bFpn}{\bbb{F}_{p^n}}
\newcommand{\ii}{\mathbf{i}}
\newcommand{\jj}{\mathbf{j}}
\newcommand{\kk}{\mathbf{k}}

\newcommand{\N}{\bN}
\newcommand{\Q}{\bQ}
\newcommand{\R}{\bR}
\newcommand{\Z}{\bZ}
\newcommand{\ZpZ}{\Z/p\Z}
\newcommand{\ZmZ}{\Z/m\Z}
\newcommand{\C}{\bC}
%\newcommand{\F}{\bF}
\newcommand{\F}{\mathbb{F}}
\newcommand{\Fp}{\bFp}
\newcommand{\Fpn}{\bFpn}

%% Calligraphic

%\newcommand*{\cal}[1]{\mathcal{#1}}
%\newcommand{\cB}{\cal{B}}
%\newcommand{\cC}{\cal{C}}
%\newcommand{\cF}{\cal{F}}
%\newcommand{\cH}{\cal{H}}
%\newcommand{\cK}{\cal{K}}
%\newcommand{\cT}{\cal{T}}

\newcommand{\mC}{\mathcal{C}}
\newcommand{\mT}{\mathcal{T}}

%% Miscellaneous

%yuk -- Kaz's preferences
%\renewcommand{\phi}{\varphi}
%\renewcommand{\theta}{\vartheta}

%% Named Functions and Operators

\newcommand{\id}{\mathrm{id}}
\newcommand{\Id}{\mathrm{Id}}
\newcommand{\rank}{\mathrm{rank}}
\newcommand{\tr}{\mathrm{tr}}
\newcommand{\Aut}{\mathrm{Aut}}
\newcommand{\Gal}{\mathrm{Gal}}
\newcommand{\Hom}{\mathrm{Hom}}
\newcommand{\im}{\mathrm{im}}
\newcommand{\lcm}{\mathrm{lcm}}
\newcommand{\Tr}{\mathrm{Tr}}
\newcommand{\GL}{\mathrm{GL}}
%\newcommand{\mod}{\textrm{ mod }}

%\DeclareMathOperator{\cod}{Codom}
%\DeclareMathOperator{\dom}{Dom}
%\DeclareMathOperator{\im}{im}
%\DeclareMathOperator{\ran}{Ran}
%\DeclareMathOperator{\supp}{supp}

%\DeclareMathOperator*{\clsp}{\overline{span}}
%\DeclareMathOperator*{\spn}{span}

%% Vectors

\usepackage{amsbsy}
\renewcommand*{\vec}[1]{\boldsymbol{#1}}
\newcommand{\va}{\vec{a}}
\newcommand{\vb}{\vec{b}}
\newcommand{\vc}{\vec{c}}
\newcommand{\vx}{\vec{x}}
\newcommand{\vy}{\vec{y}}
\newcommand{\vz}{\vec{z}}
\newcommand{\vu}{\vec{u}}
\newcommand{\vv}{\vec{v}}
\newcommand{\vw}{\vec{w}}

%% Logical Propositions

\newcommand*{\pro}[1]{\mathsf{#1}}
\newcommand{\pP}{\pro{P}}
\newcommand{\pQ}{\pro{Q}}
\newcommand{\pR}{\pro{R}}

%% Matrices and Tensors

%\DeclareMathAlphabet{\mathsfsl}{OT1}{cmss}{m}{sl}
\newcommand*{\mat}[1]{\mathsfsl{#1}}
\newcommand{\mM}{\mat{M}}
\newcommand{\mD}{\mat{D}}
\newcommand{\mI}{\mat{I}}

%% Special Numbers and Characters

\newcommand{\re}{\mathrm{e}}
\newcommand{\ri}{\mathrm{i}}
\newcommand{\rd}{\mathrm{d}} % differential
\newcommand{\dif}[1]{\,\rd{#1}} % differential
\newcommand{\df}{\dif{f}}
\newcommand{\dx}{\dif{x}}
\newcommand{\dy}{\dif{y}}
\newcommand{\dz}{\dif{z}}
\newcommand{\dt}{\dif{t}}
\newcommand{\rpi}{\pi}       % actually, pi should be upright.

\newcommand{\rE}{\mathsf{E}}
\newcommand{\rF}{\mathsf{F}}
\newcommand{\rI}{\mathsf{I}}
\newcommand{\rK}{\mathsf{K}}
\newcommand{\rL}{\mathsf{L}}
\newcommand{\rM}{\mathsf{M}}
\newcommand{\rN}{\mathsf{N}}
\newcommand{\rP}{\mathsf{P}}
\newcommand{\rT}{\mathsf{T}}
\newcommand{\rU}{\mathsf{U}}
\newcommand{\rV}{\mathsf{V}}
\newcommand{\rW}{\mathsf{W}}
\newcommand{\rX}{\mathsf{X}}
\newcommand{\rY}{\mathsf{Y}}
\newcommand{\rZ}{\mathsf{Z}}


%\newcommand*{\pig}[1]{\langle #1 \rangle}

%====================================================================

%====================================================================
%
%  mathenv.tex  math environments  version 0.00
%
%====================================================================

\usepackage{ifthen}

\theoremstyle{plain}
%\newtheorem{thm}{Theorem}[subsection]
\newtheorem{thm}{Theorem}[section]
\newtheorem{cor}[thm]{Corollary}
\newtheorem{lem}[thm]{Lemma}
\newtheorem{prop}[thm]{Proposition}
\newtheorem{ax}[thm]{Axiom}
\newtheorem*{thm*}{Theorem}
\newtheorem*{cor*}{Corollary}
\newtheorem*{lem*}{Lemma}
\newtheorem*{prop*}{Proposition}
\newtheorem*{ax*}{Axiom}

\theoremstyle{definition}
\newtheorem{defn}[thm]{Definition}
\newtheorem{problem}{Problem}
\newtheorem*{defn*}{Definition}
\newtheorem*{problem*}{Problem}
\newtheorem{ex}[thm]{Example}
\newtheorem{exs}[thm]{Examples}
\newtheorem*{ex*}{Example}
\newtheorem*{exs*}{Examples}

%\theoremstyle{remark}
\newtheorem{rem}[thm]{Remark}
\newtheorem*{rem*}{Remark}
\newtheorem{aside}[thm]{Aside}
\newtheorem*{aside*}{Aside}
\newtheorem{intuition}[thm]{Intuition}
\newtheorem*{intuition*}{Intuition}
\newtheorem{notn}[thm]{Notation}
\newtheorem*{notn*}{Notation}
\newtheorem{conv}[thm]{Convention}
\newtheorem*{conv*}{Convention}
\newtheorem{interp}[thm]{Interpretation}
\newtheorem*{interp*}{Interpretation}
\newtheorem{mnem}[thm]{Mnemonic}
\newtheorem*{mnem*}{Mnemonic}
\newtheorem{exer}[thm]{Exercise}
\newtheorem*{exer*}{Exercise}

%\numberwithin{equation}{subsection}
\numberwithin{equation}{section}

%--------------------------------------------------------------------
%
%\newenvironment{problem}[1][]
% {\noindent{\bf#1.\ }}
% {}
%
%--------------------------------------------------------------------

\newenvironment{answer}[1][Answer]
  {\begin{proof}[#1]}
  {\renewcommand{\qedsymbol}{}\end{proof}}

\newenvironment{solution}[1][Solution]
  {\begin{proof}[#1]}
  {\renewcommand{\qedsymbol}{\ensuremath{\Diamond}}\end{proof}}

\newcounter{ppart}
\newenvironment{parts}
 {\begin{list}{\textup(\alph{ppart}\textup)\hfill}
   {\usecounter{ppart}
    \settowidth{\labelwidth}{\textup(m\textup)}
    \setlength{\leftmargin}{0cm} 
    \setlength{\rightmargin}{0cm}
    \setlength{\itemindent}{2em}
    \setlength{\labelsep}{\itemindent}
    \addtolength{\labelsep}{-\labelwidth}
    \renewcommand{\makelabel}[1]{\ifthenelse
     {\equal{##1}{}}
     {\textup{(\alph{ppart})}\hfill}
     {\textup{##1}\hfill}}
   }
 }
 {\end{list}}

\renewcommand{\labelenumi}{(\roman{enumi})}

\newcommand{\secref}[1]{Section~\textup{\ref{#1}}}
\newcommand{\thmref}[1]{Theorem~\textup{\ref{#1}}}
\newcommand{\corref}[1]{Corollary~\textup{\ref{#1}}}
\newcommand{\lemref}[1]{Lemma~\textup{\ref{#1}}}
\newcommand{\propref}[1]{Proposition~\textup{\ref{#1}}}
\newcommand{\defnref}[1]{Definition~\textup{\ref{#1}}}
\newcommand{\remref}[1]{Remark~\textup{\ref{#1}}}
\newcommand{\exref}[1]{Example~\textup{\ref{#1}}}
\newcommand{\exsref}[1]{Examples~\textup{\ref{#1}}}
\newcommand{\axref}[1]{Axiom~\textup{\ref{#1}}}
\newcommand{\itemref}[1]{\textup{{\ref{#1}}}}

%====================================================================


% Override the mathenv setting which I use for prolrev:
%\numberwithin{equation}{section}
\numberwithin{equation}{subsection}

\input{xy} % XY-pic package
\xyoption{all}
\usepackage{graphicx}
\usepackage{psfrag}

\usepackage{xr}
\usepackage{hyperref}
% When using the xr package, *both* documents must use hyperref.
% This is because hyperref redefines \label.

\newcommand*{\bra}[1]{\langle #1 \mid}
\newcommand*{\ket}[1]{\mid #1 \rangle}
\newcommand*{\braket}[2]{\langle #1 \mid #2 \rangle}
\newcommand*{\braCket}[3]{\langle #1 \mid #2 \mid #3 \rangle}

\newcommand{\psitilde}{\tilde\psi}

\newcommand{\Dn}{\mathcal{D}_n}
\newcommand{\Sn}{\mathcal{S}_n}
\newcommand{\An}{\mathcal{A}_n}
\newcommand{\Vf}{\mathcal{V}_4}
\newcommand{\Qe}{\mathcal{Q}_8}
\newcommand{\veps}{\varepsilon}
\newcommand*{\pig}[1]{\langle #1 \rangle}

\newcommand*{\bfidx}[1]{\index{#1}\textbf{#1}}
\newcommand*{\bfidxtwo}[2]{\textbf{#1}\index{#2}}
\newcommand*{\bfidxaux}[2]{\index{#1}\index{#2}\textbf{#1}}
\newcommand*{\emphidx}[1]{\index{#1}\emph{#1}}
\newcommand*{\plainidx}[1]{\index{#1}#1}

\newcommand{\cle}{\preccurlyeq}
\newcommand{\vecA}{\mathbf{A}}
\newcommand{\vecB}{\mathbf{B}}
\newcommand{\vecE}{\mathbf{E}}
\newcommand{\vecF}{\mathbf{F}}
\newcommand{\vecV}{\mathbf{V}}
\newcommand{\veca}{\mathbf{a}}
\newcommand{\vecb}{\mathbf{b}}
\newcommand{\vecc}{\mathbf{c}}
\newcommand{\vecd}{\mathbf{d}}
\newcommand{\vece}{\mathbf{e}}
\newcommand{\vecf}{\mathbf{f}}
\newcommand{\vecg}{\mathbf{g}}
\newcommand{\veci}{\mathbf{i}}
\newcommand{\vecj}{\mathbf{j}}
\newcommand{\veck}{\mathbf{k}}
\newcommand{\vecn}{\mathbf{n}}
\newcommand{\vecnhat}{\mathbf{\hat{n}}}
\newcommand{\vecp}{\mathbf{p}}
\newcommand{\vecq}{\mathbf{q}}
\newcommand{\vecr}{\mathbf{r}}
\newcommand{\vecs}{\mathbf{s}}
\newcommand{\vecu}{\mathbf{u}}
\newcommand{\vecv}{\mathbf{v}}
\newcommand{\vecw}{\mathbf{w}}
\newcommand{\vecx}{\mathbf{x}}
\newcommand{\vecy}{\mathbf{y}}
\newcommand{\vecz}{\mathbf{z}}

\newcommand{\ehat}{\hat{\vece}}
\newcommand{\nhat}{\hat{\vecn}}
\newcommand{\xhat}{\hat{\vecx}}
\newcommand{\yhat}{\hat{\vecy}}
\newcommand{\zhat}{\hat{\vecz}}

%\newcommand{\ihat}{\veci}
%\newcommand{\jhat}{\vecj}
%\newcommand{\khat}{\hat\veck}

\newcommand{\ihat}{\boldsymbol{\hat{\mathrm{\imath}}}}
\newcommand{\jhat}{\boldsymbol{\hat{\mathrm{\jmath}}}}
\newcommand{\khat}{\hat\veck}

\newcommand{\omegabar}{\overline{\omega}}
\newcommand{\etabar}{\overline{\eta}}

\newcommand{\xdot}{\dot{x}}
\newcommand{\ydot}{\dot{y}}
\newcommand{\zdot}{\dot{z}}

\newcommand{\xddot}{\ddot{x}}
\newcommand{\yddot}{\ddot{y}}
\newcommand{\zddot}{\ddot{z}}

\newcommand{\myvecx}{\mathbf{x}}
\newcommand{\veclambda}{\mathbf{\lambda}}

\newcommand{\adju}{\mathrm{adju}}
\newcommand{\curl}{\mathrm{curl}}
\newcommand{\ord}{\mathrm{ord}}
\newcommand{\rGL}{\mathrm{GL}}
\newcommand{\rSL}{\mathrm{SL}}
\renewcommand{\div}{\mathrm{div}}
\newcommand{\Alt}{\mathrm{Alt}}
\newcommand{\Sym}{\mathrm{Sym}}
\newcommand{\sgn}{\mathrm{sgn}}

\newcommand{\vecnq}{\mathbf{n}_\mathbf{q}}
\newcommand{\vecuq}{\mathbf{u}_\mathbf{q}}
\newcommand{\vecvq}{\mathbf{v}_\mathbf{q}}

\newcommand{\mcA}{\mathcal{A}}
\newcommand{\mcC}{\mathcal{C}}
\newcommand{\mcD}{\mathcal{D}}
\newcommand{\mcF}{\mathcal{F}}
\newcommand{\mcH}{\mathcal{H}}
\newcommand{\mcL}{\mathcal{L}}
\newcommand{\mcS}{\mathcal{S}}
\newcommand{\mcT}{\mathcal{T}}

\newcommand{\msC}{\mathscr{C}}
\newcommand{\msD}{\mathscr{D}}

\newcommand{\D}{\partial}
\newcommand{\DA}{\partial A}
\newcommand{\DB}{\partial B}
\newcommand{\DC}{\partial C}
\newcommand{\DX}{\partial X}
\newcommand{\Dc}{\partial c}
\newcommand{\Df}{\partial f}
\newcommand{\Dg}{\partial g}
\newcommand{\DG}{\partial G}
\newcommand{\Dh}{\partial h}
\newcommand{\DM}{\partial M}
\newcommand{\Dr}{\partial r}
\newcommand{\Dx}{\partial x}
\newcommand{\Dy}{\partial y}
\newcommand{\Dz}{\partial z}
\newcommand{\Ds}{\partial s}
\newcommand{\Dt}{\partial t}

\newcommand{\grad}{\nabla}
\newcommand{\tensor}{\otimes}
\newcommand{\into}{\hookrightarrow}
\newcommand{\longto}{\longrightarrow}

\newcommand{\from}{\leftarrow}
\newcommand{\longfrom}{\longleftarrow}

% Contraction operator.  I don't know the "right" way to declare this as a
% binary operator so that LaTeX will get the intersymbol spacing right for me.
% Hence, I am just using \, as a hack.
\newcommand{\ctt}{\,\lrcorner\,}

\newcommand*{\slashD}[1]{\partial/\partial #1}
\newcommand*{\fracD}[1]{\frac{\partial}{\partial #1}}
\newcommand*{\slashDtwo}[2]{\partial #1/\partial #2}
\newcommand*{\fracDtwo}[2]{\frac{\partial #1}{\partial #2}}

\newcommand{\slashDw}{\partial/\partial w}
\newcommand{\slashDx}{\partial/\partial x}
\newcommand{\slashDy}{\partial/\partial y}
\newcommand{\slashDz}{\partial/\partial z}
\newcommand{\slashDp}{\partial/\partial p}
\newcommand{\slashDq}{\partial/\partial q}
\newcommand{\slashDr}{\partial/\partial r}
\newcommand{\slashDtheta}{\partial/\partial \theta}
\newcommand{\slashdtheta}{d/d\theta}
\newcommand{\slashDphi}{\partial/\partial \phi}
\newcommand{\slashdphi}{d/d\phi}

\newcommand{\fracDw}{\frac{\partial}{\partial w}}
\newcommand{\fracDx}{\frac{\partial}{\partial x}}
\newcommand{\fracDy}{\frac{\partial}{\partial y}}
\newcommand{\fracDz}{\frac{\partial}{\partial z}}
\newcommand{\fracDp}{\frac{\partial}{\partial p}}
\newcommand{\fracDq}{\frac{\partial}{\partial q}}
\newcommand{\fracDr}{\frac{\partial}{\partial r}}
\newcommand{\fracDtheta}{\frac{\partial}{\partial \theta}}
\newcommand{\fracdtheta}{\frac{d}{d\theta}}
\newcommand{\fracDphi}{\frac{\partial}{\partial \phi}}
\newcommand{\fracdphi}{\frac{d}{d\phi}}

\newcommand{\fracDew}{\frac{\partial e}{\partial w}}
\newcommand{\eracDex}{\frac{\partial e}{\partial x}}
\newcommand{\fracDey}{\frac{\partial e}{\partial y}}
\newcommand{\fracDez}{\frac{\partial e}{\partial z}}

\newcommand{\fracDfw}{\frac{\partial f}{\partial w}}
\newcommand{\fracDfx}{\frac{\partial f}{\partial x}}
\newcommand{\fracDfy}{\frac{\partial f}{\partial y}}
\newcommand{\fracDfz}{\frac{\partial f}{\partial z}}

\newcommand{\fracDgw}{\frac{\partial g}{\partial w}}
\newcommand{\fracDgx}{\frac{\partial g}{\partial x}}
\newcommand{\fracDgy}{\frac{\partial g}{\partial y}}
\newcommand{\fracDgz}{\frac{\partial g}{\partial z}}

\newcommand{\fracDhw}{\frac{\partial h}{\partial w}}
\newcommand{\fracDhx}{\frac{\partial h}{\partial x}}
\newcommand{\fracDhy}{\frac{\partial h}{\partial y}}
\newcommand{\fracDhz}{\frac{\partial h}{\partial z}}

\newcommand{\fracDGw}{\frac{\partial G}{\partial w}}
\newcommand{\fracDGx}{\frac{\partial G}{\partial x}}
\newcommand{\fracDGy}{\frac{\partial G}{\partial y}}
\newcommand{\fracDGz}{\frac{\partial G}{\partial z}}

\newcommand{\fracDxs}{\frac{\partial x}{\partial s}}
\newcommand{\fracDxt}{\frac{\partial x}{\partial t}}
\newcommand{\fracDys}{\frac{\partial y}{\partial s}}
\newcommand{\fracDyt}{\frac{\partial y}{\partial t}}

\newcommand{\HdR}{H_{dR}}


\newcommand*{\slashd}[1]{d/d #1}
\newcommand*{\fracd}[1]{\frac{d}{d #1}}
\newcommand*{\slashdtwo}[2]{d #1/d #2}
\newcommand*{\fracdtwo}[2]{\frac{d #1}{d #2}}

\newcommand*{\vvectwo}[2]{\begin{pmatrix} #1 \\ #2 \end{pmatrix}}
\newcommand*{\vvecthree}[3]{\begin{pmatrix} #1 \\ #2 \\ #3 \end{pmatrix}}
\newcommand*{\pickpipess}[1]{\bigg|_{\scriptsize{#1}}}
\newcommand*{\pickpipet}[1]{\bigg|_{\tiny{#1}}}
\newcommand*{\pickpipe}[1]{\pickpipet{#1}}

\newcommand*{\rowvectwo}[2]{\begin{pmatrix}#1 & #2 \end{pmatrix}}
\newcommand*{\rowvecthree}[3]{\begin{pmatrix}#1 & #2 & #3\end{pmatrix}}
%\newcommand*{\colvectwo}[2]{\begin{pmatrix}#1 \\ #2 \end{pmatrix}}
%\newcommand*{\colvecthree}[3]{\begin{pmatrix}#1 \\ #2 \\ #3\end{pmatrix}}

\newcommand*{\colvecone}[1]{\left(\begin{array}{r} #1 \end{array}\right)}
\newcommand*{\colvectwo}[2]{\left(\begin{array}{r} #1 \\ #2 \end{array}\right)}
\newcommand*{\colvecthree}[3]{\left(\begin{array}{r} #1 \\ #2 \\ #3\end{array}\right)}
\newcommand*{\colvecfour}[4]{\left(\begin{array}{r} #1 \\ #2 \\ #3 \\ #4\end{array}\right)}

\newcommand*{\collvecone}[1]{\left(\begin{array}{l} #1 \end{array}\right)}
\newcommand*{\collvectwo}[2]{\left(\begin{array}{l} #1 \\ #2 \end{array}\right)}
\newcommand*{\collvecthree}[3]{\left(\begin{array}{l} #1 \\ #2 \\ #3\end{array}\right)}
\newcommand*{\collvecfour}[4]{\left(\begin{array}{l} #1 \\ #2 \\ #3 \\ #4\end{array}\right)}

\newcommand*{\colcvecone}[1]{\left(\begin{array}{c} #1 \end{array}\right)}
\newcommand*{\colcvectwo}[2]{\left(\begin{array}{c} #1 \\ #2 \end{array}\right)}
\newcommand*{\colcvecthree}[3]{\left(\begin{array}{c} #1 \\ #2 \\ #3\end{array}\right)}
\newcommand*{\colcvecfour}[4]{\left(\begin{array}{c} #1 \\ #2 \\ #3 \\ #4\end{array}\right)}

\newcommand*{\colrvecone}[1]{\left(\begin{array}{r} #1 \end{array}\right)}
\newcommand*{\colrvectwo}[2]{\left(\begin{array}{r} #1 \\ #2 \end{array}\right)}
\newcommand*{\colrvecthree}[3]{\left(\begin{array}{r} #1 \\ #2 \\ #3\end{array}\right)}
\newcommand*{\colrvecfour}[4]{\left(\begin{array}{r} #1 \\ #2 \\ #3 \\ #4\end{array}\right)}

% pmatrix and bmatrix are nice but they don't right-flush.

\newcommand*{\mxlsqtwo}[4]{\left(\begin{array}{ll}
	#1 & #2 \\
	#3 & #4 \\
	\end{array}\right)}
\newcommand*{\mxcsqtwo}[4]{\left(\begin{array}{cc}
	#1 & #2 \\
	#3 & #4 \\
	\end{array}\right)}
\newcommand*{\mxrsqtwo}[4]{\left(\begin{array}{rr}
	#1 & #2 \\
	#3 & #4 \\
	\end{array}\right)}

\newcommand*{\mxlsqthree}[9]{\left(\begin{array}{lll}
	#1 & #2 & #3 \\
	#4 & #5 & #6 \\
	#7 & #8 & #9 \\
	\end{array}\right)}
\newcommand*{\mxcsqthree}[9]{\left(\begin{array}{ccc}
	#1 & #2 & #3 \\
	#4 & #5 & #6 \\
	#7 & #8 & #9 \\
	\end{array}\right)}
\newcommand*{\mxrsqthree}[9]{\left(\begin{array}{rrr}
	#1 & #2 & #3 \\
	#4 & #5 & #6 \\
	#7 & #8 & #9 \\
	\end{array}\right)}

% Parenthesized matrix with specified alignment.
\newcommand*{\psamatrix}[2]{
	\left(\begin{array}{#1}
		#2
	\end{array}\right)
}

% Bracketed matrix with specified alignment.
\newcommand*{\bsamatrix}[2]{
	\left[\begin{array}{#1}
		#2
	\end{array}\right]
}

% Unparenthesized matrix with specified alignment.
\newcommand*{\usamatrix}[2]{
	\begin{array}{#1}
		#2
	\end{array}
}

% Parenthesized matrix with left alignment.
\newcommand*{\plmatrix}[1]{
	\left(\begin{array}{llllllllllll}
		#1
	\end{array}\right)
}

% Bracketed matrix with left alignment.
\newcommand*{\blmatrix}[1]{
	\left[\begin{array}{llllllllllll}
		#1
	\end{array}\right]
}

% Unparenthesized matrix with left alignment.
\newcommand*{\ulmatrix}[1]{
	\begin{array}{llllllllllll}
		#1
	\end{array}
}

% Parenthesized matrix with centered alignment.
\newcommand*{\pcmatrix}[1]{
	\left(\begin{array}{cccccccccccc}
		#1
	\end{array}\right)
}

% Bracketed matrix with centered alignment.
\newcommand*{\bcmatrix}[1]{
	\left[\begin{array}{cccccccccccc}
		#1
	\end{array}\right]
}

% Unparenthesized matrix with centered alignment.
\newcommand*{\ucmatrix}[1]{
	\begin{array}{cccccccccccc}
		#1
	\end{array}
}

% Parenthesized matrix with right alignment.
\newcommand*{\prmatrix}[1]{
	\left(\begin{array}{rrrrrrrrrrrr}
		#1
	\end{array}\right)
}

% Bracketed matrix with right alignment.
\newcommand*{\brmatrix}[1]{
	\left[\begin{array}{rrrrrrrrrrrr}
		#1
	\end{array}\right]
}

% Unparenthesized matrix with right alignment.
\newcommand*{\urmatrix}[1]{
	\begin{array}{rrrrrrrrrrrr}
		#1
	\end{array}
}

%\newcommand{\uscore}{\textunderscore}
\newcommand{\uscore}{\_}
\newcommand{\cupprod}{\smallsmile}

\newcommand{\qand}{\quad\textrm{and}\quad}
\newcommand{\qqand}{\qquad\textrm{and}\qquad}
\newcommand{\qor}{\quad\textrm{or}\quad}
\newcommand{\qqor}{\qquad\textrm{or}\qquad}
\newcommand{\qie}{\quad\textrm{i.e.}\quad}
\newcommand{\qqie}{\qquad\textrm{i.e.}\qquad}

\newcommand{\TqM}{T_\vecq M}
\newcommand{\TqsM}{T^*_\vecq M}

\newcommand{\tat}{{\textasciitilde}}


%\externaldocument[prolrev-]{prolrev}
\externaldocument[prolrev-]{prolrev}

\makeindex

%--------------------------------------------------------------------
\begin{document}
\title{Geometry/topology qualifying exam solutions}
\author{John Kerl}
\date{\today}

\maketitle
\begin{abstract}
The following are some solutions for old geometry/topology qualifying exams in
the University of Arizona math department.  It is a companion to my paper
\emph{Chains, forms, and duality}, which you should find near this file as
\texttt{prolrev.pdf}.
\end{abstract}

\addcontentsline{toc}{section}{Contents}
\tableofcontents

%% ================================================================
\newpage
\section{Disclaimer}

It might be argued that you should not read such a paper as this one --- that
it is better for your character to go through the qual packet and write up your
own solutions.  But you can do both.  Following the wise advice of the current
graduate director, working the packet problems is only the \emph{start} of your
qualifier preparation.  Once you've done that, you should spend the bulk of the
summer doing \emph{all conceivable variations of all problems in the packet}.
Thus, this paper merely sheds some light on the start of your preparation
process.

%% ----------------------------------------------------------------
\section{Acknowledgements}

Several solutions are due to folks in the summer 2007 prep sessions, including
Dan Champion, David Herzog, Chol Park, Victor Piercey, Jordan Schettler, and
Ryan Smith.  Solutions are my own unless otherwise indicated.

%% ----------------------------------------------------------------
\section{Topics and theorems}

The following is a list of topics compiled by the folks in the summer 2007
qual-prep sessions.  I hope that it is exhaustive; it is certainly more
accurate than the cover page of the qual packet.

Differential geometry topics:
\begin{itemize}
\item Tensors, forms, vector fields
\item Flows of vector fields, flow coordinates
\item Derivatives, Lie derivatives, Jacobian
\item Lie bracket
\item Tangent vectors and the tangent bundle
\item Vector calculus (all classical theorems)
\item Coordinate charts, parameterizations, atlas
\item Embedded submanifolds
\item Regular and critical values of a function
\item Lagrange multipliers
\item Volume/area forms
\item Pullback, pushforward
\item Stereographic projection (circle, sphere, hypersphere)
\item Contraction of forms with vector fields
\item Closed forms
\item Exact forms
\item Cartan's magic formula
\item Change of coordinates
\item One-parameter groups of diffeomorphisms
\item Partitions of unity
\end{itemize}

Differential geometry theorems:
\begin{itemize}
\item Regular value theorem
\item Rank theorem
\item Implicit function theorem
\item Inverse function theorem
\item Stokes' theorem (classical and general)
\item Green's theorem 
\item Divergence theorem
\end{itemize}

Algebraic topology topics:
\begin{itemize}
\item Classification of compact surfaces
\item Euler characteristic
\item Connect sum
\item Homology and cohomology groups
\item Fundamental group
\item Singular/cellular/simplicial homology
\item Mayer-Vietoris long exact sequences for homology and cohomology
\item Diagram chasing
\item Degree of maps from $\S^n$ to $\S^n$
\item Orientability, compactness
\item Top-level homology and cohomology
\item Reduced homology and cohomology
\item Relative homology
\item Homotopy and homotopy invariance
\item Deformation retract
\item Retract
\item Excision
\item K\"unneth formula
\item Factoring maps
\item Fundamental theorem of algebra
\end{itemize}

Algebraic topology theorems:
\begin{itemize}
\item Brouwer fixed point theorem
\item Poincar\'e lemma
\item Poincar\'e duality
\item de Rham theorem
\item Seifer-van Kampen theorem
\end{itemize}

Covering space theory topics:
\begin{itemize}
\item Covering maps
\item Free actions
\item Properly discontinuous action
\item Universal cover
\item Correspondence between covering spaces and subgroups of the fundamental
group of the base.
\item Lifting paths
\item Homotopy lifting property
\item Deck transformations
\item The action of the fundamental group
\item Normal/regular cover
\end{itemize}

Complex analysis topics:
\begin{itemize}
\item Contour integration
\item Conformal maps 
\item Conformal equivalence
\item Meromorphic, holomorphic, analytic functions
\item Linear fractional transformations
\item Radii of convergence of power series
\item Poles
\item Singularities and removable singularities
\item Residues
\item Cauchy-Riemann equations
\item Laurent series
\item Trig functions
\item Complex numbers (arithmetic, computation, polar, \ldots)
\end{itemize}

Complex analysis theorems:
\begin{itemize}
\item Liouville's theorem
\item Cauchy's residue theorem
\item Jordan's lemma
\item ML estimate
\item Maximum modulus principle
\end{itemize}

%% ================================================================
\section{January 2007}

%% ----------------------------------------------------------------
\subsection{January 2007 \#2.}
\label{sec:J07.2}

The exponential map, $\exp:\C\to\C^*$, is a covering space map. Suppose that
$X$ is a path-connected smooth manifold and that $\phi:X \to\C^*$ is a smooth
map.  Let $z=x+iy$ be the usual complex coordinate on $\C$.

(a) Show that the one-form
$$
	\eta = \mathrm{Re}\left(\frac{dz}{2\pi iz}\right)
$$
is a generator for the de Rham cohomology $H^1(\C^*)$.

(b) Show that if $\phi^*(\eta)=0$ in $H^1(X)$ then there exists a smooth
lift $\tilde\phi:X\to \C$ such that
$$
	\phi=\exp\tilde\phi.
$$

\textbf{Solution.}  For part (a):  the punctured complex plane
deformation-retracts to the unit circle, we know the cohomology of $\S^1$, and
cohomology is homotopy invariant.  Thus $H^1(\C^*)$ is one-dimensional.  I need
to show that $\eta$ is closed and non-exact.  For closedness, note that $\eta$
is a 1-form on a 2-dimensional real manifold so we do have something to
compute.  First observe that
\begin{eqnarray*}
	\eta &=& \mathrm{Re}\left(\frac{dz}{2\pi iz}\right)
	= \mathrm{Re}\left(\frac{dx+i\,dy}{2\pi i(x+iy)}\right) \\
	&=& \mathrm{Re}\left(\frac{(dx+i\,dy)(x-iy)}{2\pi i(x^2+y^2)}\right) \\
	&=& \mathrm{Re}\left(\frac{x\,dx+y\,dy -iy\,dx + ix\,dy}
		{2\pi i(x^2+y^2)}\right) \\
	&=& \mathrm{Re}\left(\frac{-ix\,dx-iy\,dy -y\,dx + x\,dy}
		{2\pi (x^2+y^2)}\right) \\
	&=& \frac{x\,dy - y\,dx}{2\pi (x^2+y^2)}.
\end{eqnarray*}
This is of course $d\theta/2\pi$; I could leave it at that.  Or, I can work it
out:
\begin{eqnarray*}
	d\left(\frac{x\,dy - y\,dx}{2\pi (x^2+y^2)}\right)
	&=& \frac{1}{2\pi}\left[\,d\left(\frac{x}{x^2+y^2}\right)\wedge dy
		- d\left(\frac{y}{x^2+y^2}\right)\wedge dx\right].
\end{eqnarray*}
Now
\begin{eqnarray*}
	d\left(\frac{x}{x^2+y^2}\right)
		&=& \frac{(x^2+y^2)\,dx - x(2x\,dx+2y\,dy)}{(x^2+y^2)^2}
		= \frac{(-x^2+y^2)\,dx - 2xy\,dy}{(x^2+y^2)^2} \\
	d\left(\frac{x}{x^2+y^2}\right) \wedge dy
		&=& \frac{(-x^2+y^2)\,dx \wedge dy}{(x^2+y^2)^2}
\end{eqnarray*}
and likewise
\begin{eqnarray*}
	d\left(\frac{y}{x^2+y^2}\right)
		&=& \frac{(x^2+y^2)\,dy - y(2x\,dx+2y\,dy)}{(x^2+y^2)^2}
		= \frac{(x^2-y^2)\,dy - 2xy\,dx}{(x^2+y^2)^2} \\
	d\left(\frac{y}{x^2+y^2}\right) \wedge dx
		&=& \frac{(x^2-y^2)\,dy \wedge dx}{(x^2+y^2)^2}
		= \frac{(-x^2+y^2)\,dx \wedge dy}{(x^2+y^2)^2}.
\end{eqnarray*}
Thus
\begin{eqnarray*}
	d\eta
	&=& \frac{1}{2\pi}
		\left[
		\frac{(-x^2+y^2)\,dx \wedge dy}{(x^2+y^2)^2} -
		\frac{(-x^2+y^2)\,dx \wedge dy}{(x^2+y^2)^2}
		\right]
	= 0.
\end{eqnarray*}

For non-exactness, there are a couple approaches.  The quickest is to cite the
Cauchy integral formula and recall that taking the real part commutes with
taking the integral.  Thus, $\oint \eta = 1$ where the path is, say, the unit
circle counterclockwise.  The other is to explicitly compute $\oint dz/2\pi iz$
using $z=e^{it}$ for $0 \le t \le 1$.  Equivalently,  use
$\eta=(-y\,dx+x\,dy)/2\pi$ on the unit circle and write out a path integral.
In all cases the result is the same.  We then use the usual argument that if
$\eta$ were closed then by Stokes' theorem its integral around a cycle would be
zero; contrapositively, since we found the integral around a cycle to be
non-zero, $\eta$ is non-exact.

Part (b) looks like the usual factoring problem but turns out to be a bit
different.  We have the following diagram:
$$
\xymatrix{
	{} & \C \ar[d]^{\exp} \\
	X \ar[r]^{\phi} \ar@{-->}[ur]^{\tilde\phi?} & \C^*.
}
$$
The lift criterion is that we need $\phi_*(\pi_1(X)) \le \exp_*(\pi_1(\C))$.
The latter is zero:  the complex plane is simply connected; it is the universal
cover of $\C^*$.  What about the former, though?  We are given only that
$\phi^*[\eta]=0$ in $H^1(X)$.  We need only to show that $\phi_*(\pi_1(X))=0$.

Since $[\eta]$ is a generator of $H^1(X)$, $\phi^*[\eta]=0$ means that $\phi^*$
is the zero map.  We can use this information about cohomology to extract
information about homology, using the duality induced by integration.  We need
to show that $\phi_*(\pi_1(X))=0$.  Suppose to the contrary.  Then there is a
loop $c$ in $X$ whose image under $\phi_*$ is homotopic to the
around-the-origin loop, $\alpha$, in $\C^*$.  That is, $\phi_*[c]=[\alpha]$ in
homotopy.  Also recall that pairing of cycles and forms is well-defined on
homology and cohomology classes
respectively, i.e. if $[\omega_1]=[\omega_2]$ and $[c_1]=[c_2]$ then
$$
	\int_{c_1}\omega_1 =
	\int_{c_1}\omega_2 =
	\int_{c_2}\omega_1 =
	\int_{c_2}\omega_2.
$$
Then $\phi_*$ pushes $[c]$ forward from $X$ to $\C^*$; $\phi^*$ pulls back
$[\eta]$ from $C^*$ to $X$:
$$
\xymatrix{
X \ar[rrrr]^{\phi} &                          & &           & \C^* \\
H_1(X)        &[c]\ar@{|->}[rr]^{\phi_*}  & & [\alpha] & H_1(\C^*) \\
H^1(X)        &0\ar@{<-|}[rr]^{\phi^*}  & & [\eta] & H^1(\C^*) \\
&              \int_c \phi^*(\eta) = \int_c 0 = 0 & &
	\int_{\phi_*(c)}\eta = \int_{\alpha}\eta = 1.
}
$$
Since $\phi_*(c)$ and $\alpha$ are cohomotopic, they are cohomologous.  By
naturality of pullback, we expect $\int_{c}\phi^*(\eta) =
\int_{\phi_*(c)}\eta$.  We already know that on the right-hande side, the form
$\eta$ integrated around $\alpha$ in $\C^*$ is 1.  Yet also we know that
$\phi^*[\eta]=[0]$ so the integral on the left-hand side is 0.  This is the
desired contradiction.

%(b) Show that if $\phi^*(\eta)=0$ in $H^1(X)$ then there exists a smooth
%lift $\tilde\phi:X\to \C$ such that
%$$
%	\phi=\exp\tilde\phi.
%$$

%% ----------------------------------------------------------------
\subsection{January 2007 \#5.}
\label{sec:J07.5}

Let $d\theta$ denote the one-form on $\S^1$ that is the derivative of
some local choice of angle $(x,y)=(\cos\theta,\sin\theta) \in \S^1$.
Let $\S^1 \ni (x,y) \mapsto p=\frac{x}{1-y}$ denote stereographic
projection from the north pole.  This is a coordinate defined for $y\ne 1$.
Find a function $f(p)$ so that
$$
	d\theta = f(p)dp.
$$

\textbf{Solution.}  First we have
$$
	d\theta = \frac{d\theta}{dp} dp.
$$
However, $dp/d\theta$ looks easy to compute so I'll find that first.
Since $(x,y) = (\cos\theta,\sin\theta)$ we have (using the quotient rule)
\begin{eqnarray*}
	\frac{dp}{d\theta}
		&=& \frac{d}{d\theta}\left(\frac{x}{1-y}\right) \\
	&=& \frac{d}{d\theta}\left(\frac{\cos\theta}{1-\sin\theta}\right) \\
	&=& \frac{(1-\sin\theta)(-\sin\theta) - \cos\theta(-\cos\theta)}{(1-\sin\theta)^2} \\
	&=& \frac{-\sin\theta+\sin^2\theta +\cos^2\theta}{(1-\sin\theta)^2} \\
	&=& \frac{1-\sin\theta}{(1-\sin\theta)^2} = \frac{1}{1-\sin\theta} \\
	&=& \frac{1}{1-y}.
\end{eqnarray*}
This has the problem half-done:  we now know $dp/d\theta$ but we still need it
in terms of $p$.  We know that $p=x/(1-y)$.  We need to eliminate $x$ in order
to be able to solve for $1-y$.  Our only relation between $x$ and $y$ is
$x^2+y^2=1$ so we will need to square things.  We have
\begin{eqnarray*}
	p^2 &=& \frac{x^2}{(1-y)^2} \\
	&=& \frac{1-y^2}{(1-y)^2} \\
	&=& \frac{(1-y)(1+y)}{(1-y)(1-y)} \\
	&=& \frac{1+y}{1-y}.
\end{eqnarray*}
Inverting this rational function we have
\begin{eqnarray*}
	p^2-p^2y &=& 1+y \\
	(p^2+1)y &=& (p^2-1) \\
	y &=& \frac{p^2-1}{p^2+1}.
\end{eqnarray*}
Then
\begin{eqnarray*}
	\frac{1}{1-y} &=& \frac{1}{1 - \frac{p^2-1}{p^2+1}} \\
	&=& \frac{p^2+1}{(p^2+1) - (p^2-1)} \\
	&=& \frac{p^2+1}{2}.
\end{eqnarray*}
So,
$$
	f(p) = \frac{2}{p^2+1}.
$$

%% ----------------------------------------------------------------
\subsection{January 2007 \#6.}

(a) Show that the subset $M$ of $\R^3$ defined by the equation
$$
	(1-z^2)(x^2+y^2)=1
$$
is a smooth submanifold of $\R^3$.

(b) Define a vector field on $\R^3$ by
$$
	V = z^2 x \fracDx + z^2 y \fracDy + z(1-z^2) \fracDz.
$$
Show that the restriction of $V$ to $M$ is a tangent vector field to $M$.

(c) The family of maps $\phi_t(x,y,z) = (cx-sy,sx+cy,z)$ with $c=\cos t$ and
$s=\sin t$ obviously restricts to a one-parameter family of diffeomorphisms
of $M$.  For each $t$, determine the vector field $(\phi_t)_* V$ on $M$.

\textbf{Solution.} Using the regular value theorem, it suffices to show that
the Jacobian of the left-hand side represents a surjective linear
transformation for all points of $M$.  This Jacobian is
$$
	\pcmatrix{
		2x(1-z^2) &
		2y(1-z^2) &
		-2z(x^2+y^2)
	}.
$$
Full rank would be 1; anything less is 0.  So, I need to find $(x,y,z)$
such that this matrix is $(0,0,0)$.  This requires:
\begin{eqnarray*}
	x=0 &\qor& z = \pm 1 \\
	y=0 &\qor& z = \pm 1 \\
	z=0 &\qor& x=y=0.
\end{eqnarray*}
If $z=\pm 1$ then we have $0=1$ on $M$, which is absurd.  Thus for the Jacobian
to be zero we need $x=y=z=0$ or $x=y=0$.  Either case also yields $0=1$ on $M$,
so the Jacobian never vanishes on $M$.  Therefore the inverse image of 1
is either the empty set or an embedded submanifold of $\R^3$; the latter
holds since visibly $(1,0,0) \in M$.

For part (b), it suffices to show that $V$ is perpendicular to a normal field
to $M$.  The latter is given by the gradient of the function defining $M$,
and the gradient in turn is (conveniently) the transpose of the Jacobian.
This is
$$
	\vecn =
	\pcmatrix{
		2x(1-z^2) \\
		2y(1-z^2) \\
		-2z(x^2+y^2)
	}\pickpipe{\colvecthree{x}{y}{z}}
	= 2x(1-z^2) \fracDx
	+ 2y(1-z^2) \fracDy
	- 2z(x^2+y^2) \fracDz.
$$
Then at each point $(x,y,z)$ of $M$,
$$
	\vecn \cdot V =
	2x^2z^2(1-z^2) +
	2y^2z^2(1-z^2) +
	-2z^2(x^2+y^2)(1-z^2) = 0.
$$

For part (c), recall the following for maps of manifolds:  If $F:M \to N$ then
in coordinates $DF:T_qM \to T_{F(q)}N$, where $DF$ is the coordinate
representation of $F_*$.  In particular, $(\phi_t)_*$ carries $V$ footed at $q$
to $D\phi_t V$ footed at $\phi_t q$.
Let
$$
	q=\colvecthree{x}{y}{z}.
$$
Then already we are given
$$
	\phi_t(q) = \prmatrix{
		\cos(t) & -\sin(t) & 0 \\
		\sin(t) &  \cos(t) & 0 \\
		0       &          & 1
	}
	\colvecthree{x}{y}{z}.
$$
Next, $\phi_t$ is already linear (for each fixed $t$) so it is its own
derivative.  Thus
$$
	(\phi_t)_* V =
	\prmatrix{
		\cos(t) & -\sin(t) & 0 \\
		\sin(t) &  \cos(t) & 0 \\
		0       &          & 1
	}
	\colcvecthree
		{z^2 x}
		{z^2 y}
		{z(1-z^2)
	}
	\pickpipe{
		\prmatrix{
			\cos(t) & -\sin(t) & 0 \\
			\sin(t) &  \cos(t) & 0 \\
			0       &          & 1
		}
		\colvecthree{x}{y}{z}
	}.
$$

\begin{center}* * *\end{center}

However, this is less than satisfying.  To see why, let
$$
	Y = -y\;\fracDx + x\;\fracDy
$$
be the vector field corresponding to the flow of $\phi_t$.
(To derive this, compute
$$
	Y = \frac{d\phi_t}{dt}\pickpipe{t=0}\colvecthree{x}{y}{z}
$$
as usual.)  Then, on all of $\R^3$, omitting second-order derivatives which
always cancel in the Lie bracket, and grouping things vertically for visual
convenience,
\begin{eqnarray*}
	L_Y V &=& [Y,V] = YV - VY \\
	&=&
		\prmatrix{
			-&y \;\slashDx \\
			+&x \;\slashDy
		}
		\prmatrix{
			& z^2 x    \;\slashDx \\
			+&z^2 y    \;\slashDy \\
			+&z(1-z^2) \;\slashDz
		}
		\;-\;
		\prmatrix{
			& z^2 x    \;\slashDx \\
			+&z^2 y    \;\slashDy \\
			+&z(1-z^2) \;\slashDz
		}
		\prmatrix{
			-&y \;\slashDx \\
			+&x \;\slashDy
		} \\
	&=&
		\prmatrix{
			-&yz^2 \;\slashDx \\
			+&xz^2 \;\slashDy
		}
		\;-\;
		\prmatrix{
			& z^2x \;\slashDy \\
			-&z^2y \;\slashDx
		} \\
	&=& 0.
\end{eqnarray*}
Since this is true on all of $\R^3$, it is certainly true on restriction to
$M$.  Since the rotational flow preserves $V$, we should be able to write
$(\phi_t)_*V$ as $V$ somehow.

I can (and have) done this for $\R^3$, restricting to $M$ later, and
restricting to $M$ first.  In this problem, both ways work out without too much
mess (although see section \ref{sec:A06.5} where restricting first makes an
enormous simplification).  I will show the latter approach.

In order to choose appropriate coordinates for a manifold, one must first know
the homotopy class of the manifold.  (I learned this the hard way on the first
geometry mid-term my first year.)  The equation defining $M$ is
$(x^2+y^2)(1-z^2)=1$.  Now, $x^2+y^2$ is non-negative; $(1-z^2)$ will be
positive (and thus the two terms can multiply to 1) only for $-1 < z < 1$ and
$x,y\ne 0$.  Letting $r^2=x^2+y^2$ we have $r = 1/\sqrt{1-z^2}$.  Plotting some
points and forming a surface of revolution shows [xxx insert figure here] and
so $M$ visibly has the homotopy class of a cylinder.  So, cylindrical
coordinates $\theta,z$ are a natural choice.  Let $\vecq \in M$.
Parameterizing by $\theta,z$ and taking directional derivatives as usual, we
have
$$
	\vecq = \colcvecthree{x}{y}{z}
		= \colcvecthree
		{\cos\theta/\sqrt{1-z^2}}
		{\sin\theta/\sqrt{1-z^2}}
		{z}, \quad
	\fracDtheta = \colcvecthree
		{-\sin\theta/\sqrt{1-z^2}}
		{\cos\theta/\sqrt{1-z^2}}
		{0}, \quad
	\fracDz = \colcvecthree
		{z\cos\theta/(1-z^2)^{3/2}}
		{z\sin\theta/(1-z^2)^{3/2}}
		{1}.
$$
In terms of these coordinates, the vector field is
$$
	V = \colcvecthree{z^2x}{z^2y}{z(1-z^2)}
	= \colcvecthree{z^2\cos\theta/\sqrt{1-z^2}}{z^2\sin\theta/\sqrt{1-z^2}}{z(1-z^2)}
	= z(1-z^2)\; \colcvecthree{z\cos\theta/(1-z^2)^{3/2}}{z\sin\theta/(1-z^2)^{3/2}}{1}
	= z(1-z^2)\fracDz.
$$

Now replace $(x,y,z)$ with $(cx-sy,sx+cy,z)$ where $c=\cos t$ and $s=\sin t$.
We have
$$
	\vecq = \colcvecthree{\cos t x - \sin t y}{\sin t x + \cos t y}{z}
		= \colcvecthree
		{\frac{\cos t \cos\theta - \sin t \sin \theta}{\sqrt{1-z^2}}}
		{\frac{\sin t \cos\theta + \cos t \sin\theta}{\sqrt{1-z^2}}}
		{z}
$$
which we recognize from trigonometry as
$$
	\vecq = \colcvecthree
		{\cos(t+\theta)/\sqrt{1-z^2}}
		{\sin(t+\theta)/\sqrt{1-z^2}}
		{z}.
$$
Then
$$
	\fracDtheta = \colcvecthree
		{-\sin(t+\theta)/\sqrt{1-z^2}}
		{\cos(t+\theta)/\sqrt{1-z^2}}
		{0} \qqand
	\fracDz = \colcvecthree
		{z\cos(t+\theta)/(1-z^2)^{3/2}}
		{z\sin(t+\theta)/(1-z^2)^{3/2}}
		{1}.
$$
In terms of these coordinates, the vector field is
\begin{eqnarray*}
	V &=& \colcvecthree{z^2(\cos t x - \sin t y)}{z^2(\sin t x + \cos t y)}{z(1-z^2)}
	= \colcvecthree{z^2\cos(t+\theta)/\sqrt{1-z^2}}{z^2\sin(t+\theta)/\sqrt{1-z^2}}{z(1-z^2)} \\
	&=& z(1-z^2)\; \colcvecthree{z\cos(t+\theta)/(1-z^2)^{3/2}}{z\sin(t+\theta)/(1-z^2)^{3/2}}{1}
	= z(1-z^2)\fracDz
\end{eqnarray*}
which is what I wanted.

%% ----------------------------------------------------------------
\subsection{January 2007 \#7.}

Find the critical points and critical values for the function $f(x,y,z) =
x^2+y^2-z$ restricted to the two-sphere $x^2+y^2+z^2=1$.

\textbf{Solution.}  Critical points occur where the Jacobian of $f$ has less
than full rank, i.e. rank 1.  This means that the Jacobian of $f$ must have
rank 0, i.e. vanish.  Since the Jacobian of $f$ is the transpose of $f$, it
suffices to find where $\grad f$ restricted to $\S^2$ vanishes.  Lagrange
multipliers were developed for precisely this purpose.

Let $g(x,y,z)=x^2+y^2+z^2-1$.  We can use Lagrange multipliers as long as $f$
and $g$ have continuous first partials and as long as $\grad g$ vanishes
nowhere.  The first is true because $f$ and $g$ are polynomials, hence smooth;
the latter is true because $\grad g = (2x,2y,2z)^t$ which vanishes only at the
origin, which is not in $\S^2$.

We now compute $\grad f = \lambda \grad g$:
\begin{eqnarray*}
	2x &=& \lambda 2x \\
	2y &=& \lambda 2y \\
	-1 &=& \lambda 2z.
\end{eqnarray*}
Then:
\begin{itemize}
\item If $x=y=0$ then $\lambda$ can be anything, but this forces (on $\S^2$)
$z= \pm 1$.
	\begin{itemize}
	\item In the first case ($z=1$ and the critical point is the north pole),
	the critical value is $f(0,0,1)$ = $-1$.
	\item In the second case ($z=-1$ and the critical point is the south pole),
	the critical value is $f(0,0,1)$ = $1$.
	\end{itemize}

\item Otherwise, either $x$ or $y$ is non-zero and we have $\lambda=1$.  Then
$z=-1/2$, and $x$ and $y$ are free to vary at that latitude, i.e. the circle
$x^2+y^2=3/4$ at $z=-1/2$.  Here, the critical value is $f(x,y,z)=
3/4+1/2=5/4$.
\end{itemize}

%\begin{center}* * *\end{center}
%
%Here is an alternative approach.  Since $x^2+y^2+z^2=1$ on $\S^2$, we may
%write $f(z)=1-z-z^2$.  Then $f'(z)=-2z-1$, which is zero at $z=-1/2$.

%% ================================================================
\newpage
\section{August 2006}

\begin{center}
\emph{
Failure is instructive. The person who really thinks learns quite as much from
his failures as from his successes.
}
--- John Dewey.
\end{center}

%% ----------------------------------------------------------------
\subsection{August 2006 \#1.}
\textbf{Solve only one of the following two problems.}

\textbf{1A.}  Compute the following integral:
$$
	\int_0^\infty \frac{\cos(x)}{1+x^4}\,dx.
$$

\textbf{1B.}  Find a conformal mapping of the vertical semi-infinite strip
$\{0 < \mathrm{Re}(z) < 1, 0 < \mathrm{Im}(z)\}$ onto the unit disk
$|w| < 1$.

\textbf{Solution for question 1A.}  Overview:  As usual, we will include the
desired real integral as one leg of a closed loop in the complex plane, then
use Cauchy's residue theorem to evaluate the integral around that loop.

Let
$$
	I = \int_0^\infty \frac{\cos(x)}{1+x^4}\,dx
	= \frac{1}{2}\;\int_{-\infty}^\infty \frac{\cos(x)}{1+x^4}\,dx
	= \frac{1}{2}\;\int_{-\infty}^\infty \frac{e^{ix}}{1+x^4}\,dx
$$
where the first equality is due to the evenness of the integrand and the second
is due to the oddness of $\sin(x)/(x^4+1)$.

Let $C_1$ be the path along the real axis from $-R$ to $0$;
let $C_2$ be the path along the real axis from $0$ to $R$;
let $C_3$ be the semicircular arc of radius $R$ in the upper half-plane,
centered about the origin.  [xxx include figure here.]
Note that $C=C_1+C_2+C_3$ is a simple loop enclosing the two poles $z=\zeta_8,
\zeta_8^3$ of the integrand, where
$$
	\zeta_8 = \frac{1+i}{\sqrt{2}}
$$
is the principal eighth root of unity.
Let
$$
	f(z) = \frac{e^{iz}}{z^4+1}.
$$
Note that
$$
	\lim_{R \to \infty}\int_{C_1} f(z) \,dz =
	\lim_{R \to \infty}\int_{C_2} f(z) \,dz =
	I,
$$
whereas I claim that
$$
	\lim_{R \to \infty}\int_{C_3} = 0.
$$
To see this, note that in the upper half-plane, i.e. for $y\ge 0$, we have
$$
	|e^{iz}| =
	|e^{i(x+iy)}| =
	|e^{-y}e^{ix}| =
	|e^{-y}|\;|e^{ix}| =
	|e^{-y}| \le 1.
$$
Meanwhile, for large $R=|z|$ the modulus of the denominator is well estimated
by $R^4$.  Note that the length of the path $C_3$ is $\pi R$.
Using the ML estimate, we have
$$
	\int_{C_3} f \le ML = \frac{\pi R}{R^4} = \frac{\pi }{R^3} \to 0
$$
as $R \to \infty$.

Using Cauchy's residue theorem, we have
$$
	\oint_{C_1+C_2+C_3} = I+I+0 = 2I = 2\pi i \sum \Res f
$$
i.e.
$$
	I = \pi i \sum \Res f
$$
where the sum is over $z=\zeta_8, \zeta_8^3$.

Note that both $\zeta_8$ and $\zeta_8^3$ are simple poles of $f$.  Thus we may
apply the usual formula, differentiating the denominator, to obtain
$$
	\Res_{z=\zeta_8} f = \frac{e^{iz}}{4z^3}\pickpipe{z=\zeta_8}
$$
and likewise for the other pole.  Doing the first pole, we have
$$
	\Res_{z=\zeta_8} f = \frac{e^{i\zeta_8}}{4\zeta_8^3}.
$$
Recall that
$$
	\zeta_8^3=\frac{-1+i}{\sqrt 2}
	\qand
	i\zeta_8=\frac{-1+i}{\sqrt 2}.
$$
Also,
\begin{eqnarray*}
	\exp(i\zeta_8) = \exp\left(\frac{-1+i}{\sqrt 2}\right)
	= \exp\left(\frac{-1}{\sqrt 2}\right) \exp\left(\frac{i}{\sqrt 2}\right)
	= \exp\left(\frac{-1}{\sqrt 2}\right) \left(\cos(1/\sqrt 2) + i\sin(1/\sqrt 2)\right).
\end{eqnarray*}
For brevity, let
$$
	c=\cos(1/\sqrt 2) \qand s=\sin(1/\sqrt 2).
$$
Then
$$
	\Res_{z=\zeta_8} f = \frac{ e^{-1/\sqrt 2}(c+is)}{2\sqrt 2 (-1+i)}.
$$

For the other pole, we have
$$
	\Res_{z=\zeta_8^3} f = \frac{e^{i\zeta_8^3}}{4\zeta_8^9}.
$$
Notice that
$$
	\zeta_8^9=\zeta_8=\frac{1+i}{\sqrt 2}
	\qand
	i\zeta_8^3=\frac{-1-i}{\sqrt 2}.
$$
Then
\begin{eqnarray*}
	\exp(i\zeta_8^3) = \exp\left(\frac{-1-i}{\sqrt 2}\right)
	&=& \exp\left(\frac{-1}{\sqrt 2}\right) \exp\left(\frac{-i}{\sqrt 2}\right)
	= \exp\left(\frac{-1}{\sqrt 2}\right) \left(\cos(-1/\sqrt 2) + i\sin(-1/\sqrt 2)\right) \\
	&=& \exp\left(\frac{-1}{\sqrt 2}\right) \left(\cos(1/\sqrt 2) - i\sin(1/\sqrt 2)\right)
	= \exp\left(\frac{-1}{\sqrt 2}\right) (c-is).
\end{eqnarray*}
Then
$$
	\Res_{z=\zeta_8^3} f = \frac{ e^{-1/\sqrt 2}(c-is)}{2\sqrt 2 (1+i)}.
$$

The sum of residues is
\begin{eqnarray*}
	\frac{e^{-1/\sqrt 2}(c+is)}{2\sqrt 2 (-1+i)} +
	\frac{e^{-1/\sqrt 2}(c-is)}{2\sqrt 2 (1+i)}
%
	&=& \frac{e^{-1/\sqrt 2}}{2\sqrt 2} \left(
		\frac{c+is}{-1+i} + \frac{c-is}{1+i}
	\right) \\
%
	&=& \frac{e^{-1/\sqrt 2}}{2\sqrt 2} \left(
		\frac{(c+is)(1+i) \;+\; (c-is)(-1+i)}{(-1+i)(1+i)}
	\right) \\
%
	&=& \frac{-e^{-1/\sqrt 2}}{4\sqrt 2} \left(
		c+ic+is-s \; -c+ic+is+s
	\right) \\
%
	&=& \frac{-ie^{-1/\sqrt 2}}{2\sqrt 2} (c+s).
\end{eqnarray*}
Then finally
\begin{eqnarray*}
	I &=& \pi i \sum \Res f \\
	&=& \pi i \,\cdot\, \frac{-ie^{-1/\sqrt 2}}{2\sqrt 2} (c+s) \\
	&=& \frac{\pi e^{-1/\sqrt 2}}{2\sqrt 2} (\cos(1/\sqrt{2}) + \sin(1/\sqrt{2})) \\
	&\approx& 0.772
\end{eqnarray*}
which checks out numerically on my TI-83.

\textbf{Solution for question 1B.}

\ldots

%% ----------------------------------------------------------------
\subsection{August 2006 \#2.}
Compute the singular homology groups $H_*(X, \Z)$ of the space
$X=\R^3\setminus A$, where $A$ is a subset of $\R^3$ homeomorphic to the
disjoint union of two unlinked circles.

%% ----------------------------------------------------------------
\subsection{August 2006 \#3.}
Consider the following map $f:\R^3 \to \R^2$:
$$
	\colvecthree{x}{y}{z} \mapsto \colvectwo{xz-y^2}{yz-x^2}.
$$
For which values $(a,b) \in \R^2$ of $f$ is the level set $f^{-1}(a, b)$
a smooth submanifold of $\R^3$?

%% ----------------------------------------------------------------
\subsection{August 2006 \#4.}
Consider the surface $\Sigma$ obtained by identifying the edges
of a square in the following way:

[xxx figure here]

\begin{enumerate}
\item[(a)] Construct a model of the universal covering space of this surface,
indicating especially how $\pi_1(\Sigma, v)$ acts.
\item[(b)] Identify the covering space $X$ of $\Sigma$, which corresponds to
the subgroup of $\pi_1(\Sigma, v)$ generated by $a$ and describe the group
of covering automorphisms of $X$.
\end{enumerate}

%% ----------------------------------------------------------------
\subsection{August 2006 \#5.}
\label{sec:A06.5}

Consider the submanifold $\iota:M \into \R^3$ given by $x^2+y^2-z^2=1$.

\begin{enumerate}
\item[(a)] Show that the vector field $X= \frac{xz}{1+z^2} \frac{\D}{\Dx} +
\frac{yz}{1+z^2} \frac{\D}{\Dy} + \frac{\D}{\Dz}$ is tangent to $M$, i.e. that
there exists a vector field $Y$ on $M$ such that for any $m \in M$ we have
$\iota_*(Y(m)) = X(m)$.
\item[(b)] Show that the two-form
$\omega= x\,dy\wedge dz \;+\; y\,dz\wedge dx \;+\; z\,dx\wedge dy$
restricts to an area form on $M$, i.e. a two-form which never vanishes.
(\emph{Hint:}  use cylindrical coordinates.)
\item[(c)] Does the flow of $Y$ on $M$ preserve $\iota^*(\omega)$?
\end{enumerate}

\textbf{Solution.}  For part (a), it suffices to show that $X$ is perpendicular
to a normal to $M$.  A normal to $M$ is given by the gradient of the function
$(x,y,z)\mapsto x^2+y^2-z^2$, namely, $(2x, 2y, -2z)^t$.  To avoid factors of
two I may as well consider the normal field
$$
	\vecn = \colrvecthree{x}{y}{-z}\pickpipe{\colvecthree{x}{y}{z}}.
$$
Then at each point of $M$,
\begin{eqnarray*}
	\vecn \cdot X &=& \colrvecthree{x}{y}{-z} \cdot
		\colcvecthree{\frac{xz}{1+z^2}}{\frac{yz}{1+z^2}}{1} \\
	&=& \frac{x^2z}{1+z^2} + \frac{y^2z}{1+z^2} - z \\
	&=& \frac{x^2z}{1+z^2} + \frac{y^2z}{1+z^2} - \frac{z(1+z^2)}{1+z^2} \\
	&=& \frac{z(x^2+y^2-z^2-1)}{1+z^2}
\end{eqnarray*}
which vanishes on $M$.

For part (b), note that $M$ is a hyperboloid of one sheet, which is homotopy
equivalent to a cylinder.  Thus, $\theta,z$ coordinates are a good choice.  Use
$$
	\colcvecthree{x}{y}{z} = \colcvecthree{r\cos\theta}{r\sin\theta}{z}
	= \colcvecthree{\sqrt{1+z^2}\cos\theta}{\sqrt{1+z^2}\sin\theta}{z}
$$
since $r^2=x^2+y^2=1+z^2$ on $M$.  Then
\begin{eqnarray*}
	i^*(dx) &=& -\sqrt{1+z^2}\sin\theta \,d\theta + \frac{z}{\sqrt{1+z^2}}\cos\theta \,dz \\
	i^*(dy) &=&  \sqrt{1+z^2}\cos\theta \,d\theta + \frac{z}{\sqrt{1+z^2}}\sin\theta \,dz \\
	i^*(dz) &=& dz
\end{eqnarray*}
\begin{eqnarray*}
	i^*(    dy \wedge dz) &=& \sqrt{1+z^2}\cos\theta \,d\theta \wedge dz \\
	i^*(x\, dy \wedge dz) &=& (1+z^2)\cos^2\theta \,d\theta \wedge dz
\end{eqnarray*}
\begin{eqnarray*}
	i^*(    dz \wedge dx) &=& \sqrt{1+z^2}\sin\theta \,d\theta \wedge dz \\
	i^*(y\, dz \wedge dx) &=& (1+z^2)\sin^2\theta \,d\theta \wedge dz
\end{eqnarray*}
\begin{eqnarray*}
	i^*(    dx \wedge dy) &=& -z \sin^2\theta \, d\theta\wedge dz
		-z \cos^2\theta \, d\theta\wedge dz \\
	&=& -z \, d\theta\wedge dz \\
	i^*(z\, dx \wedge dy) &=& -z^2 \, d\theta\wedge dz
\end{eqnarray*}
\begin{eqnarray*}
	i^*(\omega) &=& \left( (1+z^2) -z^2 \right) \,d\theta\wedge dz \\
	&=& \,d\theta\wedge dz
\end{eqnarray*}
which is non-vanishing since its coefficient is constant 1.

For part (c), I experimented a few different ways.  I computed $L_X(\omega)$,
then tried to pull that back to $M$.  However, with or without Cartan's magic
formula, the algebra is a horrid mess.  This is the kind of thing one learns
only \emph{after} wasting an hour or two on it, and during an exam we don't
have that much time to spare.  The clue is that $X$ and $\omega$ each have
three terms in them, and $\omega$ isn't top-level so one needs to compute
$d\omega$ in the magic formula:  both these facts contribute to the mess.  On
the other hand, we've seen that $i^*(\omega)$ has only one term, and is already
top-level for $M$.  The lesson learned is that it works out much nicer to
\emph{first} pull back, \emph{then} compute the Lie derivative
$L_Y(i^*\omega)$.

To do this, though, we need to find out what $Y$ is in terms of $\slashDz$ and
$\slashDtheta$.  Recall that $X$ is
$$
	\left(x\,\fracDx \,+\, y\,\fracDy\right)\, \frac{z}{1+z^2} \,+\, \fracDz.
$$
The sum of the first two terms points straight out radially from the $z$ axis
(with magnitude dependent on $z$), and the last term points up.  Thus $X$ has
no rotational component.  Since the tangent bundle to $M$ is spanned by
$\slashDz$ and $\slashDtheta$, and since we've just seen that the coefficient
on the latter must be zero, $Y$ must be a multiple of $\slashDz$ alone.

To find out what $\slashDz$ is, proceed as usual:  parameterize $M$ by
$\theta$ and $z$, then apply the directional derivative.  We have
\begin{eqnarray*}
	\slashDz &=& \colcvecthree
		{\fracDz\left(\sqrt{1+z^2}\cos\theta\right)}
		{\fracDz\left(\sqrt{1+z^2}\sin\theta\right)}
		{\fracDz\left(z\right)} \\
	&=& \colcvecthree
		{\frac{z}{\sqrt{1+z^2}}\cos\theta}
		{\frac{z}{\sqrt{1+z^2}}\sin\theta}
		{1}.
\end{eqnarray*}

Now, $X$ is
$$
	\colcvecthree{\frac{xz}{1+z^2}}{\frac{yz}{1+z^2}}{1}.
$$
The $z$-coordinate of $\slashDz$ is the same as the $z$-coordinate of $X$,
suggesting that $Y=\slashDz$, but let's nail this to the floor by writing out
$X$ in cylindrical coordinates and restricting to $M$:
\begin{eqnarray*}
	Y &=& \colcvecthree{\frac{xz}{1+z^2}}{\frac{yz}{1+z^2}}{1}\pickpipe{M}
	= \colcvecthree
		{\frac{z\sqrt{1+z^2}\cos\theta}{1+z^2}}
		{\frac{z\sqrt{1+z^2}\sin\theta}{1+z^2}}
		{1}
	= \colcvecthree
		{\frac{z\cos\theta}{\sqrt{1+z^2}}}
		{\frac{z\sin\theta}{\sqrt{1+z^2}}}
		{1}
	= \slashDz.
\end{eqnarray*}

Given that we know what $Y$ and $i^*(\omega)$ are, showing that the flow of $Y$
preserves $i^*(\omega)$ is quick and easy:
\begin{eqnarray*}
	L_Y(i^*(\omega)) &=& \fracDz \ctt (d(d\theta\wedge dz)) + d\left(\fracDz\ctt(d\theta\wedge dz)\right) \\
	&=& d\left(\fracDz \ctt(d\theta\wedge dz)\right) \\
	&=& d\left(-d\theta)\right) \\
	&=& 0.
\end{eqnarray*}

%I will compute the Lie derivative of $\omega$ in the
%direction of $X$.  Since
%$$
%	X= \frac{xz}{1+z^2} \frac{\D}{\Dx}
%	+ \frac{yz}{1+z^2} \frac{\D}{\Dy}
%	+ \frac{\D}{\Dz},
%$$
%I have
%\begin{eqnarray*}
%	L_X(x) &=& \frac{xz}{1+z^2}, \\
%	L_X(y) &=& \frac{yz}{1+z^2}, \\
%	L_X(z) &=& 1.
%\end{eqnarray*}
%Using the product rule for Lie derivatives, and noting that the Lie
%derivative commutes with $d$ for exact forms, I have for the first term
%of $\omega$
%\begin{eqnarray*}
%	L_X(x\, dy \wedge dz)
%		&=& L_X(x)\, dy \wedge dz
%		\;+\; x\, d(L_X(y)) \wedge dz
%		\;+\; x\, dy \wedge d(L_X(z))  \\
%	%
%		&=& \frac{xz}{1+z^2}\, dy \wedge dz
%		\;+\; x\, d\left(\frac{yz}{1+z^2}\right) \wedge dz
%		\;+\; x\, dy \wedge dz \\
%	%
%		&=& \frac{xz}{1+z^2}\, dy \wedge dz
%		\;+\; \frac{xz}{1+z^2}\, dy \wedge dz
%		\;+\; x\, dy \wedge dz \\
%	%
%		&=& \frac{2xz}{1+z^2}\, dy \wedge dz
%		\;+\; x\, dy \wedge dz \\
%	%
%		&=& \frac{2xz + x+xz^2}{1+z^2}\, dy \wedge dz
%	%
%		\;=\; \frac{x(1+2z+z^2)}{1+z^2}\, dy \wedge dz \\
%	%
%		&=& \frac{x(1+z)^2}{1+z^2}\, dy \wedge dz.
%\end{eqnarray*}
%For the second term:
%\begin{eqnarray*}
%	L_X(y\, dz \wedge dx)
%		&=& L_X(y)\, dz \wedge dx
%		\;+\; y\, d(L_X(z)) \wedge dx
%		\;+\; y\, dz \wedge d(L_X(x))  \\
%	%
%		&=& \frac{yz}{1+z^2}\, dz \wedge dx
%		\;+\; y\, dz \wedge dx
%		\;+\; y\, dz \wedge d\left(\frac{xz}{1+z^2}\right)  \\
%	%
%		&=& \frac{yz}{1+z^2}\, dz \wedge dx
%		\;+\; y\, dz \wedge dx
%		\;+\; \frac{yz}{1+z^2}\, dz \wedge dx  \\
%	%
%		&=& \frac{2yz}{1+z^2}\, dz \wedge dx
%		\;+\; y\, dz \wedge dx \\
%	%
%		&=& \frac{2yz + y + yz^2}{1+z^2}\, dz \wedge dx
%	%
%		\;=\; \frac{y(1+2z+z^2)}{1+z^2}\, dz \wedge dx \\
%	%
%		&=& \frac{y(1+z)^2}{1+z^2}\, dz \wedge dx.
%\end{eqnarray*}
%For the third term:
%\begin{eqnarray*}
%	L_X(z\, dx \wedge dy)
%		&=& L_X(z)\, dx \wedge dy
%		\;+\; z\, d(L_X(x)) \wedge dy
%		\;+\; z\, dx \wedge d(L_X(y))  \\
%	%
%		&=& dx \wedge dy
%		\;+\; z\, d\left(\frac{xz}{1+z^2}\right) \wedge dy
%		\;+\; z\, dx \wedge d\left(\frac{yz}{1+z^2}\right).
%\end{eqnarray*}
%Splitting out some work, I have
%\begin{eqnarray*}
%	d\left(\frac{xz}{1+z^2}\right)
%		&=& \frac{z}{1+z^2}\, dx
%		\;+\;
%		x\, \frac{(1+z^2)-2z^2}{(1+z^2)^2}\, dz \\
%	%
%		&=& \frac{z}{1+z^2}\, dx
%		\;+\;
%		x\, \frac{1-z^2}{(1+z^2)^2}\, dz \\
%	%
%		&=& \frac{z}{1+z^2}\, dx
%		\;+\;
%		\frac{x-xz^2}{(1+z^2)^2}\, dz
%\end{eqnarray*}
%and likewise
%\begin{eqnarray*}
%	d\left(\frac{yz}{1+z^2}\right)
%		&=& \frac{z}{1+z^2}\, dy
%		\;+\;
%		y\, \frac{(1+z^2)-2z^2}{(1+z^2)^2}\, dz \\
%	%
%		&=& \frac{z}{1+z^2}\, dy
%		\;+\;
%		y\, \frac{1-z^2}{(1+z^2)^2}\, dz \\
%	%
%		&=& \frac{z}{1+z^2}\, dy
%		\;+\;
%		\frac{y-yz^2}{(1+z^2)^2}\, dz.
%\end{eqnarray*}
%Then the third term is now
%\begin{eqnarray*}
%	L_X(z\, dx \wedge dy)
%	&=& dx \wedge dy \\
%	%
%	&+& z\,
%	\left(\frac{z}{1+z^2}\, dx \;+\; \frac{x-xz^2}{(1+z^2)^2}\, dz\right)
%	\wedge dy \\
%	%
%	&+& z\, dx \wedge
%	\left(\frac{z}{1+z^2}\, dy \;+\; \frac{y-yz^2}{(1+z^2)^2}\, dz\right) \\
%&&\\%
%	&=& dx \wedge dy \\
%	%
%	&+&   \frac{z^2}{1+z^2}\, dx \wedge dy
%	\;-\; \frac{zx(1-z^2)}{(1+z^2)^2}\, dy \wedge dz \\
%	%
%	&+&   \frac{z^2}{1+z^2}\, dx \wedge dy
%	\;-\; \frac{zy(1-z^2)}{(1+z^2)^2}\, dz \wedge dx \\
%&&\\%
%	&=& \frac{1+2z^2}{1+z^2}\, dx \wedge dy \\
%	&-&   \frac{zx(1-z^2)}{(1+z^2)^2}\, dy \wedge dz
%	\;-\; \frac{zy(1-z^2)}{(1+z^2)^2}\, dz \wedge dx.
%\end{eqnarray*}
%Combining all three terms I have
%\begin{eqnarray*}
%	L_X(\omega)
%	&=& \left(
%		\frac{x(1+z)^2}{1+z^2}\, \;-\;  \frac{zx(1-z^2)}{(1+z^2)^2}
%		\right)\, dy \wedge dz \\
%	&+& \left(
%		\frac{y(1+z)^2}{1+z^2}\, \;-\; \frac{zy(1-z^2)}{(1+z^2)^2}
%		\right)\, dz \wedge dx \\
%	&+& \frac{1+2z^2}{1+z^2}\, dx \wedge dy \\
%&&\\%
%	&=& \left(
%		\frac{x(1+z)^2(1+z^2) \;-\; zx(1-z^2)}{(1+z^2)^2}
%		\right)\, dy \wedge dz \\
%	&+& \left(
%		\frac{y(1+z)^2(1+z^2) \;-\; zy(1-z^2)}{(1+z^2)^2}
%		\right)\, dz \wedge dx \\
%	&+& \frac{1+2z^2}{1+z^2}\, dx \wedge dy.
%\end{eqnarray*}
%This doesn't look like it simplifies further as a form on $\R^3$ so I
%attempt further simplification by pulling back to $M$ via
%$z^2+1 = x^2+y^2$.
%
%[xxx finish]
%
%[xxx $\theta,z$ and $L_Y(i^*\omega)$?]

%% ----------------------------------------------------------------
\subsection{August 2006 \#6.}
\label{sec:A06.6}
Prove the Poincar\'e lemma in the plane:  a closed 1-form
or 2-form on $\R^2$ is exact.

%% ----------------------------------------------------------------
\subsection{August 2006 \#7.}
Let $\T^2$ be the two-dimensional torus and let $\phi:\S^2\to \T^2$ be
 smooth map.  Show that for any top de Rham cohomology class $[\nu] \in
H^2(\T^2)$, we have $\phi^*[\nu] = 0$.

\textbf{Solution (Smith/Piercey).}  Problems like this typically involve
factoring through some third thing.  (See section \ref{sec:S05.2} for more
examples.)  The question is:  factor \emph{what} through \emph{what}?  We have
the following clues:
\begin{itemize}
\item $\S^2$ has trivial fundamental group; $\T^2$ has fundamental group
isomorphic to $\Z\oplus\Z$.

\item It is disconcerting that the question asks about second-level cohomology,
which at first glance appears unusably distinct from considerations of
fundamental groups.  (After all, $\S^2$ and $\T^2$, both being compact
orientable manifolds, have isomorphic second-level cohomology!)  However,
recall that the lifting theorems involve fundamental groups, and also maps of
manifolds (including the lifted maps!) induce maps on homology and cohomology.

\item A lifting result can be used to pass to the universal cover of $\T^2$,
thereby providing a third thing to experiment with.
\end{itemize}

$\S^2$ has trivial fundamental group, which maps by $\phi_*$ into $\T^2$'s
fundamental group $\Z \oplus \Z$, i.e.
$$
	0=\phi_*(\pi_1(\S^2)) \le \pi_1(\T^2).
$$
Since $\S^2$ is simply connected, it is its own universal cover; on the other
hand, $\R^2$ is the universal cover for $\T^2$, with covering map $p$, and
$$
	0=p_*(\pi_1(\R^2)) \le \pi_1(\T^2).
$$
(Recall that $\R^2$ in particular, and universal covers in general, have
trivial fundamental group.) Since
$$
	\phi_*(\pi_1(\S^2)) \le p_*(\pi_1(\R^2)),
$$
there exists a unique lift $\tilde\phi$ of $\phi$ from $\S^2$ to $\R^2$ such
that the following diagram commutes, i.e. $\phi = p\circ\tilde\phi$:
$$
\xymatrix{
	{} & \R^2 \ar[d]^{p} \\
	\S^2 \ar[r]^{\phi} \ar@{-->}[ur]^{\tilde\phi} & \T^2.
}
$$
Then the induced maps on cohomology give the diagram
$$
\xymatrix{
	{} & H^2(\R^2) \ar@{-->}[dl]_{\tilde\phi_*} \\
	H^2(\S^2) & H^2(\T^2) \ar[l]_{\phi^*} \ar[u]_{p^*}.
}
$$
This diagram commutes because
$$
	\phi^* = (p \circ \tilde\phi)^* = \tilde\phi^* \circ p^*.
$$
But $\R^2$ has trivial second cohomology, so $p^*[\nu]=0$ for any second
cohomology class $[\nu]$ of $\T^2$, and so
$$
	\phi^*[\nu] =
	\tilde\phi^*(p^*[0]) =
	\tilde\phi^*(0) = 0.
$$

%$$
%\xymatrix{
%	{} & \pi_1(\R^2) \ar[d]^{p_*} \\
%	\pi_1(\S^2) \ar[r]^{\phi_*} \ar@{-->}[ur]^{\tilde\phi_*} & \pi_1(\T^2) \\
%}
%$$

%% ================================================================
\newpage
\section{Winter 2005}

%% ----------------------------------------------------------------
\subsection{Winter 2005 \#2.}
\label{sec:W05.2}

Let $\S^1$ denote the unit circle in $\R^2$ and write
$$
	\S^1 \ni \colvectwo{x}{y} \mapsto p = \frac{x}{1-y},
$$
for the coordinate on $\S^1$ given by stereographic projection from the north
pole, $N$.  Let $V_p = v(p)\,\slashDp$ denote a vector field on the complement of
the north pole, $\S^1 \setminus \{N\}$.  Here $v(p)$ is a smooth function on
$\R$.  Find a condition on $v(p)$ that guarantees the vector field $V$ extends
to a smooth vector field on $\S^1$.  Does $p^3 \,\slashDp$ extend to a smooth
vector field on $\S^1$?

\textbf{Solution.}  Since $v(p)$ is already smooth everywhere except perhaps at
$N$, it suffices to give a condition for smoothness at $N$.  Recall that
stereographic projection from the south pole, $S$, is given by 
$$
	\S^1 \ni \colvectwo{x}{y} \mapsto q = \frac{x}{1+y}.
$$
Also recall that $p=1/q$.  Since vector fields are contravariant, $\slashDp$
transforms by
$$
	\fracDp = \frac{dq}{dp} \,\fracDq,
$$
where
$$
	\frac{dq}{dp} = -\frac{1}{p^2} = -q^2.
$$
Changing variable from $p$ to $q$, we have
$$
	V_{1/q} =  - v\left(\frac{1}{q}\right) \,\frac{1}{p^2}\, \fracDq
		= - q^2\, v\left(\frac{1}{q}\right) \, \fracDq.
$$
Thus, a necessary condition for smoothness of $V$ at $N$ is that
$-q^2 v(1/q)$ be smooth for $q=0$.

For example, $p\,\slashDp$ can extend smoothly to $N$:
we have
$$
	V_{1/q} = -\frac{q^2}{q} \,\slashDq = -q \,\slashDq.
$$

On the other hand, $p^3\,\slashDp$ does not extend smoothly to $N$:
we have
$$
	V_{1/q}
	= -\frac{q^2}{q^3} \,\slashDq
	= -\frac{1}{q} \,\slashDq
$$
which is singular at $q=0$.

%% ----------------------------------------------------------------
\subsection{Winter 2005 \#5.}

Suppose that $\phi:\S^1 \to \S^1$ and $\psi:\S^1 \to \S^1$ are continuous maps.
Show that the compositions $\phi\circ\psi$ and $\psi\circ\phi$ are homotopic.
(Hint:  is this true for $f(z) = z^n$ where $n \in \Z$?)

\textbf{Sketch of proof.}  View $\phi$ and $\psi$ as representatives of
homotopy classes in the fundamental group of $\S^1$.  (Technically, think of
$\S^1$ as the unit circle in the complex plane, $$\{e^{i\theta}: \theta \in
\R\}.$$ Identify $\S^1$ with $\R/\Z$:  send $[0,1] \ni t$ to $e^{2\pi i t}$.
Then $\tilde\phi: [0,1] \to \S^1$ is well-defined.) Shift $\psi$ by a constant
if necessary so that $\phi(0)=\psi(0)$ is a common basepoint.  Then $\phi$ and
$\psi$, respectively, are necessarily homotopic to $z \mapsto z^m$ and $z
\mapsto z^n$ for integer $m$ and $n$ (the winding numbers of the maps).
Representatives for the compositions are
$$
	z \mapsto (z^m)^n = z^{mn} = (z^n)^m
$$
which are identical so $\phi$ is homotopic to $\psi$.

[xxx to do:  find the right level of detail to include.]

%% ================================================================
\newpage
\section{Fall 2005}

%% ----------------------------------------------------------------
\subsection{Fall 2005 \#2.}

Let $\slashDtheta$ denote the vector field on $\S^1$ which has tangent
vector at $(x,y) = (\cos\theta, \sin\theta) \in \S^1$ determined by the
infinitesimal curve
$$
	t \mapsto (\cos(\theta+t), \sin(\theta+t))
$$
at $t=0$.  Let $\S^1 \ni (x,y) \mapsto p = x/(1-y)$ denote stereographic
projection from the north pole.  Find the function $v(p)$ so that
$$
	\fracDtheta = v(p) \,\fracDp.
$$

\emph{Solution.}  Since vector fields are first-order contravariant tensor
fields,
$$
	\fracDtheta = \frac{dp}{d\theta} \, \fracDp.
$$
Thus the desired $v(p)$ will be $dp/d\theta$, written in terms of $p$.
Since $x=\cos\theta$, $y=\sin\theta$, and $p=x/(1-y)$, we have
\begin{eqnarray*}
	\frac{dp}{d\theta}
	&=& \fracdtheta\left(\frac{\cos\theta}{1-\sin\theta}\right) \\
	&=& \frac{(1-\sin\theta)(-\sin\theta) + \cos^2\theta}{(1-\sin\theta)^2} \\
	&=& \frac{-\sin\theta + \sin^2\theta + \cos^2\theta}{(1-\sin\theta)^2} \\
	&=& \frac{1-\sin\theta}{(1-\sin\theta)^2} \\
	&=& \frac{1}{1-\sin\theta}.
\end{eqnarray*}
It remains to write this in terms of $p$.  Going back to $x$ and $y$ for
the moment, I have
\begin{eqnarray*}
	\frac{1}{1-\sin\theta}
	&=& \frac{\cos\theta}{\cos\theta(1-\sin\theta)}
	= \frac{x}{x(1-y)}
	= \frac{p}{x}.
\end{eqnarray*}
See the discussion in section \ref{sec:J07.5} for why it is that
$$
	y = \frac{p^2-1}{p^2+1}.
$$
Then, to find an expression for $x$ in terms of $p$, use 
$p=x/(1-y)$ and solve for $p$:
\begin{eqnarray*}
	x &=& p(1 - y)
	= p\left(1 - \frac{p^2-1}{p^2+1}\right)
	= p\left(\frac{2}{p^2+1}\right)
	= \frac{2p}{p^2+1}.
\end{eqnarray*}

Finally, we have
$$
	v(p) = \frac{dp}{d\theta} = \frac{p}{x}
	= p\; \left(\frac {p^2+1}{2p}\right)
	= \frac {p^2+1}{2}.
$$

\begin{center}* * *\end{center}

\textbf{Alternative solution.}  We can compute $d\theta/dp$, then reciprocate.
We have
$$
	\theta = \tan^{-1}(y/x)+k\pi
$$
with the usual due diligence in choice of $k$.  Since
$$
	x=\frac{2p}{p^2+1} \qand
	y=\frac{p^2-1}{p^2+1},
$$
we have
\begin{eqnarray*}
	\theta &=& \tan^{-1}\left(\frac{(p^2-1)/(p^2+1)}{2p/(p^2+1)}\right)+k\pi \\
	&=& \tan^{-1}\left(\frac{p^2-1}{2p}\right)+k\pi \\
	\frac{d\theta}{dp} &=&
		\frac{1}{1+\frac{(p^2-1)^2}{4p^2}} \;
		\frac{4p^2-(2p^2-2)}{4p^2} \\
	&=& \frac{2(p^2+1)}{4p^2+(p^2-1)^2} \\
	&=& \frac{2(p^2+1)}{4p^2+p^4-2p^2+1} \\
	&=& \frac{2(p^2+1)}{p^4+2p^2+1} \\
	&=& \frac{2(p^2+1)}{(p^2+1)^2} \\
	&=& \frac{2}{p^2+1}
\end{eqnarray*}
so
$$
	\frac{dp}{d\theta} = \frac {p^2+1}{2}
$$
as before.

%% ----------------------------------------------------------------
\subsection{Fall 2005 \#3.}

Let $f:D^2 \to D^2$ be a homeomorphism (where $D^2$ is the closed unit disk in
$\R^2$).  Show that $f$ must map boundary points $x \in \D D^2$ to points $f(x)
\in \D D^2$ on the boundary.

\textbf{Solution (Jordan Schettler).} Suppose that $f(q) \notin \D D^2$ for
some $q \in \D D^2$.  Then $f$ restricted to $D^2 \setminus \{q\}$ is a
homeomorphism; its image is $D^2 \setminus \{f(q)\}$ since a homeomorphism is
bijective.  Notice that $D^2 \setminus \{q\}$ has trivial fundamental group
while $D^2 \setminus \{f(q)\}$ has cyclic fundamental group.  This is a
contradiction since homeomorphisms preserve fundamental groups.

%% ================================================================
\newpage
\section{Spring 2005}

%% ----------------------------------------------------------------
\subsection{Spring 2005 \#2.}
\label{sec:S05.2}

Let $f:\S^2 \to \S^1$ be a continuous map.  Show that there is no continuous map
$g:\S^1 \to \S^2$ such that $f \circ g$ is the identity map on $\S^1$.

\textbf{Solution.}  This is a factor-through problem.  As will seen below, we
can use fundamental groups, homology, or cohomology.  The first is the most
likely guess and (happily) it works out here.  Suppose there were such a $g$.
Then
$$
	\S^1 \stackrel{g}{\to} \S^2 \stackrel{f}{\to} \S^1
$$
is the same as
$$
	\S^1 \stackrel{\id}{\to} \S^1.
$$
The induced map on fundamental groups is $(f \circ g)_* = f_* \circ g_*$ so
we have
$$
	\pi_1(\S^1,x) \stackrel{g_*}{\to}
	\pi_1(\S^2,y) \stackrel{f_*}{\to}
	\pi_1(\S^1,z)
$$
Regardless of the choice of base points, we would have (in terms of isomorphism
classes)
$$
	\Z \to 0 \to \Z
$$
being the same as the identity map.  But this is absurd; any map through zero
is the zero map.

\begin{center}* * *\end{center}
%% - - - - - - - - - - - - - - - - - - - - - - - - - - - - - - - -

\textbf{Variation (Dan Champion).}  
Let $f:\S^3 \to \T^2$ be a continuous map.  Show that there is no continuous
map $g:\T^2 \to \S^3$ such that $f \circ g$ is the identity map on $\T^2$.

\textbf{Solution.}
The argument is the same.  The point is that $\S^n$ has trivial fundamental
group for $n \ge 2$, while $\T^2$ has fundamental group of isomorphism class
$\Z^2$:  we have
$$
	\T^2 \stackrel{g}{\to} \S^3 \stackrel{f}{\to} \T^2
$$
being the same as
$$
	\T^2 \stackrel{\id}{\to} \T^2
$$
which would induce the absurd isomorphism
$$
	\Z^2 \to 0 \to \Z^2.
$$

\begin{center}* * *\end{center}
%% - - - - - - - - - - - - - - - - - - - - - - - - - - - - - - - -

\textbf{Variation (Dan Champion).}  
Let $f:\S^1 \vee \S^1 \to \T^2$ be a continuous map.  Show that there is no
continuous map $g:\T^2 \to \S^1 \vee \S^1$ such that $f \circ g$ is the identity
map on $\T^2$.

\textbf{Solution.}  We have
$$
	\T^2 \stackrel{g}{\to} \S^1 \vee \S^1 \stackrel{f}{\to} \T^2
$$
being the same as
$$
	\T^2 \stackrel{\id}{\to} \T^2.
$$
The induced map on fundamental groups is $(f \circ g)_* = f_* \circ g_*$ so
we have
$$
	\pi_1(\T^2,x) \stackrel{g_*}{\to}
	\pi_1(\S^1 \vee \S^1,y) \stackrel{f_*}{\to}
	\pi_1(\T^2,z)
$$
which in terms of isomorphism classes is
$$
	\Z^2 \to \Z * \Z \to \Z^2.
$$
The existence of such a map is, to me at least, an algebraic puzzler.
But we don't need to look at fundamental groups.  What $\S^1 \vee \S^1$
does certainly have (being one-dimensional) is trivial second homology.
The induced map is
$$
	H_2(\T^2) \stackrel{g_*}{\to}
	H_2(\S^1 \vee \S^1) \stackrel{f_*}{\to}
	H_2(\T^2)
$$
which in terms of isomorphism classes is the desired absurdity
$$
	\Z \to 0 \to \Z.
$$

\begin{center}* * *\end{center}
%% - - - - - - - - - - - - - - - - - - - - - - - - - - - - - - - -
\begin{rem*}
Questions about the existence of maps may often be attacked by (1)
factor-through arguments, or (2) lift arguments.  [xxx xref to an example of
the latter.]
\end{rem*}

%% ----------------------------------------------------------------
\subsection{Spring 2005 \#6.}

Consider $\R^3$ with coordinates $(x,y,z)$.  Write down explicit formulas
for the vector fields $X$ and $Y$ which represent the infinitesimal
generators of rotations about the $x-$ and $y-$axes respectively and
compute their Lie bracket.

\textbf{Solution.}  We should all know about rotation matrices on $\R^2$:
$$
	\colvectwo{x}{y} \mapsto
	\left(\begin{array}{rr}
		\cos(\theta) & -\sin(\theta) \\
		\sin(\theta) &  \cos(\theta) \\
	\end{array}\right)
	\colvectwo{x}{y}.
$$
With $\theta=\pi/2$ we have
$$
	\colvectwo{x}{y} \mapsto \colvectwo{-y}{x}.
$$
So, a vector footed at $(x,y)$ but perpendicular to the ray from the
origin to $(x,y)$ is $(-y,x)|_{(x,y)} = -y \slashDx + x \slashDy$.
[xxx include figure.]
Embedding into $\R^3$ we have
$$
	X = -z \fracDy + y \fracDz \qqand
	Y = -z \fracDx + x \fracDz.
$$

The Lie bracket is $[X,Y] = XY-YX$.  We do this using partial differentiation
and the chain rule, guided by the fact that second-order partials will cancel
and thus needn't be computed.  We have
\begin{eqnarray*}
	XY - YX &=& \left( -z \fracDy + y \fracDz \right)
		\left( -z \fracDx + x \fracDz \right)
	 	-\left( -z \fracDx + x \fracDz \right)
		\left( -z \fracDy + y \fracDz \right) \\
	&=& -z \fracDy \left( -z \fracDx\right)
		-z \fracDy \left( x \fracDz \right)
		+y \fracDz \left( -z \fracDx  \right)
		+y \fracDz \left( x \fracDz \right) \\
	&+& z \fracDx \left( -z \fracDy \right)
		+z \fracDx \left( +y \fracDz \right)
		-x \fracDz \left( -z \fracDy \right)
		-x \fracDz \left( +y \fracDz \right) \\
	&=& 0
		-0
		-y \fracDx
		+0 \\
	&+& 0
		+0
		+x \fracDy
		-0 \\
	&=& -y \fracDx +x \fracDy
\end{eqnarray*}
which is rotation about the $z$-axis.

%% ================================================================
\newpage
\section{Fall 2004}

%% ----------------------------------------------------------------
\subsection{Fall 2004 \#2.}

Find the image of the vertical strip $|\mathrm{Re}(z)| < \pi/2$ under the
conformal map $z \mapsto \sin(z)$.

\textbf{Solution.}  Since I am not already familiar with this map, I will do
some algebra and look for patterns.  I know that
$$
	\sin(z) = \frac{e^{iz}-e^{-iz}}{2i}.
$$
Putting $z=x+iy$, I have (attempting to simplify in terms of elementary
functions in $x$ and $y$ separately)
\begin{eqnarray*}
	\sin(x+iy) &=& \sin(x)\cos(iy) + \cos(x) \sin(iy)
		= \sin x \cosh y + i \cos x \sinh y.
\end{eqnarray*}
(See also the trig-function section of my complex-analysis notes, which you
should find nearby as \texttt{caqp.pdf}.)

Now, examine the images of horizontal line segments and vertical lines.  Let
$u+iv = w = \sin(z) = \sin(x+iy)$.  If $x=0$, then we have
$$
	u+iv = 0 + i \sinh y.
$$
Thus the imaginary axis is sent to itself.  If $x = \pi/2$ then
$$
	u+iv = \cosh(y) + 0i
$$
Likewise if $x=-\pi/2$ then
$$
	u+iv = -\cosh(y) + 0i
$$
[xxx see figure.]

For $y=0$ we have
$$
	u+iv = \sin(x)
$$
which sends that horizontal line segment to itself.  As $y$ increases,
$\sinh(y)$ and $\cosh(y)$ both increase.  Fixing $y$, when
$x=\pm\pi/2$ we have
$$
	u+iv = \pm\cosh y + 0i;
$$
when $x=0$ we have
$$
	u+iv = 0 + i\sinh y.
$$
[xxx figure.]

I now boldly claim (based on plotting which I admit I couldn't do during the
qual --- but, here were are building intuition which we \emph{can} recall
later) that the image of horizontal line segments are the upper halves of
ellipses with horizontal radius $\cosh y$ and vertical radius $\sinh y$.
Recall that the point $(u,v)$ lies on an ellipse with horizontal
radius $a$ and vertical radius $b$ if
$$
	\frac{u^2}{a^2} + \frac{v^2}{b^2} = 1.
$$
For the image of a horizontal line segment with $y$ fixed, this is
$$
	\frac{u^2}{a^2} + \frac{v^2}{b^2} =
	\frac{\sin^2 x \cosh^2 y}{\cosh^2 y} + \frac{\cos^2 x \sinh^2 y}{\sinh^2 y}
	= 1
$$
as desired.

As $y$ increases above zero, $\cosh y$ and $\sinh y$ both increase; as $y$
decreases below zero, $\cosh y$ increases while $\sinh y$ decreases.  Thus, the
image of the family of horizontal line segments is a family of concentric
ellipses.  Remember though that since $\pm\pi/2$ is not included in the domain,
the parts of the real line with $u\le -1$ and $u \ge +1$ are not in the image
of the map.

What about the images of vertical lines?  Since $\sin(z)$ is conformal, it is
angle-preserving.  So, we can sketch it as follows: [xxx figure.]

I now claim that the images of vertical lines are sheets of hyperbolae.  To
show this, I need to do some algebra as above for the ellipses.  [xxx do that.]

%% ----------------------------------------------------------------
\subsection{Fall 2004 \#5.}

Let $M$ denote the submanifold of $\R^3$ defined by $z^2-x^2-y^2=1$, with
$z>0$.  Define $\phi_t(x,y,z) = (sz+cx, y, cz+sx)$ where $c=\cosh t$ and $s =
\sinh t$.

(a) Show that $t \mapsto \phi_t$ determines a one-parameter family of
diffeomorphisms of $M$ and find the vector field associated with $t \mapsto
\phi_t$ in the coordinates for $M$ obtained by the projection $(x,y,z) \mapsto
(x,y)$.

(b) Determine $\phi_t^*(\frac{dx \wedge dy}{z})$.

\textbf{Solution.}  Showing that $t \mapsto \phi_t$ determines a one-parameter
family of diffeomorphisms of $M$ is all but identical to section
\ref{sec:F01.4}.  Note that $M$ is the top sheet of a hyperboloid of two
sheets.

Let $Y$ be the vector field associated with $t \mapsto \phi_t$ on $\R^3$, and
let $X$ be the vector field on $M$.  That is, $X$ is the restriction of $Y$ to
$M$.  The question did not ask to find $Y$, but we can.  As usual,
\begin{eqnarray*}
	Y &=& \frac{d\phi_t}{dt}\pickpipe{t=0}
		\pcmatrix{
			\cosh t & 0 & \sinh t \\
			0       & 1 & 0       \\
			\sinh t & 0 & \cosh t
		} \colvecthree{x}{y}{z} \\
	&=& \pcmatrix{
			\sinh t & 0 & \cosh t \\
			0       & 0 & 0       \\
			\cosh t & 0 & \sinh t
		} \colvecthree{x}{y}{z} \pickpipe{t=0} \\
	&=& \colvecthree{z}{0}{x} \pickpipe{\colvecthree{x}{y}{z}}
	= z\,\fracDx + x\,\fracDz.
\end{eqnarray*}

For $X$, we proceed similarly.  We are asked to use graph coordinates, so
we do so.  Note that $z = \sqrt{x^2+y^2+1}$.  Then
\begin{eqnarray*}
	\phi_t\colvectwo{x}{y} &=& \colcvectwo{cx + sz}{y}
	= \colcvectwo{\cosh(t) x + \sinh(t) \sqrt{x^2+y^2+1}}{y}
\end{eqnarray*}
and
\begin{eqnarray*}
	X &=& \frac{d\phi_t}{dt}\pickpipe{t=0}
		\colcvectwo{\cosh(t)x + \sinh(t) \sqrt{x^2+y^2+1}}{y} \\
	&=& \colcvectwo{\sinh(t)x + \cosh(t) \sqrt{x^2+y^2+1}}{0}
		\pickpipe{t=0} \\
	&=& \colcvectwo{\sqrt{x^2+y^2+1}}{0}
	= \left(\sqrt{x^2+y^2+1}\right) \fracDx.
\end{eqnarray*}

For part (b), it's not clear whether the question is asking for a pullback on
$\R^3$ or on $M$.  In the interest of exam preparation, we can do both.  Below
I'll show how this problem can also be attacked using Lie derivatives.  The
pullback on $\R^3$ (well, $\R^3 \setminus \{(0,0,0)\}$ since the form goes
undefined when $z=0$) is
$$
	\phi_t^*\left(\frac{dx \wedge dy}{z}\right)
	= \frac{d(cx+sz) \wedge dy}{sx+cz}
	= \frac{c\,dx\wedge dy - s\,dy\wedge dz}{sx+cz}
$$
and I don't see anything more that can be done with it.  On $M$, though, we can
get more mileage:
\begin{eqnarray*}
	\phi_t^*\left(\frac{dx \wedge dy}{z}\right)
	&=& \phi_t^*\left(\frac{dx \wedge dy}{\sqrt{x^2+y^2+1}}\right) \\
	&=& \frac{d\left(cx+s\sqrt{x^2+y^2+1}\right)
		\wedge dy}{sx+c\sqrt{x^2+y^2+1}} \\
	&=& \frac{c\,dx\wedge dy - s\,dy\wedge d\left(\sqrt{x^2+y^2+1}\right)}
		{sx+c\sqrt{x^2+y^2+1}} \\
	&=& \frac{c\,dx\wedge dy - s\,dy\wedge
		\left(
		\frac{x\,dx}{\sqrt{x^2+y^2+1}}+ \frac{y\,dy}{\sqrt{x^2+y^2+1}}
		\right)
		}
		{sx+c\sqrt{x^2+y^2+1}} \\
	&=& \frac{c\,dx\wedge dy +
		\left(
		\frac{sx\,dx \wedge \,dy}{\sqrt{x^2+y^2+1}}
		\right)
		}
		{sx+c\sqrt{x^2+y^2+1}} \\
	&=& \frac{c(\sqrt{x^2+y^2+1})\,dx\wedge dy + sx\,dx \wedge \,dy}
		{sx\sqrt{x^2+y^2+1}+c(x^2+y^2+1)} \\
	&=& \frac{cz\,dx\wedge dy + sx\,dx \wedge \,dy}
		{szx+cz^2} \\
	&=& \frac{(cz + sx) \,dx\wedge dy}
		{z(sx+cz)} \\
	&=& \frac{dx \wedge dy}{z}.
\end{eqnarray*}

This is all well and good, and as algebraic manipulation goes, it's not too
messy.  But it so often happens on quals that a given form is invariant under
the flow of a given vector field (as happened here) that it's worth checking
first, using the Lie derivative.  See section \ref{sec:A06.5} for caveats about
using the Lie derivative and Cartan' magic formula.  Summary:  pull back to the
manifold \emph{before} using this formula --- you will likely have far fewer
terms, making your life (or a few minutes of it) much easier.

The vector field on $M$ was found above (in graph coordinates) to be
$$
	X = \left(\sqrt{x^2+y^2+1}\right) \fracDx
$$
and the form (also in graph coordinates) was
$$
	\omega = \frac{dx\wedge dy}{\sqrt{x^2+y^2+1}}.
$$
Cartan's magic formula is
$$
	L_X\omega = X\ctt d\,\omega + d(X\ctt\omega).
$$
Since $\omega$ is top-level for $M$, $d\omega$ is immediately found to be
zero.  For the other term,
\begin{eqnarray*}
	X \ctt \omega &=&
		\left(\sqrt{x^2+y^2+1}\right) \fracDx \;\ctt\;
		\frac{dx\wedge dy}{\sqrt{x^2+y^2+1}} \\
	&=& dy.
\end{eqnarray*}
Then $d^2y =0$ so $L_X\omega=0$, so $\omega$ is invariant under the flow of $X$
--- as the above algebra found a different way, but with less effort
here.

%% ================================================================
\newpage
\section{January 2004}

%% ----------------------------------------------------------------
\subsection{January 2004 \#1.}

%% ================================================================
\newpage
\section{August 2003}

%% ----------------------------------------------------------------
\subsection{August 2003 \#2.}

Let $dA$ denote the standard area form for $\S^2$ (with the orientation
determined by the outward-pointing normal vector), and let $x,y,z$ denote the
standard rectangular coordinates for $\R^3$.  Determine the values of
$n=0,1,\ldots$ for which $\omega=z^n\,dA$ is exact.

\textbf{Solution.}  Recall that if a form $\omega$ is exact (with, say,
$\omega=d\eta$) on a compact orientable manifold $M$ without boundary, then by
Stokes theorem
$$
	\int_M \omega = \int_M d\eta = \int_{\D M} \eta = 0.
$$
I will first use the contrapositive of this statement to look for values of $n$
such that $\int_{\S^2} z^n\,dA$ is non-zero.  This will tell me those values of
$n$ for which $z^n\,dA$ is non-exact.

Recall that spherical coordinates on $\S^2$ (which miss only the north and
south poles, which form a set of zero measure in $\S^2$) are
$$
	\colvecthree{x}{y}{z} = \colvecthree
		{\sin\phi\cos\theta}
		{\sin\phi\sin\theta}
		{\cos\phi}.
$$
Also recall that
$$
	dA = \sin\phi\, d\phi\wedge d\theta
$$
(which is easily recovered, if desired, using the technique in the next
problem).
Then
\begin{eqnarray*}
	\int_{\S^2} z^n\, dA
	&=& \int_{\theta=0}^{\theta=2\pi} \left[\int_{\phi=0}^{\phi=\pi}
		\cos^n\phi \sin\phi \,d\phi\right] \,d\theta \\
	&=&
	\int_{\phi=0}^{\phi=\pi}
		\cos^n\phi \sin\phi
		\left[ \int_{\theta=0}^{\theta=2\pi} \,d\theta \right]
		\,d\phi \\
	&=& 2\pi
	\int_{\phi=0}^{\phi=\pi} \cos^n\phi \sin\phi \,d\phi.
\end{eqnarray*}
(This is justifiable by Fubini's theorem:  The integrand is continuous on a
compact domain, hence absolutely integrable.) I can do some work with
trigonometric identities, or I can proceed by symmetry.
\begin{itemize}

\item If $n=0$, then I have the surface area of the sphere, namely, $4\pi \ne
0$.  So, for $n=0$, $z^n\,dA$ is non-exact.

\item More generally, if $n$ is even, then the integrand is non-negative, with
positive area.  So, for $n$ even, $z^n\,dA$ is non-exact.

\item If $n=1$, then by examining the graphs of sine and cosine from $0$ to
$\pi$ (or by using the identity $\cos\phi\sin\phi = \frac{1}{2}\sin(2\phi)$), I
see that positive and negative areas will cancel and the value of the integral
will be zero.  So, for $n=1$, $z\,dA$ may be exact.  Below, I'll look for an
$\eta$ such that $z\,dA = d\eta$.

\item More generally, if $n$ is odd, then the portion of the graph of
$\cos^n(\phi)$ from $0$ to $\pi/2$ is non-negative and is the opposite of the
portion of the graph from $\pi/2$ to $\pi$.  Meanwhile the graph of sine is
symmetric about $\pi/2$.  So, for $n$ odd, $z^n\,dA$ may be exact.

\end{itemize}

Now for odd $n$.  I recognize from the chain rule that
$$
	d(\cos^m(\phi)) = -m \cos^{m-1}(\phi) \sin\phi\,d\phi.
$$
So, I can guess that
$$
	d\left(\left(\frac{-1}{n+1}\right)\,\cos^{n+1}\phi\,d\theta\right) =
	\cos^n\phi\sin\phi\,d\phi\wedge d\theta.
$$
Thus I want to write
$$
	\eta=\left(\frac{-1}{n+1}\right)\,\cos^{n+1}\phi\,d\theta
$$
but I need to find out why this doesn't work for even $n$.

Using spherical coordinates, $\D\S^2$ is really $S-N$ (singular one-chains on
each pole, parameterized by $\theta$, each being points) rather than
$\emptyset$, and thus
\begin{eqnarray*}
	\int_{\S^2} d\eta &=& \int_{\D\S^2}\eta \\
	&=& \left(\frac{-1}{n+1}\right)
		\int_{N-S}
	\cos^{n+1}\phi\,d\theta \\
	&=& \frac{-2\pi}{n+1}\; \left( (-1)^{n+1} - (1)^{n+1} \right)
\end{eqnarray*}
which is non-zero when $n$ is even, allaying my concern.

Thus, I've shown that $\omega$ is non-exact when $n$ is even, and is exact when
$n$ is odd.

\begin{center}* * *\end{center}

\textbf{Solution (Dan Champion).}  This approach uses almost no computation.
Note what while in general $\omega$ exact implies $\oint_M \omega = 0$, we also
have the other implication (it is an if and only if) for top-level forms.
(This fact alone would simplify the above proof.)

Use graph coordinates on the north and south poles.  Let $U$ be the (open)
upper hemisphere and $L$ be the (open) lower hemisphere.  For even $n$,
$\int_{\S^2}z^n\,dA = 2 \int_{U}z^n\,dA \ne 0$ (since $z$ is positive,
integrated over a positive area form); for odd $n$, $\int_{\S^2}z^n\,dA = 0$.

%% ----------------------------------------------------------------
\subsection{August 2003 \#3.}

Consider hyperspherical coordinates for $\S^3$.  The corresponding
parameterization, for $\theta \in (0, 2\pi)$ and $\phi,\psi \in (0, \pi)$, is
given by
$$
	\colvecthree{\theta}{\phi}{\psi} \longmapsto
	\collvecfour
		{\sin\psi\sin\phi\cos\theta}
		{\sin\psi\sin\phi\sin\theta}
		{\sin\psi\cos\phi}
		{\cos\psi}.
$$
Calculate the standard volume form in these coordinates, with respect to the
orientation given by the outward-pointing normal vector.

\textbf{Solution.}  Let
$$
	\colvecfour{w}{x}{y}{z}
$$
be rectangular coordinates for $\R^4$.  Since $\S^3$ is defined by
$$
	w^2 + x^2 + y^2 + z^2 = 1,
$$
the unit normal is given by the normalized gradient of the left-hand side,
namely,
$$
	\nhat =
	w \,\fracDw +
	x \,\fracDx +
	y \,\fracDy +
	z \,\fracDz.
$$
Meanwhile the volume form on $\R^4$ is
$$
	dw \wedge dx \wedge dy \wedge dz.
$$
Then the volume form on $\S^3$ is (using $dx_i(\slashDx_j) = \delta_{ij}$ to
compute the contractions)
\begin{eqnarray*}
	(*)\qquad\qquad
	\nhat \ctt (dw \wedge dx \wedge dy \wedge dz)
	&=& \ulmatrix{
		&  w &\slashDw \ctt (dw \wedge dx \wedge dy \wedge dz) \\
		+& x &\slashDx \ctt (dw \wedge dx \wedge dy \wedge dz) \\
		+& y &\slashDy \ctt (dw \wedge dx \wedge dy \wedge dz) \\
		+& z &\slashDz \ctt (dw \wedge dx \wedge dy \wedge dz)
		} \\
%
	&=& \ulmatrix{
		&  w& dx \wedge dy \wedge dz \\
		-& x& dw \wedge dy \wedge dz \\
		+& y& dw \wedge dx \wedge dz \\
		-& z& dw \wedge dx \wedge dy
		} \\
%
	&=& \ulmatrix{
		&  w & dx \wedge dy \wedge dz \\
		-& x & dy \wedge dz \wedge dw \\
		+& y & dz \wedge dw \wedge dx \\
		-& z & dw \wedge dx \wedge dy \\
		} \\
%
	&=& \ulmatrix{
		&  w & (dx \wedge dy) \wedge dz \\
		-& x & (dz \wedge dw) \wedge dy \\
		+& y & (dz \wedge dw) \wedge dx \\
		-& z & (dx \wedge dy) \wedge dw.
		}.
\end{eqnarray*}

I will compute the wedges one level at a time, then put the pieces together.
Note (for wedge sign convention) that I will order the variables $\psi$,
$\phi$, $\theta$.

First, the 1-forms.  These are
\begin{align*}
	dw	&= \plmatrix{
		 	&  \cos\psi & \sin\phi & \cos\theta & d\psi \\
			+& \sin\psi & \cos\phi & \cos\theta & d\phi \\
			-& \sin\psi & \sin\phi & \sin\theta & d\theta
		},
	&
	dx	&= \plmatrix{
			&  \cos\psi & \sin\phi & \sin\theta & d\psi \\
			+& \sin\psi & \cos\phi & \sin\theta & d\phi \\
			+& \sin\psi & \sin\phi & \cos\theta & d\theta
		},
	\\
	dy	&= \plmatrix{
		&  \cos\psi & \cos\phi &    & d\psi \\
		-& \sin\psi & \sin\phi &    & d\phi
		},
	&
	dz	&= \plmatrix{
		-&  \sin\psi & d\psi
		}.
\end{align*}

(Important note:  This is a tedious and error-prone problem from here on out.
If time were short on an exam, I would put equation (*), along with $w$ through
$z$ and $dw$ through $dz$ in terms of $\psi$, $\phi$, and $\theta$ as shown
just above.  Then I would state that the conceptual work is done; the rest is
repetitive, time-consuming simplification using exterior algebra.  If anyone
has a shorter solution, I'd love to see it.)

(Another note:  I am going to write forms vertically in parentheses.  This has
nothing to do with vector notation and everything to do with error avoidance.
If we are going to be doing pages of algebra on an exam, we might as well do it
correctly and quickly.)

Second, the two-forms.  These are
\begin{eqnarray*}
	dx\wedge dy &=&
	\plmatrix{
		&  \cos\psi & \sin\phi & \sin\theta & d\psi \\
		+& \sin\psi & \cos\phi & \sin\theta & d\phi \\
		+& \sin\psi & \sin\phi & \cos\theta & d\theta
	}
	\wedge
	\plmatrix{
		&  \cos\psi & \cos\phi & d\psi \\
		-& \sin\psi & \sin\phi & d\phi
	}
	\\
%
	&&\\
%
	&=&\plmatrix{
		-& \sin\psi \cos\psi &\sin^2\phi       &\sin\theta &d\psi\wedge d\phi \\
		-& \sin\psi \cos\psi &\cos^2\phi       &\sin\theta &d\psi \wedge d\phi\\
		-& \sin\psi \cos\psi &\sin\phi\cos\phi &\cos\theta &d\psi\wedge d\theta \\
		+& \sin^2\psi        &\sin^2\phi       &\cos\theta &d\phi\wedge d\theta
	} \\
%
	&&\\
%
	&=&\plmatrix{
		-& \sin\psi \cos\psi &                 &\sin\theta &d\psi \wedge d\phi \\
		-& \sin\psi \cos\psi &\sin\phi\cos\phi &\cos\theta &d\psi\wedge d\theta \\
		+& \sin^2\psi        &\sin^2\phi       &\cos\theta &d\phi\wedge d\theta
	}
\end{eqnarray*}
and
\begin{eqnarray*}
	dz\wedge dw
	&=& \plmatrix{
			-& \sin\psi &d\psi
		}
		\wedge
		\plmatrix{
			&  \cos\psi & \sin\phi & \cos\theta &d\psi \\
			+& \sin\psi & \cos\phi & \cos\theta &d\phi \\
			-& \sin\psi & \sin\phi & \sin\theta &d\theta
		} \\
	&=& \plmatrix{
		-& \sin^2\psi & \cos\phi & \cos\theta &d\psi \wedge d\phi \\
		+& \sin^2\psi & \sin\phi & \sin\theta &d\psi \wedge d\theta
		}.
\end{eqnarray*}

Next,
\begin{eqnarray*}
	w\, dx\wedge dy &=&
		\plmatrix{
			&\sin\psi &\sin\phi &\cos\theta
		}
		\plmatrix{
			-& \sin\psi \cos\psi &                 &\sin\theta &d\psi\wedge d\phi \\
			-& \sin\psi \cos\psi &\sin\phi\cos\phi &\cos\theta &d\psi\wedge d\theta\\
			+& \sin^2\psi        &\sin^2\phi       &\cos\theta &d\phi\wedge d\theta
		} \\
	%
	&=& \plmatrix{
		-& \sin^2\psi \cos\psi & \sin\phi           &\sin\theta\cos\theta &d\psi\wedge d\phi \\
		-& \sin^2\psi \cos\psi & \sin^2\phi\cos\phi &\cos^2\theta         &d\psi\wedge d\theta \\
		+& \sin^3\psi          & \sin^3\phi         &\cos^2\theta         &d\phi\wedge d\theta
		},
\end{eqnarray*}
\begin{eqnarray*}
	z\, dx\wedge dy
	&=& \plmatrix{
		\cos\psi
	}
	\plmatrix{
		-& \sin\psi \cos\psi &                 &\sin\theta &d\psi \wedge d\phi \\
		-& \sin\psi \cos\psi &\sin\phi\cos\phi &\cos\theta &d\psi\wedge d\theta \\
		+& \sin^2\psi        &\sin^2\phi       &\cos\theta &d\phi\wedge d\theta
	} \\
	&=& \plmatrix{
		-& \sin\psi\cos^2\psi &                 &\sin\theta &d\psi \wedge d\phi \\
		-& \sin\psi\cos^2\psi &\sin\phi\cos\phi &\cos\theta &d\psi\wedge d\theta \\
		+& \sin^2\psi\cos\psi &\sin^2\phi       &\cos\theta &d\phi\wedge d\theta
	},
\end{eqnarray*}
\begin{eqnarray*}
	x\,dz \wedge dw
	&=& \plmatrix{
			&\sin\psi &\sin\phi &\sin\theta
		}
		\plmatrix{
			-& \sin^2\psi &\cos\phi &\cos\theta &d\psi \wedge d\phi \\
			+& \sin^2\psi &\sin\phi &\sin\theta &d\psi \wedge d\theta
		} \\
	&=& \plmatrix{
		-& \sin^3\psi &\sin\phi\cos\phi &\sin\theta\cos\theta &d\psi\wedge d\phi \\
		+& \sin^3\psi &\sin^2\phi       &\sin^2\theta         &d\psi \wedge d\theta
		},
\end{eqnarray*}
and
\begin{eqnarray*}
	y\, dz \wedge dw
	&=& \plmatrix{
			\sin\psi&\cos\phi
		}
	\plmatrix{
		-& \sin^2\psi &\cos\phi &\cos\theta &d\psi \wedge d\phi \\
		+& \sin^2\psi &\sin\phi &\sin\theta &d\psi \wedge d\theta
	} \\
	&=& \plmatrix{
		-& \sin^3\psi &\cos^2\phi       &\cos\theta &d\psi \wedge d\phi \\
		+& \sin^3\psi &\sin\phi\cos\phi &\sin\theta &d\psi \wedge d\theta
	}.
\end{eqnarray*}

Now the 3-forms:
\begin{eqnarray*}
	(w\,dx\wedge dy) \wedge dz
	&=& \plmatrix{
			-& \sin^2\psi \cos\psi &\sin\phi           &\sin\theta\cos\theta &d\psi\wedge d\phi \\
			-& \sin^2\psi \cos\psi &\sin^2\phi\cos\phi &\cos^2\theta         &d\psi\wedge d\theta \\
			+& \sin^3\psi          &\sin^3\phi         &\cos^2\theta         &d\phi\wedge d\theta
		}
		\wedge
		\plmatrix{
			-&\sin\psi &d\psi
		} \\
	&=& \plmatrix{
		-&\sin^4\psi &\sin^3\phi &\cos^2\theta &d\psi \wedge d\phi\wedge d\theta
		};
\end{eqnarray*}
\begin{eqnarray*}
	(x\,\wedge dz \wedge dw) \wedge dy
	&=& \plmatrix {
		-& \sin^3\psi &\sin\phi\cos\phi &\sin\theta\cos\theta &d\psi\wedge d\phi \\
		+& \sin^3\psi &\sin^2\phi       &\sin^2\theta         &d\psi \wedge d\theta
		}
	\wedge
	\plmatrix{
		-& \cos\psi & \cos\phi &d\psi \\
		+& \sin\psi & \sin\phi &d\phi
	} \\
%
	&=& \plmatrix{
		&\sin^4\psi &\sin^3\phi &\sin^2\theta &d\psi \wedge d\phi \wedge d\theta
		};
\end{eqnarray*}
\begin{eqnarray*}
	\left(y\,dz \wedge dw\right) \wedge dx
	&=& \plmatrix{
		-& \sin^3\psi &\cos^2\phi       &\cos\theta &d\psi \wedge d\phi \\
		+& \sin^3\psi &\sin\phi\cos\phi &\sin\theta &d\psi \wedge d\theta
		}
	\wedge
	\plmatrix{
		&  \cos\psi &\sin\phi &\sin\theta &d\psi \\
		+& \sin\psi &\cos\phi &\sin\theta &d\phi\\
		+& \sin\psi &\sin\phi &\cos\theta &d\theta
	} \\
%
	&=& \plmatrix{
			-& \sin^4\psi &\sin\phi\cos^2\phi &\cos^2\theta &d\psi \wedge d\phi \wedge d\theta \\
			-& \sin^4\psi &\sin\phi\cos^2\phi &\sin^2\theta &d\psi \wedge d\phi \wedge d\theta
		} \\
	&=& \plmatrix{
		-&\sin^4\psi &\sin\phi\cos^2\phi &d\psi \wedge d\phi \wedge d\theta
	};
\end{eqnarray*}
\begin{eqnarray*}
	\left(z \wedge dx \wedge dy\right) \wedge dw
	&=& \plmatrix{
			-&\sin\psi\cos^2\psi &                  &\sin\theta &d\psi \wedge d\phi \\
			-&\sin\psi\cos^2\psi & \sin\phi\cos\phi &\cos\theta &d\psi\wedge d\theta \\
			+&\sin^2\psi\cos\psi & \sin^2\phi       &\cos\theta &d\phi\wedge d\theta
		}
	\wedge
	\plmatrix{
		&  \cos\psi &\sin\phi &\cos\theta &d\psi \\
		+& \sin\psi &\cos\phi &\cos\theta &d\phi \\
		-& \sin\psi &\sin\phi &\sin\theta &d\theta
	} \\
%
	&=& \plmatrix{
		& \sin^2\psi\cos^2\psi &\sin\phi           &\sin^2\theta \\
		+&\sin^2\psi\cos^2\psi &\sin\phi\cos^2\phi &\cos^2\theta \\
		+&\sin^2\psi\cos^2\psi &\sin^3\phi         &\cos^2\theta
		}
		d\psi\wedge d\phi \wedge d\theta \\
%
	&=& \plmatrix{
		&  \sin^2\psi\cos^2\psi &\sin\phi           &\sin^2\theta \\
		+& \sin^2\psi\cos^2\psi &\sin\phi\cos^2\phi &\cos^2\theta \\
		+& \sin^2\psi\cos^2\psi &\sin\phi\sin^2\phi &\cos^2\theta
		}
		d\psi\wedge d\phi \wedge d\theta \\
%
	&=& \plmatrix{
		& \sin^2\psi\cos^2\psi &\sin\phi &\sin^2\theta \\
		+&\sin^2\psi\cos^2\psi &\sin\phi &\cos^2\theta
		}
		d\psi \wedge d\phi \wedge d\theta \\
%
	&=& \plmatrix{
		&\sin^2\psi\cos^2\psi &\sin\phi &d\psi \wedge d\phi \wedge d\theta
		}.
\end{eqnarray*}

Last,
\begin{eqnarray*}
	\nhat \ctt (dw \wedge dx \wedge dy \wedge dz)
	&=& \ulmatrix{
		&  w & (dx \wedge dy) \wedge dz \\
		-& x & (dz \wedge dw) \wedge dy \\
		+& y & (dz \wedge dw) \wedge dx \\
		-& z & (dx \wedge dy) \wedge dw.
		} \\
%
	&=& -\;\plmatrix{
		&  \sin^4\psi           &\sin^3\phi          &\cos^2\theta  \\
		+& \sin^4\psi           &\sin^3\phi          &\sin^2\theta  \\
		+& \sin^4\psi           &\sin\phi\cos^2\phi  & \\
		+& \sin^2\psi\cos^2\psi &\sin\phi            &
		} \;d\psi \wedge d\phi \wedge d\theta \\
%
	&=& -\;\plmatrix{
		&  \sin^4\psi           &\sin\phi\sin^2\phi \\
		+& \sin^4\psi           &\sin\phi\cos^2\phi \\
		+& \sin^2\psi\cos^2\psi &\sin\phi
		} \;d\psi \wedge d\phi \wedge d\theta \\
%
	&=& -\;\plmatrix{
		&  \sin^2\psi\sin^2\psi &\sin\phi \\
		+& \sin^2\psi\cos^2\psi &\sin\phi
		} \;d\psi \wedge d\phi \wedge d\theta \\
%
	&=& - \sin^2\psi \sin\phi \;d\psi \wedge d\phi \wedge d\theta.
\end{eqnarray*}

%% ----------------------------------------------------------------
\subsection{August 2003 \#4.}

Let $X=\{z^2-x^2-y^2=-1\}$ in $\R^3$.
\begin{itemize}
\item[(a)]  Show that $X$ is an embedded submanifold.
\item[(b)]  Compute the homology of $X$.
\item[(c)]  Find an explicit one-form on $X$ which is closed and not exact.
\end{itemize}

\textbf{Solution.}  First re-write $X$ as $\{x^2+y^2-z^2=1\}$ in $\R^3$.

For part (a) I can use the regular value theorem.  Let $G:\R^3\to\R: (x,y,z)
\mapsto x^2+y^2-z^2$.  Then $X$ is the level set of $G$ and 1.  If 1 is a
regular value of $G$ then $G^{-1}(1)$ is empty or an embedded submanifold of
$\R^3$.  I compute
$$
	DG = (2x \; 2y \; -2z)
$$
which fails to be surjective when $x=y=z=0$ which is not in $X$.  Note that
$G^{-1}(1)$ is non-empty since it contains (at least) the point $(1,0,0)$.

For part (b), first note that in cylindrical coordinates we have $r^2-z^2=1$ so
this is a hyperboloid of one sheet.  This deformation-retracts to $\S^1$ and
homology is homotopy invariant, so $X$ has the homology of the circle.  Namely,
$H_0(X)$ has rank 1 ($X$ has one connected component), $H_1(X)$ has rank one
(for the equatorial loop), and the higher-order groups are zero.

For part (c), use $\theta$ and $z$ coordinates.  The one-form is $d\theta$.
It is closed because it is (locally) the exterior derivative of $\theta$,
and $d^2 = 0$.  It is non-exact because $\int_c d\theta = 2\pi\ne 0$ where
$c$ is the counterclockwise path around the circle formed by the intersection
of $X$ and the $xy$-plane.

%% ----------------------------------------------------------------
\subsection{August 2003 \#5.}
\label{sec:A03.5}

Consider the vector field $V$ on $\S^2$ which is given by
\begin{equation}
\label{eqn:A03.5.eqn1}
	V_{\colvecthree{x}{y}{z}} = \colrvecthree{-zy}{zx}{0},
\end{equation}
where we have identitifed $T\S^2|_q$ with $q^\perp$, the orthogonal complement.
Find an explicit expression for the flow of this vector field, and graph the
trajectories.

\textbf{Solution.}  We need to solve the system of ODEs
$$
	\colvecthree{\xdot}{\ydot}{\zdot} = \colrvecthree{-zy}{zx}{0}.
$$
Immediately we have $z=z_0$, i.e. the flow will preserve latitudes.
Taking second derivatives to obtain univariate ODEs we get
$$
	\colvecthree{\xddot}{\yddot}{\zddot}
	= \colrvecthree{-\zdot y - z\ydot}{\zdot x + z \xdot}{0}
	= \colrvecthree{-z^2x}{-z^2y}{0}.
$$
The univariate ODEs are
$$
	\xddot + z^2 x = 0 \qqand \yddot + z^2 y = 0
$$
from which
\begin{equation}
\label{eqn:A03.5.eqn2}
	x = a \cos(zt) + b \sin(zt) \qqand
	y = c \cos(zt) + d \sin(zt).
\end{equation}
Applying initial conditions for $x$ and $y$ gives
$$
	x(0) = x_0 = a \qqand
	y(0) = y_0 = c.
$$
We can look at $\xdot$ and $\ydot$ in terms of equations
\ref{eqn:A03.5.eqn2} as well as in terms of the original equations
\ref{eqn:A03.5.eqn1}:
$$
	\usamatrix{rrrrrrrrr}{
\xdot &=&-az \sin(zt) + bz \cos(zt) &=& -zy &=& -cz \cos(zt) &-& dz \sin(zt)) \\
\ydot &=&-cz \sin(zt) + dz \cos(zt) &=&  zx &=&  az \cos(zt) &+& bz \sin(zt))
	}
$$
so at $t=0$
$$
	\xdot(0) = bz = -cz \qqand
	\ydot(0) = dz =  az.
$$
When $z = 0$ the flow is stationary from equation \ref{eqn:A03.5.eqn1};
for $z\ne 0$ we can divide to obtain $b=-c$ and $d=a$.  Then we have
\begin{eqnarray*}
	x &=& x_0 \cos(zt) - y_0 \sin(zt) \\
	y &=& y_0 \cos(zt) + x_0 \sin(zt) \\
	z &=& z_0.
\end{eqnarray*}

This is stationary at the poles ($x_0=y_0=0$) as well as at the equator.  For
$z>0$, the flow is counterclockwise; for $z<0$, the flow is clockwise.  To
graph the flow,  note that the rotation speed depends only on $z$ and so it
suffices to see how the Greenwich meridian ($y=0$ and $x>0)$ flows.  What is
the magnitude of the velocity vector?  From \ref{eqn:A03.5.eqn1} we have
$$
	\|(\xdot, \ydot, \zdot)\| = \sqrt{z^2(x^2+y^2)} = \sqrt{z^2(1-z^2)}
	= |z(1-z)|.
$$
This takes its maximum at latitude $\pm 30^\circ$.
[xxx to do:  insert a nice Matlab plot here.]

%% ================================================================
\newpage
\section{Fall 2002}

%% ----------------------------------------------------------------
\subsection{Fall 2002 \#5.}

Let $\alpha$ be a closed two-form on $\S^4$.  Prove that $\alpha \wedge \alpha$
vanishes at some point.

\textbf{Solution (David Herzog).}  Since $\S^4$ has trivial $k$-th level
cohomology except for $k=0$ and $k=4$, $\alpha$ is exact: say, $\alpha =
d\beta$.  Then $\alpha\wedge\alpha$ is also exact:
$$
	d(\beta \wedge d\beta) = d\beta \wedge d\beta = \alpha \wedge \alpha.
$$
Suppose to the contrary that $\alpha \wedge \alpha$ vanishes nowhere.  Since it
is a top-level form on a compact orientable manifold, it is an orientation
form.  Then
\begin{eqnarray*}
	\int_{\S^4} \alpha \wedge \alpha &\ne& 0
\end{eqnarray*}
while on the other hand
\begin{eqnarray*}
	\int_{\S^4} \alpha \wedge \alpha
	= \int_{\S^4} d(\beta \wedge d\beta)
	= \int_{\D\S^4} \beta \wedge d\beta
	= 0
\end{eqnarray*}
by Stokes' theorem.  This is the desired contradiction.

%% ----------------------------------------------------------------
\subsection{Fall 2002 \#6.}

Let $f:\C \to \C$ be given by $f(z) = e^z$.  Does there exist a two-form $\eta$
on $\C$ such that $f^*\eta = dx\wedge dy$, where $z=x+iy$?

\textbf{Solution (Chol Park).}  There does not.

First note that in terms of $x$ and $y$ we have
\begin{eqnarray*}
	f(z) &=& e^{x+iy} = e^x\cos y + ie^x\sin y.
\end{eqnarray*}
Let
$$
	\eta = g(x,y)dx\wedge dy
$$
be such a two-form.  Then
\begin{eqnarray*}
	f^*\eta &=& g (e^x\cos y, e^x\sin y) \; d(e^x\cos y)\wedge d(e^x\sin y)
\end{eqnarray*}
Now
\begin{eqnarray*}
	d(e^x\cos y) &=& e^x\cos y\,dx - e^x\sin y\,dy
\end{eqnarray*}
and
\begin{eqnarray*}
	d(e^x\sin y) &=& e^x\sin y\,dx + e^x\cos y\,dy
\end{eqnarray*}
so
\begin{eqnarray*}
	d(e^x\cos y) \wedge d(e^x\sin y)
	&=& \left(e^x\cos y\,dx - e^x\sin y\,dy\right)
		\wedge \left(e^x\sin y\,dx + e^x\cos y\,dy\right) \\
	&=& e^{2x}\,dx\wedge dy.
\end{eqnarray*}
Then
\begin{eqnarray*}
	f^*\eta &=& g (e^x\cos y, e^x\sin y) e^{2x}\,dx\wedge dy = dx\wedge dy
\end{eqnarray*}
forces
$$
	g (e^x\cos y, e^x\sin y) = e^{-2x}
$$
Now observe that
\begin{eqnarray*}
	|e^z|^2 &=& |e^x \cos y + i e^x \sin y| \\
	&=& (e^x \cos y + i e^x \sin y) (e^x \cos y - i e^x \sin y) \\
	&=& e^{2x}
\end{eqnarray*}
so
\begin{eqnarray*}
	g (e^x\cos y, e^x\sin y) = \frac{1}{|e^z|^2}
\end{eqnarray*}
from which
$$
	g(x,y) = \frac{1}{|z|^2}
$$
which is not defined at $z=0$.

%% ----------------------------------------------------------------
\subsection{Fall 2002 \#7.}

Let $\T^2$ be the standard two-dimensional torus with $\Z$-periodic
coordinates $(x,y)$.  Consider the vector field
$$
	V = \sin(2\pi y)\fracDx.
$$
Prove that any vector field $Y$ which commutes with $V$, i.e. $[Y,V]=0$,
is collinear with $V$, i.e. has zero coefficient in front of $\slashDy$.

\textbf{Solution (David Herzog and Jordan Schettler).}  Let $Y$ be a general
vector field, i.e.
$$
	Y = f(x,y) \fracDx + g(x,y) \fracDy.
$$
The commutativity condition is then
\begin{eqnarray*}
	0 &=& [Y,V] = YV-VY \\
	%
	&=& \left(f(x,y) \fracDx + g(x,y) \fracDy\right) \sin(2\pi y)\fracDx
	-   \sin(2\pi y)\fracDx  \left(f(x,y) \fracDx + g(x,y) \fracDy\right) \\
	%
	&=& 2 \pi g(x,y) \cos(2\pi y)\fracDx
	-   \sin(2\pi y) \frac{\Df}{\Dx} \fracDx
	-   \sin(2\pi y) \frac{\Dg}{\Dx} \fracDy.
\end{eqnarray*}
Equating coefficients on $\slashDx$ and $\slashDy$ gives
\begin{eqnarray*}
	2 \pi g(x,y) \cos(2\pi y) &=& \sin(2\pi y) \frac{\Df}{\Dx} \\
	\sin(2\pi y) \frac{\Dg}{\Dx} &=& 0.
\end{eqnarray*}
The second equation forces $g$ to be constant in $x$, i.e. $g=g(y)$.  (When
$\sin(2\pi y) \ne 0$, $\Dg/\Dx = 0$; the case $\sin(2\pi y)=0$ forces
$\Dg/\Dx=0$ by continuity.)
The first equation gives
$$
	2 \pi\cot(2\pi y) g(y) = \frac{\Df}{\Dx}
$$
i.e. $\Df/\Dx$ is some function of $y$.  We are asked to show that $g=0$, and
it suffices to show that $\Df/\Dx = 0$.  Taking an antiderivative gives
$$
	f(x,y) = x F(y) + G(y)
$$
for some $F$ and $G$.  But due to periodicity of $f$, this is also
$$
	f(x,y) = x F(y) + G(y) = (x+1) F(y) + G(y)
$$
so, subtracting the two,
\begin{eqnarray*}
	x F(y) &=& (x+1) F(y) \\
	F(y) &=& 0.
\end{eqnarray*}
Then $f(x,y) = G(y)$ which implies $\Df/\Dx = 0$, so $g=0$ as desired.

\begin{center}* * *\end{center}

\textbf{Variations.}  Note that the particular form of $\sin(2\pi y)$ was not
important:  all that mattered was that it was not dependent on y, and was zero
only on a measure-zero subset of the torus.

%% ================================================================
\newpage
\section{Spring 2002}

%% ----------------------------------------------------------------
\subsection{Spring 2002 \#6.}

Does there exist a differentiable map $F: \T^4 \to \T^4$ of the
four-dimensional torus $\T^4 = \R^4/\Z^4$ such that
\begin{eqnarray*}
	F^*[dx_1 \wedge dx_2] &=& [dx_2 \wedge dx_3] \qand \\
	F^*[dx_1 \wedge dx_3] &=& [dx_1 \wedge dx_4],
\end{eqnarray*}
where $[\cdot]$ stands for the de Rham cohomology class and $F^*$ is the
map induced by $F$ in cohomology?

\textbf{Solution (Victor Piercey).}  This is a simple trick and computation,
but it first requires a lemma, namely, showing that the wedge product on forms
is well-defined with respect to cohomology classes, i.e. $[\omega] \wedge
[\eta] = [\omega \wedge \eta]$.  I will prove the lemma below.  (Note:  The
wedge on the left-hand side is sometimes called the \emph{cup product}, written
$[\omega] \cupprod [\eta]$, using \texttt{{\textbackslash}cupprod} in
{\LaTeX}.)

The insight behind the trick is that the expressions
$[dx_1 \wedge dx_2]$ and
$[dx_2 \wedge dx_3]$ repeat $dx_2$, while
$[dx_2 \wedge dx_3]$ and
$[dx_1 \wedge dx_4]$ do not.
Suppose there exists such an $F$.  Let
$$
	\omegabar = [dx_1 \wedge \dx_2]
	\qqand
	\etabar = [dx_1 \wedge \dx_3].
$$
Then, given the lemma,
\begin{eqnarray*}
	F^*(\omegabar \wedge \etabar) &=& F^*(\omegabar) \wedge F^*(\etabar) \\
	F^*([dx_1 \wedge dx_2] \wedge [dx_1 \wedge dx_3]])
		&=& [dx_2 \wedge dx_3] \wedge [dx_1 \wedge dx_4] \\
	F^*([dx_1 \wedge dx_2 \wedge dx_1 \wedge dx_3])
		&=& [dx_2 \wedge dx_3 \wedge dx_1 \wedge dx_4] \\
	F^*(0) &=& [dx_1 \wedge dx_2 \wedge dx_3 \wedge dx_4] \\
	0 &=& [dx_1 \wedge dx_2 \wedge dx_3 \wedge dx_4].
\end{eqnarray*}
Now, $\T^4$ is a compact orientable 4-dimensional manifold; the right-hand side
is a volume form and hence non-vanishing, but here we have it vanishing which
is the desired contradiction.

\begin{lem*}
Define a wedge product on cohomology classes by
$$
	[\omega] \wedge [\eta] = [\omega \wedge \eta].
$$
This product is well-defined.
\begin{proof}
By symmetry, it suffices to show that the product is well-defined with respect
to choice of representative for the equivalence class of $\omega$.  That is, if
$$
	[\omega_1] = [\omega_2],
$$
then I need to show that
$$
	[\omega_1 \wedge \eta] = [\omega_2 \wedge \eta].
$$

Let $k$ be the order of $\omega$, and let $\theta$ be a $(k-1)$-form.  (I.e. I
am using $\omega_1=\omega$ and $\omega_2=\omega+d\theta$.) Then
$$
	[\omega] = [\omega + d\theta]
$$
by definition of cohomology classes:  $(\omega+d\theta)-\omega$ is exact.  Then
since $\eta$ is closed, $d(\theta \wedge \eta) = d(\theta \wedge \eta)$.
Since $d(\theta\wedge\eta)$ is visibly exact, we have
\begin{eqnarray*}
	{}[d(\theta \wedge \eta)] &=& [0] \\
	{}[d\theta \wedge \eta] &=& [0] \\
	{}[\omega\wedge\eta + d\theta \wedge \eta] &=& [\omega\wedge\eta] \\
	{}[(\omega + d\theta) \wedge\eta] &=& [\omega\wedge\eta]
\end{eqnarray*}
as desired.
\end{proof}
\end{lem*}

%% ----------------------------------------------------------------
\subsection{Spring 2002 \#7.}

Assume that a 6-element group $\Gamma$, isomorphic to the group of
permutations on three letters $\mcS_3$, acts on $X=\S^3$ freely.
Compute $\pi_1(Y)$ and $H_1(Y,\Z)$, where $Y=X/\Gamma$.

\textbf{Solution (Jordan Schettler).}  Since $\mcS_3$ is finite, it acts
properly discontinuously.  We are given that $\mcS_3$ acts freely, and $\S^3$
is a Hausdorff space.  Thus we are justified in saying that $X$ is a normal
covering space with the canonical projection $p: X \to X/\Gamma$ being the
covering-space map.  Also note that $X$ is path-connected and locally
path-connected.  Since $X$ is simply connected, it is a universal cover and so
$$
	\pi_1(Y) / p_*(\pi_1(X)) \cong \mcS_3,
$$
but
$$
	p_*(\pi_1(X)) = p_*(0) = 0.
$$
Thus
$$
	\pi_1(Y) \cong \mcS_3.
$$
By the Hurewicz theorem, $H_1(Y,\Z)$ is the abelianization of $\mcS_3$,
namely,
$$
	\mcS_3/\mcS_3' = \mcS_3/\mcA_3 \cong \Z/2\Z
$$
since $[\mcS_3:\mcA_3] = 2$.

\textbf{Variations:}  Change the group.

%% ================================================================
\newpage
\section{Fall 2001}

%% ----------------------------------------------------------------
\subsection{Fall 2001 \#3.}
\label{sec:F01.3}

Consider the surface $X$ in $\R^3$ which is defined in cylindrical coordinates
by the equation
$$
	(r-2)^2 + z^2 = 1.
$$
(Recall that in cylindrical coordinates $x=r\cos\theta$, $y=r\sin\theta$, and
$z=z$.)

(a) Find an explicit basis for the homology (in all degrees) of this surface.

(b) By introducing coordinates or parameterizing the surface, find explicit
expressions for forms which represent the corresponding dual basis in de Rham
cohomology.

\textbf{Solution.} For part (a), we could use Mayer-Vietoris.  However, it is
simpler than that.  This surface is the one-holed torus.  It has one connected
component, so
$$
	H_0(X)= \Z[q]
$$
for some choice of $q \in X$.  It is compact, orientable, and two-dimensional,
so
$$
	H_2(X)=\Z[X].
$$
The usual Seifert-van Kampen argument shows that
$$
	\pi_1(X,q) = \Z[\alpha]\oplus\Z[\beta]
$$
where $\alpha$ is the small loop and $\beta$ is the large loop.  [xxx insert
figure here.]  The fundamental group is already abelian, so its abelianization
is
$$
	H_1(X)=\Z[\alpha]\oplus\Z[\beta].
$$

Now for part (b).  By duality, since none of the homology groups have torsion,
we know ahead of time that $H^0(X)$, $H^1(X)$, and $H^2(X)$ have dimensions 1,
2, and 1, respectively.  To obtain explicit generators, let $\phi$ be the angle
through the cross-sectional circle and let $\theta$ be the
surface-of-revolution angle.  [xxx insert figure here.] Notice that both $\phi$
and $\theta$ run from $0$ to $2\pi$.  Then
$$
	r = 2+\cos\phi \qand z = \sin\phi.
$$
The torus is parameterized by $\theta$ and $\phi$, into ambient $r, z, \theta$
coordinates, by
\begin{eqnarray*}
	r &=& 2+\cos\phi \\
	z &=& \sin\phi \\
	\theta &=& \theta.
\end{eqnarray*}
To find explicit tangent vectors for $X$, we apply the directional derivatives
to the parameterization as usual:
\begin{eqnarray*}
	\fracDtheta &=&
		\colcvecthree
			{\slashDtheta(2+\cos\phi)}
			{\slashDtheta(\sin\phi)}
			{\slashDtheta(\theta)}
	= \colcvecthree{0}{0}{1}
	= \fracDtheta
	\\
	\fracDphi &=&
		\colcvecthree
			{\slashDphi(2+\cos\phi)}
			{\slashDphi(\sin\phi)}
			{\slashDphi(\theta)}
	= \colcvecthree
		{-\sin\phi}
		{\cos\phi}
		{0}
	= -\sin\phi\,\fracDr +\sin\phi\,\fracDz.
\end{eqnarray*}
(In the first line, the $\slashDtheta$ on the left is for $X$ and the one on
the right is for $\R^3$.) Then $d\theta$ and $d\phi$ are the duals of
$\slashDtheta$ and $\slashDphi$, respectively.  These are by construction
linearly independent, given the linear independence of $\slashDphi$ and
$\slashDtheta$ which follows from their orthogonality.

To find the area form, we contract the normal field with the Euclidean volume
form on $\R^3$ as usual.  Since $X$ is defined by $(r-2)^2+z^2=1$, a normal is
found by taking the Jacobian of the left-hand side:
\begin{eqnarray*}
	\vecn &=& \pcmatrix{
		\slashDr((r-2)^2+z^2) \\
		\slashDz((r-2)^2+z^2) \\
		\slashD\theta((r-2)^2+z^2)
	}
	= \pcmatrix{
		2(r-2) \\
		2z \\
		0
	}.
\end{eqnarray*}
The factor of two is irrelevant so we may as well write
\begin{eqnarray*}
	\vecnhat
	&=& \pcmatrix{
		r-2 \\
		z \\
		0
	}
	= \pcmatrix{
		2+\cos\phi-2 \\
		\sin\phi \\
		0
	}
	= \pcmatrix{
		\cos\phi \\
		\sin\phi \\
		0
	}
\end{eqnarray*}
where the last two steps furnish a visual sanity check.  The standard volume
form on $\R^3$ is, in cylindrical coordinates,
$$
	dV = r \,dr\,dz\,d\theta.
$$
Contracting, we obtain
\begin{eqnarray*}
	\vecnhat \ctt dV
	&=& \left((r-2)\,\fracDr + z\,\fracDz \right)
		\ctt (r \,dr\,dz\,d\theta) \\
	&=& r(r-2)\,dz\wedge d\theta - rz\,dr\wedge d\theta.
\end{eqnarray*}
Pulling this back to $X$, we obtain the area form for the torus:
\begin{eqnarray*}
	dA &=& i^*(\vecnhat \ctt dV) \\
	&=& (2+\cos\phi)\cos\phi \,d(\sin\phi) \wedge d\theta
		- (2+\cos\phi) \sin\phi \,d(2+\cos\phi) \wedge d\theta \\
	&=& (2+\cos\phi)\cos^2\phi \,d\phi \wedge d\theta
		+ (2+\cos\phi) \sin^2\phi \,d\phi \wedge d\theta \\
	&=& (2+\cos\phi) \,d\phi \wedge d\theta.
\end{eqnarray*}

These are all forms of the appropriate degrees.  To finish off, we need to to
show that $d\phi$, $d\theta$, and $dA$ generate cohomology classes, i.e. that
they are closed but not exact.  The area form is closed because it is
top-level; the 1-forms are closed since they are locally $d$ of $0$ forms.  For
exactness, we use the usual argument: If a $k$-form $\omega$ is exact, say
$\omega=d\eta$ for a $(k-1)$-form $\eta$, and if $c$ is a $k$-cycle (i.e. a
$k$-chain without boundary), then by Stokes' theorem
$$
	\int_c \omega = \int_c d\eta = \int_{\D c} \eta = \int_0 \eta = 0.
$$
The contrapositive is that if the integral of $\omega$ over cycle is non-zero,
then $\omega$ is exact.  For $d\phi$, integrate over the circle parameterized
by $\phi=0$ to $2\pi$ with $\theta=0$:
$$
	\int_{\phi=0}^{\phi=2\pi} d\phi = 2\pi \ne 0.
$$
Similarly,
$$
	\int_{\theta=0}^{\theta=2\pi} d\theta = 2\pi \ne 0.
$$
Last, integrate the area form over the torus (missing the seams which are
zero-measure):
$$
	\int_{\phi=0}^{\phi=2\pi}
	\int_{\theta=0}^{\theta=2\pi}
	(2+\cos\phi) \,d\theta\,d\phi
	= 8\pi \ne 0.
$$

In summary,
\begin{eqnarray*}
	H^0_{dR}(X) &=& \R[q]\\
	H^1_{dR}(X) &=& \R[d\phi]\oplus \R[d\theta]\\
	H^2_{dR}(X) &=& \R[(2+\cos\phi)d\phi\wedge d\theta].
\end{eqnarray*}

%% ----------------------------------------------------------------
\subsection{Fall 2001 \#4.}
\label{sec:F01.4}

For each $t \in \R$, let $\phi_t$ denote the map of $\S^2$ into itself which is
defined by
$$
	\phi_t: \S^2 \to \S^2: \colvecthree{x}{y}{z} \mapsto
	\colvecthree{\cos(t)x - \sin(t) y}{\sin(t)x + \cos(t) y}{z}.
$$
Show that $\phi_t$ is a one-parameter group of diffeomorphisms, and compute and
graph the vector field $\vecv$ on $\S^2$ for which $\phi_t$ is the corresponding
flow.

Remark:  In defining the vector field $\vecv$, please specify how you are
viewing the tangent bundle of $\S^2$.

\textbf{Solution.}  To show that $\phi_t$ is a one-parameter group of
diffeomorphisms, we need to show:
\begin{itemize}
\item Each $\phi_t$ is bijective on $\S^2$.
\item Each $\phi_t$ is smooth with smooth inverse.
\item For each $s$ and $t$ we have $\phi_s \circ \phi_t = \phi_{s+t}$.
\end{itemize}
First note that the map $\phi_t$ may be written in matrix form as
$$
	\colvecthree{x}{y}{z} \stackrel{\phi_t}{\to}
	\mxrsqthree
		{\cos(t)}{-\sin(t)}{0}
		{\sin(t)}{ \cos(t)}{0}
		{0}{0}{1}
	\colvecthree{x}{y}{z}.
$$
The matrix has determinant $\cos^2(t) + \sin^2(t)=1 \ne 0$ so it is invertible,
hence bijective.  In particular its inverse is
$$
	\colvecthree{x}{y}{z} \stackrel{\phi_t^{-1}}{\to}
	\mxrsqthree
		{\cos(t)}{\sin(t)}{0}
		{-\sin(t)}{\cos(t)}{0}
		{0}{0}{1}
	\colvecthree{x}{y}{z}.
$$

This much shows that $\phi_t$ is bijective on $\R^3$.  To show bijectivity on
$\S^2$, we need to show that $\phi_t(\vecq) \in \S^2$ for all $q \in \S^2$, and
likewise for $\phi_t^{-1}$.  Now, we've already shown that $\phi_t^{-1}$ is
rotation by $-t$ (since $\cos(-t)=\cos(t)$ and $\sin(-t)=\sin(t)$) so it
suffices to show this for $\phi_t$ only.  Namely, the coordinates of $(u, v, w)
= \phi_t(x,y,z)$ should satisfy $u^2+v^2+w^2=1$.  Check:
\begin{eqnarray*}
	u^2 + v^2 + w^2 &=& 
		(\cos(t)x - \sin(t)y)^2 + (\sin(t)x + \cos(t)y)^2 + z^2 \\
&&\\
	&=& \cos^2(t) x - \cos(t)\sin(t)xy + \sin^2(t)y^2 \\
	&+& \sin^2(t)x^2 + \cos(t)\sin(t)xy + \cos^2(t)y^2 \\
	&+& z^2 \\
&&\\
	&=& x^2 + y^2 + z^2 = 1.
\end{eqnarray*}

The transformation is linear, hence it is its own derivative and thus eminently
differentiable; likewise for the inverse.  (Remember we're checking smoothness
for fixed $t$.  Smoothness with respect to $t$ is a different question.) For
the composition property, we compute (using the sum formulas for sine and
cosine)
\begin{eqnarray*}
	\phi_{s+t}
	\colvecthree{x}{y}{z}
	%
	&=&
	\mxrsqthree
		{\cos(s+t)}{-\sin(s+t)}{0}
		{\sin(s+t)}{ \cos(s+t)}{0}
		{0}{0}{1}
	\colvecthree{x}{y}{z} \\
	%
	&=&
	\mxrsqthree
		{\cos(s)\cos(t)-\sin(s)\sin(t),}{-\sin(s)\cos(t)-\cos(s)\sin(t),}{0}
		{\sin(s)\cos(t)+\cos(s)\sin(t),}{ \cos(s)\cos(t)-\sin(s)\sin(t),}{0}
		{0,}{0,}{1}
	\colvecthree{x}{y}{z}.
\end{eqnarray*}
On the other hand,
\begin{eqnarray*}
	\phi_s\circ\phi_t
	\colvecthree{x}{y}{z}
	&=&
	\mxrsqthree
		{\cos(s)}{-\sin(s)}{0}
		{\sin(s)}{ \cos(s)}{0}
		{0}{0}{1}
	\mxrsqthree
		{\cos(t)}{-\sin(t)}{0}
		{\sin(t)}{ \cos(t)}{0}
		{0}{0}{1}
	\colvecthree{x}{y}{z} \\
	&=&
	\mxrsqthree
		{\cos(s) \cos(t)-\sin(s)\sin(t),}{-\cos(s)\sin(t)-\sin(s)\cos(t),}{0}
		{\sin(s) \cos(t)+\cos(s)\sin(t),}{-\sin(s)\sin(t)+\cos(s)\cos(t),}{0}
		{0}{0}{1}
	\colvecthree{x}{y}{z}
\end{eqnarray*}
which is the same as desired.  This was the last item needed to prove
that the $\phi_t$'s form a one-parameter group of diffeomorphisms.

Recalling the necessary formula for the corresponding vector field, we have
\begin{eqnarray*}
	\vecv\pickpipe{\colvecthree{x}{y}{z}}
	&=& \frac{\D \phi_t}{\D t}\pickpipe{t=0}\colvecthree{x}{y}{z} \\
	&=& \frac{\D }{\D t}\pickpipe{t=0}
		\colvecthree{\cos(t)x - \sin(t) y}{\sin(t)x + \cos(t) y}{z} \\
	&=& \colvecthree{-\sin(t)x - \cos(t) y}{\cos(t)x - \sin(t) y}{0}
		\pickpipe{t=0} \\
	&=& \colvecthree{-y}{x}{0}\pickpipe{\colvecthree{x}{y}{z}} \\
	&=& -y\,\slashDx + x\,\slashDy.
\end{eqnarray*}
This is of course rigid rotation about the $z$-axis.  [xxx insert figure here.]
Here, we are viewing the tangent bundle as directional derivatives.

This is a vector field on $\R^3$.  [xxx to do:  convert to graph coordinates, or spherical.]

\begin{center}* * *\end{center}

\textbf{Variation:}  Let $\omega = x\,dy\wedge dz +
y\,dz\wedge dx + z\, dx\wedge dy$.  Compute $\phi_t^*(\omega)$.

\textbf{Solution.}  Let $c=\cos(t)$ and $s=\sin(t)$.  Then, replacing $x$ with
$cx-sy$ and $y$ with $sx+cy$, we have
\begin{eqnarray*}
	\phi_t^*(\omega)
		&=& (cx-sy)\, d(sx+cy)\wedge dz \\
		&+& (sx+cy)\, dz\wedge d(cx-sy) \\
		&+& z\, d(cx-sy)\wedge d(sx+cy) \\
	&&\\
		&=& (cx-sy)\, (sdx\wedge dz +c dy\wedge dz) \\
		&+& (sx+cy)\, (c dz\wedge dx - s dz \wedge dy) \\
		&+& z\, (cdx-sdy)\wedge (sdx+cdy) \\
	&&\\
		&=& (cx-sy)\, (-sdz\wedge dx +c dy\wedge dz) \\
		&+& (sx+cy)\, (c dz\wedge dx + s dy \wedge dz) \\
		&+& z\,(c^2+s^2) dx \wedge dy \\
	&&\\
		&=& -cs\,x\, dz\wedge dx
		+c^2\,x\, dy \wedge dz
		+s^2\,y\, dz \wedge dx
		-cs\,y\, dy\wedge dz \\
	&+& cs\,x\, dz\wedge dx
		+s^2\,x\, dy \wedge dz
		+c^2\,y\, dz \wedge dx
		+cs\,y\, dy\wedge dz \\
		&+& z\, dx \wedge dy \\
	&&\\
		&=& x\, dy \wedge dz +y\, dz \wedge dx + z\, dx \wedge dy \\
	&=& \omega.
\end{eqnarray*}
Note that this is a bit messy, only because I chose a form with three terms in
it.

\textbf{Variation:}  Show that the two-form $\omega = x\,dy\wedge dz +
y\,dz\wedge dx + z\, dx\wedge dy$ is invariant under the flow of $\phi_t$.

\textbf{Solution.}  One option is to do what we just did.  Another option is to
see whether the Lie derivative of $\omega$ in the direction of $\vecv$ is zero.
Since
$$
	\vecv = -y\, \fracDx + x\, \fracDy,
$$
we have
$$
	L_\vecv(x) = -y, \quad
	L_\vecv(y) =  x, \quad
	L_\vecv(z) =  0.
$$
Recall that $L_\vecv$ is a derivation, i.e. it follows the product rule through
wedges.  Also, it commutes with $d$ for exact forms.  For the first term of
$\omega$, namely $x\, dy\wedge dz$, we then have
\begin{eqnarray*}
	L_\vecv(x\, dy \wedge dz)
		&=& L_\vecv(x) dy \wedge dz
		\;+\; x \, d(L_\vecv(y)) \wedge dz
		\;+\; x \, dy \wedge d(L_\vecv(z)) \\
	%
		&=& -y\, dy \wedge dz
		\;+\; x\, dx \wedge dz
		\;+\; x \, dy \wedge 0 \\
	%
		&=& -y\, dy \wedge dz \;-\; x\, dz \wedge dx.
\end{eqnarray*}
For the second term:
\begin{eqnarray*}
	L_\vecv(y\, dz \wedge dx)
		&=& L_\vecv(y) dz \wedge dx
		\;+\; y \, d(L_\vecv(z)) \wedge dx
		\;+\; y \, dz \wedge d(L_\vecv(x)) \\
	%
		&=& x\, dz \wedge dx
		\;+\; y \, 0 \wedge dx
		\;-\; y \, dz \wedge dy \\
	%
		&=& x\, dz \wedge dx \;+\; y \, dy \wedge dz.
\end{eqnarray*}
For the third term:
\begin{eqnarray*}
	L_\vecv(z\, dx \wedge dy)
		&=& L_\vecv(z) dx \wedge dy
		\;+\; z \, d(L_\vecv(x)) \wedge dy
		\;+\; z \, dx \wedge d(L_\vecv(y)) \\
	%
		&=& 0\, dx \wedge dy
		\;-\; z \, dy \wedge dy
		\;+\; z \, dx \wedge dx \\
	%
		&=& 0.
\end{eqnarray*}
Combining these, we have
\begin{eqnarray*}
	L_\vecv(\omega)
	&=& -y\, dy \wedge dz \;-\; x\, dz \wedge dx \\
	&+& x\, dz \wedge dx \;+\; y \, dy \wedge dz \\
	&=& 0.
\end{eqnarray*}
which shows that $\omega$ is invariant under the flow of $\phi_t$.

\begin{center}* * *\end{center}

An alternative (and sometimes quicker) solution is to use Cartan's magic
formula:
$$
	L_\vecv(\omega) = \vecv \ctt (d\omega) + d(\vecv \ctt \omega).
$$
Since $\omega$ is an order-2 (i.e. top-level) form on a 2-dimensional manifold,
$d\omega=0$.  For the other term, we have
\begin{eqnarray*}
	\vecv \ctt \omega
	&=&   -y\, \fracDx \ctt
		\left(
			x\,dy\wedge dz +
			y\,dz\wedge dx +
			z\, dx\wedge dy
		\right)
	\;+\;  x\, \fracDy \ctt
		\left(
			x\,dy\wedge dz +
			y\,dz\wedge dx +
			z\, dx\wedge dy
		\right) \\
	&=&   -y\, \fracDx \ctt (y\, dz\wedge dx)
	\;-\;  y\, \fracDx \ctt (z\, dx\wedge dy)
	\;+\;  x\, \fracDy \ctt (x\, dy\wedge dz)
	\;+\;  x\, \fracDy \ctt (z\, dx\wedge dy) \\
%
	&=&    y^2 \, dz
	\;-\;  yz  \, dy
	\;+\;  x^2 \, dz
	\;-\;  xz  \, dx.
\end{eqnarray*}
Then
\begin{eqnarray*}
	d(\vecv \ctt \omega)
	&=&    d(y^2 \, dz)
	\;-\;  d(yz  \, dy)
	\;+\;  d(x^2 \, dz)
	\;-\;  d(xz  \, dx) \\
%
	&=&    2y \,dy \wedge dz
	\;-\;   y \,dz \wedge dy
	\;+\;  2x \,dx \wedge dz
	\;-\;   x \,dz \wedge dx \\
%
	&=&    2y \,dy \wedge dz
	\;+\;   y \,dy \wedge dz
	\;-\;  2x \,dz \wedge dx
	\;-\;   x \,dz \wedge dx \\
%
	&=&    3y \,dy \wedge dz
	\;-\;  3x \,dz \wedge dx.
\end{eqnarray*}
Here, it's not quite so obvious that this is zero.  This is one of the perils
of working with ambient (too many) coordinates:  the relations (deriving
ultimately from $x^2+y^2+z^2=1$) aren't always immediately apparent.  My
resolution of this problem is to use good coordinates, namely, $\phi$ and
$\theta$.  Plugging in $x=\sin\phi\cos\theta$, $y=\sin\phi\sin\theta$, and
$z=\cos\phi$, after simplification I get
\begin{eqnarray*}
	y \,dy\wedge dz &=& \sin^3\phi\sin\theta\cos\theta d\phi\wedge d\theta\\
	x \,dz\wedge dx &=& \sin^3\phi\sin\theta\cos\theta d\phi\wedge d\theta
\end{eqnarray*}
so they indeed cancel out.

%\begin{eqnarray*}
%	d\omega
%		&=& dx \wedge dy \wedge dz
%		\;+\; dy \wedge dz \wedge dx
%		\;+\; dz \wedge dx \wedge dy
%		\;=\; 3 dx \wedge dy \wedge dz \\
%	\vecv \ctt d\omega &=&
%		-3 y\, \fracDx \ctt (dx \wedge dy \wedge dz)
%		+3 x\, \fracDy \ctt (dx \wedge dy \wedge dz) \\
%	&=&
%		-3 y\, (dy \wedge dz)
%		-3 x\, (dx \wedge dz).
%\end{eqnarray*}

%% ----------------------------------------------------------------
\subsection{Fall 2001 \#5.}
\label{sec:F01.5}

(a)  Let $\C^*$ denote the nonzero complex numbers.  Determine all the covering
spaces (up to isomorphism) of $\C^*$ and their automorphism groups.

(b) Let $G$ denote a finite subgroup of $\bH_1$, the group of unit quaternions
(which you may identify with $\mathrm{SU}(2,\C)$ if you prefer).  Using
covering space theory, explain why $G$ is isomorphic to the fundamental group
of the coset space $\bH_1/G$.

\textbf{Solution.}  For part (a), \ldots [xxx type me up].

For part (b), note that the unit quaternions are
$$
	\left\{a + b\veci + c\vecj + d\veck : a^2+b^2+c^2+d^2=1\right\},
$$
namely, the 3-sphere, which is simply connected and thus has trivial
fundamental group.

The 3-sphere is a Hausdorff space since it is a metric space (inheriting the
metric from $\R^4$).  The action of $G$ is free on $\bH_1$: to see this, let $h
\in \bH_1$ and $g \in G$.  Then, since $G$ is a subgroup of $\bH_1$, $g \cdot h
= h$ implies $g=1$.  (This result holds whenever a group is acted on by a
subgroup of itself.)  Since $G$ is finite, it automatically acts properly
discontinuously.  Given these three facts, the action of $G$ on $\bH_1$ is a
covering-space action.

Let $p$ be the canonical projection from $\bH_1$ to $\bH_1/G$.  Since
$p_*(0)=0$ which is normal in $\pi_1(\bH_1/G)$, we know the automorphism group
of $\bH_1$ over $\bH_1/G$ is isomorphic to $\pi_1(\bH_1)/0$ which in turn is
isomorphic to $\pi_1(\bH_1/G)$.  It remains to show that, in turn, $G$ is
isomorphic to $\Aut(\bH_1, \bH_1/G)$.  But this is true because the action of
$G$ is a covering-space action.

%% ----------------------------------------------------------------
\subsection{Fall 2001 \#7.}

Consider the 1-form $\alpha = x\,dy - y\,dx$ in $\R^3$.  Prove the following:
if $f(x,y,z)\in C^\infty(\R^3)$ and $f\alpha$ is a closed 1-form, then $f$ is
identically zero.  (Hint:  use cylindrical coordinates.)

\textbf{Solution (Victor Piercey).}  First, convert $\alpha$ to cylindrical
coordinates.  Since $x=r\cos\theta$ and $y=r\cos\theta$,
\begin{eqnarray*}
	\alpha &=& (r^2\cos^2\theta \,d\theta + r^2\sin^2\theta \,d\theta)
	= r^2\, d\theta \\
	d\alpha &=& 2r\, dr \wedge d\theta.
\end{eqnarray*}
Also
$$
	df = \frac{\Df}{\Dr}\,dr
	+ \frac{\Df}{\D\theta}\,d\theta
	+ \frac{\Df}{\Dz}\,dz.
$$
Since $f\alpha$ is closed, we have
\begin{eqnarray*}
	0 &=& d(f\alpha) \\
	&=& df\wedge \alpha + fd\alpha \\
%
	&=& \left(
		\frac{\Df}{\Dr}\,dr +
		\frac{\Df}{\D\theta}\,d\theta +
		\frac{\Df}{\Dz}\,dz
		\right) \wedge r^2\, d\theta
	+ 2 r f(r,\theta,z) \,dr \wedge d\theta \\
%
	&=& r^2 \frac{\Df}{\Dr}\,dr \wedge d\theta
	\;+\; r^2 \frac{\Df}{\Dz}\,dz \wedge d\theta
	\;+\; 2 r f(r,\theta,z) \,dr \wedge d\theta \\
%
	&=& \left(r^2 \frac{\Df}{\Dr}
		\;+\; 2 r f(r,\theta,z) \right) \,dr \wedge d\theta
	\;+\; r^2 \frac{\Df}{\Dz}\,dz \wedge d\theta.
\end{eqnarray*}
Equating coefficients gives the two equations
\begin{eqnarray*}
	r^2 \frac{\Df}{\Dr} + 2rf(r,\theta,z) &=& 0 \\
	r^2 \frac{\Df}{\Dz} &=& 0.
\end{eqnarray*}
For $r \ne 0$ (and at $r=0$ by continuity), the second equation shows that $f$
does not depend on $z$, i.e. $f=f(r,\theta)$.  The first equation is
\begin{eqnarray*}
	\frac{\D}{\Dr} \left( r^2 f(r,\theta) \right) &=& 0.
\end{eqnarray*}
This means that $r^2f$ has no $r$-dependence, i.e. is a function of $\theta$
alone:
\begin{eqnarray*}
	r^2f(r,\theta) &=& G(\theta) \\
	f(r,\theta) &=& \frac{G(\theta)}{r^2}.
\end{eqnarray*}
For $f$ to be well-defined at the origin, $G(\theta)$ must be constant; for $f$
to be in $C^\infty(\R^3)$, that constant must be zero.  This forces $f=0$ as
desired.

%% ================================================================
\newpage
\section{January 2001}

%% ----------------------------------------------------------------
\subsection{January 2001 \#3.}

For the function $g:\S^2 \to \R$ given by
$$
	\colvecthree{x}{y}{z} \mapsto y^2 - z,
$$
determine the critical points, the critical values, and qualitatively
describe the level sets.

\textbf{Solution.}  Let $f(x,y,z)=x^2+y^2+z^2$; $\S^2$ is the level set
of $f$ and 1.  Then
$$
	\grad f = \colvecthree{2x}{2y}{2z}
$$
which is non-zero on all of $\S^2$ (it vanishes only at the origin which, is
not contained in the two-sphere.) The method of Lagrange multipliers then
applies.  We solve
\begin{eqnarray*}
	\grad g &=& \lambda \grad f \\
	\colvecthree{0}{2y}{-1}
		&=& \colvecthree{2\lambda x}{2\lambda y}{2\lambda z}
\end{eqnarray*}
which gives the system of three equations
\begin{eqnarray*}
	2\lambda x &=& 0 \\
	2\lambda y &=& 2y \\
	2\lambda z &=& -1.
\end{eqnarray*}
If $\lambda=0$ then the third equation is contradicted.  Thus, we may divide by
$\lambda$ in the first equation to conclude $x=0$.  Now $y=0$ or $y\ne 0$:
\begin{itemize}
\item If $y=0$, then $x=y=0$ forces $z=\pm 1$.
\item If $y\ne 0$, then from the second equation $\lambda=1$, yielding
	$z=-1/2$ from the third equation.  Then $x^2+y^2+z^2=1$ forces $y=\pm
	\sqrt{3}/2$.
\end{itemize}
Thus there are four critical points:
$$
	\colvecthree{0}{0}{1},
	\colvecthree{0}{0}{-1},
	\colvecthree{0}{\sqrt{3}/2}{-1/2}, \qand
	\colvecthree{0}{-\sqrt{3}/2}{-1/2},
$$

The critical values are found by $y^2-z$.  We have
$$
	-1,
	1,
	5/4, \qand
	5/4,
$$
respectively.

Here is a pair of Matlab plots of $g$ on the sphere.  (See
\texttt{figures/y2z.m} relative to the directory where you found this file.)
The first plot is a top view (positive $z$ is up); the second plot is a bottom
view (negative $z$ is up).
\begin{center}
\psfragscanon
\includegraphics{figures/y2z.eps}
\includegraphics{figures/y2zb.eps}
\end{center}
For the qualifying exams, this is a tool we will not have.  However, I think
it's important to be able to see, at least once, in a vivid way, what these
things really look like.  Also, the picture helps us to sanity-check our work.

As for the level sets, at the critical values we have
\begin{eqnarray*}
	z &=& y^2 + 1 \\
	z &=& y^2 - 1 \\
	z &=& y^2 - 5/4.
\end{eqnarray*}
These level sets are doubly constrained:  by $y^2-z=c$, as well as by
$x^2+y^2+z^2=1$.  Since there are three variables constrained by two equations,
we expect one-dimensional level sets.  Also note that these will be isotherms
of the Matlab plot (i.e. curves of constant color).

For the first, when $y=0$ we have $z=1$; for $y\ne 0$ we have $z > 1$.  Thus
the first level set is just the north pole.

For the second, consider the following table.  Varying $y$, $z=y^2-1$ gives me
$z$; then, $x^2+y^2+z^2=1$ gives me $x$.
$$
	\begin{tabular}{|r|r|r|}
	\hline $y$ & $z$ & $x$ \\
	\hline
	\hline $ 1  $ & $ 0  $ & $0$ \\
	\hline $ 1/2$ & $-3/4$ & $\pm\sqrt{3}/4$ \\
	\hline $ 0  $ & $-1  $ & $0$ \\
	\hline $-1/2$ & $-3/4$ & $\pm\sqrt{3}/4$ \\
	\hline $-1  $ & $ 0  $ & $0$ \\
	\hline
	\end{tabular}
$$
Refining this table suggests that this second level set is a figure-eight drawn
on the bottom of the sphere:  one loop ends on the equator at $y=1$, another
ends on the equator at $y=-1$, and the two loops cross at the south pole.

For the third level set, I have (for loss of a better idea)
$z=y^2-5/4$ so $y^2=z+5/4$, giving me
\begin{eqnarray*}
	x^2 + y^2 + z^2 &=& 1 \\
	x^2 + z^2 + z + 5/4 &=& 1 \\
	x^2 + z^2 + z + 1/4 &=& 0 \\
	x^2 + (z + 1/2)^2 &=& 0.
\end{eqnarray*}
The left-hand side is non-negative, and can be zero only when $x=0$ and
$z=-1/2$.  Then $y=\pm\sqrt{3}/2$, i.e. this level set consists of the two
critical points other than the north and south poles.

It now remains only to describe the level sets for regular values.  By the
regular-value theorem, preimages of regular values are either empty, or are
embedded submanifolds of $\S^2$.  (From the plot I am guessing that in the blue
regions, the level sets are homotopic to circles, and at the bottom, they are
homotopic to a pair of unlinked circles inside the lobes of the figure eight.)
Proceeding as before, varying $c$ now and completing the square, I have
\begin{eqnarray*}
	y^2 - z &=& c \\
	y^2 &=& z + c \\
	x^2 + y^2 + z^2 &=& 1 \\
	x^2 + z^2 + z + c &=& 1 \\
	x^2 + z^2 + z + 1/4 &=& 5/4 - c \\
	x^2 + (z +1/2)^2 &=& 5/4 - c \\
\end{eqnarray*}
The best way I know to think of this is using $x,z$ graph coordinates.  Note
that as $c$ varies from $5/4$ at the non-pole critical points, to $+1$ at the
south pole, to $-1$ at the north pole, the value $5/4-c$ varies from 0 up to
$1/4$ up to $9/4$.  These are the squares of $0$, $1/2$, and $3/2$
respectively.  In $x,z$ coordinates, $x^2 + (z +1/2)^2 = a^2$ is a circle of
radius $a$ centered at $x=0$, $z=-1/2$.  (Note that these are circles only on
an $x,z$ graph; they are oblong on the sphere.)  Then:
\begin{itemize}
\item When $a=0$, we have a single point at front and back.
\item When $0<a<1/2$, we have a circle of radius $a$ on the
front and back.
\item When $a=1/2$, the two circles touch at the south pole.
\item When $1/2<a<3/2$, there are really only two \emph{arcs} of circles,
touching one another to form a single loop.
\item When $a=3/2$, there is a single point at the north pole.
\end{itemize}

%% ----------------------------------------------------------------
\subsection{January 2001 \#4.}

Consider the vector field $\vecv$ on $\R^2$ which is given by
$$
	\vecv\pickpipe{\colvectwo{x}{y}} = -y\, \fracDx + x \,\fracDy.
$$
Find an explicit expression for the flow of this vector field and graph the
trajectories.

\textbf{Solution.}  See section \ref{sec:A03.5}, which is similar although this
problem is easier.  (One finds rigid rotation rather than $z$-dependent
rotation.)

%% ----------------------------------------------------------------
\subsection{January 2001 \#5.}

(a) Determine whether the two-form $\omega = z\, dx \wedge dy$ is
exact in $\R^3$.

(b) Let $M$ denote the embedded submanifold of $\R^3$ given by
$M=\{z-x^2-y^2=1\}$.  Determine whether the restriction of $\omega$ to $M$ is
exact.

\textbf{Solution.}  For part (a), note that 
$$
	d\omega = dz \wedge dx \wedge dy = dx \wedge dy \wedge dz
$$
which is non-zero.  If $\omega$ were exact, i.e. $\omega=d\eta$, then $d\omega
= d^2\eta = 0$ which it visibly is not.  Therefore $\omega$ is not exact.

For part (b), $\omega=z\,dx\wedge dy = (x^2+y^2+1)\, dx \wedge dy$ is
top-level, so $d\omega = 0$, so the argument from part (a) does not apply.
Note that $M$ is a paraboloid; this is homeomorphic to the plane.  The
Poincar\'e lemma applies:  any closed form (we just saw $d\omega=0$) is exact.

That is enough, but I will go ahead and find an explicit 1-form $\eta$ such
that $d\eta=\omega$.  The proof of the Poincar\'e lemma (see section
\ref{sec:A06.6}) is constructive.  Namely, let
$$
	\eta = f(x,y)\,dx + g(x,y)\,dy.
$$
We want $d\eta=\omega$, i.e.
$$
	\left(\frac{\Dg}{\Dx} - \frac{\Df}{\Dy}\right) dx \wedge dy
		= \left(x^2 + y^2 + 1\right) \,dx\wedge dy.
$$
The trick, as before, is that it suffices to take only one term from within the
parentheses.  We have (letting the integration constant be zero:  we don't need
\emph{all} $\eta$, just \emph{any} $\eta$):
\begin{eqnarray*}
	\frac{\Dg}{\Dx} &=& x^2 + y^2 + 1 \\
	g(x,y) &=& \int (x^2 + y^2 + 1) \,dx \\
	&=& \frac{x^3}{3} + xy^2 + x \\
	\eta = &=& \left(\frac{x^3}{3} + xy^2 + x\right) \,dy.
\end{eqnarray*}
We can check this:
\begin{eqnarray*}
	d\eta = &=& (x^2  + y^2 + 1) \,dx \wedge dy
\end{eqnarray*}
as desired.

%% ================================================================
\newpage
\section{Fall 2000}

%% ----------------------------------------------------------------
\subsection{Fall 2000 \#3.}

Let $M=\{(x,y)|y^2-x^2=1,y>0\} \subset \R^2$.  Then $M$ is a one-dimensional
manifold with the global coordinate function $x$.  For $a \in \R$ define
$$
	R_a\colvectwo{x}{y} = \prmatrix{\cosh a & \sinh a \\ \sinh a & \cosh a}
	\colvectwo{x}{y}.
$$
Then it is easy to check that $R_a:M \to M$ and also $R_a R_b = R_{a+b}$.
(Recall that $\cosh a = \frac{1}{2}(e^a+e^{-a})$ and $\sinh a =
\frac{1}{2}(e^aie^{-a})$.)

Let $p_0=\colvectwo{0}{1}$ be the point in $M$ with $x$ coordinate equal to 0.
Every point $p$ in $M$ can be expressed in the form $p=R_a p_0$ for a unique
choice of $a \in \R$.  For $p=R_a p_0$ define a vector field $V(p) =
dR_{a,0}(\slashDx|_{x=0})$ where $dR_{a,0}:T_{p_0}(M) \to T_p(M)$ is the
derivative of the map $R_a$ at $p_0$ and $\slashDx|_{x=0}$ is the vector field
associated with the coordinate function $x$ evaluated at $p_0$.  Show that
$V(R_a p)=dR_aV(p)$ and find the expression $v(x)\slashDx$ for the vector field
$V$ in the coordinate system $x$.

\textbf{Solution.}  This author's notation is bizarre.  By way of decoding it,
note the following:

\begin{itemize}

\item $M$ is nothing more than the upper branch of a hyperbola whose asymptotes
are the lines $y=x$ and $y=-x$.

\item The ``it is easy to check'' statement is a roundabout way of saying
that the $R_a$'s form a one-parameter family of diffeomorphisms.

\item $R_a$ is hyperbolic rotation.  It carries $p$ to $R_a p$; what we call
$R_{a*}$, or $DR_a$ in coordinates (and what the author appears to call
$dR_a$), carries a tangent vector footed at $p$ to a tangent vector footed at
$R_a p$.

\item As usual, we convert from a flow $\phi_t$ to a vector field $X$ via
$$
	X_p = \frac{d\phi_t}{dt}\pickpipe{t=0}(p).
$$
Using the notation of this problem, this is
$$
	V(p) = \frac{dR_a}{da}\pickpipe{a=0}(p).
$$
\end{itemize}

Since $R_a$ is already linear, it is its own Jacobian:
\begin{eqnarray*}
	R_a(x,y)  &=& \colvectwo{\cosh a x + \sinh a y}{\sinh a x + \cosh a y} \\
	DR_a(x,y) &=& \prmatrix{\cosh a & \sinh a \\ \sinh a & \cosh a}
\end{eqnarray*}

To find $V(p)$ in $\R^2$ coordinates, use the definition:
\begin{eqnarray*}
	V(p) &=& \frac{dR_a}{da}\pickpipe{a=0}\colvectwo{x}{y} \\
	&=& \frac{d}{da}\pickpipe{a=0}
		\prmatrix{\cosh a & \sinh a \\ \sinh a & \cosh a}
		\colvectwo{x}{y} \\
	&=& \prmatrix{\sinh a & \cosh a \\ \cosh a & \sinh a}
		\colvectwo{x}{y} \pickpipe{a=0} \\
	&=& \prmatrix{0 & 1 \\ 1 & 0}
		\colvectwo{x}{y} \\
	&=& \colvectwo{y}{x}\pickpipe{\colvectwo{x}{y}}.
	%&=& y\fracDx + x \fracDy.
\end{eqnarray*}
Then $V(R_a p)$ is found by sending $x$ to $\cosh a x + \sinh a y$
and $y$ to $\sinh a x + \cosh a y$:
$$
	V(R_a p) =
	\colvectwo{\sinh a x + \cosh a y}{\cosh a x + \sinh a y}
		\pickpipe
	{\colvectwo{\cosh a x + \sinh a y}{\sinh a x + \cosh a y}}
$$
On the other hand, $DR_a V(p)$ is
\begin{eqnarray*}
	DR_a V(p) &=& \prmatrix{\cosh a & \sinh a \\ \sinh a & \cosh a}
		\colvectwo{y}{x}
			\pickpipe{
			\prmatrix{\cosh a & \sinh a \\ \sinh a & \cosh a}
			\colvectwo{x}{y}} \\
	&=& \colvectwo{\sinh a x + \cosh a y}{\cosh a x + \sinh a y}
			\pickpipe{
			\colvectwo{\cosh a x + \sinh a y}{\sinh a x + \cosh a y}}
\end{eqnarray*}
which is the same as $V(R_a p)$ as desired.

To find $\slashDx$ in the $x$ coordinate system, switch from $\R^2$
to graph coordinates using $y=\sqrt{x^2+1}$.  Then
\begin{eqnarray*}
	p &=& \colvectwo{x}{y} = \colcvectwo{x}{\sqrt{x^2+1}} \\
	\slashDx &=& \colcvectwo{1}{\frac{x}{\sqrt{x^2+1}}}.
\end{eqnarray*}
To find $V(p)$ in terms of this, remember that $V(p)$ was
\begin{eqnarray*}
	\colcvectwo{y}{x}\pickpipe{\colcvectwo{x}{y}}
	&=& \colcvectwo{\sqrt{x^2+1}}{x}\pickpipe{\colcvectwo{x}{y}} \\
	&=& \sqrt{x^2+1}\; \colcvectwo{1}{\frac{x}{\sqrt{x^2+1}}}
		\pickpipe{\colcvectwo{x}{y}} \\
	&=& \sqrt{x^2+1}\;\fracDx.
\end{eqnarray*}
That is, $v(x) = \sqrt{x^2+1}$.

%% ----------------------------------------------------------------
\subsection{Fall 2000 \#6.}

Let $p:\R\to\S^1$ be defined by $p(t)=e^{2\pi i t} = \cos(2\pi t) + i \sin(2\pi
t)$.  Note that $p$ is a covering space map.  Prove or give a counterexample to
the following statement:  if $f:\R\P^2\to \S^1$ is continuous then there exists
a continuous lift $\tilde f:\R\P^2 \to \R$ so that $f=p\circ \tilde f$.

\textbf{Solution.}  The lifting lemma tells us that there is a lift iff
$f_*(\pi_1(\R\P^2)) \le p_*(\pi_1(\R))$.  The former is cyclic-two; the latter
is trivial.  Therefore the lift does not exist.

%% ----------------------------------------------------------------
\subsection{Fall 2005 \#7.}

Let $X$ be the set of $2 \times 2$ upper-triangular complex matrices with
determinant 1.  Note that $X$ is a 4-dimensional dfferentiable real manifold.
For which integers $i$ do there exist closed $i$ forms on $X$ which are not
exact?

\textbf{Solution (Victor Piercey).}  First note what the question is really
asking:  for which $i$ is $H^i_{dR}(X)$ non-trivial?  Second, write down
what $X$ looks like:
\begin{eqnarray*}
	X &=& \left\{ \pcmatrix{a & b \\ 0 & c} : a,b,c\in \C; ac=1 \right\} \\
	&=& \left\{ \pcmatrix{a & b \\ 0 & 1/a} : a,b \in \C; a \ne 0 \right\} \\
	&=& \C^* \times \C.
\end{eqnarray*}
Now, $\C^*$ retracts to $\S^1$ and $\C$ retracts to a point.  Deformation
retracts preserve homology and cohomology, and we know that
$H^0_{dR}(\S^1)\cong \R$, $H^1_{dR}(\S^1)\cong \R$, and $H^i_{dR}(\S^1)\cong 0$
for all other $i$.  Thus, $X$ has non-trivial cohomology only for
$i=0$ and $i=1$.

%% ================================================================
\newpage
\section{Spring 2000}

%% ----------------------------------------------------------------
\subsection{Spring 2000 \#2.}

Let $M$ and $N$ be $C^\infty$ manifolds of the same dimension.  If $f:M \to N$
is a local diffeomorphism, show that $f(M)$ is open in $N$.

\textbf{Solution (Piercey/Schettler).} For each point $q \in M$ take an open
neighborhood $U_q$, which exists since $M$ is a manifold and hence locally
Euclidean.  For each $U_q$, restrict $f$ to $U_q$.  A local diffeomorphism is a
homeomorphism, hence an open map, taking open sets to open sets.  Unions are
respected by $f$, and the arbitrary union of open sets is again open.

%% ----------------------------------------------------------------
\subsection{Spring 2000 \#3.}

For the following vector fields $X$ and $Y$ and differential forms $\alpha$ and
$\beta$ on $\R^3$, calculate the Lie bracket $[X,Y]$ and the Lie derivatives
$L_X \alpha$, $L_Y \beta$, and $L_{[X,Y]}(\alpha\wedge\beta)$:
$$
	X = x\,\fracDx - z^2\,\fracDy, \quad
	Y = z\,\fracDy + x^3\,\fracDz, \quad
	\alpha = e^x\,dx + y\,dy + z\,dz, \quad
	\beta = dx\wedge dy \wedge dz.
$$

\textbf{Solution.}  Writing things vertically for visual convenience, and
omitting second-order derivatives which always cancel in the Lie bracket, I
have
\begin{eqnarray*}
	[X,Y] = XY - YX
%
	&=& \prmatrix{
		&x   \,\slashDx \\
		-&z^2\,\slashDy
	}
	\prmatrix{
		&z   \,\slashDy \\
		+&x^3\,\slashDz
	}
	-
	\prmatrix{
		&z   \,\slashDy \\
		+&x^3\,\slashDz
	}
	\prmatrix{
		&x   \,\slashDx \\
		-&z^2\,\slashDy
	} \\
%
	&=& 3x^3  \,\slashDz
	+   2x^3z \,\slashDy.
\end{eqnarray*}
Next, using Cartan's magic formula,
\begin{eqnarray*}
	L_X(\alpha) &=& d(X\ctt\alpha) + X\ctt(d\alpha) \\
	X\ctt\alpha &=&
		\prmatrix{
			&  x   \,\slashDx \\
			-& z^2 \,\slashDy \\
		} \ctt
		\prmatrix{
			&  e^x\,dx \\
			+& y  \,dy \\
			+& z  \,dz
		} \\
	&=& xe^x - z^2 y \\
	d(X\ctt \alpha) &=& (1+x)e^x\,dx - z^2 \,dy - 2yz\,dz \\
	d(\alpha) = 0
\end{eqnarray*}
and so
$$
	L_X(\alpha) = (1+x)e^x\,dx - z^2 \,dy - 2yz\,dz.
$$
Likewise,
\begin{eqnarray*}
	L_Y(\beta) &=& d(Y\ctt\beta) + Y\ctt(d\beta) \\
	Y\ctt\beta &=&
		\prmatrix{
			&  z   \,\slashDy \\
			+& x^3 \,\slashDz \\
		} \ctt
		\prmatrix{
			dx\wedge dy \wedge dz
		} \\
	&=& -z dx\wedge dz + x^3 \,dx \wedge dy \\
	d(Y\ctt \beta) &=& 0 \\
	d(\beta) &=& 0
\end{eqnarray*}
and so
$$
	L_Y(\beta) = 0.
$$
Last, $\alpha \wedge \beta = 0$ since there is $dx\wedge dx$,
$dy\wedge dy$, or $dz\wedge dz$ in each term.  Therefore
$$
	L_{[X,Y]}(\alpha\wedge \beta) =
	L_{[X,Y]}(0) = 0.
$$

%% ----------------------------------------------------------------
\subsection{Spring 2000 \#4.}

(a) Show that if $n>1$, the $2n$-dimensional sphere does not admit any
closed 2-form $\omega$ such that $\omega^n$ (the $n$th wedge power of $\omega$)
is a volume form.

(b) Exhibit a volume form for the 2-sphere $\S^2$.

\textbf{Solution.}  For part (a), note that a volume form (being top-level)
must be closed, yet must not be exact.  To see this, suppose otherwise.  Then
$\omega^n=d\eta$ for some $2n-1$ form $\eta$.  By Stokes' theorem, since
$\S^{2n}$ has no boundary,
$$
	\int_{\S^{2n}} \omega^n =
	\int_{\S^{2n}} d\eta =
	\int_{\D\S^{2n}} \eta = 0.
$$
But this is absurd since $\S^{2n}$ is compact and orientable, hence has
non-zero area.

It remains to use this intermediate result by showing that $\omega^n$ is exact,
hence unsuitable for a volume form.  Finding an $\eta$ such that
$d\eta=\omega^n$ may seem at first blush like guesswork.  However, since there
are only so many symbols to go around, and since $\eta$ has degree $2n-1$ while
$\omega$ has degree 2, I want to try
$$
	\eta=\omega^{n-1}\wedge \theta
$$
for some $1$-form $\theta$.  What else do I know about $\theta$?  Let's compute
$d\eta$.  This is
\begin{eqnarray*}
	d\eta &=& d\omega \wedge \omega \wedge \cdots \wedge \omega\wedge\theta \\
	&&+ \omega \wedge d\omega \wedge \cdots \wedge \omega\wedge\theta \\
	&&+ \cdots \\
	&&+ \omega \wedge \omega \wedge \cdots \wedge d\omega\wedge\theta \\
	&&+ \omega \wedge \omega \wedge \cdots \wedge \omega\wedge d\theta.
\end{eqnarray*}
(Recall that $d$ works through wedges as $d(\alpha\wedge\beta)
=\alpha\wedge\beta + (-1)^{\ord(\alpha)}\beta\wedge\alpha$, but $\omega$ has
degree 2, so I am not missing any minus signs.) Since $\omega$ is closed,
$d\omega=0$ and all but the last term disappears:
$$
	d\eta = \omega^{n-1} \wedge d\theta.
$$
Now I see that I need $d\theta=\omega$.  What guarantee do I have of that?
Well, for $n>1$, $\S^{2n}$ has trivial de Rham cohomology, so any closed
form (which $\omega$ is) is exact (so such a $\theta$ exists).
Then
$$
	d\eta = \omega^{n-1}\wedge\theta = \omega^{n-1}\wedge d\theta = \omega^n,
$$
so $\omega^n$ is exact, which is all that remained to be proved.

For part (b), see section \ref{sec:F99.7}.

%% ================================================================
\newpage
\section{Fall 1999}

%% ----------------------------------------------------------------
\subsection{Fall 1999 \#1.}

Find a linear fractional transformation from the upper half plane to the unit
disk sending $i$ to $0$, $0$ to $1$, and $\infty$ to $-1$.  Does there exist
such a linear fractional transformation that also sends $-1$ to $i$?

\textbf{Solution.}  A general linear fractional transformation is
$$
	f(z) = \frac{az+b}{cz+d}
$$
with $ad-bc\ne 0$.  There is an implicit form we can use; alternatively, we can
simply solve a system of three equations.  (Note that although there appear to
be four unknowns, the same LFT results if numerator and denominator are scaled
by the same non-zero constant.  Thus there are really only three degrees of
freedom.)  For the first point we have
$$
	\frac{ai+b}{ci+d} = 0 \;\implies\;
	ai+b = 0 \;\implies\;
	b = -ai
$$
so
$$
	f(z) = \frac{az-ai}{cz+d}.
$$
Using the second point,
$$
	\frac{a\cdot 0-ai}{c\cdot 0+d} = 1 \;\implies\;
	-ai = d
$$
so
$$
	f(z) = \frac{az-ai}{cz-ai}.
$$
For the third point, recall that we define
$$
	\lim_{z\to\infty}f(z) = \lim_{z\to 0}f(1/z).
$$
Then
$$
	f(1/z)
	= \frac{a/z-ai}{c/z-ai}
	= \frac{a-aiz}{c-aiz}; \qquad
	\lim_{z\to 0} f(1/z) = \frac{a}{c} = -1
$$
so
$$
	c = -a.
$$
Then
$$
	f(z) = \frac{az-ai}{-az-ai}
$$
for arbitrary non-zero $a$, so we may as well take $a=1$ and write
$$
	f(z) = \frac{z-i}{-z-i}
	= \frac{i-z}{i+z}.
$$

Recall that an LFT is uniquely specified by its image on three distinct
points, so if this LFT does not send $-1$ to $i$ as well as satisfying
the above three constraints, then no other LFT does either.  We compute
$$
	f(-1)
	= \frac{i+1}{i-1}
	= \left(\frac{i+1}{i-1}\right) \left(\frac{i+1}{i+1}\right)
	= \frac{2i}{-2} = -i \ne i.
$$

Furthermore, since an LFT is uniquely specified by its image on three distinct
points, if it does not send the upper half plane to the unit disk then there is
no fixing it.  To check that the upper half plane does in fact land in the unit
disk, though, let $z=x+iy$ with $y \ge 0$.  Then it suffices to show that
$|f(z)| \le 1$:
\begin{eqnarray*}
	\left| \frac{i-x-iy}{i+x+iy} \right|
	= \frac{|i-x-iy|}{|i+x+iy|}
	= \frac{|-x+(1-y)i|}{|x+(1+y)i|}
	= \sqrt{
		\frac{x^2 + (1-y)^2}{x^2 + (1+y)^2}
		}.
\end{eqnarray*}
This is less than or equal to 1 if
$$
	\frac{x^2 + (1-y)^2}{x^2 + (1+y)^2} \le 1
$$
which is true if
$$
	x^2 + (1-y)^2 \le x^2 + (1+y)^2
$$
which is true if
$$
	(1-y)^2 \le (1+y)^2
$$
which is true if
$$
	1 - 2y + y^2 \le 1 + 2y + y^2
$$
which is true if
$$
	- 2y \le 2y
$$
which is true since $y \ge 0$.


%% ----------------------------------------------------------------
\subsection{Fall 1999 \#3.}

Let $M$ be a smooth manifold and let $X$ be a vector field on $M$.  Show that
for any closed $k$-form $\omega$ on $M$, the Lie derivative $L_X\omega$ of
$\omega$ in the direction of $X$ is exact.

\textbf{Solution.}  Writing down the Lie derivative using Cartan's magic
formula, we have
$$
	L_X\omega = X\ctt(d\omega) + d(X\ctt\omega).
$$
Since $\omega$ is exact, $d\omega=0$; linearity of the contraction operator
gives $X\ctt 0 = 0$.  What remains is
$$
	L_X\omega = d(X\ctt\omega)
$$
which is what was to be shown.

%% ----------------------------------------------------------------
\subsection{Fall 1999 \#7.}
\label{sec:F99.7}

Let $\alpha$ be the two-form on $\R^3\setminus\{0\}$ given by
$$
	\alpha=\frac
		{x\,dy\wedge dz + y\,dx\wedge dz + z\,dx\wedge dy}
		{(x^2+y^2+z^2)^{3/2}}.
$$
Let $i: \S^2\to\R^3\setminus\{0\}$ be the inclusion map.  Evaluate
$$
	\int_{\S^2} i^*\alpha.
$$
Is $i^*\alpha$ exact?  Why?

\textbf{Solution.}  On $\S^2$ we have $x^2+y^2+z^2=1$ so, in ambient
coordinates,
$$
	i^*\alpha= {x\,dy\wedge dz + y\,dx\wedge dz + z\,dx\wedge dy}.
$$
Now observe that the unit normal field $\S^2$ (obtained from half its
gradient as usual) is $(x,y,z)$ footed at each point $(x,y,z)$,
i.e.
$$
	\nhat = x\,\slashDx + y\,\slashDy + z\,\slashDz,
$$
while the standard volume form on $\R^3$ is $dx\wedge dy \wedge dz$.
Contracting we obtain
\begin{eqnarray*}
	\nhat \ctt (dx \wedge dy \wedge dz) &=&
		x\,\slashDx \ctt (dx \wedge dy \wedge dz) +
		y\,\slashDy \ctt (dx \wedge dy \wedge dz) +
		z\,\slashDz \ctt (dx \wedge dy \wedge dz) \\
	&=&
		x\,dy \wedge dz - y\,dx \wedge dz + z\,dx \wedge dy
\end{eqnarray*}
which is to say that $\nhat \ctt (dx \wedge dy \wedge dz)$ is the standard area
form on $\S^2$.  But this is the same as $i^*\alpha$.

To evaluate $\int_{\S^2} i^*\alpha \ne 0$, I could state that the area of the
two-sphere is $4\pi$.  Or, I can compute this using spherical coordinates.  We
have
\begin{eqnarray*}
	\colvecthree{x}{y}{z} &=& \colvecthree
		{\sin\phi\cos\theta}
		{\sin\phi\sin\theta}
		{\cos\phi} \\
	i^*(dx) &=& \cos\phi\cos\theta\,d\phi - \sin\phi\sin\theta\,d\theta \\
	i^*(dy) &=& \cos\phi\sin\theta\,d\phi + \sin\phi\cos\theta\,d\theta \\
	i^*(dz) &=& -\sin\phi \,d\phi \\
	i^*(x\, dy \wedge dz) &=& \sin^3\phi\cos^2\theta \,d\phi\wedge d\theta\\
	i^*(-y\,dx \wedge dz) &=& \sin^3\phi\sin^2\theta\,d\phi\wedge d\theta \\
	i^*(z\, dx \wedge dy) &=&
		(\sin\phi\cos^2\phi\cos^2\theta + \sin\phi\cos^2\phi\sin^2\theta)
		\,d\phi\wedge d\theta \\
	&=& \sin\phi\cos^2\phi \,d\phi\wedge d\theta \\
	i^*(\alpha) &=&
		(\sin^3\phi\cos^2\theta +
		\sin^3\phi \sin^2\theta +
		\sin\phi\cos^2\phi) \,d\phi\wedge d\theta \\
	&=& (\sin^3\phi +\sin\phi\cos^2\phi) \,d\phi\wedge d\theta \\
	&=& \sin\phi(\sin^2\phi + \cos^2\phi) \,d\phi\wedge d\theta \\
	&=& \sin\phi \,d\phi\wedge d\theta.
\end{eqnarray*}
Then by Tonelli's theorem, using the non-negativity of the integrand, we have
\begin{eqnarray*}
	\int_{\S^2} i^*\alpha &=&
		\int_{\theta=0}^{\theta=2\pi} \left(\int_{\phi=0}^{\phi=\pi} \sin\phi d\phi \right) d\theta \\
	&=& 2\pi \left[ -\cos\phi\right]_{\phi=0}^{\phi=\pi} \\
	&=& 4\pi.
\end{eqnarray*}

Since $4\pi \ne 0$, this shows $i^*\alpha$ is not exact:  If it were, say
$\alpha=d\beta$ for some one-form $\beta$ on $\S^2$, then by Stokes' theorem
and since $\S^2$ has empty boundary,
$$
	\int_{\S^2} \alpha = \int_{\S^2} d\beta \int_{\D \S^2} \beta = 0.
$$
Contrapositively, since we found the integral to be non-zero, $\alpha$ is not exact.

%% ================================================================
\newpage
\section{Spring 1999}

%% ----------------------------------------------------------------
\subsection{Spring 1999 \#2.}

Explain why you need at least two coordinate charts to cover a compact
manifold.

\textbf{Solution.}  A coordinate chart is a homeomorpmism from within $M$ to an
open subset of $\R^m$ for some $m$.  If a single coordinate chart covered $M$,
then $M$ would be (globally) homeomorphic to $\R^m$.  Homeomorphisms preserve
compactness, yet $M$ is compact while $\R^m$ (for $m>0$) is not.  This is
absurd.

%% ----------------------------------------------------------------
\subsection{Spring 1999 \#3.}

Construct a nowhere-vanishing vector field $X$ on $\S^1 = \{(x,y)\in
\R:x^2+y^2=1\}$.  For the function $f(x,y)=xy^2$ on $\S^1$, calculate the
function $Xf$.

\textbf{Solution.} First, the construction.  The circle has unit normal given
by its normalized gradient, namely,
$$
	\nhat = \colvectwo{x}{y}\pickpipe{\colvectwo{x}{y}}.
$$
We may write down the vector field
$$
	X = \colvectwo{-y}{x}\pickpipe{\colvectwo{x}{y}}.
$$
This is tangent to $\S^1$ since it is perpendicular to the normal field, as
evidenced by the fact that $\nhat \cdot X = -xy+xy=0$.  Furthermore, $X$ is
nowhere vanishing since its norm is $\sqrt{(-y)^2+x^2}=\sqrt{x^2+y^2}=1$.

\begin{center}* * *\end{center}

Using ambient coordinates, we have from above
$$
	X = -y \fracDx \;+\; x \fracDy.
$$
Then
$$
	Xf = \left(-y \fracDx \;+\; x \fracDy\right)\; xy^2
	= -y^3 + 2x^2y.
$$

Using polar coordinates, we have
$$
	x = \cos\theta,\quad
	y = \sin\theta,\quad
	X = \fracDtheta = \colvectwo
		{\slashDtheta(\cos\theta)}
		{\slashDtheta(\sin\theta)}
	= \colvectwo
		{-\sin\theta}
		{\cos\theta}
	= \colvectwo{-y}{x}\pickpipe{\colvectwo{x}{y}}.
$$
Then
$$
	Xf = \fracDtheta\left(\cos\theta\sin^2\theta\right)
	= -\sin^3\theta + 2\cos^2\theta\sin\theta.
$$

%% ----------------------------------------------------------------
\subsection{Spring 1999 \#4.}

Let $M$ be an $n$-dimensional manifold and let $\omega$ be a differential form
on $M$ of even degree.  Show that the form $\omega \wedge d\omega$ is always
exact.

\textbf{Solution.}  Let $k$ be the degree of $\omega$.  Then $\omega \wedge
d\omega$ has degree $2k+1$.  If $2k+1 > n$ then $\omega\wedge d\omega = 0$
which is $d$ of 0, hence exact.

For the more interesting case $2k+1 \le n$, well, there are only finitely many
symbols here so I will guess $d(\omega\wedge\omega)$.  Writing this out I have
\begin{eqnarray*}
	d(\omega \wedge \omega) &=& d\omega \wedge \omega + (-1)^k \omega \wedge d\omega \\
	&=& d\omega \wedge \omega + \omega \wedge d\omega \\
	&=& 2 \omega \wedge d\omega
\end{eqnarray*}
since $k$ is even.  (Recall that for a $k$-form $\eta$ and an $\ell$-form
$\theta$, $\theta\wedge\eta = (-1)^{k\ell} \eta\wedge\theta$.  Since here we
have $k$ even, $d\omega\wedge\omega=\omega\wedge d\omega$.)  I overguessed by a
factor of two; I amend that to
$$
	d\left(\frac{\omega\wedge\omega}{2}\right) = \omega\wedge d\omega.
$$

%% ----------------------------------------------------------------
\subsection{Spring 1999 \#6.}

For $n \ge 1$, construct an $n$-form on the $n$-sphere $\S^n$ that represents a
non-trivial de Rham cohomology class in $H^n_{dR}(\S^n)$.  Explain your work.

\textbf{Solution.}  Let $\omega$ be the form, explicitly constructed as
$$
	\omega = \nhat \ctt dV
$$
where $\nhat$ is the unit normal to $\S^n$ and
$$
	dV=dx_1 \wedge \cdots \wedge dx_{n+1}
$$
is the volume form on $\R^{n+1}$.  Since $\S^n$ is the level set of
$$
	x_1^2 + \ldots + x_{n+1}^2 = 1,
$$
a normal is found by the gradient of the left-hand side, namely,
$$
	\colvecthree{2 x_1}{\vdots}{2 x_{n+1}}.
$$
The unit normal is then
$$
	\colvecthree{x_1}{\vdots}{x_{n+1}}\pickpipe{\colvecthree{x_1}{\vdots}{x_{n+1}}}
	= x_1 \fracDx_1 + \ldots + x_{n+1} \fracDx_{n+1}.
$$
Since
$$
	\fracDx_j \ctt (dx_1 \wedge \cdots \wedge dx_{n+1})
	= (-1)^{j-1}
	dx_1 \wedge \cdots \wedge \widehat{dx_j} \wedge\cdots \wedge dx_{n+1},
$$
where the overhat indicates omission, the contraction is
$$
	\omega = 
	x_1 dx_2 \wedge\cdots\wedge dx_{n+1}
	\;-\; x_2 dx_1 \wedge dx_3 \wedge \cdots\wedge dx_{n+1}
	\;+\; \cdots
	\;+\; (-1)^{n}
	x_{n+1} dx_1 \wedge dx_3 \wedge \cdots\wedge dx_n.
$$

For $\omega$ to represent a non-trivial de Rham cohomology class, it must be
closed but not exact.  It is closed since it is top-level on $\S^n$.
If it were exact, say $\omega=d\eta$, then by Stokes' theorem
$$
	\int_{\S^n}\omega =
	\int_{\S^n}d\eta =
	\int_{\D\S^n}\eta = 0
$$
since $\S^n$ has empty boundary.  But $\S^n$ is a compact, orientable manifold
and thus has a one-dimensional space of volume forms, of which we have found a
specific non-zero element [xxx can't cite linear independence with ambient
coordinates \ldots].  Thus $\int_{\S^n}\omega \ne 0$ which is a contradiction.
Therefore there is no such $\eta$.  This is all that remained to be shown.

% x dy - y dx
% x dy dz - y dx dz + z dx dy

\end{document}
